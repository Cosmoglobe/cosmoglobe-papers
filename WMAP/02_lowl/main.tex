\documentclass[twocolumn]{../../common/aa}
%\documentclass[referee]{aa}

\usepackage{graphicx}
\usepackage{amsmath,amsfonts,amssymb}
\usepackage{txfonts}
\usepackage{color}
\usepackage{natbib}
\usepackage{float}
%\usepackage{stfloats}
\usepackage{dblfloatfix}
\usepackage{afterpage}
\usepackage{ifthen}
\usepackage[morefloats=12]{morefloats}
\usepackage{placeins}
\usepackage{multicol}
\bibpunct{(}{)}{;}{a}{}{,}
\usepackage[switch]{lineno}
\definecolor{linkcolor}{rgb}{0.6,0,0}
\definecolor{citecolor}{rgb}{0,0,0.75}
\definecolor{urlcolor}{rgb}{0.12,0.46,0.7}
\usepackage[breaklinks, colorlinks, urlcolor=urlcolor,
    linkcolor=linkcolor,citecolor=citecolor,pdfencoding=auto]{hyperref}
\hypersetup{linktocpage}
\usepackage{bold-extra}
\usepackage{xcolor}

%\usepackage[grid,
%  gridcolor=red!20,
%  subgridcolor=green!20,
%  gridunit=cm]{eso-pic}
%



\def\setsymbol#1#2{\expandafter\def\csname #1\endcsname{#2}}
\def\getsymbol#1{\csname #1\endcsname}

\def\Planck{\textit{Planck}}

\def\HeJT{$^4$He-JT}

\def\allearlypapers{\nocite{planck2011-1.1, planck2011-1.3, planck2011-1.4, planck2011-1.5, planck2011-1.6, planck2011-1.7, planck2011-1.10, planck2011-1.10sup, planck2011-5.1a, planck2011-5.1b, planck2011-5.2a, planck2011-5.2b, planck2011-5.2c, planck2011-6.1, planck2011-6.2, planck2011-6.3a, planck2011-6.4a, planck2011-6.4b, planck2011-6.6, planck2011-7.0, planck2011-7.2, planck2011-7.3, planck2011-7.7a, planck2011-7.7b, planck2011-7.12, planck2011-7.13}}

\def\alltwentythirteenresultspapers{\nocite{planck2013-p01, planck2013-p02, planck2013-p02a, planck2013-p02d, planck2013-p02b, planck2013-p03, planck2013-p03c, planck2013-p03f, planck2013-p03d, planck2013-p03e, planck2013-p01a, planck2013-p06, planck2013-p03a, planck2013-pip88, planck2013-p08, planck2013-p11, planck2013-p12, planck2013-p13, planck2013-p14, planck2013-p15, planck2013-p05b, planck2013-p17, planck2013-p09, planck2013-p09a, planck2013-p20, planck2013-p19, planck2013-pipaberration, planck2013-p05, planck2013-p05a, planck2013-pip56, planck2013-p06b, planck2013-p01a}}

\def\alltwentyfifteenresultspapers{\nocite{planck2014-a01, planck2014-a03, planck2014-a04, planck2014-a05, planck2014-a06, planck2014-a07, planck2014-a08, planck2014-a09, planck2014-a11, planck2014-a12, planck2014-a13, planck2014-a14, planck2014-a15, planck2014-a16, planck2014-a17, planck2014-a18, planck2014-a19, planck2014-a20, planck2014-a22, planck2014-a24, planck2014-a26, planck2014-a28, planck2014-a29, planck2014-a30, planck2014-a31, planck2014-a35, planck2014-a36, planck2014-a37, planck2014-ES}}

\newbox\tablebox    \newdimen\tablewidth
\def\leaderfil{\leaders\hbox to 5pt{\hss.\hss}\hfil}
\def\endPlancktable{\tablewidth=\columnwidth 
    $$\hss\copy\tablebox\hss$$
    \vskip-\lastskip\vskip -2pt}
\def\endPlancktablewide{\tablewidth=\textwidth 
    $$\hss\copy\tablebox\hss$$
    \vskip-\lastskip\vskip -2pt}
\def\tablenote#1 #2\par{\begingroup \parindent=0.8em
    \abovedisplayshortskip=0pt\belowdisplayshortskip=0pt
    \noindent
    $$\hss\vbox{\hsize\tablewidth \hangindent=\parindent \hangafter=1 \noindent
    \hbox to \parindent{$^#1$\hss}\strut#2\strut\par}\hss$$
    \endgroup}
\def\doubleline{\vskip 3pt\hrule \vskip 1.5pt \hrule \vskip 5pt}

\def\L2{\ifmmode L_2\else $L_2$\fi}
\def\dtt{\Delta T/T}
\def\DeltaT{\ifmmode \Delta T\else $\Delta T$\fi}
\def\deltat{\ifmmode \Delta t\else $\Delta t$\fi}
\def\fknee{\ifmmode f_{\rm knee}\else $f_{\rm knee}$\fi}
\def\Fmax{\ifmmode F_{\rm max}\else $F_{\rm max}$\fi}
\def\solar{\ifmmode{\rm M}_{\mathord\odot}\else${\rm M}_{\mathord\odot}$\fi}
\def\Msolar{\ifmmode{\rm M}_{\mathord\odot}\else${\rm M}_{\mathord\odot}$\fi}
\def\Lsolar{\ifmmode{\rm L}_{\mathord\odot}\else${\rm L}_{\mathord\odot}$\fi}
\def\inv{\ifmmode^{-1}\else$^{-1}$\fi}
\def\mo{\ifmmode^{-1}\else$^{-1}$\fi}
\def\sup#1{\ifmmode ^{\rm #1}\else $^{\rm #1}$\fi}
\def\expo#1{\ifmmode \times 10^{#1}\else $\times 10^{#1}$\fi}
\def\,{\thinspace}
\def\lsim{\mathrel{\raise .4ex\hbox{\rlap{$<$}\lower 1.2ex\hbox{$\sim$}}}}
\def\gsim{\mathrel{\raise .4ex\hbox{\rlap{$>$}\lower 1.2ex\hbox{$\sim$}}}}
\let\lea=\lsim
\let\gea=\gsim
\def\simprop{\mathrel{\raise .4ex\hbox{\rlap{$\propto$}\lower 1.2ex\hbox{$\sim$}}}}
\def\deg{\ifmmode^\circ\else$^\circ$\fi}
\def\pdeg{\ifmmode $\setbox0=\hbox{$^{\circ}$}\rlap{\hskip.11\wd0 .}$^{\circ}
          \else \setbox0=\hbox{$^{\circ}$}\rlap{\hskip.11\wd0 .}$^{\circ}$\fi}
\def\arcs{\ifmmode {^{\scriptstyle\prime\prime}}
          \else $^{\scriptstyle\prime\prime}$\fi}
\def\arcm{\ifmmode {^{\scriptstyle\prime}}
          \else $^{\scriptstyle\prime}$\fi}
\newdimen\sa  \newdimen\sb
\def\parcs{\sa=.07em \sb=.03em
     \ifmmode \hbox{\rlap{.}}^{\scriptstyle\prime\kern -\sb\prime}\hbox{\kern -\sa}
     \else \rlap{.}$^{\scriptstyle\prime\kern -\sb\prime}$\kern -\sa\fi}
\def\parcm{\sa=.08em \sb=.03em
     \ifmmode \hbox{\rlap{.}\kern\sa}^{\scriptstyle\prime}\hbox{\kern-\sb}
     \else \rlap{.}\kern\sa$^{\scriptstyle\prime}$\kern-\sb\fi}
\def\ra[#1 #2 #3.#4]{#1\sup{h}#2\sup{m}#3\sup{s}\llap.#4}
\def\dec[#1 #2 #3.#4]{#1\deg#2\arcm#3\arcs\llap.#4}
\def\deco[#1 #2 #3]{#1\deg#2\arcm#3\arcs}
\def\rra[#1 #2]{#1\sup{h}#2\sup{m}}
\def\page{\vfill\eject}
\def\dots{\relax\ifmmode \ldots\else $\ldots$\fi}
\def\WHzsr{\ifmmode $W\,Hz\mo\,sr\mo$\else W\,Hz\mo\,sr\mo\fi}
\def\mHz{\ifmmode $\,mHz$\else \,mHz\fi}
\def\GHz{\ifmmode $\,GHz$\else \,GHz\fi}
\def\mKs{\ifmmode $\,mK\,s$^{1/2}\else \,mK\,s$^{1/2}$\fi}
\def\muKs{\ifmmode \,\mu$K\,s$^{1/2}\else \,$\mu$K\,s$^{1/2}$\fi}
\def\muKRJs{\ifmmode \,\mu$K$_{\rm RJ}$\,s$^{1/2}\else \,$\mu$K$_{\rm RJ}$\,s$^{1/2}$\fi}
\def\muKHz{\ifmmode \,\mu$K\,Hz$^{-1/2}\else \,$\mu$K\,Hz$^{-1/2}$\fi}
\def\MJysr{\ifmmode \,$MJy\,sr\mo$\else \,MJy\,sr\mo\fi}
\def\MJysrmK{\ifmmode \,$MJy\,sr\mo$\,mK$_{\rm CMB}\mo\else \,MJy\,sr\mo\,mK$_{\rm CMB}\mo$\fi}
\def\microns{\ifmmode \,\mu$m$\else \,$\mu$m\fi}
\def\micron{\microns}
\def\muK{\ifmmode \,\mu$K$\else \,$\mu$\hbox{K}\fi}
\def\microK{\ifmmode \,\mu$K$\else \,$\mu$\hbox{K}\fi}
\def\muW{\ifmmode \,\mu$W$\else \,$\mu$\hbox{W}\fi}
\def\kms{\ifmmode $\,km\,s$^{-1}\else \,km\,s$^{-1}$\fi}
\def\kmsMpc{\ifmmode $\,\kms\,Mpc\mo$\else \,\kms\,Mpc\mo\fi}

\providecommand{\sorthelp}[1]{}


\def\WMAP{\emph{WMAP}}
\def\WMAPnine{\emph{WMAP9}}
\def\COBE{\emph{COBE}}
\def\wmap{\emph{WMAP}}
\def\planck{\emph{Planck}}
\def\Planck{\emph{Planck}}
\def\LCDM{$\Lambda$CDM}
\def\ffp{FFP6}
\def\unionmask{U73}
\def\nside{N_{\mathrm{side}}}

\def\healpix{\texttt{HEALPix}}
\def\commander{\texttt{Commander}}
\def\commanderone{\texttt{Commander1}}
\def\commandertwo{\texttt{Commander2}}
\def\commanderthree{\texttt{Commander3}}
\def\ruler{\texttt{Ruler}}
\def\comrul{\texttt{Commander-Ruler}}
\def\CR{\texttt{C-R}}
\def\nilc{\texttt{NILC}}
\def\gnilc{\texttt{GNILC}}
\def\sevem{\texttt{SEVEM}}
\def\smica{\texttt{SMICA}}
\def\CamSpec{\texttt{CamSpec}}
\def\Plik{\texttt{Plik}}
\def\XFaster{\texttt{XFaster}}
\def\sroll2{\texttt{SRoll2}}

\renewcommand{\d}[0]{\vec{d}}
\renewcommand{\t}[0]{\vec{t}}
\newcommand{\A}[0]{\mathrm{A}}
\newcommand{\B}[0]{\mathrm{B}}
\newcommand{\Y}[0]{\tens{Y}}
\newcommand{\n}[0]{\vec{n}}
\newcommand{\red}[0]{\color{red}}
\newcommand{\green}[0]{\color{green}}
\newcommand{\s}[0]{\vec{s}}
%\renewcommand{\a}[0]{\vec{a}}
\newcommand{\m}[0]{\vec{m}}
\newcommand{\f}[0]{\vec{f}}
\newcommand{\F}[0]{\tens{F}}
\newcommand{\T}[0]{\tens{T}}
\newcommand{\Cp}[0]{\tens{C}}
\renewcommand{\L}[0]{\tens{L}}
\newcommand{\g}[0]{\vec{g}}
\newcommand{\N}[0]{\tens{N}}
\newcommand{\M}[0]{\tens{M}}
\newcommand{\iN}[0]{\tens{N}^{-1}}
\newcommand{\iM}[0]{\tens{M}^{-1}}
\newcommand{\w}[0]{\vec{w}}
\renewcommand{\S}[0]{\tens{S}}
\renewcommand{\r}[0]{\vec{r}}
\renewcommand{\u}[0]{\vec{u}}
\newcommand{\q}[0]{\vec{q}}
\renewcommand{\v}[0]{\vec{v}}
\renewcommand{\P}[0]{\tens{P}}
\newcommand{\dt}[0]{d_t}
\newcommand{\di}[0]{d_i}
\newcommand{\nt}[0]{n_t}
\newcommand{\st}[0]{s_t}
\newcommand{\mt}[0]{m_t}
\newcommand{\ft}[0]{f_t}
\newcommand{\Te}[0]{T_{\rm e}}
\newcommand{\EM}[0]{\rm EM}
\newcommand{\mathsc}[1]{{\normalfont\textsc{#1}}}
\newcommand{\hi}{\ensuremath{\mathsc {Hi}}}
\newcommand{\bpbold}{\bfseries{\scshape{BeyondPlanck}}}
\newcommand{\BP}{\textsc{BeyondPlanck}}
\newcommand{\bp}{\textsc{BeyondPlanck}}
\newcommand{\cosmoglobe}{\textsc{Cosmoglobe}}
\newcommand{\Cosmoglobe}{\textsc{Cosmoglobe}}
\newcommand{\lfi}[0]{LFI}
\newcommand{\hfi}[0]{HFI}
\newcommand{\npipe}[0]{\texttt{NPIPE}}
\newcommand{\K}[0]{\textit K}
\newcommand{\Ka}[0]{\textit{Ka}}
\newcommand{\Q}[0]{\textit Q}
\newcommand{\V}[0]{\textit V}
\newcommand{\W}[0]{\textit W}
\newcommand{\e}{\mathrm e}
\newcommand{\cvar}{\ensuremath{c(\vartheta, \varphi, \psi)}}

\usepackage[T1]{fontenc}


\def\bC{\tens{C}}
\def\ba{\vec{a}}
\def\ncha{N_\mathrm{cha}}
\def\nfg{N_\mathrm{fg}}

\newcommand{\ncorr}{\vec n_\mathrm{corr}}
\newcommand{\Dbp}{\Delta_\mathrm{bp}}

%\modulolinenumbers[5]
%\linenumbers

\newcommand{\includegraphicsdpi}[3]{
    \pdfimageresolution=#1  % Change the dpi of images
    \includegraphics[#2]{#3}
    \pdfimageresolution=72  % Change it back to the default
}

\renewcommand{\topfraction}{1.0}	% max fraction of floats at top
    \renewcommand{\bottomfraction}{1.0}	% max fraction of floats at bottom
    %   Parameters for TEXT pages (not float pages):
    \setcounter{topnumber}{2}
    \setcounter{bottomnumber}{2}
    \setcounter{totalnumber}{4}     % 2 may work better
    \setcounter{dbltopnumber}{2}    % for 2-column pages
    \renewcommand{\dbltopfraction}{0.9}	% fit big float above 2-col. text
    \renewcommand{\textfraction}{0.04}	% allow minimal text w. figs
    %   Parameters for FLOAT pages (not text pages):
    \renewcommand{\floatpagefraction}{0.9}	% require fuller float pages
	% N.B.: floatpagefraction MUST be less than topfraction !!
    \renewcommand{\dblfloatpagefraction}{0.9}	% require fuller float pages

\def\adj{^{\dagger}}
\def\tp{^{\rm T}}
\def\inv{^{-1}}
\def\lm{{\ell m}}

\begin{document}

\title{\bfseries{\Cosmoglobe\ II. Preliminary implications for large-scale\\ CMB polarization with improved WMAP sky maps}}
%This author list corresponds to \title{Author list for L04\_CMB\_Foregrounds\_Extraction}
%Prepared by M. Lopez-Caniego (Marcos.Lopez.Caniego@sciops.esa.int), ESAC/ESA
%This version is from Thu Jul 12 18:11:48 2018 CET
%\subtitle{There are 152 co-authors in this list}
\newcommand{\oslo}[0]{1}
%\newcommand{\MIT}[0]{2}
\newcommand{\milanoA}[0]{2}
\newcommand{\milanoB}[0]{3}
\newcommand{\milanoC}[0]{4}
\newcommand{\triesteB}[0]{5}
\newcommand{\planetek}[0]{6}
\newcommand{\princeton}[0]{7}
\newcommand{\jpl}[0]{8}
\newcommand{\helsinkiA}[0]{9}
\newcommand{\helsinkiB}[0]{10}
\newcommand{\nersc}[0]{11}
\newcommand{\haverford}[0]{12}
\newcommand{\mpa}[0]{13}
\newcommand{\triesteA}[0]{14}
\newcommand{\iia}[0]{2}

\author{\small
J.~R.~Eskilt\inst{\oslo}\thanks{Corresponding author: J.~R.~Eskilt; \url{j.r.eskilt@astro.uio.no}}
\and
K.~Lee\inst{\oslo}
\and
D.~J.~Watts\inst{\oslo}
\and
S.~Nerval\inst{\oslo}
\and
et al.
}
\institute{\small
        Institute of Theoretical Astrophysics, University of Oslo, Blindern, Oslo, Norway \goodbreak
}

%\authorrunning{From BeyondPlanck to Cosmoglobe}
\authorrunning{Watts et al.}
\titlerunning{\cosmoglobe\ \wmap{}  analysis}

\abstract{
    We present the first joint analysis of \textit{WMAP} and \textit{Planck} LFI data, presenting maps that have been generated from a fully consistent joint treatment, including the sampling of sky signals and instrumental properties. The joint analysis approach yields improved \textit{WMAP} data with better treatment of poorly constrained modes, as well as the first fully optimal sampling of all nine years of data.  We also improve on the \textsc{BeyondPlanck} analysis, by reducing poorly measured modes in LFI polarization. In particular, we find a $\sim4\,\mathrm{\mu K}$ change in the 30~GHz channel as a result of including the higher signal-to-noise \WMAP\ \textit K-band maps. The \WMAP\ maps we present are free of previously documented systematic effects, and have an $x$\% reduction in the white noise level. As the first release of \textsc{Cosmoglobe} products, the maps from this analysis should be considered both a considerable improvement over previous analyses, as well as the first iteration of future joint analyses with other data, including, e.g., the ground-based QUIET experiment and the DIRBE instrument aboard \textit{COBE}.
}

\keywords{ISM: general -- Cosmology: observations, polarization,
    cosmic microwave background, diffuse radiation -- Galaxy:
    general}

\maketitle

%\hypersetup{linkcolor=black}
\tableofcontents
%\hypersetup{linkcolor=red} 




\section{Introduction}
\label{sec:introduction}

A to-do list:

\begin{itemize}
	\item Find the time it takes for each beam to cross itself.
	\item Fix AME model (\textit{I'm not sure what motivated this, perhaps not necessary?}
	\item Fix noise model (\textit{Explained because of the Bessel filter plus linear trend})
\end{itemize}

A table to include
\begin{itemize}
	\item Spin rate -- 0.464\,rpm (7.57 mHz), but translations to $2.6$ degrees per second in boresight?
	\item Precession -- 1 rev/hour (0.3 mHz)
	\item Signal bandwidth extends from 0.008--8 Hz \citep{jarosik2003a}
	\item Beam size in degrees -- 0.88, 0.66, 0.51, 0.35, 0.22.
\end{itemize}


The cosmic microwave background (CMB) is the most direct probe of the initial state of the Universe. Since the initial discovery of the CMB \citep{penzias:1965}, subsequent experiments have continually refined the measurements, to the extent that the \WMAP\ results are generally considered bringing cosmology into the regime of precision science \citep{bennett2012}. Prior to \WMAP, it was common for CMB experiments to be superseded by more sensitive successors, with the noteworthy exceptions of \COBE/FIRAS and \COBE/DIRBE.

The \planck\ experiment, rather than superseding \WMAP, consistently used \WMAP\ data in its calibration, component separation, and cosmological analyses. The most direct comparison between \WMAP\ and \Planck\ is through analysis of the two experiments' frequency maps, as \WMAP's \K, \Ka, \Q, \V, and \W\ maps are interleaved by the \Planck\ LFI's 30, 44, and 70\,GHz bands. Since the initial \Planck\ data release, there have been several analyses comparing the two experiments by members of the \WMAP\ team \citep{larson2014,addison:2016,huang:2018,weiland:2018,weiland:2022} and by the \Planck\ team \citep{planck2014-a13,planck2016-l06,planck2016-l05}.

While the \WMAP\ low-level analysis has remained stable since \citet{bennett2012}, there has been continued work on \Planck\ time-ordered data processing, notably \bp\ for the LFI instrument \citep{bp01}, \sroll2\ for the HFI instrument \citep{delouis:2019}, and \Planck\ DR4 for both LFI and HFI \citep[\npipe,][]{npipe}. The LFI instrument in particular has had several systematics mitigated by improved analysis, particularly a smoothed gain solution and an improved noise model \citep{npipe,bp06,bp07,bp10}. When comparing \WMAP\ \K-band with the \Planck\ LFI data, the residuals are mainly characterized by \WMAP's poorly measured modes, which can be seen clearly in Figures 50 and 51 of \citet{npipe} and Figures 4 and 7 of \citet{bp14}.

One of the primary outcomes of the \BP\ project is that end-to-end analysis of a dataset with poorly measured modes can be mitigated by a joint analysis with another dataset that measures these modes well. In particular, \Planck\ LFI had large scale polarizated modes aligned with the instrument's scan strategy, induced by relative errors between different polarization-sensitive radiometers \citep{bp07}. The \bp\ project mitigated this by using \WMAP's polarized \Ka--\V\ maps for component separation, where these modes were well-measured. In order to properly combine these datasets, the polarized maps were the $N_\mathrm{side}=16$ \healpix\footnote{\url{http://healpix.sourceforge.net} \citep{gorski2005}} products with a pixel-pixel covariance matrix that explicitly projected out the poorly measured modes.

In principle, the \Planck\ experiment can be used to identify \WMAP's poorly measured modes in the same way that \WMAP\ removed \Planck's poorly measured modes. This was shown in \citet{bp17}, in which \WMAP\ data was calibrated against the \bp\ sky model, and the resulting maps differed from the \WMAPnine\ products mainly through the lack of the poorly measured modes. This work mainly functioned as a demonstration that the \commanderthree\ framework could be applied to the \WMAP\ dataset, and was not a true end-to-end analysis.

In this work, we present the first joint TOD analysis in the \cosmoglobe\footnote{\url{cosmoglobe.uio.no}} framework, in which we analyze the full \WMAP\ dataset along with time-ordered \Planck\ LFI data. In Sect.~\ref{sec:methods}, we review the \cosmoglobe\ statistical framework and the data processing for \Planck\ LFI and \WMAP\ in the \commanderthree\ pipeline. In Sect.~\ref{sec:freqmaps}, we present the \Planck\ and \WMAP\ joint frequency maps, and compare these frequency maps with the fiducial analyses in Sect.~\ref{sec:comparison}. We discuss outstanding systematic errors and the propagation of uncertainty in Sect.~\ref{sec:systematics}. We summarize our results and lay a path forward in Sect.~\ref{sec:conclusions}.



\textbf{Quotes regarding low-multipole null-test failures.}

\begin{itemize}
	\item \citet{page2007}, Sect.~5.2 -- 
		``When the W band and its cross-spectra are added to the mix, we find $\mathrm{PTE}<0.03$ for $l=5,7,9$, although all other values of $l$ give reasonable values. For BB, all frequency combinations yield reasonable PTEs for all $l$. Thus, there is a residual signal in our power spectra that we do not yet understand. It is evident in the W band in EE at $l=7$ and to a lesser degree at $l=5$ and $l=9$. We see no clear evidence of it anywhere else.''
		
		Also, due to the scan strategy (Sect.~5), ``\ldots Note in particular that we expect relatively larger error bars on $l=2,5,7$ in EE and on $l=3$ in BB.''
	\item \citet{hinshaw2009}, Sect.~6.1 -- 
		``Both spectrum estimates show excess power relative to the model, with the most puzzling multipole being $l=7$ (EE)\ldots It is worth recalling that $l=7$ EE, like $l=3$ BB, is a mode that is relatively poorly measured\ldots The W-band BB data also exhibit unusual behavior at $l=2,3$.''
	\item \citet{jarosik2010}, Sect.~4.1 --
		``\ldots the overall results do no (\textit{sic}) fully explain the excess variance observed in all the W-band $\ell\leq 7$ polarization multipoles. The low-$\ell$ W-band polarization data therefore continue to be excluded from cosmological analysis.''
\end{itemize}

\section{Constraining poorly measured modes in WMAP with \Cosmoglobe}
\label{sec:methods}



\begin{table}
\newdimen\tblskip \tblskip=5pt
\caption{Difference map $\chi^2$ statistics.  }
\label{tab:chisq}
\vskip -4mm
\footnotesize
\setbox\tablebox=\vbox{
 \newdimen\digitwidth
 \setbox0=\hbox{\rm 0}
 \digitwidth=\wd0
 \catcode`*=\active
 \def*{\kern\digitwidth}
%
  \newdimen\dpwidth
  \setbox0=\hbox{.}
  \dpwidth=\wd0
  \catcode`!=\active
  \def!{\kern\dpwidth}
%
  \halign{\hbox to 2.5cm{#\leaderfil}\tabskip 2em&
    \hfil$#$\hfil \tabskip 2em&
    \hfil$#$\hfil \tabskip 2em&    
    \hfil$#$\hfil \tabskip 0em\cr
\noalign{\doubleline}
\omit\hfil\sc Difference \hfil& \chi^2_{\mathrm{uncorr}} & \chi^2_{\mathrm{corr}} & \Delta \chi^2\cr
\noalign{\vskip 3pt\hrule\vskip 5pt}
$0.32\times$K1 $-$ Ka1  & 4291  & 4287  & **4 \cr
Q1 $-$ Q2         & 4500  & 4380  & 120 \cr
V1 $-$ V2         & 4490  & 4429  & *61 \cr
W1 $-$ W2         & 4328  & 4270  & *68 \cr
W3 $-$ W4         & 4257  & 4145  & 112 \cr
\noalign{\vskip 5pt\hrule\vskip 5pt}}}
\endPlancktablewide
\end{table}


\begin{figure*}[t]
  \centering
        \includegraphics[width=0.16\linewidth]{figures/CG_K_n16_Q.pdf}
        \includegraphics[width=0.16\linewidth]{figures/WMAP_K_n16_Q.pdf}
        \includegraphics[width=0.16\linewidth]{figures/diff_K_n16_10deg_Q.pdf}\hspace*{2mm}
        \includegraphics[width=0.16\linewidth]{figures/CG_K_n16_U.pdf}
        \includegraphics[width=0.16\linewidth]{figures/WMAP_K_n16_U.pdf}
        \includegraphics[width=0.16\linewidth]{figures/diff_K_n16_10deg_U.pdf}\\
        \includegraphics[width=0.16\linewidth]{figures/CG_Ka_n16_Q.pdf}
        \includegraphics[width=0.16\linewidth]{figures/WMAP_Ka_n16_Q.pdf}
        \includegraphics[width=0.16\linewidth]{figures/diff_Ka_n16_10deg_Q.pdf}\hspace*{2mm}
        \includegraphics[width=0.16\linewidth]{figures/CG_Ka_n16_U.pdf}
        \includegraphics[width=0.16\linewidth]{figures/WMAP_Ka_n16_U.pdf}
        \includegraphics[width=0.16\linewidth]{figures/diff_Ka_n16_10deg_U.pdf}\\
        \includegraphics[width=0.16\linewidth]{figures/CG_Q1_n16_Q.pdf}
        \includegraphics[width=0.16\linewidth]{figures/WMAP_Q1_n16_Q.pdf}
        \includegraphics[width=0.16\linewidth]{figures/diff_Q1_n16_10deg_Q.pdf}\hspace*{2mm}
        \includegraphics[width=0.16\linewidth]{figures/CG_Q1_n16_U.pdf}
        \includegraphics[width=0.16\linewidth]{figures/WMAP_Q1_n16_U.pdf}
        \includegraphics[width=0.16\linewidth]{figures/diff_Q1_n16_10deg_U.pdf}\\
        \includegraphics[width=0.16\linewidth]{figures/CG_Q2_n16_Q.pdf}
        \includegraphics[width=0.16\linewidth]{figures/WMAP_Q2_n16_Q.pdf}
        \includegraphics[width=0.16\linewidth]{figures/diff_Q2_n16_10deg_Q.pdf}\hspace*{2mm}
        \includegraphics[width=0.16\linewidth]{figures/CG_Q2_n16_U.pdf}
        \includegraphics[width=0.16\linewidth]{figures/WMAP_Q2_n16_U.pdf}
        \includegraphics[width=0.16\linewidth]{figures/diff_Q2_n16_10deg_U.pdf}\\
        \includegraphics[width=0.16\linewidth]{figures/CG_V1_n16_Q.pdf}
        \includegraphics[width=0.16\linewidth]{figures/WMAP_V1_n16_Q.pdf}
        \includegraphics[width=0.16\linewidth]{figures/diff_V1_n16_10deg_Q.pdf}\hspace*{2mm}
        \includegraphics[width=0.16\linewidth]{figures/CG_V1_n16_U.pdf}
        \includegraphics[width=0.16\linewidth]{figures/WMAP_V1_n16_U.pdf}
        \includegraphics[width=0.16\linewidth]{figures/diff_V1_n16_10deg_U.pdf}\\
        \includegraphics[width=0.16\linewidth]{figures/CG_V2_n16_Q.pdf}
        \includegraphics[width=0.16\linewidth]{figures/WMAP_V2_n16_Q.pdf}
        \includegraphics[width=0.16\linewidth]{figures/diff_V2_n16_10deg_Q.pdf}\hspace*{2mm}
        \includegraphics[width=0.16\linewidth]{figures/CG_V2_n16_U.pdf}
        \includegraphics[width=0.16\linewidth]{figures/WMAP_V2_n16_U.pdf}
        \includegraphics[width=0.16\linewidth]{figures/diff_V2_n16_10deg_U.pdf}\\
        \includegraphics[width=0.16\linewidth]{figures/CG_W1_n16_Q.pdf}
        \includegraphics[width=0.16\linewidth]{figures/WMAP_W1_n16_Q.pdf}
        \includegraphics[width=0.16\linewidth]{figures/diff_W1_n16_10deg_Q.pdf}\hspace*{2mm}
        \includegraphics[width=0.16\linewidth]{figures/CG_W1_n16_U.pdf}
        \includegraphics[width=0.16\linewidth]{figures/WMAP_W1_n16_U.pdf}
        \includegraphics[width=0.16\linewidth]{figures/diff_W1_n16_10deg_U.pdf}\\
        \includegraphics[width=0.16\linewidth]{figures/CG_W2_n16_Q.pdf}
        \includegraphics[width=0.16\linewidth]{figures/WMAP_W2_n16_Q.pdf}
        \includegraphics[width=0.16\linewidth]{figures/diff_W2_n16_10deg_Q.pdf}\hspace*{2mm}
        \includegraphics[width=0.16\linewidth]{figures/CG_W2_n16_U.pdf}
        \includegraphics[width=0.16\linewidth]{figures/WMAP_W2_n16_U.pdf}
        \includegraphics[width=0.16\linewidth]{figures/diff_W2_n16_10deg_U.pdf}\\
        \includegraphics[width=0.16\linewidth]{figures/CG_W3_n16_Q.pdf}
        \includegraphics[width=0.16\linewidth]{figures/WMAP_W3_n16_Q.pdf}
        \includegraphics[width=0.16\linewidth]{figures/diff_W3_n16_10deg_Q.pdf}\hspace*{2mm}
        \includegraphics[width=0.16\linewidth]{figures/CG_W3_n16_U.pdf}
        \includegraphics[width=0.16\linewidth]{figures/WMAP_W3_n16_U.pdf}
        \includegraphics[width=0.16\linewidth]{figures/diff_W3_n16_10deg_U.pdf}\\
        \includegraphics[width=0.16\linewidth]{figures/CG_W4_n16_Q.pdf}
        \includegraphics[width=0.16\linewidth]{figures/WMAP_W4_n16_Q.pdf}
        \includegraphics[width=0.16\linewidth]{figures/diff_W4_n16_10deg_Q.pdf}\hspace*{2mm}
        \includegraphics[width=0.16\linewidth]{figures/CG_W4_n16_U.pdf}
        \includegraphics[width=0.16\linewidth]{figures/WMAP_W4_n16_U.pdf}
        \includegraphics[width=0.16\linewidth]{figures/diff_W4_n16_10deg_U.pdf}\\                
	\caption{Sky maps}
	\label{fig:skymaps}
\end{figure*}



\begin{table}
\newdimen\tblskip \tblskip=5pt
\caption{Transmission imbalance template amplitudes for each \WMAP\ radiometer as estimated by fitting the official templates to low-resolution difference maps between \cosmoglobe\ and \WMAP. The templates are provided in mK, and the template amplitudes are therefore dimensionless. The fourth column lists the relative decrease in standard deviation, $\sqrt{\sigma_{\mathrm{raw}}^2 - \sigma_{\mathrm{corr}}^2}/\sigma_{\mathrm{raw}}$,  after subtracting the best-fit templates in percent. }
\label{tab:transmission}
\vskip -4mm
\footnotesize
\setbox\tablebox=\vbox{
 \newdimen\digitwidth
 \setbox0=\hbox{\rm 0}
 \digitwidth=\wd0
 \catcode`*=\active
 \def*{\kern\digitwidth}
%
  \newdimen\dpwidth
  \setbox0=\hbox{.}
  \dpwidth=\wd0
  \catcode`!=\active
  \def!{\kern\dpwidth}
%
  \halign{\hbox to 1.8cm{#\leaderfil}\tabskip 2em&
    \hfil$#$\hfil \tabskip 2em&
    \hfil$#$\hfil \tabskip 2em&    
    \hfil$#$\hfil \tabskip 0em\cr
\noalign{\doubleline}
\omit\hfil\sc DA \hfil& a_1 & a_2 & \Delta \sigma [\%]\cr
\noalign{\vskip 3pt\hrule\vskip 5pt}
K1 &   -27.5*  &  -50.6* & 30 \cr
Ka1 &   -1.4  &   -1.9 & 25 \cr
Q1 &   -30.0* &  -71.6* & 11 \cr
Q2 &   -7.1  &  -1.5 & 20 \cr
V1 &   -32.8*  &  -53.4* & *6 \cr
V2 &   *8.8  &  -4.1 & 16 \cr
W1 &   -2.8  &  *4.6 & *8 \cr
W2 &   -6.9  & -3.5 & 11 \cr
W3 &   29.1  & 53.4 & 12 \cr
W4 &   15.5  & -6.8 & 52 \cr
\noalign{\vskip 5pt\hrule\vskip 5pt}}}
\endPlancktablewide
\end{table}



\begin{figure*}[t]
  \centering
        \includegraphics[width=0.16\linewidth]{figures/diff_K1_Q.pdf}
        \includegraphics[width=0.16\linewidth]{figures/temp_corr_K1_Q.pdf}
        \includegraphics[width=0.16\linewidth]{figures/res_loss_K1_Q.pdf}\hspace*{2mm}
        \includegraphics[width=0.16\linewidth]{figures/diff_K1_U.pdf}
        \includegraphics[width=0.16\linewidth]{figures/temp_corr_K1_U.pdf}
        \includegraphics[width=0.16\linewidth]{figures/res_loss_K1_U.pdf}\\
        \includegraphics[width=0.16\linewidth]{figures/diff_Ka1_Q.pdf}
        \includegraphics[width=0.16\linewidth]{figures/temp_corr_Ka1_Q.pdf}
        \includegraphics[width=0.16\linewidth]{figures/res_loss_Ka1_Q.pdf}\hspace*{2mm}
        \includegraphics[width=0.16\linewidth]{figures/diff_Ka1_U.pdf}
        \includegraphics[width=0.16\linewidth]{figures/temp_corr_Ka1_U.pdf}
        \includegraphics[width=0.16\linewidth]{figures/res_loss_Ka1_U.pdf}\\
        \includegraphics[width=0.16\linewidth]{figures/diff_Q1_Q.pdf}
        \includegraphics[width=0.16\linewidth]{figures/temp_corr_Q1_Q.pdf}
        \includegraphics[width=0.16\linewidth]{figures/res_loss_Q1_Q.pdf}\hspace*{2mm}
        \includegraphics[width=0.16\linewidth]{figures/diff_Q1_U.pdf}
        \includegraphics[width=0.16\linewidth]{figures/temp_corr_Q1_U.pdf}
        \includegraphics[width=0.16\linewidth]{figures/res_loss_Q1_U.pdf}\\
        \includegraphics[width=0.16\linewidth]{figures/diff_Q2_Q.pdf}
        \includegraphics[width=0.16\linewidth]{figures/temp_corr_Q2_Q.pdf}
        \includegraphics[width=0.16\linewidth]{figures/res_loss_Q2_Q.pdf}\hspace*{2mm}
        \includegraphics[width=0.16\linewidth]{figures/diff_Q2_U.pdf}
        \includegraphics[width=0.16\linewidth]{figures/temp_corr_Q2_U.pdf}
        \includegraphics[width=0.16\linewidth]{figures/res_loss_Q2_U.pdf}\\
        \includegraphics[width=0.16\linewidth]{figures/diff_V1_Q.pdf}
        \includegraphics[width=0.16\linewidth]{figures/temp_corr_V1_Q.pdf}
        \includegraphics[width=0.16\linewidth]{figures/res_loss_V1_Q.pdf}\hspace*{2mm}
        \includegraphics[width=0.16\linewidth]{figures/diff_V1_U.pdf}
        \includegraphics[width=0.16\linewidth]{figures/temp_corr_V1_U.pdf}
        \includegraphics[width=0.16\linewidth]{figures/res_loss_V1_U.pdf}\\
        \includegraphics[width=0.16\linewidth]{figures/diff_V2_Q.pdf}
        \includegraphics[width=0.16\linewidth]{figures/temp_corr_V2_Q.pdf}
        \includegraphics[width=0.16\linewidth]{figures/res_loss_V2_Q.pdf}\hspace*{2mm}
        \includegraphics[width=0.16\linewidth]{figures/diff_V2_U.pdf}
        \includegraphics[width=0.16\linewidth]{figures/temp_corr_V2_U.pdf}
        \includegraphics[width=0.16\linewidth]{figures/res_loss_V2_U.pdf}\\
        \includegraphics[width=0.16\linewidth]{figures/diff_W1_Q.pdf}
        \includegraphics[width=0.16\linewidth]{figures/temp_corr_W1_Q.pdf}
        \includegraphics[width=0.16\linewidth]{figures/res_loss_W1_Q.pdf}\hspace*{2mm}
        \includegraphics[width=0.16\linewidth]{figures/diff_W1_U.pdf}
        \includegraphics[width=0.16\linewidth]{figures/temp_corr_W1_U.pdf}
        \includegraphics[width=0.16\linewidth]{figures/res_loss_W1_U.pdf}\\
        \includegraphics[width=0.16\linewidth]{figures/diff_W2_Q.pdf}
        \includegraphics[width=0.16\linewidth]{figures/temp_corr_W2_Q.pdf}
        \includegraphics[width=0.16\linewidth]{figures/res_loss_W2_Q.pdf}\hspace*{2mm}
        \includegraphics[width=0.16\linewidth]{figures/diff_W2_U.pdf}
        \includegraphics[width=0.16\linewidth]{figures/temp_corr_W2_U.pdf}
        \includegraphics[width=0.16\linewidth]{figures/res_loss_W2_U.pdf}\\
        \includegraphics[width=0.16\linewidth]{figures/diff_W3_Q.pdf}
        \includegraphics[width=0.16\linewidth]{figures/temp_corr_W3_Q.pdf}
        \includegraphics[width=0.16\linewidth]{figures/res_loss_W3_Q.pdf}\hspace*{2mm}
        \includegraphics[width=0.16\linewidth]{figures/diff_W3_U.pdf}
        \includegraphics[width=0.16\linewidth]{figures/temp_corr_W3_U.pdf}
        \includegraphics[width=0.16\linewidth]{figures/res_loss_W3_U.pdf}\\
        \includegraphics[width=0.16\linewidth]{figures/diff_W4_Q.pdf}
        \includegraphics[width=0.16\linewidth]{figures/temp_corr_W4_Q.pdf}
        \includegraphics[width=0.16\linewidth]{figures/res_loss_W4_Q.pdf}\hspace*{2mm}
        \includegraphics[width=0.16\linewidth]{figures/diff_W4_U.pdf}
        \includegraphics[width=0.16\linewidth]{figures/temp_corr_W4_U.pdf}
        \includegraphics[width=0.16\linewidth]{figures/res_loss_W4_U.pdf}\\
        \includegraphics[width=0.30\linewidth]{figures/colourbar_3uK.pdf}\\         
	\caption{Transmission imbalance templates}
	\label{fig:imbal}
\end{figure*}





\section{Low-resolution sky maps}
\label{sec:skymaps}


\begin{figure}[t]
  \centering
        \includegraphics[width=0.49\linewidth]{figures/wtr39yrKaQVmapq.pdf}
        \includegraphics[width=0.49\linewidth]{figures/wtr39yrKaQVmapu.pdf}\\
        \includegraphics[width=0.49\linewidth]{figures/wmap_reprod_KaQV_9yr_q_scale.pdf}
        \includegraphics[width=0.49\linewidth]{figures/wmap_reprod_KaQV_9yr_u_scale.pdf}\\
        \includegraphics[width=0.49\linewidth]{figures/wmap_reprod_tempcorr_KaQV_9yr_q_scale.pdf}
        \includegraphics[width=0.49\linewidth]{figures/wmap_reprod_tempcorr_KaQV_9yr_u_scale.pdf}
	\caption{Noise-weighted likelihood input maps.}
	\label{fig:likelihood_maps}
\end{figure}



\section{Likelihood analysis}
\label{sec:likelihood}


\begin{figure}[t]
  	\centering
	\includegraphics[width=\linewidth]{figures/lnL_EE_v1.pdf}
        \includegraphics[width=\linewidth]{figures/lnL_BB_v1.pdf}
	\caption{Likelihood slices}
	\label{fig:lnL}
\end{figure}




\section{Conclusions}
\label{sec:conclusions}









%\input{BP_wmap_acknowledgments.tex}

\bibliographystyle{../../common/aa}

\bibliography{../../common/Planck_bib,../../common/BP_bibliography}


\end{document}
