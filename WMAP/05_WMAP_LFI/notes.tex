
In Figure~\ref{fig:relgain}, we compare the gain solution from the \commanderthree\ output to the official \WMAPnine\ gain solution, obtained by linearly regressing the calibrated and uncalibrated delivered \WMAP\ TODs. With the exception of \K-band, the gains agree to a sub-per cent level. The \K-band deviation is not entirely unexpected, as it is the TOD band with the largest synchrotron contribution, and hence in a joint gain/synchrotron amplitude sampling setup there is exists a degeneracy between \K-band's absolute gain the synchrotron spectral index.


\section{Comparison with BP10 and \textit{WMAP9}}
\label{sec:comparison}

Much of the \Cosmoglobe\ processing of \WMAP\ data followed the procedures detailed in the \WMAP\ suite of publications, as well as much of the original IDL code provided by the team.\footnote{\url{https://lambda.gsfc.nasa.gov/product/wmap/dr5/m_sw.html}} However, the use of a sky model allowed for more detailed characterization of the data themselves. The use of a sky model that is sampled allows for a full characterization of the joint probability distribution of instrumental parameters that can be updated when new data are included.

\subsection{Instrument characterization}

The \WMAP\ analysis estimated the noise properties of the timestreams by using a sky-subtracted TOD \citep{jarosik2007}, similar to the approach in \Cosmoglobe\ and \BP\ \citep{bp06}. The fiducial analysis parameterizes each radiometer's autocorrelation function on a year-by-year basis, as the instrument is very stable on long timescales. The \commanderthree\ pipeline estimates the noise power spectrum once per TOD period, on the order of three days. In general, the noise parameters are stable over one-year periods, as shown in Fig.~\ref{fig:inst_params}. However, some radiometers, e.g., \Q2 and \V2, display smooth variation in $f_\mathrm{knee}$ that changes on the order of 10\% over a one-year period, while \W1, \W2, and \W4 display much sharper variations.

The \Cosmoglobe\ white noise levels also estimated on a per-TOD basis using a standard trick from radio astronomy, in which adjacent samples from a residual sky signal $\boldsymbol r =\boldsymbol d-g\boldsymbol s^\mathrm{tot}-\boldsymbol n^\mathrm{corr}$ are differenced, i.e.,
\begin{equation}
	\sigma_0^2=\frac{\mathrm{Var}(r_{i+1}-r_i)}2.
	\label{eq:sigma0}
\end{equation}
This effectively estimates the power spectrum at the sampling frequency, and is a good estimate of the white noise level if the power spectrum is flat. However, as described in \citet{jarosik2003:MAP} and \citet{jarosik2007}, there is a lag-1 anticorrelation between adjacent samples that must be taken into account before estimating the white noise level. This was achieved in the \WMAPnine\ pipeline by pre-whitening the data using the optimal time-domain filter described above. Therefore, the method of \eqref{eq:sigma0} gives an inaccurate estimate of the TOD white noise level, depending on the deviation from flatness at high temporal frequencies. We find that the estimate of a simple $1/f$ noise signal is enough to make the goodness-of-fit $\chi^2$ significantly nonzero.

[Insert Ihle figure here]

The gain is estimated using an identical procedure to that in \citep{bp07}, i.e., using the orbital dipole for constant gain across time and all detectors, while calibrating the detector- time-variation against a sky model. Similarly, we estimate imbalance parameters $x_\mathrm{im}$ by comparing to the total expected common-mode signal, $T_\mathrm A+T_\mathrm B$. For this, we use the entire sky model, whereas the \WMAP\ team associated the Galactic sky signal as a smooth component that cannot be distinguished from other long-term drifts \citep{hinshaw2003a}. We fit to the total sky signal, and find good agreement with the \WMAPnine\ results. A summary of our imbalance parameters and their deviations from the \WMAPnine\ results can be found in Table \ref{tab:table2}.

\subsection{Handling of poorly-measured modes}
