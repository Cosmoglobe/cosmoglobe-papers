\documentclass[twocolumn]{../../common/aa}
%\documentclass[referee]{aa}

\usepackage{graphicx}
\usepackage{amsmath,amsfonts,amssymb}
\usepackage{txfonts}
\usepackage{color}
\usepackage{natbib}
\usepackage{float}
%\usepackage{stfloats}
\usepackage{dblfloatfix}
\usepackage{afterpage}
\usepackage{ifthen}
\usepackage[morefloats=12]{morefloats}
\usepackage{placeins}
\usepackage{multicol}
\bibpunct{(}{)}{;}{a}{}{,}
\usepackage[switch]{lineno}
\definecolor{linkcolor}{rgb}{0.6,0,0}
\definecolor{citecolor}{rgb}{0,0,0.75}
\definecolor{urlcolor}{rgb}{0.12,0.46,0.7}
\usepackage[breaklinks, colorlinks, urlcolor=urlcolor,
    linkcolor=linkcolor,citecolor=citecolor,pdfencoding=auto]{hyperref}
\hypersetup{linktocpage}
\usepackage{bold-extra}
\usepackage{xcolor}

%\usepackage[grid,
%  gridcolor=red!20,
%  subgridcolor=green!20,
%  gridunit=cm]{eso-pic}
%

%Planck style file, to be used with A&A style to produce Planck papers for publication.
%
% version 28 September 2010 --- useful macros --- CRL
% version 17 October 2010   --- first cut at important instrument values, from Daniele Mennella and
%                               Francois Bouchet, 13 October 2010 --- CRL
% version 18 October 2010   --- LFI FWHM changed to one value per feed, rather than M & S separately
%                               LFI FWHM uncertainties added for individual feeds.  Corrections made
%                               to LFI values. --- Andrea Zacchei
% version 24 October 2010   --- added to and corrected definitions.  No changes made to instrument
%                               quantities. --- CRL 
% version 31 October 2010   --- added definition of \muKHz. --- CRL
%
% version 15 November 2010  --- fixed conflict with aa.cls in definition of \endtable
%                               by naming the command below "\endPlancktable".  See section
%                               13.16 of the Style Guide.
%
% version 06 December 2010  --- Set up names with and without units.
%                               Add \allearlypapers command to ensure that all early papers are
%                               included in the reference list.
%                               Define macro for the name of the 4He JT cooler.
%
% version 07 December 2010  --- removed extraneous "planck2011-1.2" entry in \allearlypapers
%
% version 12 December 2010  --- added \endPlancktablewide command to set tablenotes to the full
%                               page width in the \begin{table*}...\end{table*} environment when
%                               the ``twocolumn'' option is specified in the \documentclass command.
%                               (It would be more elegant to extract the appropriate width from the
%                               aa.cls system at the time of execution, but that is buried more
%                               deeply in the system than I investigated.)
%
% version 05 January 2011   --- added unit \MJysr.  HFI performance values updated per FRB email
%                               01/05/2011 02:38-0800, and Brendan Crill email 01/05/2011 18:08 -0800.
%
% version 06 January 2011   --- changed \scriptscriptstyle primes to \scriptstyle, to better match the
%                               tx fonts used by A&A.
%
% version 07 January 2011   --- modified \allearlypapers to correspond with final early paper list.  
%                               Fixed 545 GHz center frequency.
%
% version 07 January 2011b  --- changed LFI white-noise sensitivity numbers to correct problem with units
%
% version 05 July 2011      --- added \Msol and \Lsol to get the symbols for solar mass and luminosity.
%                               Deleted previous definitions of \solar and \sol, which were equivalent
%                               to the new \Msol.
%
% version 16 August 2011    --- changed comments on \endPlancktable and \endPlancktablewide for clarity
%
% version 11 September 2011 --- changed definition of \tablenote to make footnote labels italic, as per A\&A
%
% version 26 April 2011     --- changed definition of \Planck to agree with what is said in the Style Guide (!)
%
% version 04 Dec 2013       --- included 2013 results references
%
% version 17 Jan 2014       --- included fix to bibtex file v4.3, i.e. \providecommand{\sorthelp}[1]{}
%
% version 26 Jul 2014       --- fixed incompatibility problem with aa.cls v8.0 and v8.2.  v8.2 should now be used
%                               for all Planck papers.
%                           --- fixed problem in definition of "\all2013resultspapers" that introduced a blanck
%                               into the reference to p06b.
%                           --- removed all the parameter definition stuff at the end.  We weren't using it, and
%                               it took up a lot of space.
%
% version 28 Jan 2015       --- added "\alltwentyfiftennresultspapers" and corrected "\all2013resultspapers" to
%                               "\all20thirteenresultspapers",
%
% Usage:  after the \documentclass[traditabstract]{aa} command in the La\TeX\ input file,
%         add this command:      \input Planck.tex


\def\setsymbol#1#2{\expandafter\def\csname #1\endcsname{#2}}
\def\getsymbol#1{\csname #1\endcsname}

%-----------------------------------------------------------------------
% Planck
%-----------------------------------------------------------------------
\def\Planck{\textit{Planck}}

%-----------------------------------------------------------------------
% The Planck Helium-4 JT cooler
%-----------------------------------------------------------------------
\def\HeJT{$^4$He-JT}

%-----------------------------------------------------------------------
% To include all Planck Early Results papers in the reference lists
%-----------------------------------------------------------------------
\def\allearlypapers{\nocite{planck2011-1.1, planck2011-1.3, planck2011-1.4, planck2011-1.5, planck2011-1.6, planck2011-1.7, planck2011-1.10, planck2011-1.10sup, planck2011-5.1a, planck2011-5.1b, planck2011-5.2a, planck2011-5.2b, planck2011-5.2c, planck2011-6.1, planck2011-6.2, planck2011-6.3a, planck2011-6.4a, planck2011-6.4b, planck2011-6.6, planck2011-7.0, planck2011-7.2, planck2011-7.3, planck2011-7.7a, planck2011-7.7b, planck2011-7.12, planck2011-7.13}}

%-----------------------------------------------------------------------
% To include all Planck 2013 Results papers in the reference lists
%-----------------------------------------------------------------------
\def\alltwentythirteenresultspapers{\nocite{planck2013-p01, planck2013-p02, planck2013-p02a, planck2013-p02d, planck2013-p02b, planck2013-p03, planck2013-p03c, planck2013-p03f, planck2013-p03d, planck2013-p03e, planck2013-p01a, planck2013-p06, planck2013-p03a, planck2013-pip88, planck2013-p08, planck2013-p11, planck2013-p12, planck2013-p13, planck2013-p14, planck2013-p15, planck2013-p05b, planck2013-p17, planck2013-p09, planck2013-p09a, planck2013-p20, planck2013-p19, planck2013-pipaberration, planck2013-p05, planck2013-p05a, planck2013-pip56, planck2013-p06b, planck2013-p01a}}

%-----------------------------------------------------------------------
% To include all Planck 2015 Results papers in the reference lists
%-----------------------------------------------------------------------
\def\alltwentyfifteenresultspapers{\nocite{planck2014-a01, planck2014-a03, planck2014-a04, planck2014-a05, planck2014-a06, planck2014-a07, planck2014-a08, planck2014-a09, planck2014-a11, planck2014-a12, planck2014-a13, planck2014-a14, planck2014-a15, planck2014-a16, planck2014-a17, planck2014-a18, planck2014-a19, planck2014-a20, planck2014-a22, planck2014-a24, planck2014-a26, planck2014-a28, planck2014-a29, planck2014-a30, planck2014-a31, planck2014-a35, planck2014-a36, planck2014-a37, planck2014-ES}}

%-----------------------------------------------------------------------
% Tables
%-----------------------------------------------------------------------
\newbox\tablebox    \newdimen\tablewidth
\def\leaderfil{\leaders\hbox to 5pt{\hss.\hss}\hfil}
%
% use the following definition of \endPlancktable for ApJ style notes to tables, set to the 
%         width of the table
% \def\endPlancktable{\tablewidth=\wd\tablebox 
%
% use the following definitions of \endPlancktable and \endPlancktablewide for A&A style notes 
% set to one-column  or full-page width, respectively
\def\endPlancktable{\tablewidth=\columnwidth 
    $$\hss\copy\tablebox\hss$$
    \vskip-\lastskip\vskip -2pt}
\def\endPlancktablewide{\tablewidth=\textwidth 
    $$\hss\copy\tablebox\hss$$
    \vskip-\lastskip\vskip -2pt}
\def\tablenote#1 #2\par{\begingroup \parindent=0.8em
    \abovedisplayshortskip=0pt\belowdisplayshortskip=0pt
    \noindent
    $$\hss\vbox{\hsize\tablewidth \hangindent=\parindent \hangafter=1 \noindent
    \hbox to \parindent{$^#1$\hss}\strut#2\strut\par}\hss$$
    \endgroup}
\def\doubleline{\vskip 3pt\hrule \vskip 1.5pt \hrule \vskip 5pt}

%-----------------------------------------------------------------------
% useful macros
%-----------------------------------------------------------------------
%
\def\L2{\ifmmode L_2\else $L_2$\fi}
%
\def\dtt{\Delta T/T}
\def\DeltaT{\ifmmode \Delta T\else $\Delta T$\fi}
\def\deltat{\ifmmode \Delta t\else $\Delta t$\fi}
\def\fknee{\ifmmode f_{\rm knee}\else $f_{\rm knee}$\fi}
\def\Fmax{\ifmmode F_{\rm max}\else $F_{\rm max}$\fi}
%
\def\solar{\ifmmode{\rm M}_{\mathord\odot}\else${\rm M}_{\mathord\odot}$\fi}
\def\Msolar{\ifmmode{\rm M}_{\mathord\odot}\else${\rm M}_{\mathord\odot}$\fi}
\def\Lsolar{\ifmmode{\rm L}_{\mathord\odot}\else${\rm L}_{\mathord\odot}$\fi}
%
\def\inv{\ifmmode^{-1}\else$^{-1}$\fi}
\def\mo{\ifmmode^{-1}\else$^{-1}$\fi}
\def\sup#1{\ifmmode ^{\rm #1}\else $^{\rm #1}$\fi}
\def\expo#1{\ifmmode \times 10^{#1}\else $\times 10^{#1}$\fi}
%
\def\,{\thinspace}
\def\lsim{\mathrel{\raise .4ex\hbox{\rlap{$<$}\lower 1.2ex\hbox{$\sim$}}}}
\def\gsim{\mathrel{\raise .4ex\hbox{\rlap{$>$}\lower 1.2ex\hbox{$\sim$}}}}
\let\lea=\lsim
\let\gea=\gsim
\def\simprop{\mathrel{\raise .4ex\hbox{\rlap{$\propto$}\lower 1.2ex\hbox{$\sim$}}}}
%
\def\deg{\ifmmode^\circ\else$^\circ$\fi}
\def\pdeg{\ifmmode $\setbox0=\hbox{$^{\circ}$}\rlap{\hskip.11\wd0 .}$^{\circ}
          \else \setbox0=\hbox{$^{\circ}$}\rlap{\hskip.11\wd0 .}$^{\circ}$\fi}
\def\arcs{\ifmmode {^{\scriptstyle\prime\prime}}
          \else $^{\scriptstyle\prime\prime}$\fi}
\def\arcm{\ifmmode {^{\scriptstyle\prime}}
          \else $^{\scriptstyle\prime}$\fi}
\newdimen\sa  \newdimen\sb
\def\parcs{\sa=.07em \sb=.03em
     \ifmmode \hbox{\rlap{.}}^{\scriptstyle\prime\kern -\sb\prime}\hbox{\kern -\sa}
     \else \rlap{.}$^{\scriptstyle\prime\kern -\sb\prime}$\kern -\sa\fi}
\def\parcm{\sa=.08em \sb=.03em
     \ifmmode \hbox{\rlap{.}\kern\sa}^{\scriptstyle\prime}\hbox{\kern-\sb}
     \else \rlap{.}\kern\sa$^{\scriptstyle\prime}$\kern-\sb\fi}
%
\def\ra[#1 #2 #3.#4]{#1\sup{h}#2\sup{m}#3\sup{s}\llap.#4}
\def\dec[#1 #2 #3.#4]{#1\deg#2\arcm#3\arcs\llap.#4}
\def\deco[#1 #2 #3]{#1\deg#2\arcm#3\arcs}
\def\rra[#1 #2]{#1\sup{h}#2\sup{m}}
%
\def\page{\vfill\eject}
\def\dots{\relax\ifmmode \ldots\else $\ldots$\fi}
%
%-----------------------------------------------------------------------
% units
%-----------------------------------------------------------------------
%
\def\WHzsr{\ifmmode $W\,Hz\mo\,sr\mo$\else W\,Hz\mo\,sr\mo\fi}
\def\mHz{\ifmmode $\,mHz$\else \,mHz\fi}
\def\GHz{\ifmmode $\,GHz$\else \,GHz\fi}
\def\mKs{\ifmmode $\,mK\,s$^{1/2}\else \,mK\,s$^{1/2}$\fi}
\def\muKs{\ifmmode \,\mu$K\,s$^{1/2}\else \,$\mu$K\,s$^{1/2}$\fi}
\def\muKRJs{\ifmmode \,\mu$K$_{\rm RJ}$\,s$^{1/2}\else \,$\mu$K$_{\rm RJ}$\,s$^{1/2}$\fi}
\def\muKHz{\ifmmode \,\mu$K\,Hz$^{-1/2}\else \,$\mu$K\,Hz$^{-1/2}$\fi}
\def\MJysr{\ifmmode \,$MJy\,sr\mo$\else \,MJy\,sr\mo\fi}
\def\MJysrmK{\ifmmode \,$MJy\,sr\mo$\,mK$_{\rm CMB}\mo\else \,MJy\,sr\mo\,mK$_{\rm CMB}\mo$\fi}
\def\microns{\ifmmode \,\mu$m$\else \,$\mu$m\fi}
\def\micron{\microns}
\def\muK{\ifmmode \,\mu$K$\else \,$\mu$\hbox{K}\fi}
\def\microK{\ifmmode \,\mu$K$\else \,$\mu$\hbox{K}\fi}
\def\muW{\ifmmode \,\mu$W$\else \,$\mu$\hbox{W}\fi}
\def\kms{\ifmmode $\,km\,s$^{-1}\else \,km\,s$^{-1}$\fi}
\def\kmsMpc{\ifmmode $\,\kms\,Mpc\mo$\else \,\kms\,Mpc\mo\fi}
%
%
%----------------------------------------------------------------------
% set up machinery to list Planck papers in roman numeral order.
%----------------------------------------------------------------------

\providecommand{\sorthelp}[1]{}


\def\WMAP{\emph{WMAP}}
\def\WMAPnine{\emph{WMAP9}}
\def\COBE{\emph{COBE}}
\def\wmap{\emph{WMAP}}
\def\planck{\emph{Planck}}
\def\Planck{\emph{Planck}}
\def\LCDM{$\Lambda$CDM}
\def\ffp{FFP6}
\def\unionmask{U73}
\def\nside{N_{\mathrm{side}}}

\def\healpix{\texttt{HEALPix}}
\def\commander{\texttt{Commander}}
\def\commanderone{\texttt{Commander1}}
\def\commandertwo{\texttt{Commander2}}
\def\commanderthree{\texttt{Commander3}}
\def\ruler{\texttt{Ruler}}
\def\comrul{\texttt{Commander-Ruler}}
\def\CR{\texttt{C-R}}
\def\nilc{\texttt{NILC}}
\def\gnilc{\texttt{GNILC}}
\def\sevem{\texttt{SEVEM}}
\def\smica{\texttt{SMICA}}
\def\CamSpec{\texttt{CamSpec}}
\def\Plik{\texttt{Plik}}
\def\XFaster{\texttt{XFaster}}
\def\sroll2{\texttt{SRoll2}}

\renewcommand{\d}[0]{\vec{d}}
\renewcommand{\t}[0]{\vec{t}}
\newcommand{\A}[0]{\mathrm{A}}
\newcommand{\B}[0]{\mathrm{B}}
\newcommand{\Y}[0]{\tens{Y}}
\newcommand{\n}[0]{\vec{n}}
\newcommand{\red}[0]{\color{red}}
\newcommand{\green}[0]{\color{green}}
\newcommand{\s}[0]{\vec{s}}
%\renewcommand{\a}[0]{\vec{a}}
\newcommand{\m}[0]{\vec{m}}
\newcommand{\f}[0]{\vec{f}}
\newcommand{\F}[0]{\tens{F}}
\newcommand{\T}[0]{\tens{T}}
\newcommand{\Cp}[0]{\tens{C}}
\renewcommand{\L}[0]{\tens{L}}
\newcommand{\g}[0]{\vec{g}}
\newcommand{\N}[0]{\tens{N}}
\newcommand{\M}[0]{\tens{M}}
\newcommand{\iN}[0]{\tens{N}^{-1}}
\newcommand{\iM}[0]{\tens{M}^{-1}}
\newcommand{\w}[0]{\vec{w}}
\renewcommand{\S}[0]{\tens{S}}
\renewcommand{\r}[0]{\vec{r}}
\renewcommand{\u}[0]{\vec{u}}
\newcommand{\q}[0]{\vec{q}}
\renewcommand{\v}[0]{\vec{v}}
\renewcommand{\P}[0]{\tens{P}}
\newcommand{\dt}[0]{d_t}
\newcommand{\di}[0]{d_i}
\newcommand{\nt}[0]{n_t}
\newcommand{\st}[0]{s_t}
\newcommand{\mt}[0]{m_t}
\newcommand{\ft}[0]{f_t}
\newcommand{\Te}[0]{T_{\rm e}}
\newcommand{\EM}[0]{\rm EM}
\newcommand{\mathsc}[1]{{\normalfont\textsc{#1}}}
\newcommand{\hi}{\ensuremath{\mathsc {Hi}}}
\newcommand{\bpbold}{\bfseries{\scshape{BeyondPlanck}}}
\newcommand{\BP}{\textsc{BeyondPlanck}}
\newcommand{\bp}{\textsc{BeyondPlanck}}
\newcommand{\cosmoglobe}{\textsc{Cosmoglobe}}
\newcommand{\Cosmoglobe}{\textsc{Cosmoglobe}}
\newcommand{\lfi}[0]{LFI}
\newcommand{\hfi}[0]{HFI}
\newcommand{\npipe}[0]{\texttt{NPIPE}}
\newcommand{\K}[0]{\textit K}
\newcommand{\Ka}[0]{\textit{Ka}}
\newcommand{\Q}[0]{\textit Q}
\newcommand{\V}[0]{\textit V}
\newcommand{\W}[0]{\textit W}
\newcommand{\e}{\mathrm e}
\newcommand{\cvar}{\ensuremath{c(\vartheta, \varphi, \psi)}}

\usepackage[T1]{fontenc}


\def\bC{\tens{C}}
\def\ba{\vec{a}}
\def\ncha{N_\mathrm{cha}}
\def\nfg{N_\mathrm{fg}}

\newcommand{\ncorr}{\vec n_\mathrm{corr}}
\newcommand{\Dbp}{\Delta_\mathrm{bp}}

%\modulolinenumbers[5]
%\linenumbers

\newcommand{\includegraphicsdpi}[3]{
    \pdfimageresolution=#1  % Change the dpi of images
    \includegraphics[#2]{#3}
    \pdfimageresolution=72  % Change it back to the default
}

\renewcommand{\topfraction}{1.0}	% max fraction of floats at top
    \renewcommand{\bottomfraction}{1.0}	% max fraction of floats at bottom
    %   Parameters for TEXT pages (not float pages):
    \setcounter{topnumber}{2}
    \setcounter{bottomnumber}{2}
    \setcounter{totalnumber}{4}     % 2 may work better
    \setcounter{dbltopnumber}{2}    % for 2-column pages
    \renewcommand{\dbltopfraction}{0.9}	% fit big float above 2-col. text
    \renewcommand{\textfraction}{0.04}	% allow minimal text w. figs
    %   Parameters for FLOAT pages (not text pages):
    \renewcommand{\floatpagefraction}{0.9}	% require fuller float pages
	% N.B.: floatpagefraction MUST be less than topfraction !!
    \renewcommand{\dblfloatpagefraction}{0.9}	% require fuller float pages

\def\adj{^{\dagger}}
\def\tp{^{\rm T}}
\def\inv{^{-1}}
\def\lm{{\ell m}}

\begin{document}

\title{\bfseries{\Cosmoglobe\ II. Improved consistency of \emph{Planck} LFI and \emph{WMAP} frequency maps}}
%This author list corresponds to \title{Author list for L04\_CMB\_Foregrounds\_Extraction}
%Prepared by M. Lopez-Caniego (Marcos.Lopez.Caniego@sciops.esa.int), ESAC/ESA
%This version is from Thu Jul 12 18:11:48 2018 CET
%\subtitle{There are 152 co-authors in this list}
\newcommand{\oslo}[0]{1}
\newcommand{\iiabangalore}[0]{2}

\author{\small
D.~J.~Watts\inst{\ref{uio}}\thanks{Corresponding author: D.~J.~Watts; \url{duncanwa@astro.uio.no}}
\and
A.~Basyrov\inst{\ref{uio}}
\and
H.~T.~Ihle\inst{\ref{uio}}
\and
S.~Paradiso\inst{\ref{waterloo}}
\and
F.~Rahman\inst{\ref{iiabangalore}}
\and
H.~Thommesen\inst{\ref{uio}}
\and
M.~Bersanelli\inst{\ref{milan}}
\and
L.~A.~Bianchi\inst{\ref{milan}}
\and
M.~Brilenkov\inst{\ref{uio}}
\and
L.~P.~L.~Colombo\inst{\ref{milan}}
\and
H.~K.~Eriksen\inst{\ref{uio}}
\and
J.~R.~Eskilt\inst{\ref{uio},\ref{imperial}}
\and
K.~S.~F.~Fornazier\inst{\ref{saopaulo}}
\and
C.~Franceschet\inst{\ref{milan}}
\and
U.~Fuskeland\inst{\ref{uio}}
\and
M.~Galloway\inst{\ref{uio}}
\and
E.~Gjerl\o w\inst{\ref{uio}}
\and
B.~Hensley\inst{\ref{princeton}}
\and
L.~T.~Hergt\inst{\ref{ubc}}
\and
D.~Herman\inst{\ref{uio}}
\and
G.~A.~Hoerning\inst{\ref{saopaulo}}
\and
K.~Lee\inst{\ref{uio}}
\and
J.~G.~S.~Lunde\inst{\ref{uio}}
\and
A.~Marins\inst{\ref{saopaulo},\ref{ustofc}}
\and
S.~K.~Nerval\inst{\ref{dunlap1},\ref{dunlap2}}
\and
S.~K.~Patel\inst{\ref{iit_bhu}}
\and
M.~Regnier\inst{\ref{apc}}
\and
M.~San\inst{\ref{uio}}
\and
S.~Sanyal\inst{\ref{iit_bhu}}
\and
N.-O.~Stutzer\inst{\ref{uio}}
\and
A.~Verma\inst{\ref{iit_bhu}}
\and
I.~K.~Wehus\inst{\ref{uio}}
\and
Y.~Zhou\inst{\ref{berkeley}}
}
\institute{\small
Institute of Theoretical Astrophysics, University of Oslo, Blindern, Oslo, Norway\label{uio}
\and
Waterloo Centre for Astrophysics, University of Waterloo, Waterloo, ON N2L 3G1, Canada\label{waterloo}
\and
Indian Institute of Astrophysics, Koramangala II Block, Bangalore, 560034, India\label{iiabangalore}
\and
Dipartimento di Fisica, Università degli Studi di Milano, Via Celoria, 16, Milano, Italy\label{milan}
\and
Imperial Centre for Inference and Cosmology, Department of Physics, Imperial College London, Blackett Laboratory, Prince Consort Road, London SW7 2AZ, United Kingdom\label{imperial}
\and
Instituto de Física, Universidade de São Paulo - C.P. 66318, CEP: 05315-970, São Paulo, Brazil\label{saopaulo}
\and
Department of Astrophysical Sciences, Princeton University, 4 Ivy Lane, Princeton, NJ 08540\label{princeton}
\and
Department of Physics and Astronomy, University of British Columbia, 6224 Agricultural Road, Vancouver BC, V6T1Z1, Canada\label{ubc}
\and
Department of Astronomy,  University of Science and Technology of China, Hefei, China\label{ustofc}
\and
David A. Dunlap Department of Astronomy \& Astrophysics, University of Toronto, 50 St. George Street, Toronto, ON M5S 3H4, Canada\label{dunlap1}
\and
Dunlap Institute for Astronomy \& Astrophysics, University of Toronto, 50 St. George Street, Toronto, ON M5S 3H4, Canada\label{dunlap2}
\and
Department of Physics, Indian Institute of Technology (BHU), Varanasi - 221005, India\label{iit_bhu}
\and
Laboratoire Astroparticule et Cosmologie (APC), Université Paris-Cité, Paris, France\label{apc}
\and
Department of Physics, UC Berkeley\label{berkeley}
}

 %\author{V.~Arsenijevic\inst{\ref{inst1}}\and S.~Fabbro\inst{\ref{inst2}}\and
%A.~M.~Mour\~ao\inst{\ref{inst3}}\and A.~J.~Rica da Silva\inst{\ref{inst1}}}
%
%\institute{Multidisciplinar de Astrof\'{\i}sica, IST, Avenida Rovisco Pais, 1049
%Lisbon, Portugal\email{...}\label{inst1} \and < Multidisciplinar de Astrof\'{\i}sica, IST, Avenida Rovisco Pais, 1049 Lisbon, Portugal\email{...}\label{inst2}
%\and
%Multidisciplinar de Astrof\'{\i}sica, IST, Avenida Rovisco Pais, 1049
%Lisbon, Portugal\email{...}\label{inst3}
%} 

%\authorrunning{From BeyondPlanck to Cosmoglobe}
\authorrunning{Watts et al.}
\titlerunning{\cosmoglobe\ \wmap{}  analysis}

\abstract{
We improve on the \textsc{BeyondPlanck} analysis, by reducing poorly measured modes in LFI polarization. In particular, we find a $\sim4\,\mathrm{\mu K}$ change in the 30~GHz channel as a result of including the higher signal-to-noise \WMAP\ \textit K-band maps. 
}

\keywords{ISM: general -- Cosmology: observations, polarization,
    cosmic microwave background, diffuse radiation -- Galaxy:
    general}

\maketitle

%\hypersetup{linkcolor=black}
\tableofcontents
%\hypersetup{linkcolor=red} 




\section{Introduction}
\label{sec:introduction}

A to-do list:

\begin{itemize}
	\item Find the time it takes for each beam to cross itself.
	\item Fix AME model (\textit{I'm not sure what motivated this, perhaps not necessary?}
	\item Fix noise model (\textit{Explained because of the Bessel filter plus linear trend})
\end{itemize}

A table to include
\begin{itemize}
	\item Spin rate -- 0.464\,rpm (7.57 mHz), but translations to $2.6$ degrees per second in boresight?
	\item Precession -- 1 rev/hour (0.3 mHz)
	\item Signal bandwidth extends from 0.008--8 Hz \citep{jarosik2003a}
	\item Beam size in degrees -- 0.88, 0.66, 0.51, 0.35, 0.22.
\end{itemize}


The cosmic microwave background (CMB) is the most direct probe of the initial state of the Universe. Since the initial discovery of the CMB \citep{penzias:1965}, subsequent experiments have continually refined the measurements, to the extent that the \WMAP\ results are generally considered bringing cosmology into the regime of precision science \citep{bennett2012}. Prior to \WMAP, it was common for CMB experiments to be superseded by more sensitive successors, with the noteworthy exceptions of \COBE/FIRAS and \COBE/DIRBE.

The \planck\ experiment, rather than superseding \WMAP, consistently used \WMAP\ data in its calibration, component separation, and cosmological analyses. The most direct comparison between \WMAP\ and \Planck\ is through analysis of the two experiments' frequency maps, as \WMAP's \K, \Ka, \Q, \V, and \W\ maps are interleaved by the \Planck\ LFI's 30, 44, and 70\,GHz bands. Since the initial \Planck\ data release, there have been several analyses comparing the two experiments by members of the \WMAP\ team \citep{larson2014,addison:2016,huang:2018,weiland:2018,weiland:2022} and by the \Planck\ team \citep{planck2014-a13,planck2016-l06,planck2016-l05}.

While the \WMAP\ low-level analysis has remained stable since \citet{bennett2012}, there has been continued work on \Planck\ time-ordered data processing, notably \bp\ for the LFI instrument \citep{bp01}, \sroll2\ for the HFI instrument \citep{delouis:2019}, and \Planck\ DR4 for both LFI and HFI \citep[\npipe,][]{npipe}. The LFI instrument in particular has had several systematics mitigated by improved analysis, particularly a smoothed gain solution and an improved noise model \citep{npipe,bp06,bp07,bp10}. When comparing \WMAP\ \K-band with the \Planck\ LFI data, the residuals are mainly characterized by \WMAP's poorly measured modes, which can be seen clearly in Figures 50 and 51 of \citet{npipe} and Figures 4 and 7 of \citet{bp14}.

One of the primary outcomes of the \BP\ project is that end-to-end analysis of a dataset with poorly measured modes can be mitigated by a joint analysis with another dataset that measures these modes well. In particular, \Planck\ LFI had large scale polarizated modes aligned with the instrument's scan strategy, induced by relative errors between different polarization-sensitive radiometers \citep{bp07}. The \bp\ project mitigated this by using \WMAP's polarized \Ka--\V\ maps for component separation, where these modes were well-measured. In order to properly combine these datasets, the polarized maps were the $N_\mathrm{side}=16$ \healpix\footnote{\url{http://healpix.sourceforge.net} \citep{gorski2005}} products with a pixel-pixel covariance matrix that explicitly projected out the poorly measured modes.

In principle, the \Planck\ experiment can be used to identify \WMAP's poorly measured modes in the same way that \WMAP\ removed \Planck's poorly measured modes. This was shown in \citet{bp17}, in which \WMAP\ data was calibrated against the \bp\ sky model, and the resulting maps differed from the \WMAPnine\ products mainly through the lack of the poorly measured modes. This work mainly functioned as a demonstration that the \commanderthree\ framework could be applied to the \WMAP\ dataset, and was not a true end-to-end analysis.

In this work, we present the first joint TOD analysis in the \cosmoglobe\footnote{\url{cosmoglobe.uio.no}} framework, in which we analyze the full \WMAP\ dataset along with time-ordered \Planck\ LFI data. In Sect.~\ref{sec:methods}, we review the \cosmoglobe\ statistical framework and the data processing for \Planck\ LFI and \WMAP\ in the \commanderthree\ pipeline. In Sect.~\ref{sec:freqmaps}, we present the \Planck\ and \WMAP\ joint frequency maps, and compare these frequency maps with the fiducial analyses in Sect.~\ref{sec:comparison}. We discuss outstanding systematic errors and the propagation of uncertainty in Sect.~\ref{sec:systematics}. We summarize our results and lay a path forward in Sect.~\ref{sec:conclusions}.



\textbf{Quotes regarding low-multipole null-test failures.}

\begin{itemize}
	\item \citet{page2007}, Sect.~5.2 -- 
		``When the W band and its cross-spectra are added to the mix, we find $\mathrm{PTE}<0.03$ for $l=5,7,9$, although all other values of $l$ give reasonable values. For BB, all frequency combinations yield reasonable PTEs for all $l$. Thus, there is a residual signal in our power spectra that we do not yet understand. It is evident in the W band in EE at $l=7$ and to a lesser degree at $l=5$ and $l=9$. We see no clear evidence of it anywhere else.''
		
		Also, due to the scan strategy (Sect.~5), ``\ldots Note in particular that we expect relatively larger error bars on $l=2,5,7$ in EE and on $l=3$ in BB.''
	\item \citet{hinshaw2009}, Sect.~6.1 -- 
		``Both spectrum estimates show excess power relative to the model, with the most puzzling multipole being $l=7$ (EE)\ldots It is worth recalling that $l=7$ EE, like $l=3$ BB, is a mode that is relatively poorly measured\ldots The W-band BB data also exhibit unusual behavior at $l=2,3$.''
	\item \citet{jarosik2010}, Sect.~4.1 --
		``\ldots the overall results do no (\textit{sic}) fully explain the excess variance observed in all the W-band $\ell\leq 7$ polarization multipoles. The low-$\ell$ W-band polarization data therefore continue to be excluded from cosmological analysis.''
\end{itemize}

\section{Constraining poorly measured modes in WMAP with \Cosmoglobe}
\label{sec:methods}







\section{Low-resolution sky maps}
\label{sec:skymaps}








\section{Likelihood analysis}
\label{sec:likelihood}






\section{Conclusions}
\label{sec:conclusions}









%\input{BP_wmap_acknowledgments.tex}

\bibliographystyle{../../common/aa}

\bibliography{../../common/Planck_bib,../../common/BP_bibliography}


\end{document}
