\documentclass[twocolumn]{../../common/aa}
%\documentclass[referee]{aa}

\usepackage{graphicx}
\usepackage{amsmath,amsfonts,amssymb}
\usepackage{txfonts}
\usepackage{color}
\usepackage{natbib}
\usepackage{float}
%\usepackage{stfloats}
\usepackage{dblfloatfix}
\usepackage{afterpage}
\usepackage{ifthen}
\usepackage[morefloats=12]{morefloats}
\usepackage{placeins}
\usepackage{multicol}
\usepackage{multirow}
\bibpunct{(}{)}{;}{a}{}{,}
\usepackage[switch]{lineno}
\definecolor{linkcolor}{rgb}{0.6,0,0}
\definecolor{citecolor}{rgb}{0,0,0.75}
\definecolor{urlcolor}{rgb}{0.12,0.46,0.7}
\usepackage[breaklinks, colorlinks, urlcolor=urlcolor,
    linkcolor=linkcolor,citecolor=citecolor,pdfencoding=auto]{hyperref}
\hypersetup{linktocpage}
\usepackage{bold-extra}
\usepackage{xcolor}
\newcommand{\gray}[0]{\color{gray}}

%\usepackage[grid,
%  gridcolor=red!20,
%  subgridcolor=green!20,
%  gridunit=cm]{eso-pic}




\def\setsymbol#1#2{\expandafter\def\csname #1\endcsname{#2}}
\def\getsymbol#1{\csname #1\endcsname}

\def\Planck{\textit{Planck}}

\def\HeJT{$^4$He-JT}

\def\allearlypapers{\nocite{planck2011-1.1, planck2011-1.3, planck2011-1.4, planck2011-1.5, planck2011-1.6, planck2011-1.7, planck2011-1.10, planck2011-1.10sup, planck2011-5.1a, planck2011-5.1b, planck2011-5.2a, planck2011-5.2b, planck2011-5.2c, planck2011-6.1, planck2011-6.2, planck2011-6.3a, planck2011-6.4a, planck2011-6.4b, planck2011-6.6, planck2011-7.0, planck2011-7.2, planck2011-7.3, planck2011-7.7a, planck2011-7.7b, planck2011-7.12, planck2011-7.13}}

\def\alltwentythirteenresultspapers{\nocite{planck2013-p01, planck2013-p02, planck2013-p02a, planck2013-p02d, planck2013-p02b, planck2013-p03, planck2013-p03c, planck2013-p03f, planck2013-p03d, planck2013-p03e, planck2013-p01a, planck2013-p06, planck2013-p03a, planck2013-pip88, planck2013-p08, planck2013-p11, planck2013-p12, planck2013-p13, planck2013-p14, planck2013-p15, planck2013-p05b, planck2013-p17, planck2013-p09, planck2013-p09a, planck2013-p20, planck2013-p19, planck2013-pipaberration, planck2013-p05, planck2013-p05a, planck2013-pip56, planck2013-p06b, planck2013-p01a}}

\def\alltwentyfifteenresultspapers{\nocite{planck2014-a01, planck2014-a03, planck2014-a04, planck2014-a05, planck2014-a06, planck2014-a07, planck2014-a08, planck2014-a09, planck2014-a11, planck2014-a12, planck2014-a13, planck2014-a14, planck2014-a15, planck2014-a16, planck2014-a17, planck2014-a18, planck2014-a19, planck2014-a20, planck2014-a22, planck2014-a24, planck2014-a26, planck2014-a28, planck2014-a29, planck2014-a30, planck2014-a31, planck2014-a35, planck2014-a36, planck2014-a37, planck2014-ES}}

\newbox\tablebox    \newdimen\tablewidth
\def\leaderfil{\leaders\hbox to 5pt{\hss.\hss}\hfil}
\def\endPlancktable{\tablewidth=\columnwidth 
    $$\hss\copy\tablebox\hss$$
    \vskip-\lastskip\vskip -2pt}
\def\endPlancktablewide{\tablewidth=\textwidth 
    $$\hss\copy\tablebox\hss$$
    \vskip-\lastskip\vskip -2pt}
\def\tablenote#1 #2\par{\begingroup \parindent=0.8em
    \abovedisplayshortskip=0pt\belowdisplayshortskip=0pt
    \noindent
    $$\hss\vbox{\hsize\tablewidth \hangindent=\parindent \hangafter=1 \noindent
    \hbox to \parindent{$^#1$\hss}\strut#2\strut\par}\hss$$
    \endgroup}
\def\doubleline{\vskip 3pt\hrule \vskip 1.5pt \hrule \vskip 5pt}

\def\L2{\ifmmode L_2\else $L_2$\fi}
\def\dtt{\Delta T/T}
\def\DeltaT{\ifmmode \Delta T\else $\Delta T$\fi}
\def\deltat{\ifmmode \Delta t\else $\Delta t$\fi}
\def\fknee{\ifmmode f_{\rm knee}\else $f_{\rm knee}$\fi}
\def\Fmax{\ifmmode F_{\rm max}\else $F_{\rm max}$\fi}
\def\solar{\ifmmode{\rm M}_{\mathord\odot}\else${\rm M}_{\mathord\odot}$\fi}
\def\Msolar{\ifmmode{\rm M}_{\mathord\odot}\else${\rm M}_{\mathord\odot}$\fi}
\def\Lsolar{\ifmmode{\rm L}_{\mathord\odot}\else${\rm L}_{\mathord\odot}$\fi}
\def\inv{\ifmmode^{-1}\else$^{-1}$\fi}
\def\mo{\ifmmode^{-1}\else$^{-1}$\fi}
\def\sup#1{\ifmmode ^{\rm #1}\else $^{\rm #1}$\fi}
\def\expo#1{\ifmmode \times 10^{#1}\else $\times 10^{#1}$\fi}
\def\,{\thinspace}
\def\lsim{\mathrel{\raise .4ex\hbox{\rlap{$<$}\lower 1.2ex\hbox{$\sim$}}}}
\def\gsim{\mathrel{\raise .4ex\hbox{\rlap{$>$}\lower 1.2ex\hbox{$\sim$}}}}
\let\lea=\lsim
\let\gea=\gsim
\def\simprop{\mathrel{\raise .4ex\hbox{\rlap{$\propto$}\lower 1.2ex\hbox{$\sim$}}}}
\def\deg{\ifmmode^\circ\else$^\circ$\fi}
\def\pdeg{\ifmmode $\setbox0=\hbox{$^{\circ}$}\rlap{\hskip.11\wd0 .}$^{\circ}
          \else \setbox0=\hbox{$^{\circ}$}\rlap{\hskip.11\wd0 .}$^{\circ}$\fi}
\def\arcs{\ifmmode {^{\scriptstyle\prime\prime}}
          \else $^{\scriptstyle\prime\prime}$\fi}
\def\arcm{\ifmmode {^{\scriptstyle\prime}}
          \else $^{\scriptstyle\prime}$\fi}
\newdimen\sa  \newdimen\sb
\def\parcs{\sa=.07em \sb=.03em
     \ifmmode \hbox{\rlap{.}}^{\scriptstyle\prime\kern -\sb\prime}\hbox{\kern -\sa}
     \else \rlap{.}$^{\scriptstyle\prime\kern -\sb\prime}$\kern -\sa\fi}
\def\parcm{\sa=.08em \sb=.03em
     \ifmmode \hbox{\rlap{.}\kern\sa}^{\scriptstyle\prime}\hbox{\kern-\sb}
     \else \rlap{.}\kern\sa$^{\scriptstyle\prime}$\kern-\sb\fi}
\def\ra[#1 #2 #3.#4]{#1\sup{h}#2\sup{m}#3\sup{s}\llap.#4}
\def\dec[#1 #2 #3.#4]{#1\deg#2\arcm#3\arcs\llap.#4}
\def\deco[#1 #2 #3]{#1\deg#2\arcm#3\arcs}
\def\rra[#1 #2]{#1\sup{h}#2\sup{m}}
\def\page{\vfill\eject}
\def\dots{\relax\ifmmode \ldots\else $\ldots$\fi}
\def\WHzsr{\ifmmode $W\,Hz\mo\,sr\mo$\else W\,Hz\mo\,sr\mo\fi}
\def\mHz{\ifmmode $\,mHz$\else \,mHz\fi}
\def\GHz{\ifmmode $\,GHz$\else \,GHz\fi}
\def\mKs{\ifmmode $\,mK\,s$^{1/2}\else \,mK\,s$^{1/2}$\fi}
\def\muKs{\ifmmode \,\mu$K\,s$^{1/2}\else \,$\mu$K\,s$^{1/2}$\fi}
\def\muKRJs{\ifmmode \,\mu$K$_{\rm RJ}$\,s$^{1/2}\else \,$\mu$K$_{\rm RJ}$\,s$^{1/2}$\fi}
\def\muKHz{\ifmmode \,\mu$K\,Hz$^{-1/2}\else \,$\mu$K\,Hz$^{-1/2}$\fi}
\def\MJysr{\ifmmode \,$MJy\,sr\mo$\else \,MJy\,sr\mo\fi}
\def\MJysrmK{\ifmmode \,$MJy\,sr\mo$\,mK$_{\rm CMB}\mo\else \,MJy\,sr\mo\,mK$_{\rm CMB}\mo$\fi}
\def\microns{\ifmmode \,\mu$m$\else \,$\mu$m\fi}
\def\micron{\microns}
\def\muK{\ifmmode \,\mu$K$\else \,$\mu$\hbox{K}\fi}
\def\microK{\ifmmode \,\mu$K$\else \,$\mu$\hbox{K}\fi}
\def\muW{\ifmmode \,\mu$W$\else \,$\mu$\hbox{W}\fi}
\def\kms{\ifmmode $\,km\,s$^{-1}\else \,km\,s$^{-1}$\fi}
\def\kmsMpc{\ifmmode $\,\kms\,Mpc\mo$\else \,\kms\,Mpc\mo\fi}

\providecommand{\sorthelp}[1]{}


\def\WMAP{\emph{WMAP}}
\def\WMAPnine{\emph{WMAP9}}
\def\COBE{\emph{COBE}}
\def\wmap{\emph{WMAP}}
\def\planck{\emph{Planck}}
\def\Planck{\emph{Planck}}
\def\LCDM{$\Lambda$CDM}
\def\ffp{FFP6}
\def\unionmask{U73}
\def\nside{N_{\mathrm{side}}}

\def\healpix{\texttt{HEALPix}}
\def\commander{\texttt{Commander}}
\def\commanderone{\texttt{Commander1}}
\def\commandertwo{\texttt{Commander2}}
\def\commanderthree{\texttt{Commander3}}
\def\ruler{\texttt{Ruler}}
\def\comrul{\texttt{Commander-Ruler}}
\def\CR{\texttt{C-R}}
\def\nilc{\texttt{NILC}}
\def\gnilc{\texttt{GNILC}}
\def\sevem{\texttt{SEVEM}}
\def\smica{\texttt{SMICA}}
\def\CamSpec{\texttt{CamSpec}}
\def\Plik{\texttt{Plik}}
\def\XFaster{\texttt{XFaster}}
\def\sroll2{\texttt{SRoll2}}

\newcommand{\phm}{\phantom{-}}
\renewcommand{\d}[0]{\vec{d}}
\renewcommand{\t}[0]{\vec{t}}
\newcommand{\A}[0]{\mathrm{A}}
\newcommand{\B}[0]{\mathrm{B}}
\newcommand{\Y}[0]{\tens{Y}}
\newcommand{\n}[0]{\vec{n}}
\newcommand{\red}[0]{\color{red}}
\newcommand{\green}[0]{\color{green}}
\newcommand{\s}[0]{\vec{s}}
\renewcommand{\a}[0]{\vec{a}}
\newcommand{\m}[0]{\vec{m}}
\newcommand{\f}[0]{\vec{f}}
\newcommand{\F}[0]{\tens{F}}
\newcommand{\T}[0]{\tens{T}}
\newcommand{\Cp}[0]{\tens{C}}
\renewcommand{\L}[0]{\tens{L}}
\newcommand{\g}[0]{\vec{g}}
\newcommand{\N}[0]{\tens{N}}
\newcommand{\M}[0]{\tens{M}}
\newcommand{\iN}[0]{\tens{N}^{-1}}
\newcommand{\iM}[0]{\tens{M}^{-1}}
\newcommand{\w}[0]{\vec{w}}
\renewcommand{\S}[0]{\tens{S}}
\renewcommand{\r}[0]{\vec{r}}
\renewcommand{\u}[0]{\vec{u}}
\newcommand{\q}[0]{\vec{q}}
\renewcommand{\v}[0]{\vec{v}}
\renewcommand{\P}[0]{\tens{P}}
\newcommand{\dt}[0]{d_t}
\newcommand{\di}[0]{d_i}
\newcommand{\nt}[0]{n_t}
\newcommand{\st}[0]{s_t}
\newcommand{\mt}[0]{m_t}
\newcommand{\ft}[0]{f_t}
\newcommand{\Te}[0]{T_{\rm e}}
\newcommand{\EM}[0]{\rm EM}
\newcommand{\mathsc}[1]{{\normalfont\textsc{#1}}}
\newcommand{\hi}{\ensuremath{\mathsc {Hi}}}
\newcommand{\bpbold}{\bfseries{\scshape{BeyondPlanck}}}
\newcommand{\BP}{\textsc{BeyondPlanck}}
\newcommand{\bp}{\textsc{BeyondPlanck}}
\newcommand{\cosmoglobe}{\textsc{Cosmoglobe}}
\newcommand{\Cosmoglobe}{\textsc{Cosmoglobe}}
\newcommand{\lfi}[0]{LFI}
\newcommand{\hfi}[0]{HFI}
\newcommand{\npipe}[0]{\texttt{NPIPE}}
\newcommand{\K}[0]{\textit K}
\newcommand{\Ka}[0]{\textit{Ka}}
\newcommand{\Q}[0]{\textit Q}
\newcommand{\V}[0]{\textit V}
\newcommand{\W}[0]{\textit W}
\newcommand{\e}{\mathrm e}
\newcommand{\cvar}{\ensuremath{c(\vartheta, \varphi, \psi)}}

\usepackage[T1]{fontenc}


\def\bC{\tens{C}}
\def\ba{\vec{a}}
\def\ncha{N_\mathrm{cha}}
\def\nfg{N_\mathrm{fg}}

\newcommand{\ncorr}{\vec n_\mathrm{corr}}
\newcommand{\Dbp}{\Delta_\mathrm{bp}}

%\modulolinenumbers[5]
%\linenumbers

\newcommand{\includegraphicsdpi}[3]{
    \pdfimageresolution=#1  % Change the dpi of images
    \includegraphics[#2]{#3}
    \pdfimageresolution=72  % Change it back to the default
}

\renewcommand{\topfraction}{1.0}	% max fraction of floats at top
    \renewcommand{\bottomfraction}{1.0}	% max fraction of floats at bottom
    %   Parameters for TEXT pages (not float pages):
    \setcounter{topnumber}{2}
    \setcounter{bottomnumber}{2}
    \setcounter{totalnumber}{4}     % 2 may work better
    \setcounter{dbltopnumber}{2}    % for 2-column pages
    \renewcommand{\dbltopfraction}{0.9}	% fit big float above 2-col. text
    \renewcommand{\textfraction}{0.04}	% allow minimal text w. figs
    %   Parameters for FLOAT pages (not text pages):
    \renewcommand{\floatpagefraction}{0.9}	% require fuller float pages
	% N.B.: floatpagefraction MUST be less than topfraction !!
    \renewcommand{\dblfloatpagefraction}{0.9}	% require fuller float pages

\def\adj{^{\dagger}}
\def\tp{^{\rm T}}
\def\inv{^{-1}}
\def\lm{{\ell m}}

\begin{document}

\title{\bfseries{\Cosmoglobe\ I. Improved \emph{Wilkinson Microwave Anisotropy Probe} frequency maps through Bayesian end-to-end analysis}}
%This author list corresponds to \title{Author list for L04\_CMB\_Foregrounds\_Extraction}
%Prepared by M. Lopez-Caniego (Marcos.Lopez.Caniego@sciops.esa.int), ESAC/ESA
%This version is from Thu Jul 12 18:11:48 2018 CET
%\subtitle{There are 152 co-authors in this list}
\newcommand{\oslo}[0]{1}
%\newcommand{\MIT}[0]{2}
\newcommand{\milanoA}[0]{2}
\newcommand{\milanoB}[0]{3}
\newcommand{\milanoC}[0]{4}
\newcommand{\triesteB}[0]{5}
\newcommand{\planetek}[0]{6}
\newcommand{\princeton}[0]{7}
\newcommand{\jpl}[0]{8}
\newcommand{\helsinkiA}[0]{9}
\newcommand{\helsinkiB}[0]{10}
\newcommand{\nersc}[0]{11}
\newcommand{\haverford}[0]{12}
\newcommand{\mpa}[0]{13}
\newcommand{\triesteA}[0]{14}
\newcommand{\iia}[0]{2}

\author{\small
J.~R.~Eskilt\inst{\oslo}\thanks{Corresponding author: J.~R.~Eskilt; \url{j.r.eskilt@astro.uio.no}}
\and
K.~Lee\inst{\oslo}
\and
D.~J.~Watts\inst{\oslo}
\and
S.~Nerval\inst{\oslo}
\and
et al.
}
\institute{\small
        Institute of Theoretical Astrophysics, University of Oslo, Blindern, Oslo, Norway \goodbreak
}

%\authorrunning{From BeyondPlanck to Cosmoglobe}
\authorrunning{Watts et al.}
\titlerunning{\cosmoglobe\ \wmap{}  analysis}

\abstract{
    We present the first joint analysis of \textit{WMAP} and \textit{Planck} LFI data, presenting maps that have been generated from a fully consistent joint treatment, including the sampling of sky signals and instrumental properties. The joint analysis approach yields improved \textit{WMAP} data with better treatment of poorly constrained modes, as well as the first fully optimal sampling of all nine years of data.  We also improve on the \textsc{BeyondPlanck} analysis, by reducing poorly measured modes in LFI polarization. In particular, we find a $\sim4\,\mathrm{\mu K}$ change in the 30~GHz channel as a result of including the higher signal-to-noise \WMAP\ \textit K-band maps. The \WMAP\ maps we present are free of previously documented systematic effects, and have an $x$\% reduction in the white noise level. As the first release of \textsc{Cosmoglobe} products, the maps from this analysis should be considered both a considerable improvement over previous analyses, as well as the first iteration of future joint analyses with other data, including, e.g., the ground-based QUIET experiment and the DIRBE instrument aboard \textit{COBE}.
}

\keywords{ISM: general -- Cosmology: observations, polarization,
    cosmic microwave background, diffuse radiation -- Galaxy:
    general}

\maketitle

%\hypersetup{linkcolor=black}
\tableofcontents
%\hypersetup{linkcolor=red} 




\section{Introduction}
\label{sec:introduction}

\textbf{Describe WMAP highlights, establishment of firm LCDM model. Brief summary of Planck, Commander, BeyondPlanck, and then Cosmoglobe; WMAP is first major extension of the end-to-end framework, and marks the transition from classical single-experiment TOD analysis to multi-experiment TOD analysis. A brave new world.}

A to-do list:

\begin{itemize}
	\item Find the time it takes for each beam to cross itself.
	\item Fix AME model (\textit{I'm not sure what motivated this, perhaps not necessary?}
	\item Fix noise model (\textit{Explained because of the Bessel filter plus linear trend})
\end{itemize}

A table to include
\begin{itemize}
	\item Spin rate -- 0.464\,rpm (7.57 mHz), but translations to $2.6$ degrees per second in boresight?
	\item Precession -- 1 rev/hour (0.3 mHz)
	\item Signal bandwidth extends from 0.008--8 Hz \citep{jarosik2003a}
	\item Beam size in degrees -- 0.88, 0.66, 0.51, 0.35, 0.22.
\end{itemize}


The cosmic microwave background (CMB) is the most direct probe of the initial state of the Universe. Since the initial discovery of the CMB \citep{penzias:1965}, subsequent experiments have continually refined the measurements, to the extent that the \WMAP\ results are generally considered bringing cosmology into the regime of precision science \citep{bennett2012}. Prior to \WMAP, it was common for CMB experiments to be superseded by more sensitive successors, with the noteworthy exceptions of \COBE/FIRAS and \COBE/DIRBE.

The \planck\ experiment, rather than superseding \WMAP, consistently used \WMAP\ data in its calibration, component separation, and cosmological analyses. The most direct comparison between \WMAP\ and \Planck\ is through analysis of the two experiments' frequency maps, as \WMAP's \K, \Ka, \Q, \V, and \W\ maps are interleaved by the \Planck\ LFI's 30, 44, and 70\,GHz bands. Since the initial \Planck\ data release, there have been several analyses comparing the two experiments by members of the \WMAP\ team \citep{larson2014,addison:2016,huang:2018,weiland:2018,weiland:2022} and by the \Planck\ team \citep{planck2014-a13,planck2016-l06,planck2016-l05}.

While the \WMAP\ low-level analysis has remained stable since \citet{bennett2012}, there has been continued work on \Planck\ time-ordered data processing, notably \bp\ for the LFI instrument \citep{bp01}, \sroll2\ for the HFI instrument \citep{delouis:2019}, and \Planck\ DR4 for both LFI and HFI \citep[\npipe,][]{npipe}. The LFI instrument in particular has had several systematics mitigated by improved analysis, particularly a smoothed gain solution and an improved noise model \citep{npipe,bp06,bp07,bp10}. When comparing \WMAP\ \K-band with the \Planck\ LFI data, the residuals are mainly characterized by \WMAP's poorly measured modes, which can be seen clearly in Figures 50 and 51 of \citet{npipe} and Figures 4 and 7 of \citet{bp14}.

One of the primary outcomes of the \BP\ project is that end-to-end analysis of a dataset with poorly measured modes can be mitigated by a joint analysis with another dataset that measures these modes well. In particular, \Planck\ LFI had large scale polarizated modes aligned with the instrument's scan strategy, induced by relative errors between different polarization-sensitive radiometers \citep{bp07}. The \bp\ project mitigated this by using \WMAP's polarized \Ka--\V\ maps for component separation, where these modes were well-measured. In order to properly combine these datasets, the polarized maps were the $N_\mathrm{side}=16$ \healpix\footnote{\url{http://healpix.sourceforge.net} \citep{gorski2005}} products with a pixel-pixel covariance matrix that explicitly projected out the poorly measured modes.

In principle, the \Planck\ experiment can be used to identify \WMAP's poorly measured modes in the same way that \WMAP\ removed \Planck's poorly measured modes. This was shown in \citet{bp17}, in which \WMAP\ data was calibrated against the \bp\ sky model, and the resulting maps differed from the \WMAPnine\ products mainly through the lack of the poorly measured modes. This work mainly functioned as a demonstration that the \commanderthree\ framework could be applied to the \WMAP\ dataset, and was not a true end-to-end analysis.

In this work, we present the first joint TOD analysis in the \cosmoglobe\footnote{\url{cosmoglobe.uio.no}} framework, in which we analyze the full \WMAP\ dataset along with time-ordered \Planck\ LFI data. In Sect.~\ref{sec:methods}, we review the \cosmoglobe\ statistical framework and the data processing for \Planck\ LFI and \WMAP\ in the \commanderthree\ pipeline. In Sect.~\ref{sec:freqmaps}, we present the \Planck\ and \WMAP\ joint frequency maps, and compare these frequency maps with the fiducial analyses in Sect.~\ref{sec:comparison}. We discuss outstanding systematic errors and the propagation of uncertainty in Sect.~\ref{sec:systematics}. We summarize our results and lay a path forward in Sect.~\ref{sec:conclusions}.



\section{End-to-end Bayesian \WMAP\ analysis}
\label{sec:methods}

\subsection{Brief overview of official \WMAP\ pipeline}

\subsection{\WMAP\ instrument model}

\textbf{Write down data model; refer to WMAP and BP papers extensively}

In the \cosmoglobe\ paradigm, it is simplest to characterize the data's goodness-of-fit to a model,
\begin{equation}
	\label{eq:model}
	\boldsymbol d =\mathsf G\mathsf P[\mathsf B^\mathrm{symm}\mathsf M\boldsymbol a+\mathsf B^\mathrm{4\pi}(\boldsymbol s^\mathrm{orb}
	+\boldsymbol s^\mathrm{fsl})] + \boldsymbol s^\mathrm{inst}+\boldsymbol n^\mathrm{corr}+\boldsymbol n^\mathrm w,
\end{equation}
where $\mathsf G$ is the time-dependent gain in the form of the matrix $\mathrm{diag}(g_t)$, $\mathsf P$ is the $n_p\times n_t$ pointing matrix, 
$\mathsf B^\mathrm{symm}$ and $\mathsf B^{4\pi}$ are the symmetrized and full symmetric beam, respectively, $\mathsf M$ is the mixing matrix between a given component $c$ with spectral energy distribution $f_c(\nu/\nu_{0,c})$ and a detector $j$ with bandpass $\tau_j(\nu)$, given by
\begin{equation}
	\mathsf M_{cj}=\int d\nu\,\tau_j(\nu)f_c(\nu/\nu_{c,0}).
\end{equation}
The maps $\boldsymbol a$ are the component amplitudes, $\boldsymbol s^\mathrm{orb}$ is the orbital dipole induced by the motion of the telescope with respect to the Sun, and $\boldsymbol s^\mathrm{fsl}$ is the time-dependent far sidelobe signal. In the \commanderthree\ \citep{bp03} implementation, $\boldsymbol n^\mathrm{corr}$ is a realization of the correlated noise component whose SED is parametrized by $P(f\mid\boldsymbol\xi_n)$, where $\boldsymbol\xi_n$ generally includes $f_\mathrm{knee}$, a slope $\alpha$, and whose amplitude is fixed by the white noise $\sigma_0$. This model is often augmented, as we will discuss in Sects.~\ref{ssec:lfi} and \ref{ssec:wmap}. Similarly, each experiment has particular signals that are specific to the instrument in question, e.g., the 1\,Hz spike in \Planck\ LFI or the horn imbalance in \WMAP, which can be modeled by $\boldsymbol s^\mathrm{inst}$.

\subsection{Astrophysical sky model: A revised AME SED model}

\begin{figure}
	\centering
	\includegraphics[width=\linewidth]{figures/tempfit_AME_single_v1.pdf}
	\caption{AME exponential}
\end{figure}

\subsection{Posterior distribution and Gibbs sampling}

As shown in \citet{bp01}, this allows us to write down a total model for the data, $\boldsymbol d=\boldsymbol s^\mathrm{tot}(\boldsymbol\omega)+\boldsymbol n^\mathrm w$, where $\boldsymbol s^\mathrm{tot}$ encompasses all of the terms in Eq.~\eqref{eq:model} except for the white noise term. Assuming that all instrumental effects have been modelled, the data should be Gaussian distributed with a mean of $\boldsymbol s^\mathrm{tot}(\boldsymbol\omega)$ and variance $\boldsymbol \sigma_0^2$. Given this model, we can evaluate the likelihood for arbitrary chunks of time-ordered data in the context of the entire model, so that individual chunks of data with poor fits can be more easily identified. In general, the likelihood is written
\begin{equation}
	P(\boldsymbol d\mid\boldsymbol\omega)\propto\exp\left(-\frac12\sum_t\frac{(d_t-s^\mathrm{tot}_t(\boldsymbol\omega))^2}{\sigma_0^2}
	\right).
\end{equation}
If $\boldsymbol d\sim\mathcal N(\boldsymbol s^\mathrm{tot},\boldsymbol\sigma_0^2)$ is the correct model for the data, the argument of the exponent is proportional to a $\chi^2$-distribution with $n_\mathrm{TOD}$ degrees of freedom. In the limit of large $n$, a $\chi^2$ distribution is well-approximated by a Gaussian with mean $n$ and variance $2n$. Therefore we define and use the reduced-$\chi^2$ statistic,
\begin{equation}
	\chi^2\equiv \frac{\sum_t((d_t-s_t^\mathrm{tot})^2/\sigma_0^2 - n_\mathrm{TOD}}{\sqrt{2n_\mathrm{TOD}}},
\end{equation}
which is approximately drawn from the standard normal distribution $\mathcal N(0,1)$.

The \cosmoglobe\ Gibbs chain is given by
\begin{alignat}{10}
\label{eq:gain_samp_dist}\g &\,\leftarrow          P(\g&\,               \mid \d, &\,    &          &\,\boldsymbol\xi_n,  &\,\boldsymbol s^\mathrm{inst}, &\,\boldsymbol\beta, &\,\boldsymbol a, &\,C_{\ell},&\,\boldsymbol\theta)\\
\label{eq:ncorr_samp_dist} \ncorr &\,\leftarrow    P(\ncorr&\,           \mid \d, &\,\g, &\,        &\,\boldsymbol\xi_n,  &\,\boldsymbol s^\mathrm{inst}, &\,\boldsymbol\beta, &\,\boldsymbol a, &\,C_{\ell},&\,\boldsymbol\theta)\\ 
\label{eq:xi_samp_dist} \boldsymbol\xi_n &\,\leftarrow        P(\boldsymbol\xi_n&\,            \mid \d, &\,\g, &\,\ncorr, &\,        &\,\boldsymbol s^\mathrm{inst}, &\,\boldsymbol\beta, &\,\boldsymbol a, &\,C_{\ell},&\,\boldsymbol\theta)\\
\label{eq:samp_inst}\boldsymbol s^\mathrm{inst} &\,\leftarrow                                 P(\boldsymbol s^\mathrm{inst}&\,             \mid \d, &\,\g, &\,\ncorr, &\,\boldsymbol\xi_n,  &\,      &\,\boldsymbol\beta, &\,\boldsymbol a, &\,C_{\ell},&\,\boldsymbol\theta)\\
\label{eq:beta_samp}\boldsymbol\beta &\,\leftarrow                     P(\boldsymbol\beta&\, \mid \d, &\,\g, &\,\ncorr, &\,\boldsymbol\xi_n,  &\,\boldsymbol s^\mathrm{inst}, &\,       &\,    &\,C_{\ell},&\,\boldsymbol\theta)\\
\boldsymbol a &\,\leftarrow                                   P(\boldsymbol a&\,               \mid \d, &\,\g, &\,\ncorr, &\,\boldsymbol\xi_n,  &\,\boldsymbol s^\mathrm{inst}, &\,\boldsymbol\beta, &\,    &\,C_{\ell},&\,\boldsymbol\theta)\\
C_{\ell} &\,\leftarrow                             P(C_{\ell}&\,         \mid \d, &\,\g, &\,\ncorr, &\,\boldsymbol\xi_n,  &\,\boldsymbol s^\mathrm{inst}, &\,\boldsymbol\beta, &\,\boldsymbol a,&\,\phantom{C_{\ell}}&\,\boldsymbol\theta)&\label{eq:cl_sampling}\\
\boldsymbol\theta &\,\leftarrow                             P(\boldsymbol\theta&\,         \mid \d, &\,\g, &\,\ncorr, &\,\boldsymbol\xi_n,  &\,\boldsymbol s^\mathrm{inst}, &\,\boldsymbol\beta, &\,\boldsymbol a,&\,C_\ell\phantom{,}&\phantom{\,\boldsymbol\theta})\label{eq:param_samp},
\end{alignat}
with each step requiring its own dedicated sampling algorithm, and in the case of \bp, its own publication. The \commanderthree\ pipeline is designed so that results of each Gibbs sample can be easily passed to each other, and that the internal calculations of each step do not directly depend on the inner workings of each other. Therefore, in order to add another data set to the Gibbs chain, one must implement Eqs.~\eqref{eq:gain_samp_dist}--\eqref{eq:samp_inst} for each instrument, as was done in \citet{bp01} and \citet{bp10} for \Planck\ LFI and in \citet{bp17} for \WMAP, or simply pass processed maps with beam, mask, and noise information to Eqs.~\eqref{eq:beta_samp}--\eqref{eq:param_samp}, as was done for the Haslam 408\,MHz map \citep{haslam1982,remazeilles2014} and the \Planck\ 353 and 857\,GHz maps.

Before we discuss the results of this Gibbs chain as applied to the \Planck\ LFI and \WMAP\ data, we summarize the TOD processing steps in Sects.~\ref{ssec:lfi} and \ref{ssec:wmap}.

\begin{table*}[t]
  \begingroup
  \newdimen\tblskip \tblskip=5pt
  \caption{Computational resources required for end-to-end
    \cosmoglobe\ processing. All times correspond to CPU hours. All reported times are
    averaged over more than 100 samples, and vary by $\lesssim\,5\,\%$ from sample to
    sample. \textbf{Preliminary numbers, not all accurate}}
  \label{tab:resources}
  \nointerlineskip
  \vskip -3mm
  \footnotesize
  \setbox\tablebox=\vbox{
    \newdimen\digitwidth
    \setbox0=\hbox{\rm 0}
    \digitwidth=\wd0
    \catcode`*=\active
    \def*{\kern\digitwidth}
    %
    \newdimen\signwidth
    \setbox0=\hbox{-}
    \signwidth=\wd0
    \catcode`!=\active
    \def!{\kern\signwidth}
    %
 \halign{
      \hbox to 5.0cm{#\leaderfil}\tabskip 1em&
      \hfil#\tabskip 1em&
      \hfil#\tabskip 1em&
      \hfil#\tabskip 1em&
      \hfil#\tabskip 1em&
      \hfil#\tabskip 1em&
      \hfil#\tabskip 1em&
      \hfil#\tabskip 1em&
      \hfil#\tabskip 1em&
      \hfil#\tabskip 1em&
      \hfil#\tabskip 1em&
      \hfil#\tabskip 1em&
      \hfil#\tabskip 1em&
      \hfil#\tabskip 1em&
      %\hfil#\tabskip 2em&
      #\tabskip 0em\hfil\cr
    \noalign{\doubleline}
      \omit\textsc{Item}\hfil&
      \omit\hfil\textsc{30}\hfil&
      \omit\hfil\textsc{44}\hfil&
      \omit\hfil\textsc{70}\hfil&
      \omit\hfil\K\hfil&
      \omit\hfil\Ka\hfil&
      \omit\hfil\Q1\hfil&
      \omit\hfil\Q2\hfil&
      \omit\hfil\V1\hfil&
      \omit\hfil\V2\hfil&
      \omit\hfil\W1\hfil&
      \omit\hfil\W2\hfil&
      \omit\hfil\W3\hfil&
      \omit\hfil\W4\hfil&
      \omit\hfil\textsc{Sum}\hfil\cr
      %\omit\hfil\textsc{Sum}\hfil&
      %\omit\hfil\textsc{Reference}\hfil\cr
      \noalign{\vskip 4pt\hrule\vskip 4pt}
      %\noalign{\vskip 8pt}
      %\noalign{\vskip 5pt\hrule\vskip 5pt}
      %  \multispan5\textit{Data volume}\hfil\cr
      %  \noalign{\vskip 2pt}
      %  \hskip 10pt Uncompressed TOD volume                 & {\gray 761 GB} &
      %  {\gray 1\,633 GB} & {\gray 5\,522 GB} & {\gray 7\,915 GB}& \cr
      %  \hskip 10pt Compressed TOD volume & **86
      %  GB & *178 GB & **597 GB & ***861 GB& \cr
      %  \hskip 10pt Non-TOD-related RAM usage &   &  &  & ***659 GB& \cr
      %  \hskip 10pt {\bf Total RAM requirements} &  &  &  & **{\bf1\,520 GB}& \cr      
      %  \noalign{\vskip 2pt}      
      \multispan5\textit{Processing time (cost per run)}\hfil\cr
      \hskip 10pt TOD initialization/IO time                    &1.8& 2.5& 9.3& 0.3& 0.3& 0.4& 0.4& 0.4& 0.4& 0.5& 0.5& 0.5& 0.5& 17.8\cr
      \hskip 10pt Other initialization                                &  &  &  &  & & & & & & & & & & 13.4\cr
      \hskip 10pt {\bf Total initialization}                          &  &  &  &  & & & & & & & & & & {\bf 31.2}\cr
      \noalign{\vskip 2pt}      
      \multispan5\textit{Gibbs sampling steps (cost per sample)}\hfil\cr
      %\noalign{\vskip 2pt}
      \hskip 10pt Huffman decompression                            & 1.1& 2.1& 10.5& 0.9& 0.8& 1.0& 1.0& 1.3& 1.3& 1.8& 1.8& 1.8& 1.8& 27.2\cr
      \hskip 10pt TOD projection ($\P$ operation)                  & 0.4& 0.9&  4.2& 2.6& 2.6& 3.3& 3.4& 4.3& 4.3& 6.4& 6.3& 6.3& 6.4& 54.0\cr
      \hskip 10pt Sidelobe evaluation                              & 1.0& 2.1&  7.6& 2.9& 2.9& 3.5& 3.5& 4.7& 4.8& 7.0& 6.9& 6.9& 6.9& 60.7\cr
      \hskip 10pt Orbital dipole                                   & 0.9& 1.9& 7.1& 1.3& 1.3& 1.7& 1.7& 2.2& 2.3& 3.4& 3.3& 3.3& 3.3& 33.7\cr
      \hskip 10pt Gain sampling                                    & 0.5& 0.8& 1.9& 0.8& 0.8& 0.5& 0.5& 0.9& 0.9& 0.7& 0.7& 0.7& 0.7& 10.4\cr
      \hskip 10pt 1\,Hz spike sampling                             & 0.3& 0.4& 1.6& & & & & & & & & & & 2.4\cr      
      \hskip 10pt Correlated noise sampling                        & 2.0& 4.0& 21.7& 2.8& 2.9& 3.3& 3.6& 5.1& 5.4& 8.0& 7.7& 7.2& 8.5& 81.3\cr
      \hskip 10pt Correlated noise PSD sampling                    & 4.8& 5.9& 1.5& 0.2& 0.2& 0.3& 0.3 & 0.5& 0.4& 0.7& 0.6& 0.6& 0.7& 16.7\cr
      \hskip 10pt TOD binning ($\P^t$ operation)                   & 0.1& 0.1& 4.0& 0.5& 0.5& 0.7& 0.8& 0.8& 0.8& 1.2& 12& 1.2& 1.2& 13.1\cr
      \hskip 10pt Mapmaking                                        & & & & 6.4& 7.0& 8.9& 8.1& 11.1& 9.5& 14.4& 14.3& 15.3& 16.4& 119.5\cr
      \hskip 10pt Sum of other TOD processing                      & 4.4& 8.6& 44.4& 14.7& 4.6& 5.1& 5.0& 9.4& 7.7& 8.1& 6.8& 8.6& 8.7& 136.1\cr
      \hskip 10pt {\bf TOD processing cost per sample}             & {\bf 15.5}& {\bf 26.8}& {\bf 104.5}&  {\bf 23.0}& {\bf 24.1}& {\bf 27.6}& {\bf 27.9}& {\bf 40.3}& {\bf 37.4}& {\bf 51.7}& {\bf 50.6}& {\bf 51.9}& {\bf 54.6}& {\bf 535.9}\cr
      \noalign{\vskip 2pt}
      \hskip 10pt Amplitude sampling  &&&&&&&&&&&   &  &  & 14.0\cr
      \hskip 10pt Spectral index sampling  &&&&&&&&&&&   &  &  & 25.5\cr
      \noalign{\vskip 2pt}
      \hskip 10pt {\bf Total cost per sample}                  &&&&&&&&&& &   &  &  &  {\bf 581.2}\cr
      \noalign{\vskip 4pt\hrule\vskip 5pt} } }
  \endPlancktablewide \endgroup
\end{table*}


\subsection{Sampling algorithms}

\subsubsection{Review of previous work}
\label{ssec:oldsamplers}


\subsubsection{Differential mapmaking}
\label{ssec:mapmaking}

\textbf{Mapmaking; review/summary of BP17}


\subsubsection{Transmission imbalance sampling}
\label{ssec:mapmaking}

\textbf{transmission imbalance; review/summary of BP17}

\subsubsection{Baseline sampling}
\label{ssec:baseline}

\textbf{baselines; review/summary of BP17}

%\subsubsection{Other things we have added?}
%\label{ssec:baseline}



%\subsection{\planck\ LFI data processing}
%\label{ssec:lfi}

Summary of \citet{bp01}, \citet{bp10}.
Artem was here...

%\subsection{\wmap\ data processing}
%\label{ssec:wmap}

As summarized in \citet{bp17}, the TOD processing for \WMAP\ has been
implemented in the \commanderthree\ Gibbs sampling framework.  The process of
implementing this module in \commanderthree\ generally involved implementing
the procedures outlined in \citet{jarosik2003a}, \citet{hinshaw2003a},
\citet{jarosik2007}, and \citet{wmapexsupp}, and implementing a sampling
algorithm whenever an estimate based off of the data was obtained. \citet{bp17}
found excellent agreement between our processed maps and the official
\WMAPnine\ products. The most visually obvious exception is that the
\citet{bp17} maps have no trace of the poorly-measured modes induced by
transmission imbalance uncertainties. Although \citet{bp17} focused on the
\Q-band maps, we show in Sect.~\ref{sec:freqmaps} that this is representative
of all \WMAP\ bands, with the notable exception of \K-band.

The data model adopted in \citet{hinshaw2003a} can be written in raw digital units (du) as
\begin{equation}
	\d = \mathsf{GPBM}\boldsymbol a+\boldsymbol n+\boldsymbol b,
\end{equation}
where $\boldsymbol b$ is the instrumental baseline and $\boldsymbol n$ is the
total instrumental noise. As detailed in \citet{bp06}, \commanderthree\ divides
the noise into $\n=\n^\mathrm w+\n^\mathrm{corr}$, a white noise term and a
correlated noise term. By definition, the white noise does not have any
correlations between adjacent pixels, so that any pixel-pixel covariance should
be fully described by realizations of the $\n^\mathrm{corr}$ timestream.

Three notable deviations from the official \WMAPnine\ pipeline and the
\cosmoglobe\ approach are the gain estimation, the correlated noise treatment,
and the data fraction used.  The \WMAPnine\ gain solution was a parametric fit
to the orbital dipole measurements as a function of the instrumental
housekeeping parameters, as detailed in Appendix~A of \citet{wmapexsupp}. This
approach avoids the dependence of a robust Galactic foreground model, and
requires both excellent knowledge of the radiometers and a clean orbital dipole
measurement which can be compared to the known signal. \commanderthree's
sampling approach, as summarized in \citet{bp07}, fits the average gain over
all detectors and TOD samples to the orbital dipole. In addition,
\commanderthree\ calibrates deviations from the average gain as a function of
detector and time to the total sky model.  This approach is highly dependent on
using a sky model that accurately reflects the data, but makes fewer
assumptions regarding the behavior of the detectors in relation to the
housekeeping parameters. 


The second main deviation is in the treatment the noise power spectra. As shown
in Sect~2.5 of \citet{jarosik2007}, the noise autocorrelation spectrum is fit
on a year-by-year basis to a polynomial in $\log(\Delta t)$, where $\Delta t$
is the time lag between data points. This method is very similar to the
\commanderthree\ approach, which fits for the power spectrum in Fourier space
using a correlated noise model of the form
$\sigma_0^2(f/f_\mathrm{knee})^{\alpha}$. Properly parameterized, these two
approaches should yield similar results, albeit with different levels of
uncertainty and time resolution. However, we have confirmed that in many cases
the simple $1/f$ noise model does not fit the signal-subtracted TOD, yielding
$\chi^2$ values that are up to $10\sigma$ discrepant from their expected
values.  [Show, discuss figure with the PSDs, residual spectrum, and Bessel
filter.]

Deviations from the $1/f$ model consist either of a linear increase or downturn
above 10\,Hz. This can be partially explained by the use of a two pole Bessel
low-pass filter just prior to signal quantization, which introduces a 2.62\%
correlation between 25.6\,ms sample integrations
\citep[][Sect.~5.3]{jarosik2003:MAP}. The exact form of the Bessel filter was
not used on flight data, but rather the parametric fit as discussed above.
However, the filter is designed to reduce the signal by half at 100\,Hz, and as
such has a negligible effect. 








The third main difference in the \Cosmoglobe\ treatment is the amount of data used, and the size of individual scans used. 
As shown in \citet{bp03}, a large fraction of \commander's computational time is spent performing FFTs on individual scans. Rather than truncating datastreams to have lengths equal to ``magic numbers'' for which \texttt{FFTW} \citep{FFTW05} is fastest, as in \citet{bp03}, 
we split the data into scans of length $2^N$, where $N=22$ for \K--\Q, $N=23$ for \V--\W. This yields scans with lengths of 6.21 days for \K- and \Ka-band, 4.97 days for \Q-band, 7.46 days for \V-band, and 4.97 days for \W-band.
These datastream lengths are short enough to be processed quickly and distributed efficiently across multiple processors, while being long enough to properly characterize the noise properties of the timestreams, whose $f_\mathrm{knee}$'s are on the order $1\,\mathrm{mHz}$.

\citet{wmapexsupp} lists a series of flagged events, which we isolated to their own TOD segments. When we encountered these events, TOD segments that were interrupted by the event were appended to the previous TOD, in most cases creating TODs with lengths $>2^N$. We found that events of length $<2^N$ were too short to accurately estimate the noise PSD parameters. This criterion led to unflagged data to be discarded. A similar problem occurred when $>10\%$ of the TOD was flagged. Together, these two effects led to $\simeq1\%$ of the data to be discarded despite on their face acceptable. We present the full statistics for our maps in Table~\ref{table:flagged_data}. In total, the \Cosmoglobe\ maps use slightly less data than the \textit{WMAP9} official products, which had a total efficiency of $\simeq98.4\%$ \citep{bennett2012}. The total difference in data can be entirely accounted for by the cuts described in this paragraph.


%Compare with Table 2 of \citet{bennett2003a}, Table 1 of \citet{hinshaw2007}, Table 2 of \citet{hinshaw2009}. Total observing efficiency was 98.4\% \citep{bennett2012}

%\subsection{Calibration}

In order to get the raw data into the sky units $\mathrm{mK_{CMB}}$, it is necessary to make a quantitative assessment of the data's goodness-of-fit. To do this, we use mapped time-domain residuals of the timestream based on an initial sky model. Following the procedure of \citet{bp06}, we create a smoothed map of the absolute deviation of the model from the sky signal. The residual maps were first smoothed to $3^\circ$, the absolute value was taken, and then the map was smoothed again with a $15^\circ$ beam. This procedure gave us an indication of the regions of the sky that are not fully characterized by the sky model. We then evaluated the \texttt{Cosmoglobe} Sky model\footnote{\url{cosmoglobe.readthedocs.io}} associated with the \BP\ data release\footnote{\url{beyondplanck.science}} to create a map of the expected thermal dust, AME, free-free, and radio emission at each \WMAP\ frequency. We thresholded using the 80th percentile of this map to match the residual map. The combination of this sky model-informed mask and the residual-informed mask formed the main processing mask for the calibration and correlated noise analysis.
Outside of these mask, the data are fit will by the sky model, so the calibration and the correlated noise estimates should both be unbiased.

Emphasize, we use the orbital dipole for the absolute calibration, but use the full sky model from the rest of the Gibbs chain for the rest of the gain and the imbalance parameters.


\section{Data}
\label{sec:data}

\subsection{Time-ordered data}
\label{sec:tod}

\subsection{Survey of fixed \WMAP\ data products}
\label{sec:fixedproducts}

\textbf{Summarize everything we assume from the WMAP analysis: Beams, bandpasses, pointing, etc.}



\section{Data selection, Markov chains and resources}
\label{sec:markov_chains}

\subsection{Data selection}

\begin{table}
\caption{Flagging statistics}              % title of Table
\label{table:flagged_data}      % is used to refer this table in the text
\centering                                      % used for centering table
\begin{tabular}{c r r r}          % centered columns (4 columns)
\hline\hline                        % inserts double horizontal lines
	Band & Flagged (\%) & Discarded (\%) & Used (\%) \\    % table heading
\hline                                   % inserts single horizontal line
	\K  &  1.72 & 0.87 & 97.4\\
	\Ka &  1.64 & 0.88 & 97.5\\      % inserting body of the table
	\Q1 &  1.84 & 0.84 & 96.5\\
	\Q2 &  1.62 & 0.81 & 97.6\\
	\V1 &  1.62 & 1.10 & 97.3\\
	\V2 &  1.61 & 1.01 & 97.4\\
	\W1 &  1.76 & 1.03 & 97.2\\
	\W2 &  1.60 & 0.81 & 97.6\\
	\W3 &  1.61 & 0.87 & 97.5\\
	\W4 &  1.60 & 0.81 & 97.6\\
\hline                                             %inserts single line
\end{tabular}
\end{table}

djfksajf lasj fkdjas fklsaj lfjdsl kfkj jasdlkfj aslkfj klsadjf lsdajf lkasdj flksdj kflsajdk f
djfksajf lasj fkdjas fklsaj lfjdsl kfkj jasdlkfj aslkfj klsadjf lsdajf lkasdj flksdj kflsajdk f
djfksajf lasj fkdjas fklsaj lfjdsl kfkj jasdlkfj aslkfj klsadjf lsdajf lkasdj flksdj kflsajdk f
jfksajf lasj fkdjas fklsaj lfjdsl kfkj jasdlkfj aslkfj klsadjf lsdajf lkasdj flksdj kflsajdk f
djfksajf lasj fkdjas fklsaj lfjdsl kfkj jasdlkfj aslkfj klsadjf lsdajf lkasdj flksdj kflsajdk f

djfksajf lasj fkdjas fklsaj lfjdsl kfkj jasdlkfj aslkfj klsadjf lsdajf lkasdj flksdj kflsajdk f
jfksajf lasj fkdjas fklsaj lfjdsl kfkj jasdlkfj aslkfj klsadjf lsdajf lkasdj flksdj kflsajdk f
djfksajf lasj fkdjas fklsaj lfjdsl kfkj jasdlkfj aslkfj klsadjf lsdajf lkasdj flksdj kflsajdk f
djfksajf lasj fkdjas fklsaj lfjdsl kfkj jasdlkfj aslkfj klsadjf lsdajf lkasdj flksdj kflsajdk f
jfksajf lasj fkdjas fklsaj lfjdsl kfkj jasdlkfj aslkfj klsadjf lsdajf lkasdj flksdj kflsajdk f

djfksajf lasj fkdjas fklsaj lfjdsl kfkj jasdlkfj aslkfj klsadjf lsdajf lkasdj flksdj kflsajdk f
djfksajf lasj fkdjas fklsaj lfjdsl kfkj jasdlkfj aslkfj klsadjf lsdajf lkasdj flksdj kflsajdk f
jfksajf lasj fkdjas fklsaj lfjdsl kfkj jasdlkfj aslkfj klsadjf lsdajf lkasdj flksdj kflsajdk f
djfksajf lasj fkdjas fklsaj lfjdsl kfkj jasdlkfj aslkfj klsadjf lsdajf lkasdj flksdj kflsajdk f
djfksajf lasj fkdjas fklsaj lfjdsl kfkj jasdlkfj aslkfj klsadjf lsdajf lkasdj flksdj kflsajdk f
jfksajf lasj fkdjas fklsaj lfjdsl kfkj jasdlkfj aslkfj klsadjf lsdajf lkasdj flksdj kflsajdk f

djfksajf lasj fkdjas fklsaj lfjdsl kfkj jasdlkfj aslkfj klsadjf lsdajf lkasdj flksdj kflsajdk f
djfksajf lasj fkdjas fklsaj lfjdsl kfkj jasdlkfj aslkfj klsadjf lsdajf lkasdj flksdj kflsajdk f
jfksajf lasj fkdjas fklsaj lfjdsl kfkj jasdlkfj aslkfj klsadjf lsdajf lkasdj flksdj kflsajdk f
djfksajf lasj fkdjas fklsaj lfjdsl kfkj jasdlkfj aslkfj klsadjf lsdajf lkasdj flksdj kflsajdk f
djfksajf lasj fkdjas fklsaj lfjdsl kfkj jasdlkfj aslkfj klsadjf lsdajf lkasdj flksdj kflsajdk f


\subsection{Trace plots}
\label{sec:traceplots}

\subsection{Parameter correlations}
\label{sec:correlations}

djfksajf lasj fkdjas fklsaj lfjdsl kfkj jasdlkfj aslkfj klsadjf lsdajf lkasdj flksdj kflsajdk f
djfksajf lasj fkdjas fklsaj lfjdsl kfkj jasdlkfj aslkfj klsadjf lsdajf lkasdj flksdj kflsajdk f
djfksajf lasj fkdjas fklsaj lfjdsl kfkj jasdlkfj aslkfj klsadjf lsdajf lkasdj flksdj kflsajdk f
jfksajf lasj fkdjas fklsaj lfjdsl kfkj jasdlkfj aslkfj klsadjf lsdajf lkasdj flksdj kflsajdk f
djfksajf lasj fkdjas fklsaj lfjdsl kfkj jasdlkfj aslkfj klsadjf lsdajf lkasdj flksdj kflsajdk f

djfksajf lasj fkdjas fklsaj lfjdsl kfkj jasdlkfj aslkfj klsadjf lsdajf lkasdj flksdj kflsajdk f
jfksajf lasj fkdjas fklsaj lfjdsl kfkj jasdlkfj aslkfj klsadjf lsdajf lkasdj flksdj kflsajdk f
djfksajf lasj fkdjas fklsaj lfjdsl kfkj jasdlkfj aslkfj klsadjf lsdajf lkasdj flksdj kflsajdk f
djfksajf lasj fkdjas fklsaj lfjdsl kfkj jasdlkfj aslkfj klsadjf lsdajf lkasdj flksdj kflsajdk f
jfksajf lasj fkdjas fklsaj lfjdsl kfkj jasdlkfj aslkfj klsadjf lsdajf lkasdj flksdj kflsajdk f

djfksajf lasj fkdjas fklsaj lfjdsl kfkj jasdlkfj aslkfj klsadjf lsdajf lkasdj flksdj kflsajdk f
djfksajf lasj fkdjas fklsaj lfjdsl kfkj jasdlkfj aslkfj klsadjf lsdajf lkasdj flksdj kflsajdk f
jfksajf lasj fkdjas fklsaj lfjdsl kfkj jasdlkfj aslkfj klsadjf lsdajf lkasdj flksdj kflsajdk f
djfksajf lasj fkdjas fklsaj lfjdsl kfkj jasdlkfj aslkfj klsadjf lsdajf lkasdj flksdj kflsajdk f
djfksajf lasj fkdjas fklsaj lfjdsl kfkj jasdlkfj aslkfj klsadjf lsdajf lkasdj flksdj kflsajdk f
jfksajf lasj fkdjas fklsaj lfjdsl kfkj jasdlkfj aslkfj klsadjf lsdajf lkasdj flksdj kflsajdk f

djfksajf lasj fkdjas fklsaj lfjdsl kfkj jasdlkfj aslkfj klsadjf lsdajf lkasdj flksdj kflsajdk f
djfksajf lasj fkdjas fklsaj lfjdsl kfkj jasdlkfj aslkfj klsadjf lsdajf lkasdj flksdj kflsajdk f
jfksajf lasj fkdjas fklsaj lfjdsl kfkj jasdlkfj aslkfj klsadjf lsdajf lkasdj flksdj kflsajdk f
djfksajf lasj fkdjas fklsaj lfjdsl kfkj jasdlkfj aslkfj klsadjf lsdajf lkasdj flksdj kflsajdk f
djfksajf lasj fkdjas fklsaj lfjdsl kfkj jasdlkfj aslkfj klsadjf lsdajf lkasdj flksdj kflsajdk f

\subsection{Computational resources}
\label{sec:resources}




\section{Low-level posterior distributions}
\label{sec:lowlevel}

\begin{figure}[t]
  	\centering
	\includegraphics[width=\linewidth]{figures/instpar_CG_K111_v1.pdf}
	\caption{Overview of K113}
	\label{fig:inst_K113}
\end{figure}


\subsection{Gain and baselines}

djfksajf lasj fkdjas fklsaj lfjdsl kfkj jasdlkfj aslkfj klsadjf lsdajf lkasdj flksdj kflsajdk f
djfksajf lasj fkdjas fklsaj lfjdsl kfkj jasdlkfj aslkfj klsadjf lsdajf lkasdj flksdj kflsajdk f
djfksajf lasj fkdjas fklsaj lfjdsl kfkj jasdlkfj aslkfj klsadjf lsdajf lkasdj flksdj kflsajdk f
jfksajf lasj fkdjas fklsaj lfjdsl kfkj jasdlkfj aslkfj klsadjf lsdajf lkasdj flksdj kflsajdk f
djfksajf lasj fkdjas fklsaj lfjdsl kfkj jasdlkfj aslkfj klsadjf lsdajf lkasdj flksdj kflsajdk f

djfksajf lasj fkdjas fklsaj lfjdsl kfkj jasdlkfj aslkfj klsadjf lsdajf lkasdj flksdj kflsajdk f
jfksajf lasj fkdjas fklsaj lfjdsl kfkj jasdlkfj aslkfj klsadjf lsdajf lkasdj flksdj kflsajdk f
djfksajf lasj fkdjas fklsaj lfjdsl kfkj jasdlkfj aslkfj klsadjf lsdajf lkasdj flksdj kflsajdk f
djfksajf lasj fkdjas fklsaj lfjdsl kfkj jasdlkfj aslkfj klsadjf lsdajf lkasdj flksdj kflsajdk f
jfksajf lasj fkdjas fklsaj lfjdsl kfkj jasdlkfj aslkfj klsadjf lsdajf lkasdj flksdj kflsajdk f

djfksajf lasj fkdjas fklsaj lfjdsl kfkj jasdlkfj aslkfj klsadjf lsdajf lkasdj flksdj kflsajdk f
djfksajf lasj fkdjas fklsaj lfjdsl kfkj jasdlkfj aslkfj klsadjf lsdajf lkasdj flksdj kflsajdk f
jfksajf lasj fkdjas fklsaj lfjdsl kfkj jasdlkfj aslkfj klsadjf lsdajf lkasdj flksdj kflsajdk f
djfksajf lasj fkdjas fklsaj lfjdsl kfkj jasdlkfj aslkfj klsadjf lsdajf lkasdj flksdj kflsajdk f
djfksajf lasj fkdjas fklsaj lfjdsl kfkj jasdlkfj aslkfj klsadjf lsdajf lkasdj flksdj kflsajdk f
jfksajf lasj fkdjas fklsaj lfjdsl kfkj jasdlkfj aslkfj klsadjf lsdajf lkasdj flksdj kflsajdk f

djfksajf lasj fkdjas fklsaj lfjdsl kfkj jasdlkfj aslkfj klsadjf lsdajf lkasdj flksdj kflsajdk f
djfksajf lasj fkdjas fklsaj lfjdsl kfkj jasdlkfj aslkfj klsadjf lsdajf lkasdj flksdj kflsajdk f
jfksajf lasj fkdjas fklsaj lfjdsl kfkj jasdlkfj aslkfj klsadjf lsdajf lkasdj flksdj kflsajdk f
djfksajf lasj fkdjas fklsaj lfjdsl kfkj jasdlkfj aslkfj klsadjf lsdajf lkasdj flksdj kflsajdk f
djfksajf lasj fkdjas fklsaj lfjdsl kfkj jasdlkfj aslkfj klsadjf lsdajf lkasdj flksdj kflsajdk f


\subsection{Transmission imbalance}

\begin{figure}[t]
  	\centering
	\includegraphics[width=\linewidth]{figures/x_im_CG_v1.pdf}
	\caption{Transmission imbalance}
	\label{fig:x_im}
\end{figure}

\begin{table}
\newdimen\tblskip \tblskip=5pt
\caption{Transmission imbalance parameters for each \WMAP\ radiometer as estimated in the current analysis (\emph{second column}) and in the official 9-year \WMAP\ analysis (\emph{third column}). Our uncertainties indicate $1\,\sigma$ marginal posterior standard deviations. }
\label{tab:dipole}
\vskip -8mm
\footnotesize
\setbox\tablebox=\vbox{
 \newdimen\digitwidth
 \setbox0=\hbox{\rm 0}
 \digitwidth=\wd0
 \catcode`*=\active
 \def*{\kern\digitwidth}
%
  \newdimen\dpwidth
  \setbox0=\hbox{.}
  \dpwidth=\wd0
  \catcode`!=\active
  \def!{\kern\dpwidth}
%
  \halign{\hbox to 1.8cm{#\leaderfil}\tabskip 2em&
    \hfil$#$\hfil \tabskip 2em&
    \hfil$#$\hfil \tabskip 0em\cr
\noalign{\doubleline}
\omit\hfil\sc Radiometer \hfil& x_{\mathrm{im}}^{\mathrm{CG}}& x_{\mathrm{im}}^{\mathrm{WMAP}}\cr
\noalign{\vskip 3pt\hrule\vskip 5pt}
K11 &   0.00018 \pm  0.00013  &  -0.00067 \pm  0.00017 \cr
K12 &   0.00388 \pm  0.00015  &   0.00536 \pm  0.00014 \cr
Ka11 &   0.00339 \pm  0.00012  &   0.00353 \pm  0.00017 \cr
Ka12 &   0.00150 \pm  0.00010  &   0.00154 \pm  0.00008 \cr
Q11 &   0.00081 \pm  0.00016  &  -0.00013 \pm  0.00046 \cr
Q12 &   0.00517 \pm  0.00027  &   0.00414 \pm  0.00025 \cr
Q21 &   0.00985 \pm  0.00042  &   0.00756 \pm  0.00052 \cr
Q22 &   0.01235 \pm  0.00011  &   0.00986 \pm  0.00115 \cr
V11 &   0.00012 \pm  0.00041  &   0.00053 \pm  0.00020 \cr
V12 &   0.00212 \pm  0.00089  &   0.00250 \pm  0.00057 \cr
V21 &   0.00246 \pm  0.00012  &   0.00352 \pm  0.00033 \cr
V22 &   0.00323 \pm  0.00070  &   0.00245 \pm  0.00098 \cr
W11 &   0.01169 \pm  0.00105  &   0.01134 \pm  0.00199 \cr
W12 &   0.00442 \pm  0.00109  &   0.00173 \pm  0.00036 \cr
W21 &   0.01595 \pm  0.00052  &   0.01017 \pm  0.00216 \cr
W22 &   0.01540 \pm  0.00167  &   0.01142 \pm  0.00121 \cr
W31 &  -0.00089 \pm  0.00010  &  -0.00122 \pm  0.00062 \cr
W32 &   0.00354 \pm  0.00084  &   0.00463 \pm  0.00041 \cr
W41 &   0.02734 \pm  0.00219  &   0.02311 \pm  0.00380 \cr
W42 &   0.01882 \pm  0.00282  &   0.02054 \pm  0.00202 \cr
\noalign{\vskip 5pt\hrule\vskip 5pt}}}
\endPlancktablewide
\end{table}


\subsection{Noise characterization}



\begin{table*}
\newdimen\tblskip \tblskip=5pt
\caption{Summary of noise properties. }
\label{tab:noise}
\vskip -4mm
\footnotesize
\setbox\tablebox=\vbox{
 \newdimen\digitwidth
 \setbox0=\hbox{\rm 0}
 \digitwidth=\wd0
 \catcode`*=\active
 \def*{\kern\digitwidth}
%
  \newdimen\dpwidth
  \setbox0=\hbox{.}
  \dpwidth=\wd0
  \catcode`!=\active
  \def!{\kern\dpwidth}
%
  \halign{\hbox to 2.cm{#\leaderfil}\tabskip 2em&
    \hfil#\hfil \tabskip 1em&
    \hfil$#$\hfil \tabskip 0.5em&
    \hfil$#$\hfil \tabskip 0.5em&
    \hfil$#$\hfil \tabskip 2em&    
    \hfil$#$\hfil \tabskip 0.5em&
    \hfil$#$\hfil \tabskip 0.5em&
    \hfil$#$\hfil \tabskip 2em&
    \hfil$#$\hfil \tabskip 0em\cr
\noalign{\doubleline}
\omit&&\multispan3\hfil Sensitivity, $\sigma_0$ (mK\,$\sqrt{\mathrm{s}}$) \hfil&
\multispan3\hfil Knee frequency, $f_{\mathrm{knee}}$ (mHz) \hfil& \cr
\noalign{\vskip -3pt}
\omit&&\multispan3\hrulefill&\multispan3\hrulefill&\omit\cr
\noalign{\vskip 3pt} 
Radiometer & Diode & \textrm{GSFC} & \textrm{WMAP} & \textrm{CG}/\sqrt{2} & \textrm{GSFC} & \textrm{WMAP} & \textrm{CG}/\sqrt{2} & \mathrm{Slope}, \alpha \cr
\noalign{\vskip 3pt\hrule\vskip 5pt}
K11  &  1  &  0.72  &  0.66  &   0.704 \pm  0.002  &  6.13  &  0.4  &    0.82 \pm   0.20  &   -1.01 \pm   0.10 \cr
\omit &  2  &  &  &   0.708 \pm  0.003  &  &  &    0.63 \pm   0.14  &   -0.95 \pm   0.10 \cr
K12  &  1  &  0.87  &  0.75  &   0.796 \pm  0.004  &  5.37  &  0.51  &    0.42 \pm   0.19  &   -0.93 \pm   0.12 \cr
\omit &  2  &  &  &   0.780 \pm  0.005  &  &  &    0.71 \pm   0.15  &   -1.02 \pm   0.10 \cr
Ka11  &  1  &  0.75  &  0.71  &   0.788 \pm  0.001  &  1.66  &  0.71  &    1.20 \pm   0.22  &   -1.02 \pm   0.09 \cr
\omit &  2  &  &  &   0.777 \pm  0.001  &  &  &    1.19 \pm   0.22  &   -1.02 \pm   0.09 \cr
Ka12  &  1  &  0.77  &  0.72  &   0.788 \pm  0.003  &  1.29  &  0.32  &    0.62 \pm   0.16  &   -0.99 \pm   0.11 \cr
\omit &  2  &  &  &   0.784 \pm  0.001  &  &  &    0.63 \pm   0.13  &   -1.01 \pm   0.11 \cr
Q11  &  1  &  0.99  &  0.92  &   0.998 \pm  0.002  &  3.21  &  1.09  &    1.06 \pm   0.16  &   -1.09 \pm   0.09 \cr
\omit &  2  &  &  &   0.992 \pm  0.002  &  &  &    1.06 \pm   0.16  &   -1.10 \pm   0.09 \cr
Q12  &  1  &  0.95  &  1.02  &   1.159 \pm  0.007  &  3.13  &  0.35  &    0.45 \pm   0.47  &   -0.98 \pm   0.11 \cr
\omit &  2  &  &  &   1.146 \pm  0.007  &  &  &    0.83 \pm   0.14  &   -1.00 \pm   0.09 \cr
Q21  &  1  &  0.89  &  0.85  &   0.908 \pm  0.002  &  1.92  &  5.76  &    2.88 \pm   0.37  &   -1.10 \pm   0.07 \cr
\omit &  2  &  &  &   0.906 \pm  0.002  &  &  &    3.22 \pm   0.56  &   -1.10 \pm   0.06 \cr
Q22  &  1  &  1.04  &  0.99  &   1.074 \pm  0.004  &  4.61  &  8.62  &    3.95 \pm   0.54  &   -1.11 \pm   0.06 \cr
\omit &  2  &  &  &   1.064 \pm  0.003  &  &  &    4.05 \pm   0.64  &   -1.11 \pm   0.06 \cr
V11  &  1  &  1.25  &  1.22  &   1.551 \pm  0.003  &  2.56  &  0.09  &    1.27 \pm   0.15  &   -0.90 \pm   0.06 \cr
\omit &  2  &  &  &   1.539 \pm  0.003  &  &  &    1.19 \pm   0.14  &   -0.89 \pm   0.06 \cr
V12  &  1  &  1.07  &  1.11  &   1.398 \pm  0.002  &  4.49  &  1.41  &    2.11 \pm   0.20  &   -0.97 \pm   0.05 \cr
\omit &  2  &  &  &   1.432 \pm  0.002  &  &  &    1.88 \pm   0.17  &   -0.96 \pm   0.05 \cr
V21  &  1  &  1.01  &  0.97  &   1.241 \pm  0.298  &  2.43  &  0.88  &    1.50 \pm   0.24  &   -0.95 \pm   0.07 \cr
\omit &  2  &  &  &   1.217 \pm  0.294  &  &  &    1.60 \pm   0.26  &   -0.97 \pm   0.06 \cr
V22  &  1  &  1.13  &  1.1  &   1.443 \pm  0.300  &  3.06  &  8.35  &    4.01 \pm   0.85  &   -1.00 \pm   0.08 \cr
\omit &  2  &  &  &   1.415 \pm  0.316  &  &  &    3.08 \pm   0.65  &   -1.01 \pm   0.08 \cr
W11  &  1  &  1.18  &  1.35  &   1.938 \pm  0.005  &  16.2  &  7.88  &    5.59 \pm   0.53  &   -0.94 \pm   0.05 \cr
\omit &  2  &  &  &   1.895 \pm  0.005  &  &  &    8.99 \pm   0.85  &   -0.95 \pm   0.04 \cr
W12  &  1  &  1.41  &  1.61  &   2.301 \pm  0.005  &  15.1  &  0.66  &    3.91 \pm   0.42  &   -0.89 \pm   0.05 \cr
\omit &  2  &  &  &   2.345 \pm  0.006  &  &  &    4.81 \pm   0.53  &   -0.89 \pm   0.05 \cr
W21  &  1  &  1.38  &  1.61  &   2.225 \pm  0.007  &  1.76  &  9.02  &   13.57 \pm   1.47  &   -0.89 \pm   0.03 \cr
\omit &  2  &  &  &   2.292 \pm  0.006  &  &  &    5.06 \pm   0.95  &   -0.93 \pm   0.05 \cr
W22  &  1  &  1.44  &  1.72  &   2.291 \pm  0.006  &  0.77  &  7.47  &    3.02 \pm   0.53  &   -0.98 \pm   0.05 \cr
\omit &  2  &  &  &   2.232 \pm  0.007  &  &  &    7.26 \pm   1.05  &   -0.95 \pm   0.04 \cr
W31  &  1  &  1.47  &  1.65  &   2.328 \pm  0.005  &  1.84  &  0.93  &    1.30 \pm   0.46  &   -0.99 \pm   0.07 \cr
\omit &  2  &  &  &   2.322 \pm  0.006  &  &  &    1.97 \pm   0.28  &   -0.98 \pm   0.06 \cr
W32  &  1  &  1.69  &  1.86  &   2.707 \pm  0.015  &  2.39  &  0.28  &    1.59 \pm   0.29  &   -0.98 \pm   0.07 \cr
\omit &  2  &  &  &   2.579 \pm  0.015  &  &  &    1.40 \pm   0.39  &   -1.00 \pm   0.07 \cr
W41  &  1  &  1.6  &  1.71  &   2.519 \pm  0.010  &  8.46  &  46.5  &   26.81 \pm   1.83  &   -0.92 \pm   0.04 \cr
\omit &  2  &  &  &   2.479 \pm  0.009  &  &  &   24.75 \pm   1.63  &   -0.92 \pm   0.04 \cr
W42  &  1  &  1.43  &  1.65  &   2.221 \pm  0.017  &  5.31  &  26.0  &   16.10 \pm   1.09  &   -0.94 \pm   0.04 \cr
\omit &  2  &  &  &   2.202 \pm  0.015  &  &  &   17.11 \pm   1.19  &   -0.94 \pm   0.04 \cr
\noalign{\vskip 5pt\hrule\vskip 5pt}}}
\endPlancktablewide
\end{table*}


\subsection{Goodness-of-fit}

djfksajf lasj fkdjas fklsaj lfjdsl kfkj jasdlkfj aslkfj klsadjf lsdajf lkasdj flksdj kflsajdk f
djfksajf lasj fkdjas fklsaj lfjdsl kfkj jasdlkfj aslkfj klsadjf lsdajf lkasdj flksdj kflsajdk f
djfksajf lasj fkdjas fklsaj lfjdsl kfkj jasdlkfj aslkfj klsadjf lsdajf lkasdj flksdj kflsajdk f
jfksajf lasj fkdjas fklsaj lfjdsl kfkj jasdlkfj aslkfj klsadjf lsdajf lkasdj flksdj kflsajdk f
djfksajf lasj fkdjas fklsaj lfjdsl kfkj jasdlkfj aslkfj klsadjf lsdajf lkasdj flksdj kflsajdk f

djfksajf lasj fkdjas fklsaj lfjdsl kfkj jasdlkfj aslkfj klsadjf lsdajf lkasdj flksdj kflsajdk f
jfksajf lasj fkdjas fklsaj lfjdsl kfkj jasdlkfj aslkfj klsadjf lsdajf lkasdj flksdj kflsajdk f
djfksajf lasj fkdjas fklsaj lfjdsl kfkj jasdlkfj aslkfj klsadjf lsdajf lkasdj flksdj kflsajdk f
djfksajf lasj fkdjas fklsaj lfjdsl kfkj jasdlkfj aslkfj klsadjf lsdajf lkasdj flksdj kflsajdk f
jfksajf lasj fkdjas fklsaj lfjdsl kfkj jasdlkfj aslkfj klsadjf lsdajf lkasdj flksdj kflsajdk f

djfksajf lasj fkdjas fklsaj lfjdsl kfkj jasdlkfj aslkfj klsadjf lsdajf lkasdj flksdj kflsajdk f
djfksajf lasj fkdjas fklsaj lfjdsl kfkj jasdlkfj aslkfj klsadjf lsdajf lkasdj flksdj kflsajdk f
jfksajf lasj fkdjas fklsaj lfjdsl kfkj jasdlkfj aslkfj klsadjf lsdajf lkasdj flksdj kflsajdk f
djfksajf lasj fkdjas fklsaj lfjdsl kfkj jasdlkfj aslkfj klsadjf lsdajf lkasdj flksdj kflsajdk f
djfksajf lasj fkdjas fklsaj lfjdsl kfkj jasdlkfj aslkfj klsadjf lsdajf lkasdj flksdj kflsajdk f
jfksajf lasj fkdjas fklsaj lfjdsl kfkj jasdlkfj aslkfj klsadjf lsdajf lkasdj flksdj kflsajdk f

djfksajf lasj fkdjas fklsaj lfjdsl kfkj jasdlkfj aslkfj klsadjf lsdajf lkasdj flksdj kflsajdk f
djfksajf lasj fkdjas fklsaj lfjdsl kfkj jasdlkfj aslkfj klsadjf lsdajf lkasdj flksdj kflsajdk f
jfksajf lasj fkdjas fklsaj lfjdsl kfkj jasdlkfj aslkfj klsadjf lsdajf lkasdj flksdj kflsajdk f
djfksajf lasj fkdjas fklsaj lfjdsl kfkj jasdlkfj aslkfj klsadjf lsdajf lkasdj flksdj kflsajdk f
djfksajf lasj fkdjas fklsaj lfjdsl kfkj jasdlkfj aslkfj klsadjf lsdajf lkasdj flksdj kflsajdk f


\section{Frequency maps}
\label{sec:maps}

\subsection{Posterior mean maps}

\begin{figure*}
	\centering
	\includegraphics[width=0.75\textwidth]{figures/023-WMAP_K_mu_I.pdf}
	\includegraphics[width=0.75\textwidth]{figures/023-WMAP_K_mu_Q.pdf}
	\includegraphics[width=0.75\textwidth]{figures/023-WMAP_K_mu_U.pdf}
	\caption{\K-band}
\end{figure*}
\begin{figure*}
	\centering
	\includegraphics[width=0.75\textwidth]{figures/030-WMAP_Ka_mu_I.pdf}
	\includegraphics[width=0.75\textwidth]{figures/030-WMAP_Ka_mu_Q.pdf}
	\includegraphics[width=0.75\textwidth]{figures/030-WMAP_Ka_mu_U.pdf}
	\caption{\Ka-band}
\end{figure*}
\begin{figure*}
	\centering
	\includegraphics[width=0.75\textwidth]{figures/Q_mu_I.pdf}
	\includegraphics[width=0.75\textwidth]{figures/Q_mu_Q.pdf}
	\includegraphics[width=0.75\textwidth]{figures/Q_mu_U.pdf}
	\caption{\Q-band}
\end{figure*}
\begin{figure*}
	\centering
	\includegraphics[width=0.75\textwidth]{figures/V_mu_I.pdf}
	\includegraphics[width=0.75\textwidth]{figures/V_mu_Q.pdf}
	\includegraphics[width=0.75\textwidth]{figures/V_mu_U.pdf}
	\caption{\V-band}
\end{figure*}
\begin{figure*}
	\centering
	\includegraphics[width=0.75\textwidth]{figures/W_mu_I.pdf}
	\includegraphics[width=0.75\textwidth]{figures/W_mu_Q.pdf}
	\includegraphics[width=0.75\textwidth]{figures/W_mu_U.pdf}
	\caption{\W-band}
\end{figure*}


\subsection{Posterior uncertainties}

\begin{figure*}[p]
	\centering
	\includegraphics[width=0.9\textwidth]{figures/023-WMAP_K_rms.pdf}
	\includegraphics[width=0.9\textwidth]{figures/030-WMAP_Ka_rms.pdf}
	\includegraphics[width=0.9\textwidth]{figures/040-WMAP_Q1_rms.pdf}
	\includegraphics[width=0.9\textwidth]{figures/040-WMAP_Q2_rms.pdf}
	\includegraphics[width=0.9\textwidth]{figures/060-WMAP_V1_rms.pdf}
	\includegraphics[width=0.9\textwidth]{figures/060-WMAP_V2_rms.pdf}
	\includegraphics[width=0.9\textwidth]{figures/090-WMAP_W1_rms.pdf}
	\includegraphics[width=0.9\textwidth]{figures/090-WMAP_W2_rms.pdf}
	\includegraphics[width=0.9\textwidth]{figures/090-WMAP_W3_rms.pdf}
	\includegraphics[width=0.9\textwidth]{figures/090-WMAP_W4_rms.pdf}
	\caption{RMS maps}
        \label{fig:rms}
\end{figure*}

\begin{figure*}[p]
	\centering
	\includegraphics[width=0.9\textwidth]{figures/023-WMAP_K_std.pdf}
	\includegraphics[width=0.9\textwidth]{figures/030-WMAP_Ka_std.pdf}
	\includegraphics[width=0.9\textwidth]{figures/040-WMAP_Q1_std.pdf}
	\includegraphics[width=0.9\textwidth]{figures/040-WMAP_Q2_std.pdf}
	\includegraphics[width=0.9\textwidth]{figures/060-WMAP_V1_std.pdf}
	\includegraphics[width=0.9\textwidth]{figures/060-WMAP_V2_std.pdf}
	\includegraphics[width=0.9\textwidth]{figures/090-WMAP_W1_std.pdf}
	\includegraphics[width=0.9\textwidth]{figures/090-WMAP_W2_std.pdf}
	\includegraphics[width=0.9\textwidth]{figures/090-WMAP_W3_std.pdf}
	\includegraphics[width=0.9\textwidth]{figures/090-WMAP_W4_std.pdf}
	\caption{STD std}
        \label{fig:std}
\end{figure*}
\begin{figure*}
	\centering
	\includegraphics[width=0.7\textwidth]{figures/023-WMAP_K_sampdiff.pdf}
	\includegraphics[width=0.7\textwidth]{figures/030-WMAP_Ka_sampdiff.pdf}
	\includegraphics[width=0.7\textwidth]{figures/040-WMAP_Q1_sampdiff.pdf}
	\includegraphics[width=0.7\textwidth]{figures/040-WMAP_Q2_sampdiff.pdf}
	\includegraphics[width=0.7\textwidth]{figures/060-WMAP_V1_sampdiff.pdf}
	\includegraphics[width=0.7\textwidth]{figures/060-WMAP_V2_sampdiff.pdf}
	\includegraphics[width=0.7\textwidth]{figures/090-WMAP_W1_sampdiff.pdf}
	\includegraphics[width=0.7\textwidth]{figures/090-WMAP_W2_sampdiff.pdf}
	\includegraphics[width=0.7\textwidth]{figures/090-WMAP_W3_sampdiff.pdf}
	\includegraphics[width=0.7\textwidth]{figures/090-WMAP_W4_sampdiff.pdf}
	\caption{Differences between two samples}
        \label{fig:sampdiff}
\end{figure*}


\subsection{Angular power spectra}

Will combine these spectra shortly

\begin{figure*}
	\centering
	\includegraphics[width=0.8\textwidth]{figures/TT_spectra.pdf}
	\caption{TT power spectra}
\end{figure*}
\begin{figure*}
	\centering
	\includegraphics[width=0.8\textwidth]{figures/EE_spectra.pdf}
	\caption{EE power spectra}
\end{figure*}
\begin{figure*}
	\centering
	\includegraphics[width=0.8\textwidth]{figures/BB_spectra.pdf}
	\caption{BB power spectra}
\end{figure*}
%\begin{figure*}
%	\centering
%	\includegraphics[width=0.8\textwidth]{figures/TT_spectra_zoom.pdf}
%	\caption{TT power spectra zoom}
%\end{figure*}
%\begin{figure*}
%	\centering
%	\includegraphics[width=0.8\textwidth]{figures/EE_spectra_zoom.pdf}
%	\caption{EE power spectra zoom}
%\end{figure*}
%\begin{figure*}
%	\centering
%	\includegraphics[width=0.8\textwidth]{figures/BB_spectra_zoom.pdf}
%	\caption{BB power spectra zoom}
%\end{figure*}
\begin{figure*}
	\centering
	\includegraphics[width=0.8\textwidth]{figures/TT_spectra_ratio.pdf}
	\caption{TT ratios}
\end{figure*}
\begin{figure*}
	\centering
	\includegraphics[width=0.8\textwidth]{figures/EE_spectra_ratio.pdf}
	\caption{EE ratios}
\end{figure*}
\begin{figure*}
	\centering
	\includegraphics[width=0.8\textwidth]{figures/BB_spectra_ratio.pdf}
	\caption{BB ratios}
\end{figure*}


%\section{Preliminary astrophysical results}
%\subsection{Diffuse foregrounds}
%\subsection{CMB constraints}
%\subsubsection{Solar dipole}
%\subsubsection{Temperature power spectrum}


\clearpage


Want to compare the $QU$ correlation in WMAP and Planck LFI, get a quantitative number. Point out that the polarization solution itself is much better, but the covariance between pixels themselves is much higher. This wasn't an issue for LFI, so we had to take that into account here.

I also want to put a bit here on why the low-$\ell$ approach needed to be done separately, how correlated noise sampling addresses it, to what extent it's mitigated, etc.

Note that LFI's 30 and 70\,GHz $QU$ correlation is $\sim0.1$, whereas 44\,GHz is much larger, $\sim0.5$. This discrepancy is due to the number of horns with differing polarization orientation. Both 30 and 70\,GHz have an even number of horns, allowing for pairs of datastreams to be combined to give independent polarization measurements. Conversely, 44\,GHz has one horn pair and an unpaired horn, the latter of which induces more correlation in the $QU$ observation matrix. An example for \Ka\ and 30\,GHz is shown in Fig.~\ref{fig:rho_qu}. Aside from the obvious morphological changes due to the two experiments' different observing strategies, the magnitude of \WMAP's correlation is much larger than \Planck's.

The \BP\ project took this covariance structure into account using the dense $N_\mathrm{side}=16$ noise covariance matrix provided by the \WMAP\ team.\footnote{\url{https://lambda.gsfc.nasa.gov/product/wmap/dr5/ninv_info.html}} Properly sampled correlated noise only leaves white nosie in the maps, so the noise properties of each map's sample do not require a dense pixel-pixel covariance, even at low resolution \citep{bp01, bp10}. 
The \WMAPnine\ inverse noise covariance matrices were computed using the full time-space noise matrix $\mathsf N=\mathsf N^\mathrm w+\mathsf N^\mathrm{corr}$, so the full pixel-pixel covariance matrix $\Sigma^{-1}=\mathsf P^T\mathsf N^{-1}\mathsf P$ took into account the correlation between neighboring samples. The \cosmoglobe\ maps, by subtracting a realization of correlated noise before mapmaking, estimates an inerse noise covariance matrix
\begin{equation}
	\Sigma^{-1}_{pp'}
	=\sum_{t_1,t_2}\mathsf P_{t_1,p_1}^T\mathsf N_{t_1,t_2}^{-1}\mathsf P_{t_2,p_2}
	=\sum_t\mathsf P_{t,p_1}^T\mathsf N_{t,t}^{-1}\mathsf P_{t,p_2}
\end{equation}

How much off-diagonal pixel covariance is there here?

dense noise covariance matrix also explicitly projected out the poorly-measured imbalance modes, but because we find no trace of these modes in our sky maps or residual maps, this treatment is not necessary in our approach. However, the correlation between Stokes $Q$ and $U$ was not taken into account in the \BP\ LFI analysis. This was not a significant oversight in the LFI analysis because the 30 and 70\,GHz maps only had a 10\% correlation, and 44\,GHz's 50\% correlation was subdominant to other systematic effects. We have updated \commanderthree\ to take $QU$ correlation into account for LFI.


\section{Comparison with official products}

\begin{figure*}
	\centering
	\includegraphics[width=0.93\linewidth]{figures/megaplot.pdf}
	\caption{Mean of \textit{WMAP}+LFI bands}
\end{figure*}

\subsection{Frequency difference maps}

\begin{figure*}
	\centering
	\includegraphics[width=0.93\linewidth]{figures/megadiff.pdf}
	\caption{Differences between Commander-processed chains and official maps (BP10 and WMAP9)}
\end{figure*}


Need to have the things that go directly into likelihood, the template-corrected WMAP data, theirs versus ours.

Show the low-resolution one. Plotting the W3-W4, W1-W2, V1-V2, Q1-Q2, Ka-0.32K.

We need to have a brief overview of the sky model.

Also need to have the sky model added.

Include the SED from the 03 paper, reference it, remove the power law.

Resources table, as from Q1.

\begin{figure*}
	\centering
	\includegraphics[width=0.24\textwidth]{figures/K30_deltaQ.pdf}
	\includegraphics[width=0.24\textwidth]{figures/K30_W_deltaQ.pdf}
	\includegraphics[width=0.24\textwidth]{figures/K30_deltaU.pdf}
	\includegraphics[width=0.24\textwidth]{figures/K30_W_deltaU.pdf}
	\includegraphics[width=0.24\textwidth]{figures/30Ka_deltaQ.pdf}
	\includegraphics[width=0.24\textwidth]{figures/30Ka_W_deltaQ.pdf}
	\includegraphics[width=0.24\textwidth]{figures/30Ka_deltaU.pdf}
	\includegraphics[width=0.24\textwidth]{figures/30Ka_W_deltaU.pdf}
	\includegraphics[width=0.24\textwidth]{figures/Q_deltaQ.pdf}
	\includegraphics[width=0.24\textwidth]{figures/Q_W_deltaQ.pdf}
	\includegraphics[width=0.24\textwidth]{figures/Q_deltaU.pdf}
	\includegraphics[width=0.24\textwidth]{figures/Q_W_deltaU.pdf}
	\includegraphics[width=0.24\textwidth]{figures/V_deltaQ.pdf}
	\includegraphics[width=0.24\textwidth]{figures/V_W_deltaQ.pdf}
	\includegraphics[width=0.24\textwidth]{figures/V_deltaU.pdf}
	\includegraphics[width=0.24\textwidth]{figures/V_W_deltaU.pdf}
	\includegraphics[width=0.24\textwidth]{figures/W_deltaQ.pdf}
	\includegraphics[width=0.24\textwidth]{figures/W_W_deltaQ.pdf}
	\includegraphics[width=0.24\textwidth]{figures/W_deltaU.pdf}
	\includegraphics[width=0.24\textwidth]{figures/W_W_deltaU.pdf}
	\caption{Half-difference maps, smoothed by $10^\circ$.}
\end{figure*}

djfksajf lasj fkdjas fklsaj lfjdsl kfkj jasdlkfj aslkfj klsadjf lsdajf lkasdj flksdj kflsajdk f
djfksajf lasj fkdjas fklsaj lfjdsl kfkj jasdlkfj aslkfj klsadjf lsdajf lkasdj flksdj kflsajdk f
djfksajf lasj fkdjas fklsaj lfjdsl kfkj jasdlkfj aslkfj klsadjf lsdajf lkasdj flksdj kflsajdk f
jfksajf lasj fkdjas fklsaj lfjdsl kfkj jasdlkfj aslkfj klsadjf lsdajf lkasdj flksdj kflsajdk f
djfksajf lasj fkdjas fklsaj lfjdsl kfkj jasdlkfj aslkfj klsadjf lsdajf lkasdj flksdj kflsajdk f

djfksajf lasj fkdjas fklsaj lfjdsl kfkj jasdlkfj aslkfj klsadjf lsdajf lkasdj flksdj kflsajdk f
jfksajf lasj fkdjas fklsaj lfjdsl kfkj jasdlkfj aslkfj klsadjf lsdajf lkasdj flksdj kflsajdk f
djfksajf lasj fkdjas fklsaj lfjdsl kfkj jasdlkfj aslkfj klsadjf lsdajf lkasdj flksdj kflsajdk f
djfksajf lasj fkdjas fklsaj lfjdsl kfkj jasdlkfj aslkfj klsadjf lsdajf lkasdj flksdj kflsajdk f
jfksajf lasj fkdjas fklsaj lfjdsl kfkj jasdlkfj aslkfj klsadjf lsdajf lkasdj flksdj kflsajdk f

djfksajf lasj fkdjas fklsaj lfjdsl kfkj jasdlkfj aslkfj klsadjf lsdajf lkasdj flksdj kflsajdk f
djfksajf lasj fkdjas fklsaj lfjdsl kfkj jasdlkfj aslkfj klsadjf lsdajf lkasdj flksdj kflsajdk f
jfksajf lasj fkdjas fklsaj lfjdsl kfkj jasdlkfj aslkfj klsadjf lsdajf lkasdj flksdj kflsajdk f
djfksajf lasj fkdjas fklsaj lfjdsl kfkj jasdlkfj aslkfj klsadjf lsdajf lkasdj flksdj kflsajdk f
djfksajf lasj fkdjas fklsaj lfjdsl kfkj jasdlkfj aslkfj klsadjf lsdajf lkasdj flksdj kflsajdk f
jfksajf lasj fkdjas fklsaj lfjdsl kfkj jasdlkfj aslkfj klsadjf lsdajf lkasdj flksdj kflsajdk f

djfksajf lasj fkdjas fklsaj lfjdsl kfkj jasdlkfj aslkfj klsadjf lsdajf lkasdj flksdj kflsajdk f
djfksajf lasj fkdjas fklsaj lfjdsl kfkj jasdlkfj aslkfj klsadjf lsdajf lkasdj flksdj kflsajdk f
jfksajf lasj fkdjas fklsaj lfjdsl kfkj jasdlkfj aslkfj klsadjf lsdajf lkasdj flksdj kflsajdk f
djfksajf lasj fkdjas fklsaj lfjdsl kfkj jasdlkfj aslkfj klsadjf lsdajf lkasdj flksdj kflsajdk f
djfksajf lasj fkdjas fklsaj lfjdsl kfkj jasdlkfj aslkfj klsadjf lsdajf lkasdj flksdj kflsajdk f

%\begin{figure*}
%	\centering
%	\includegraphics[width=0.93\linewidth]{figures/megaplot_official.pdf}
%  \caption{Published \textit{WMAP}+LFI bands (WMAP9 and BP10)}
%\end{figure*}



\subsection{Channel difference maps}

djfksajf lasj fkdjas fklsaj lfjdsl kfkj jasdlkfj aslkfj klsadjf lsdajf lkasdj flksdj kflsajdk f
djfksajf lasj fkdjas fklsaj lfjdsl kfkj jasdlkfj aslkfj klsadjf lsdajf lkasdj flksdj kflsajdk f
djfksajf lasj fkdjas fklsaj lfjdsl kfkj jasdlkfj aslkfj klsadjf lsdajf lkasdj flksdj kflsajdk f
jfksajf lasj fkdjas fklsaj lfjdsl kfkj jasdlkfj aslkfj klsadjf lsdajf lkasdj flksdj kflsajdk f
djfksajf lasj fkdjas fklsaj lfjdsl kfkj jasdlkfj aslkfj klsadjf lsdajf lkasdj flksdj kflsajdk f

djfksajf lasj fkdjas fklsaj lfjdsl kfkj jasdlkfj aslkfj klsadjf lsdajf lkasdj flksdj kflsajdk f
jfksajf lasj fkdjas fklsaj lfjdsl kfkj jasdlkfj aslkfj klsadjf lsdajf lkasdj flksdj kflsajdk f
djfksajf lasj fkdjas fklsaj lfjdsl kfkj jasdlkfj aslkfj klsadjf lsdajf lkasdj flksdj kflsajdk f
djfksajf lasj fkdjas fklsaj lfjdsl kfkj jasdlkfj aslkfj klsadjf lsdajf lkasdj flksdj kflsajdk f
jfksajf lasj fkdjas fklsaj lfjdsl kfkj jasdlkfj aslkfj klsadjf lsdajf lkasdj flksdj kflsajdk f

djfksajf lasj fkdjas fklsaj lfjdsl kfkj jasdlkfj aslkfj klsadjf lsdajf lkasdj flksdj kflsajdk f
djfksajf lasj fkdjas fklsaj lfjdsl kfkj jasdlkfj aslkfj klsadjf lsdajf lkasdj flksdj kflsajdk f
jfksajf lasj fkdjas fklsaj lfjdsl kfkj jasdlkfj aslkfj klsadjf lsdajf lkasdj flksdj kflsajdk f
djfksajf lasj fkdjas fklsaj lfjdsl kfkj jasdlkfj aslkfj klsadjf lsdajf lkasdj flksdj kflsajdk f
djfksajf lasj fkdjas fklsaj lfjdsl kfkj jasdlkfj aslkfj klsadjf lsdajf lkasdj flksdj kflsajdk f
jfksajf lasj fkdjas fklsaj lfjdsl kfkj jasdlkfj aslkfj klsadjf lsdajf lkasdj flksdj kflsajdk f

djfksajf lasj fkdjas fklsaj lfjdsl kfkj jasdlkfj aslkfj klsadjf lsdajf lkasdj flksdj kflsajdk f
djfksajf lasj fkdjas fklsaj lfjdsl kfkj jasdlkfj aslkfj klsadjf lsdajf lkasdj flksdj kflsajdk f
jfksajf lasj fkdjas fklsaj lfjdsl kfkj jasdlkfj aslkfj klsadjf lsdajf lkasdj flksdj kflsajdk f
djfksajf lasj fkdjas fklsaj lfjdsl kfkj jasdlkfj aslkfj klsadjf lsdajf lkasdj flksdj kflsajdk f
djfksajf lasj fkdjas fklsaj lfjdsl kfkj jasdlkfj aslkfj klsadjf lsdajf lkasdj flksdj kflsajdk f

\begin{figure*}
  \centering
	\includegraphics[height=0.15\textheight]{figures/diff_18_DR5_Q.pdf}
	\includegraphics[height=0.15\textheight]{figures/diff_18_DR5_U.pdf}
	\newline
	\includegraphics[height=0.15\textheight]{figures/diff_NPIPE_DR5_Q.pdf}
	\includegraphics[height=0.15\textheight]{figures/diff_NPIPE_DR5_U.pdf}
	\newline
	\includegraphics[height=0.15\textheight]{figures/diff_BP_DR5_Q.pdf}
	\includegraphics[height=0.15\textheight]{figures/diff_BP_DR5_U.pdf}
	\newline
	\includegraphics[height=0.15\textheight]{figures/diff_CG_Q.pdf}
	\includegraphics[height=0.15\textheight]{figures/diff_CG_U.pdf}\\
        \hspace*{-26mm}\includegraphics[width=0.25\textheight]{figures/colourbar_10uK.pdf}        
	\caption{Difference, $\pm10\,\mathrm{\mu K}$.}
\end{figure*}



\subsection{Internal half-difference maps}

\subsection{Comparison of noise rms maps}

%\begin{figure}
%	\includegraphics[width=\columnwidth]{figures/rho_QU.pdf}
%	\caption{$QU$ correlation for \WMAP\ \Ka-band \textit{(top)} and \Planck\ 30\,GHz \textit{(bottom)}. Note the differing dynamic range in the two maps.}
%	\label{fig:rho_qu}
%\end{figure}

\subsection{Angular power spectra}

djfksajf lasj fkdjas fklsaj lfjdsl kfkj jasdlkfj aslkfj klsadjf lsdajf lkasdj flksdj kflsajdk f
djfksajf lasj fkdjas fklsaj lfjdsl kfkj jasdlkfj aslkfj klsadjf lsdajf lkasdj flksdj kflsajdk f
djfksajf lasj fkdjas fklsaj lfjdsl kfkj jasdlkfj aslkfj klsadjf lsdajf lkasdj flksdj kflsajdk f
jfksajf lasj fkdjas fklsaj lfjdsl kfkj jasdlkfj aslkfj klsadjf lsdajf lkasdj flksdj kflsajdk f
djfksajf lasj fkdjas fklsaj lfjdsl kfkj jasdlkfj aslkfj klsadjf lsdajf lkasdj flksdj kflsajdk f

djfksajf lasj fkdjas fklsaj lfjdsl kfkj jasdlkfj aslkfj klsadjf lsdajf lkasdj flksdj kflsajdk f
jfksajf lasj fkdjas fklsaj lfjdsl kfkj jasdlkfj aslkfj klsadjf lsdajf lkasdj flksdj kflsajdk f
djfksajf lasj fkdjas fklsaj lfjdsl kfkj jasdlkfj aslkfj klsadjf lsdajf lkasdj flksdj kflsajdk f
djfksajf lasj fkdjas fklsaj lfjdsl kfkj jasdlkfj aslkfj klsadjf lsdajf lkasdj flksdj kflsajdk f
jfksajf lasj fkdjas fklsaj lfjdsl kfkj jasdlkfj aslkfj klsadjf lsdajf lkasdj flksdj kflsajdk f

djfksajf lasj fkdjas fklsaj lfjdsl kfkj jasdlkfj aslkfj klsadjf lsdajf lkasdj flksdj kflsajdk f
djfksajf lasj fkdjas fklsaj lfjdsl kfkj jasdlkfj aslkfj klsadjf lsdajf lkasdj flksdj kflsajdk f
jfksajf lasj fkdjas fklsaj lfjdsl kfkj jasdlkfj aslkfj klsadjf lsdajf lkasdj flksdj kflsajdk f
djfksajf lasj fkdjas fklsaj lfjdsl kfkj jasdlkfj aslkfj klsadjf lsdajf lkasdj flksdj kflsajdk f
djfksajf lasj fkdjas fklsaj lfjdsl kfkj jasdlkfj aslkfj klsadjf lsdajf lkasdj flksdj kflsajdk f
jfksajf lasj fkdjas fklsaj lfjdsl kfkj jasdlkfj aslkfj klsadjf lsdajf lkasdj flksdj kflsajdk f

djfksajf lasj fkdjas fklsaj lfjdsl kfkj jasdlkfj aslkfj klsadjf lsdajf lkasdj flksdj kflsajdk f
djfksajf lasj fkdjas fklsaj lfjdsl kfkj jasdlkfj aslkfj klsadjf lsdajf lkasdj flksdj kflsajdk f
jfksajf lasj fkdjas fklsaj lfjdsl kfkj jasdlkfj aslkfj klsadjf lsdajf lkasdj flksdj kflsajdk f
djfksajf lasj fkdjas fklsaj lfjdsl kfkj jasdlkfj aslkfj klsadjf lsdajf lkasdj flksdj kflsajdk f
djfksajf lasj fkdjas fklsaj lfjdsl kfkj jasdlkfj aslkfj klsadjf lsdajf lkasdj flksdj kflsajdk f


\section{Systematic error corrections and uncertainties}
\label{sec:systematics}

\subsection{Sky map corrections}



\subsection{Power spectrum residuals}

\begin{figure*}
  \center	
  \includegraphics[width=0.98\linewidth]{figures/components_power_spectrum_std_masked.pdf}
  \caption{Pseudo-spectrum standard deviation for each instrumental
    systematic correction shown in
    Figs.~\ref{fig:corrmaps30}--\ref{fig:corrmaps70} (\emph{thin
      colored lines}). For comparison, thick black lines show spectra
    for the full coadded frequency map; thick red lines show the
    standard deviation of the same (i.e., the full systematic
    uncertainty); gray lines show white noise; and dashed black lines
    show the best-fit \Planck\ 2018 $\Lambda$CDM power spectrum
    convolved with the instrument beam. Columns show results for 30,
    44 and 70\,GHz, respectively, while rows show results for each of
    the six polarization states ($TT$, $EE$, $BB$, $TE$, $TB$, and
    $EB$). All spectra have been derived outside the CMB confidence
    mask presented by \citet{bp13} using the HEALPix \texttt{anafast}
    utility, correcting only for sky fraction and not for mask mode
    coupling. }
  \label{fig:corrmap_powspec_stddev}
\end{figure*}


\section{Conclusions}
\label{sec:conclusions}









%\input{BP_wmap_acknowledgments.tex}

\bibliographystyle{../../common/aa}

\bibliography{../../common/Planck_bib,../../common/BP_bibliography}


\appendix

\section{Survey of instrumental parameters}

\begin{figure*}[p]
  	\centering
	\includegraphics[width=\textwidth]{figures/instpar_CG_baseline_v1.pdf}
	\caption{baseline.}
	\label{fig:baseline}
\end{figure*}

\begin{figure*}[p]
  	\centering
	\includegraphics[width=\textwidth]{figures/instpar_CG_baseslope_v1.pdf}
	\caption{baseline slopes.}
	\label{fig:baseslope}
\end{figure*}

\begin{figure*}[p]
  	\centering
	\includegraphics[width=\textwidth]{figures/instpar_CG_gain_v1.pdf}
	\caption{Gain.}
	\label{fig:gain}
\end{figure*}


\begin{figure*}[p]
  	\centering
	\includegraphics[width=\textwidth]{figures/instpar_CG_sigma0_v1.pdf}
	\caption{sigma0.}
	\label{fig:sigma0}
\end{figure*}

\begin{figure*}[p]
  	\centering
	\includegraphics[width=\textwidth]{figures/instpar_CG_fknee_v1.pdf}
	\caption{Fknee.}
	\label{fig:fknee}
\end{figure*}

\begin{figure*}[p]
  	\centering
	\includegraphics[width=\textwidth]{figures/instpar_CG_alpha_v1.pdf}
	\caption{alpha.}
	\label{fig:alpha}
\end{figure*}


\begin{figure*}[p]
  	\centering
	\includegraphics[width=\textwidth]{figures/instpar_CG_chisq_v1.pdf}
	\caption{chisq.}
	\label{fig:chisq}
\end{figure*}


\end{document}
