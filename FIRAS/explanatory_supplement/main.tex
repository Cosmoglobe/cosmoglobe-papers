\documentclass[notitlepage]{report}

\usepackage{titling} % for keeping title and abstract on the same page
\usepackage{xcolor} % lets you color text

\newcommand{\todo}[1]{\textcolor{red}{\textbf{TODO:} #1}}

\title{Cosmic Background Explorer (COBE)\\Far Infrared Absolute Spectrophotometer (FIRAS)\\ Explanatory Supplement}
\author{Edited by Ana Isabel Silva Martins}
\date{Version 5\\\textit{FIRAS} Project Dataset (Pass 5) Release\\\today}

\begin{document}
\begin{titlingpage}
    \maketitle
    \begin{abstract}
        This is an explanatory supplement.
    \end{abstract}
\end{titlingpage}

\tableofcontents

\newpage

\chapter{Introduction}

\chapter{Instrument Description}

\section{Data Collection and Observing Modes}

FIRAS had four different observing modes, depending on the length of the sweep of the MTM (short or long) and the speed at which it moved (slow or fast). The four modes are summarized in Table~\ref{tab:observing_modes}.

\begin{table}[ht]
\label{tab:observing_modes}
\centering
\caption{FIRAS Observing Modes}
\begin{tabular}{lc}
Mode  & Length (cm)  \\ \hline
Short & 1.76         \\
Long  & 7.07         \\ \hline
Mode  & Speed (cm/s) \\ \hline
Slow  & 0.8          \\
Fast  & 1.2          \\ \hline
\end{tabular}
\end{table}

There was not high enough broadband aboard COBE in order to send down each individual interferogram that was taken at each sweep of the MTM. As such, the interferograms were coadded in groups of four (for the long scans) or 16 (for the short scans) before being sent to the ground. \todo{how long does each mode take to collect data? telemetered. turn-around time too}

\todo{long scans give better spectral resolution (how much?)}



\chapter{Data Model}


\section{Self-heating}

A summary of the self-heating problem.

The thermometer works by running a current through the metal;V=IR(T)where R(T) is some table that has been computed ahead of time.

The fun part is that there is Joule heating, $P=I^{2}R$, that increases the temperature itself. For the sake of sanity, we assume that it's a very quick response. Assuming the current is known, we measure the voltage, which gives us R(T). 

The trick is that $T=T_{\mathrm{env}}+T_{\mathrm{Joule}}$. 

\[
R(T)\simeq R(T_{\mathrm{env}})+T_{\mathrm{Joule}}\frac{\partial R}{\partial T}
\]
then ``all you need to do'' is get
\[
T_{\mathrm{env}}=R^{-1}\left(R(T)-T_{\mathrm{Joule}}\frac{\partial R}{\partial T}\right)
\]

The negative exponential self-heating might be because of the same effect as explained in section 3.2.6 of the \href{https://www.ir.isas.jaxa.jp/AKARI/Observation/support/FIS/IDUM/FIS_IDUM_1.3.pdf}{AKARI Explanatory Supplement}.

\end{document}
