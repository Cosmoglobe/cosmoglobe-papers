\documentclass[notitlepage]{report}

\usepackage{titling} % for keeping title and abstract on the same page

\title{Cosmic Background Explorer (COBE)\\Far Infrared Absolute Spectrophotometer (FIRAS)\\ Explanatory Supplement}
\author{Edited by Ana Isabel Silva Martins}
\date{Version 5\\\textit{FIRAS} Project Dataset (Pass 5) Release\\\today}

\begin{document}
\begin{titlingpage}
    \maketitle
    \begin{abstract}
        This is an explanatory supplement.
    \end{abstract}
\end{titlingpage}

\tableofcontents

\newpage

\chapter{Introduction}

\chapter{Instrument Description}

\section{Data Collection and Observing Modes}

FIRAS had four different observing modes, depending on the length of the sweep of the MTM (short or long) and the speed at which it moved (slow or fast). The four modes are summarized in Table~\ref{tab:observing_modes}.

\begin{table}[ht]
\label{tab:observing_modes}
\centering
\caption{FIRAS Observing Modes}
\begin{tabular}{lc}
Mode  & Length (cm)  \\ \hline
Short & 1.76         \\
Long  & 7.07         \\ \hline
Mode  & Speed (cm/s) \\ \hline
Slow  & 0.8          \\
Fast  & 1.2          \\ \hline
\end{tabular}
\end{table}

There was not high enough broadband aboard COBE in order to send down each individual interferogram that was taken at each sweep of the MTM. As such, the interferograms were coadded in groups of four (for the long scans) or 16 (for the short scans) before being sent to the ground.

\chapter{Data Model}

\end{document}