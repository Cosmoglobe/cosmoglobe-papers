%                                                                 aa.dem
% AA vers. 9.1, LaTeX class for Astronomy & Astrophysics
% demonstration file
%                                                       (c) EDP Sciences
%-----------------------------------------------------------------------
%
% \documentclass[referee]{aa} % for a referee version
%\documentclass[onecolumn]{aa} % for a paper on 1 column  
%\documentclass[longauth]{aa} % for the long lists of affiliations 
%\documentclass[letter]{aa} % for the letters 
%\documentclass[bibyear]{aa} % if the references are not structured 
%                              according to the author-year natbib style

%

\documentclass{aa}  

%
\usepackage{graphicx}
\usepackage{amsmath,amsfonts,amssymb}
\usepackage{natbib}


%%%%%%%%%%%%%%%%%%%%%%%%%%%%%%%%%%%%%%%%
\usepackage{txfonts}
\usepackage{xcolor}

\usepackage{blindtext}
%%%%%%%%%%%%%%%%%%%%%%%%%%%%%%%%%%%%%%%%
% \usepackage[options]{hyperref}
% To add links in your PDF file, use the package "hyperref"
% with options according to your LaTeX or PDFLaTeX drivers.
\usepackage{float}
%\usepackage{stfloats}
\usepackage{dblfloatfix}
\usepackage{afterpage}
\usepackage{ifthen}
\usepackage[morefloats=12]{morefloats}

\usepackage{placeins}
\usepackage{multicol}
%\usepackage[breaklinks,colorlinks,citecolor=blue]{hyperref}
\bibpunct{(}{)}{;}{a}{}{,}
\usepackage[switch]{lineno}
\definecolor{linkcolor}{rgb}{0.6,0,0}
\definecolor{citecolor}{rgb}{0,0,0.75}
\definecolor{urlcolor}{rgb}{0.12,0.46,0.7}
\usepackage[breaklinks, colorlinks, urlcolor=urlcolor,
    linkcolor=linkcolor,citecolor=citecolor,pdfencoding=auto]{hyperref}
\hypersetup{linktocpage}
\usepackage{bold-extra}

\usepackage[nameinlink,capitalise]{cleveref}
\Crefname{section}{Sect.}{Sects.}
\Crefname{table}{Table}{Tables}
\Crefname{equation}{Eq.}{Eqs.}
\Crefname{appsec}{appendix}{appendices}




\def\setsymbol#1#2{\expandafter\def\csname #1\endcsname{#2}}
\def\getsymbol#1{\csname #1\endcsname}

\def\Planck{\textit{Planck}}

\def\HeJT{$^4$He-JT}

\def\allearlypapers{\nocite{planck2011-1.1, planck2011-1.3, planck2011-1.4, planck2011-1.5, planck2011-1.6, planck2011-1.7, planck2011-1.10, planck2011-1.10sup, planck2011-5.1a, planck2011-5.1b, planck2011-5.2a, planck2011-5.2b, planck2011-5.2c, planck2011-6.1, planck2011-6.2, planck2011-6.3a, planck2011-6.4a, planck2011-6.4b, planck2011-6.6, planck2011-7.0, planck2011-7.2, planck2011-7.3, planck2011-7.7a, planck2011-7.7b, planck2011-7.12, planck2011-7.13}}

\def\alltwentythirteenresultspapers{\nocite{planck2013-p01, planck2013-p02, planck2013-p02a, planck2013-p02d, planck2013-p02b, planck2013-p03, planck2013-p03c, planck2013-p03f, planck2013-p03d, planck2013-p03e, planck2013-p01a, planck2013-p06, planck2013-p03a, planck2013-pip88, planck2013-p08, planck2013-p11, planck2013-p12, planck2013-p13, planck2013-p14, planck2013-p15, planck2013-p05b, planck2013-p17, planck2013-p09, planck2013-p09a, planck2013-p20, planck2013-p19, planck2013-pipaberration, planck2013-p05, planck2013-p05a, planck2013-pip56, planck2013-p06b, planck2013-p01a}}

\def\alltwentyfifteenresultspapers{\nocite{planck2014-a01, planck2014-a03, planck2014-a04, planck2014-a05, planck2014-a06, planck2014-a07, planck2014-a08, planck2014-a09, planck2014-a11, planck2014-a12, planck2014-a13, planck2014-a14, planck2014-a15, planck2014-a16, planck2014-a17, planck2014-a18, planck2014-a19, planck2014-a20, planck2014-a22, planck2014-a24, planck2014-a26, planck2014-a28, planck2014-a29, planck2014-a30, planck2014-a31, planck2014-a35, planck2014-a36, planck2014-a37, planck2014-ES}}

\newbox\tablebox    \newdimen\tablewidth
\def\leaderfil{\leaders\hbox to 5pt{\hss.\hss}\hfil}
\def\endPlancktable{\tablewidth=\columnwidth 
    $$\hss\copy\tablebox\hss$$
    \vskip-\lastskip\vskip -2pt}
\def\endPlancktablewide{\tablewidth=\textwidth 
    $$\hss\copy\tablebox\hss$$
    \vskip-\lastskip\vskip -2pt}
\def\tablenote#1 #2\par{\begingroup \parindent=0.8em
    \abovedisplayshortskip=0pt\belowdisplayshortskip=0pt
    \noindent
    $$\hss\vbox{\hsize\tablewidth \hangindent=\parindent \hangafter=1 \noindent
    \hbox to \parindent{$^#1$\hss}\strut#2\strut\par}\hss$$
    \endgroup}
\def\doubleline{\vskip 3pt\hrule \vskip 1.5pt \hrule \vskip 5pt}

\def\L2{\ifmmode L_2\else $L_2$\fi}
\def\dtt{\Delta T/T}
\def\DeltaT{\ifmmode \Delta T\else $\Delta T$\fi}
\def\deltat{\ifmmode \Delta t\else $\Delta t$\fi}
\def\fknee{\ifmmode f_{\rm knee}\else $f_{\rm knee}$\fi}
\def\Fmax{\ifmmode F_{\rm max}\else $F_{\rm max}$\fi}
\def\solar{\ifmmode{\rm M}_{\mathord\odot}\else${\rm M}_{\mathord\odot}$\fi}
\def\Msolar{\ifmmode{\rm M}_{\mathord\odot}\else${\rm M}_{\mathord\odot}$\fi}
\def\Lsolar{\ifmmode{\rm L}_{\mathord\odot}\else${\rm L}_{\mathord\odot}$\fi}
\def\inv{\ifmmode^{-1}\else$^{-1}$\fi}
\def\mo{\ifmmode^{-1}\else$^{-1}$\fi}
\def\sup#1{\ifmmode ^{\rm #1}\else $^{\rm #1}$\fi}
\def\expo#1{\ifmmode \times 10^{#1}\else $\times 10^{#1}$\fi}
\def\,{\thinspace}
\def\lsim{\mathrel{\raise .4ex\hbox{\rlap{$<$}\lower 1.2ex\hbox{$\sim$}}}}
\def\gsim{\mathrel{\raise .4ex\hbox{\rlap{$>$}\lower 1.2ex\hbox{$\sim$}}}}
\let\lea=\lsim
\let\gea=\gsim
\def\simprop{\mathrel{\raise .4ex\hbox{\rlap{$\propto$}\lower 1.2ex\hbox{$\sim$}}}}
\def\deg{\ifmmode^\circ\else$^\circ$\fi}
\def\pdeg{\ifmmode $\setbox0=\hbox{$^{\circ}$}\rlap{\hskip.11\wd0 .}$^{\circ}
          \else \setbox0=\hbox{$^{\circ}$}\rlap{\hskip.11\wd0 .}$^{\circ}$\fi}
\def\arcs{\ifmmode {^{\scriptstyle\prime\prime}}
          \else $^{\scriptstyle\prime\prime}$\fi}
\def\arcm{\ifmmode {^{\scriptstyle\prime}}
          \else $^{\scriptstyle\prime}$\fi}
\newdimen\sa  \newdimen\sb
\def\parcs{\sa=.07em \sb=.03em
     \ifmmode \hbox{\rlap{.}}^{\scriptstyle\prime\kern -\sb\prime}\hbox{\kern -\sa}
     \else \rlap{.}$^{\scriptstyle\prime\kern -\sb\prime}$\kern -\sa\fi}
\def\parcm{\sa=.08em \sb=.03em
     \ifmmode \hbox{\rlap{.}\kern\sa}^{\scriptstyle\prime}\hbox{\kern-\sb}
     \else \rlap{.}\kern\sa$^{\scriptstyle\prime}$\kern-\sb\fi}
\def\ra[#1 #2 #3.#4]{#1\sup{h}#2\sup{m}#3\sup{s}\llap.#4}
\def\dec[#1 #2 #3.#4]{#1\deg#2\arcm#3\arcs\llap.#4}
\def\deco[#1 #2 #3]{#1\deg#2\arcm#3\arcs}
\def\rra[#1 #2]{#1\sup{h}#2\sup{m}}
\def\page{\vfill\eject}
\def\dots{\relax\ifmmode \ldots\else $\ldots$\fi}
\def\WHzsr{\ifmmode $W\,Hz\mo\,sr\mo$\else W\,Hz\mo\,sr\mo\fi}
\def\mHz{\ifmmode $\,mHz$\else \,mHz\fi}
\def\GHz{\ifmmode $\,GHz$\else \,GHz\fi}
\def\mKs{\ifmmode $\,mK\,s$^{1/2}\else \,mK\,s$^{1/2}$\fi}
\def\muKs{\ifmmode \,\mu$K\,s$^{1/2}\else \,$\mu$K\,s$^{1/2}$\fi}
\def\muKRJs{\ifmmode \,\mu$K$_{\rm RJ}$\,s$^{1/2}\else \,$\mu$K$_{\rm RJ}$\,s$^{1/2}$\fi}
\def\muKHz{\ifmmode \,\mu$K\,Hz$^{-1/2}\else \,$\mu$K\,Hz$^{-1/2}$\fi}
\def\MJysr{\ifmmode \,$MJy\,sr\mo$\else \,MJy\,sr\mo\fi}
\def\MJysrmK{\ifmmode \,$MJy\,sr\mo$\,mK$_{\rm CMB}\mo\else \,MJy\,sr\mo\,mK$_{\rm CMB}\mo$\fi}
\def\microns{\ifmmode \,\mu$m$\else \,$\mu$m\fi}
\def\micron{\microns}
\def\muK{\ifmmode \,\mu$K$\else \,$\mu$\hbox{K}\fi}
\def\microK{\ifmmode \,\mu$K$\else \,$\mu$\hbox{K}\fi}
\def\muW{\ifmmode \,\mu$W$\else \,$\mu$\hbox{W}\fi}
\def\kms{\ifmmode $\,km\,s$^{-1}\else \,km\,s$^{-1}$\fi}
\def\kmsMpc{\ifmmode $\,\kms\,Mpc\mo$\else \,\kms\,Mpc\mo\fi}

\providecommand{\sorthelp}[1]{}


% Custom definitions
\newcommand{\mathsc}[1]{{\normalfont\textsc{#1}}}
\newcommand{\dv}[0]{\vec{d}}
\newcommand{\s}[0]{\vec{s}}
\newcommand{\M}[0]{\tens{M}}
\renewcommand{\P}[0]{\tens{P}}
\newcommand{\G}[0]{\tens{G}}
\newcommand{\B}[0]{\tens{B}}
\renewcommand{\a}[0]{\vec{a}}
\newcommand{\n}[0]{\vec{n}}
\renewcommand{\t}[0]{\vec{t}}
\def\Cosmoglobe{\textsc{Cosmoglobe}}
\def\Planck{\textit{Planck}}
\def\WMAP{\textit{WMAP}}
\def\COBE{\textit{COBE}}
\def\GAIA{\textit{Gaia}}
\def\gaia{\textit{Gaia}}
\def\Gaia{\textit{Gaia}}
\def\WISE{WISE}
\def\AKARI{\textit{{AKARI}}}
\def\IRAS{\textit{{IRAS}}}
\def\nside{\ensuremath{N_{\mathrm{side}}}}
\newcommand{\cii}{\ensuremath{\mathsc{C\ ii}}}
\newcommand{\CII}{\ensuremath{\mathsc{C\ ii}}}
\newcommand{\hi}{\ensuremath{\mathsc{H\ i}}}
\newcommand{\HI}{\ensuremath{\mathsc{H\ i}}}
\def\Commander{\texttt{Commander} }
\def\commanderthree{\texttt{Commander3} }

\def\Tcmb{\ifmmode T_\mathrm{CMB}\else $T_{\mathrm{CMB}}$\fi}
\def\Tcold{\ifmmode T_\mathrm{c}\else $T_{\mathrm{c}}$\fi}
\def\Thot{\ifmmode T_\mathrm{h}\else $T_{\mathrm{h}}$\fi}
\def\Tnear{\ifmmode T_\mathrm{n}\else $T_{\mathrm{n}}$\fi}
\def\scmb{\ifmmode s_\mathrm{CMB}\else $s_{\mathrm{CMB}}$\fi}
\def\squad{\ifmmode s_\mathrm{quad}\else $s_{\mathrm{quad}}$\fi}
\def\ssynch{\ifmmode s_\mathrm{s}\else $s_\mathrm{s}$\fi}
\def\sdust{\ifmmode s_\mathrm{d}\else $s_{\mathrm{d}}$\fi}
\def\ssdust{\ifmmode s_\mathrm{sd}\else $s_{\mathrm{sd}}$\fi}
\def\same{\ifmmode s_\mathrm{AME}\else $s_{\mathrm{AME}}$\fi}
\def\ssrc{\ifmmode s_\mathrm{src}\else $s_{\mathrm{src}}$\fi}
\def\sco{\ifmmode s_\mathrm{CO}\else $s_{\mathrm{CO}}$\fi}
\def\sff{\ifmmode s_\mathrm{ff}\else $s_{\mathrm{ff}}$\fi}
\def\gff{\ifmmode g_\mathrm{ff}\else $g_{\mathrm{ff}}$\fi}
\def\fsynch{\ifmmode f_\mathrm{s}\else $f_{\mathrm{s}}$\fi}
\def\fsd{\ifmmode f_\mathrm{sd}\else $f_{\mathrm{sd}}$\fi}
\def\fame{\ifmmode f_\mathrm{AME}\else $f_{\mathrm{AME}}$\fi}
\def\alphasrc{\ifmmode \alpha_\mathrm{src}\else $\alpha_{\mathrm{src}}$\fi}
\def\bcold{\ifmmode \beta_\mathrm{c}\else $\beta_{\mathrm{c}}$\fi}
\def\bhot{\ifmmode \beta_\mathrm{h}\else $\beta_{\mathrm{h}}$\fi}
\def\bnear{\ifmmode \beta_\mathrm{n}\else $\beta_{\mathrm{n}}$\fi}
\def\bsynch{\ifmmode \beta_\mathrm{s}\else $\beta_{\mathrm{s}}$\fi} 
\def\bsun{\ifmmode \beta_\mathrm{sun}\else $\beta_{\mathrm{sun}}$\fi} 
\def\nuzeros{\ifmmode \nu_{0,\mathrm{s}}\else $\nu_{0,\mathrm{s}}$\fi} 
\def\nuzeroff{\ifmmode \nu_{0,\mathrm{ff}}\else $\nu_{0,\mathrm{ff}}$\fi} 
\def\nuzerocold{\ifmmode \nu_{0,\mathrm{c}}\else $\nu_{0,\mathrm{c}}$\fi}
\def\nuzerohot{\ifmmode \nu_{0,\mathrm{h}}\else $\nu_{0,\mathrm{h}}$\fi}
\def\nuzeronear{\ifmmode \nu_{0,\mathrm{n}}\else $\nu_{0,\mathrm{n}}$\fi} 
\def\nuzeroame{\ifmmode \nu_{0,\mathrm{AME}}\else $\nu_{0,\mathrm{AME}}$\fi} 
\def\nuzerosd{\ifmmode \nu_{0,\mathrm{}}\else $\nu_{0,\mathrm{sd}}$\fi} 
\def\nuzerosrc{\ifmmode \nu_{0,\mathrm{src}}\else $\nu_{0,\mathrm{src}}$\fi} 
\def\nup{\ifmmode \nu_{\mathrm{p}}\else $\nu_{\mathrm{p}}$\fi} 
\def\alphasd{\ifmmode \alpha_{\mathrm{sd}}\else $\alpha_{\mathrm{sd}}$\fi} 
\def\Te{\ifmmode T_{\mathrm{e}}\else $T_{\mathrm{e}}$\fi} 
\def\kB{\ifmmode k_\mathrm{B}\else $k_{\mathrm{B}}$\fi} 

\newcommand{\aism}[1]{\textcolor{red}{\textbf{AISM:} #1}}

\begin{document} 


%\title{\bfseries{\Cosmoglobe\ DR2. On the correlation between ionized carbon and thermal dust emission}}
%\title{\bfseries{\Cosmoglobe\ DR2. Thermal dust and ionized carbon emission are spatially strongly correlated}}
\title{\bfseries{\Cosmoglobe\ DR2. V. Spatial correlations between thermal dust and ionized carbon emission in \Planck\ HFI and COBE-DIRBE}}

   %This author list corresponds to \title{Author list for L04\_CMB\_Foregrounds\_Extraction}
%Prepared by M. Lopez-Caniego (Marcos.Lopez.Caniego@sciops.esa.int), ESAC/ESA
%This version is from Thu Jul 12 18:11:48 2018 CET
%\subtitle{There are 152 co-authors in this list}
\newcommand{\oslo}[0]{1}
%\newcommand{\MIT}[0]{2}
\newcommand{\milanoA}[0]{2}
\newcommand{\milanoB}[0]{3}
\newcommand{\milanoC}[0]{4}
\newcommand{\triesteB}[0]{5}
\newcommand{\planetek}[0]{6}
\newcommand{\princeton}[0]{7}
\newcommand{\jpl}[0]{8}
\newcommand{\helsinkiA}[0]{9}
\newcommand{\helsinkiB}[0]{10}
\newcommand{\nersc}[0]{11}
\newcommand{\haverford}[0]{12}
\newcommand{\mpa}[0]{13}
\newcommand{\triesteA}[0]{14}
\newcommand{\iia}[0]{2}

\author{\small
J.~R.~Eskilt\inst{\oslo}\thanks{Corresponding author: J.~R.~Eskilt; \url{j.r.eskilt@astro.uio.no}}
\and
K.~Lee\inst{\oslo}
\and
D.~J.~Watts\inst{\oslo}
\and
S.~Nerval\inst{\oslo}
\and
et al.
}
\institute{\small
        Institute of Theoretical Astrophysics, University of Oslo, Blindern, Oslo, Norway \goodbreak
}


  
   % Shortened title, author list for top of page 
   \titlerunning{Correlations between thermal dust and \CII}
   \authorrunning{Gjerløw et al.}

   \date{\today} 
   
   \abstract{We fit five tracers of thermal dust emission to ten \textit{Planck} HFI and \textit{COBE}-DIRBE frequency maps between 353\,GHz and 25\,THz, aiming to map the relative importance of each physical host environment as a function of frequency and position on the sky. Four of these correspond to classic thermal dust tracers, namely \textsc{H i} (HI4PI), CO \citep{dame2001}, H$\alpha$ (WHAM, \cite{wham:2003,2016WHAM}), and dust extinction (\textit{Gaia}, \cite{edenhofer:2024}), while the fifth is ionized carbon (\textsc{C ii}) emission as observed by \textit{COBE}-FIRAS. To our knowledge, this has until now been considered to be primarily a gas tracer, rather than a dust tracer. After smoothing all data to the common resolution of the FIRAS experiment, and subtracting subdominant astrophysical components as appropriate for each channel (cosmic microwave and infrared backgrounds, and zodiacal light), we jointly fit these five templates to each frequency channel through standard multi-variate linear regression. At frequencies higher than 1\,THz, we find that the dominant tracer is \textsc{C ii}, and above 10\,THz this component accounts for almost the entire fitted signal; at frequencies below 1\,THz, its importance is second only to \textsc{H i}. We further find that all five components are well described by a modified blackbody spectral energy density (SED) up to some component-dependent maximum frequency ranging between 1 and 5\,THz. In this interpretation, the \textsc{C ii}-correlated component is the hottest among all five with an effective temperature of about 25\,K. For comparison, the \textsc{H i} and CO components have effective temperatures of 16\,K and 12\,K, respectively. The H$\alpha$ component has a temperature of 18\,K, and, unlike the other four, is observed in absorption rather than emission. The spectral indices of the five components range between $\beta = 1.4$ (for \textsc{H i}) and 2.6 (for H$\alpha$); for the \textsc{C ii} component, $\beta=1.56$. Despite the simplicity of this model, which relies only external templates coupled to spatially isotropic SEDs, we find that it captures 98\,\% of the full signal root mean squared (RMS) below 1\,THz. At higher frequencies, which are more susceptible to non-thermal emission processes, the model still captures more than 80\,\% of the full signal RMS. This high efficiency suggests that spatial variations in the thermal dust SED, as for instance reported by \Planck\ and other experiments, may be more economically modelled on large angular scales in terms of a spatial mixing of individually isotropic physical components, than by a single uniform well-mixed interstellar medium coupled to a spatially varying temperature field, as has been the norm until now. Indeed, the results found in this paper motivate the thermal dust model adopted for the \textsc{Cosmoglobe} DR2 re-analysis of the \textit{COBE}-DIRBE data, and we believe that they may also provide inspiration for refining both current theoretical interstellar medium models and component separation algorithms in general.}
   
   \keywords{ISM: general - Zodiacal dust, Interplanetary medium - Cosmology: observations, diffuse radiation - Galaxy: general}

   \maketitle

\setcounter{tocdepth}{2}
\tableofcontents

% INTRODUCTION
%-------------------------------------------------------------------
\section{Introduction}
%\the\textwidth \the\columnwidth

The interstellar medium (ISM) plays a ubiquitous role in modern
astrophysics and cosmology across the electromagnetic spectrum (e.g., \citealt{draine2011}, \citealt{Hensley2023}). On 
one hand, understanding the composition and physics of the ISM informs
us about the structure and dynamics of both the Milky Way and distant
galaxies, and ISM studies are therefore an important and interesting
field of astronomy in their own right. On the other hand, ISM radiation
is a key contaminant for a broad range of other high-impact
science targets, for instance the search for gravitational waves in
the cosmic microwave background (CMB), dark matter annihilation in
gamma-ray observations, or dark energy constraints through
high-precision measurements of distant supernovae. Accurate ISM
modelling is therefore a key aspect in 21st century
astrophysics.

Broadly speaking, the ISM is comprised of cosmic rays
(relativistic particles), gas (atoms or small molecules), and dust
(large molecules, typically ranging in size from a few angstroms to
100\,$\mu$m). All of these emit electromagnetic radiation at various
frequencies, for instance through synchrotron, bremsstrahlung, quantum
mechanical line emission, or thermal emission. In addition, dust
grains absorb electromagnetic radiation with wavelengths that are
comparable to the grain size, which due to the grain distribution
happens most notably in wavelengths ranging from infrared to
X-rays. 

The current paper is part of a suite of seven companion papers that
describes the \Cosmoglobe\ Data Release 2 (DR2; \citealp{CG02_01}). The
primary main goal of this work is to reanalyze the \COBE-DIRBE data
\citep{hauser1998} within an global Bayesian end-to-end analysis
framework, and use the resulting data to constrain the spectrum of the
cosmic infrared background (CIB; \citealp{CG02_03}). The DIRBE
instrument observed the full sky in 10 wavelength bands covering 1.25
and 240\,$\mu$m. By virtue of covering virtually the entire infrared
regime, DIRBE is an excellent dust tracer, both originating from the
Milky Way and from within the Solar system. Indeed, while DIRBE's
original science goal was to detect and characterize the CIB spectrum
and fluctuations, the most long-lasting legacy of the survey has
arguably been to serve as unique well-calibrated full-sky tracer of
dust emission in the Milky Way 
\citep{schlegel1998,IRIS,planck2014-a12,sano2016,pysm_methods}
%(ADD LOTS OF REFERENCES)
and zodiacal
light emission \citep[e.g.,][]{K98,planck2013-pip88,2016AJ....151...71K,CG02_02,obrien2025}.% (include other recent references, CIBER for example).

One of the key questions in which the DIRBE data has played a role
regards the nature and composition of thermal dust emission in the
far-infrared regime, and how it may be modelled most
efficiently. Early studies quickly suggested that a single so-called
modified blackbody (MBB) spectrum provides a good fit to the thermal
dust spectral energy density (SED) for a broad range of frequencies
\citep{reach1995b}. An MBB spectrum has three degrees of freedom, namely
an overall amplitude (tracing the density of the medium), an effective
temperature, $T$, and a spectral index, $\beta$. To this date, the
single-component MBB model plays a key role in both microwave
modelling and infrared-based dust studies.

However, it also quickly became clear that the validity of a single
MBB spectrum is limited in frequencies. Notably, in a seminal paper by \citet{finkbeiner:1999} presented a two-component
MBB thermal dust model derived from the \IRAS\ 100\,$\mu$m and DIRBE 100
and 240\,$\mu$m data that served as a benchmark for the CMB community for more than a decade, and was
only superseded by the far more sensitive \Planck\ HFI data
\citep{planck2013-p06b}. Since that time, the combination of
\Planck, DIRBE, and \IRAS\ measurements have dominated the study of
large-scale dust emission at microwave and far-infrared frequencies.

Despite massive efforts developing increasingly more accurate
and well-defined models of thermal dust emission, combining
theoretical insights with observational constraints (ADD REFERENCES),
many critical questions plague this field to date. Furthermore,
the importance of understanding microwave dust emission has only increased in recent
years, as the attention of the CMB field has shifted to the search for
inflationary gravitational waves through $B$-mode polarization
constraints. For such experiments, polarized thermal dust emission
represents one of the most important contaminants, and major resources
are currently being invested in devising both instrumentational and
data analysis techniques to handle this challenge optimally. 

% \begin{table}
%     \centering
%     \caption{Templates used for each frequency band.}
%     \begin{tabular}{c|c|c|c|c|c}
%         \label{tab:bands}
%         Band & \CII & HI4PI & Gaia & WHAM & CO \\
%         \hline
%         \Planck\ 353\,GHz & x & x & x & x & x \\
%         \Planck\ 545\,GHz & x & x & x & x & x \\
%         \Planck\,857 GHz & x & x & x & x & x \\
%         DIRBE 240 $\mu m$ & x & x & x & x & x \\
%         DIRBE 140 $\mu m$ & x & x & x & x & x \\
%         DIRBE 100 $\mu m$ & x & x & x & x & x \\
%         DIRBE 60 $\mu m$ & x & x & x & x & \\
%         DIRBE 25 $\mu m$ & x & x & & & \\
%         DIRBE 12 $\mu m$ & x & & & & 
%     \end{tabular} 
% \end{table}

\begin{table}
 \caption{Templates used for each frequency band.}
 \label{tab:bands}
\begingroup
\newdimen\tblskip \tblskip=4pt
\nointerlineskip
\vskip -3mm
\footnotesize
\setbox\tablebox=\vbox{
\halign{
 \tabskip 0pt \hbox to 0.25\linewidth{#\hfil}
&\hbox to 0.15\linewidth{\hfil#\hfil}\tabskip 0pt
&\hbox to 0.11\linewidth{\hfil#\hfil}\tabskip 0pt
&\hbox to 0.11\linewidth{\hfil#\hfil}\tabskip 0pt
&\hbox to 0.11\linewidth{\hfil#\hfil}\tabskip 0pt
&\hbox to 0.11\linewidth{\hfil#\hfil}\tabskip 0pt
&\hbox to 0.11\linewidth{\hfil#}\tabskip 0pt\cr
\noalign{\doubleline\vskip 1pt}
\omit\hbox to 1in{Band\hfil} & Freq. (GHz) & \CII & HI4PI & Gaia & WHAM & CO \cr
\noalign{\vskip 4pt\hrule\vskip 6pt}
\noalign{\vskip 3pt}
       \Planck\ 353\,GHz & 353 & x & x & x & x & x \cr
       \noalign{\vskip 3pt}
       \Planck\ 545\,GHz & 545 & x & x & x & x & x \cr
       \noalign{\vskip 3pt}
       \Planck\,857 GHz & 857 & x & x & x & x & x \cr
       \noalign{\vskip 3pt}
       DIRBE 240 $\mu m$ & 1250 & x & x & x & x & x \cr
       \noalign{\vskip 3pt}
       DIRBE 140 $\mu m$ & 2100 & x & x & x & x & x \cr
       \noalign{\vskip 3pt}
       DIRBE 100 $\mu m$ & 3000 & x & x & x & x & x \cr
       \noalign{\vskip 3pt}
       DIRBE 60 $\mu m$ & 5000 & x & x & x & x & \cr
       \noalign{\vskip 3pt}
       DIRBE 25 $\mu m$ & 12000 & x & x & & & \cr
       \noalign{\vskip 3pt}
       DIRBE 12 $\mu m$ & 25000 & x & & & & \cr
       \noalign{\vskip 3pt}
\noalign{\vskip 3pt\hrule\vskip 4pt}}}
\endPlancktable
\endgroup
\end{table}

\begin{figure*}
  \centering
  \includegraphics[width=0.49\textwidth]{figures/CII_FIRAS_v1_n16.pdf}
  \includegraphics[width=0.49\textwidth]{figures/HI4PI_NHI_n0064_60arcmin_rescaled_T.pdf}\\
  \includegraphics[width=0.49\textwidth]{figures/extinctions_515.pdf}
  \includegraphics[width=0.49\textwidth]{figures/lambda_WHAM_1_256.pdf}\\
  \includegraphics[width=0.49\textwidth]{figures/lambda_wco_dht2001.pdf}
  \caption{Input maps of each of the following thermal dust tracers (from left to right and top to bottom): \CII\ 158\,$\mu$m line emission from \COBE-FIRAS, \HI\ line emission from HI4PI, dust extinction from \GAIA, H$\alpha$ line emission from WHAM, and CO line emission from \citet{dame:2001}. All maps are smoothed to the FIRAS beam and then used for the low-resolution template fitting at \nside=16.}
  \label{fig:tempfit_inputs}
\end{figure*}


In the current paper, and in two companion papers by \citet{CG02_06}
and \citet{CG02_07}, we revisit the question of how to model thermal
dust emission efficiently in light of the currently reprocessed DIRBE
data, jointly with archival measurements from
\Planck\ HFI. Specifically, during the course of the \Cosmoglobe\ DR2
work, a new four-component MBB model, based on five astrophysical tracers, has emerged as a particularly
compact and efficient description. Although a four-component model may
at first sight appear as significantly more complicated than either
the standard single-component MBB model adopted by the \Planck\ team,
or by the two-component MBB model pioneered by
\citet{finkbeiner:1999}, the key point in our new analysis is that all
four components are well traced by well-known dust templates coupled
with spatially isotropic SEDs. The number of degrees of freedom per
pixel is therefore very small, allowing for a very rigid and compact
statistical description.

Four of the five templates in this model are already known to be
efficient dust tracers, namely 1) HI4PI \HI\ line emission, tracing
molecular gas; 2) \GAIA\ extinction, tracing near-by dust absorption; 3)
\citet{dame:2001} CO line emission, tracing cold dust in star-forming
regions; and 4) WHAM H$\alpha$ line emission, tracing hot dust in
ionized regions. However, in this paper and its companion papers, we additionally find that ionized carbon emission, as traced by the FIRAS
\CII\ 158\,$\mu$m line emission map, is an excellent tracer of thermal
dust emission with a temperature of $\sim$30\,K. This observation may
have far-reaching consequences for future studies and models of dust
emission in the microwave and infrared regimes.

In the current paper, we perform a linear regression analysis of
\Planck\ HFI 353--857\,GHz and DIRBE 3.5--100\,$\mu$m data with this
model, which demonstrates the validity of this key finding, even with a
minimal set of assumptions. In a follow-up analysis by \citet{CG02_06},
we perform a Bayesian analysis of \Planck\ HFI data with the goal of
deriving high-resolution templates of the \HI\ and \CII\ correlated
components directly from microwave data. Finally, \citet{CG02_07}
applies this novel model to the \COBE-DIRBE data.

The rest of the paper is organized as follows:
%In \cref{sec:dr2} we give a brief overview of the data model adopted for the\Cosmoglobe\ DR2 reanalysis. 
in \cref{sec:tempfit} we discuss the template fitting algorithm and in \cref{sec:data} the data used in the
current paper; finally, in \cref{sec:results} we present the results from
these calculations, before we conclude in \cref{sec:conclusions}.

\begin{figure}
  \centering
  \includegraphics[width=\columnwidth]{figures/firas_beam.pdf}
  \caption{The \COBE-FIRAS beam, with which all maps in this analysis were smoothed. The beam diameter is around $7^\circ$ on the sky, which limits this analysis to \nside=16.}
  \label{fig:firasbeam}
\end{figure}

\section{Template fitting methodology}
\label{sec:tempfit}

As the aim of this paper is mainly to investigate the feasibility of describing
thermal dust as a linear combination of components -- components that
can be reasonably imagined to be associated with the thermal dust in the galaxy
-- we perform a fitting of such components to existing data \emph{outside} of
the \Cosmoglobe\ framework proper. That is, we here attempt to fit the
aforementioned components directly to dust-dominated \Planck\ HFI and DIRBE data,
and we do this without any joint modelling like the one used in \Cosmoglobe.

In order to perform such a fit, we first assume all components and all data on
which we wish to perform the fit are given as HEALPix maps in the same
resolution (see \cref{sec:data} for more on this). For each data set
$\dv_i$ at a given frequency band $i$ (bold notation here implies a
vector of $n_\mathrm{pix}$ HEALPix pixels), the basic data model we employ is

\begin{figure}
  \centering
  \includegraphics[width=\columnwidth]{figures/tempamp_v1.pdf}
  \caption{Template amplitudes as a function of frequency for the nine channels used in this analysis, plotted only at the channels where they are fit. The WHAM H$\alpha$ amplitudes are negative because it is an absorptive template.}
  \label{fig:tempamp}
\end{figure}

\begin{figure*}
  \centering
  \includegraphics[width=0.49\textwidth]{figures/map_tempfit_353_v1.pdf}
  \includegraphics[width=0.49\textwidth]{figures/res_tempfit_353_v2.pdf}\\
  \includegraphics[width=0.49\textwidth]{figures/map_tempfit_545_v1.pdf}
  \includegraphics[width=0.49\textwidth]{figures/res_tempfit_545_v2.pdf}\\
  \includegraphics[width=0.49\textwidth]{figures/map_tempfit_857_v1.pdf}
  \includegraphics[width=0.49\textwidth]{figures/res_tempfit_857_v2.pdf}
  \caption{Comparison between full (left column) and template cleaned (right) maps for the three dust-dominated \Planck\ HFI channels. The grey pixels are masked by the common analysis mask. All maps are convolved with the FIRAS beam. }
  \label{fig:tempfit_map_vs_res1}
\end{figure*}


\begin{figure*}
  \centering
  \includegraphics[width=0.49\textwidth]{figures/map_tempfit_1250_v1.pdf}
  \includegraphics[width=0.49\textwidth]{figures/res_tempfit_1250_v2.pdf}\\  
  \includegraphics[width=0.49\textwidth]{figures/map_tempfit_2100_v1.pdf}
  \includegraphics[width=0.49\textwidth]{figures/res_tempfit_2100_v2.pdf}\\
  \includegraphics[width=0.49\textwidth]{figures/map_tempfit_3000_v1.pdf}
  \includegraphics[width=0.49\textwidth]{figures/res_tempfit_3000_v2.pdf}
  \caption{Same as \cref{fig:tempfit_map_vs_res1} but for the three lowest frequency DIRBE channels (240$\mu$m - 100 $\mu$m).}
  \label{fig:tempfit_map_vs_res2}
\end{figure*}

\begin{figure*}
  \centering
  \includegraphics[width=0.49\textwidth]{figures/map_tempfit_5000_v1.pdf}
  \includegraphics[width=0.49\textwidth]{figures/res_tempfit_5000_v2.pdf}\\
  \includegraphics[width=0.49\textwidth]{figures/map_tempfit_12000_v1.pdf}
  \includegraphics[width=0.49\textwidth]{figures/res_tempfit_12000_v2.pdf}\\
  \includegraphics[width=0.49\textwidth]{figures/map_tempfit_25000_v1.pdf}
  \includegraphics[width=0.49\textwidth]{figures/res_tempfit_25000_v2.pdf}
  \caption{Same as \cref{fig:tempfit_map_vs_res1} but for the three intermediate frequency DIRBE channels (60 $\mu$m - 12 $\mu$m).}
  \label{fig:tempfit_map_vs_res3}
\end{figure*}

\begin{figure}
  \centering
  \includegraphics[width=\columnwidth]{figures/template_signal_rms_v1.pdf}
  \caption{Efficiency of the dust model at describing the data at at each frequency. We subtract the squared residuals from the squared map, and normalize by the amplitude of the map. The five component template-based dust model is able to recover more than 80\% of the total map power at all frequencies, including those with known residual zodiacal contamination like 25$\mu$m.}
  \label{fig:temprms}
\end{figure}

\begin{figure}
  \centering
  \includegraphics[width=\columnwidth]{figures/tempfraction_v1.pdf}
  \caption{The relative template amplitudes at each frequency, showing the power contribution of each component as a percentage of the combined model amplitude.}
  \label{fig:tempfract}
\end{figure}


\begin{equation}
    \dv_i = \sum_c a_{i, c} \t_c + m_i + \n_i,
    \label{eq:datamodel}
\end{equation}
where the sum runs over the $n$ components that we assume to be
adequate dust tracers, and where $a_{i, c}$ is the amplitude that best
describes the given component's contribution to the frequency the given data
was measured at $\t_c$ denotes the spatial template of component $c$.
In addition, we include a monopole term, $m_i$, and a white noise contribution
$\n_i$. We thus assume that the data does \emph{not} include a CMB
contribution or a dipole term (again, see \cref{sec:data} for how we
deal with these assumptions).

We allow for masking of the data and components, with the only effect being to
make each vector in the equation shorter, and $n_\mathrm{pix}$ in what follows is the
effective number of pixels after masking.

Given this model, the main goal becomes to find the set of amplitudes $a_{i,
c}$ (and monopole) that jointly minimize the residual
\begin{equation}
    \boldsymbol r_i \equiv \dv_i - \boldsymbol s_i,
\end{equation}
where we define $s_i$ to be all the terms on the right hand side of \cref{eq:datamodel} except the white noise term.

This kind of minimization for a linear problem has a well-known solution. We
annotate our solution vector at each frequency band as $\mathbf{a}_i$ (which, in
this case, is an $n+1$-dimensional vector, the extra dimension being due to the
monopole contribution), and we can then write
\begin{equation}
    \boldsymbol{a}_i = (\mathsf{T}^T \mathsf{T})^{-1} (\mathsf{T}^T\boldsymbol{d}) ,
    \label{eq:lss}
\end{equation}
where we define $\mathsf{T}$ as a $n_\mathrm{pix}$ by $n+1$ matrix such that matrix
column $c$ contains component template $\boldsymbol{t}_c$ (and the final column
is a constant, representing the monopole). Implicit in this way of solving for
the amplitudes is the assumption that the noise root mean squared (RMS) is constant from pixel to
pixel. At the resolutions used for this analysis, this is a fair assumption,
as large-scale instrumental features are dominated by systematic effects rather
than white noise. Solving this equation, the solution vector will then have as
its $n+1$ elements the best-fit amplitudes for each component (and best-fit
monopole).



\section{Data}
\label{sec:data}
As, in this paper and its companions, we are mostly concerned with economical dust modelling, we
use the most accurate data available in the frequency ranges where dust
dominates. More specifically, we use the 353, 545 and 857 GHz frequency bands from \Planck\ HFI, as well as the six longest wavelength bands 
(from 240\,$\mu\mathrm m$ to $12\,\mu\mathrm m$) from the DIRBE data.

We use five known or derived dust tracers as component templates: The
\COBE-FIRAS \CII\ observations, the HI4PI neutral hydrogen map, the Dame CO
$J=1\rightarrow0$ map, the WHAM ionized hydrogen map, and a derived map of
nearby dust structures based on GAIA observations. We note that the fitting
procedure described in \cref{sec:tempfit} does not require that we use
the same templates for all data bands, and because some of these five templates
are only expected to be relevant at lower frequencies, we will restrict the
number of templates used for each band according to \cref{tab:bands}. We
provide justification for these restriction in the following sections, on a per-component basis.

We will first describe the DIRBE and \Planck\ HFI data sets, before describing the
dust tracing templates in more detail, and for which frequencies they are
relevant. Finally, we describe the preprocessing
and masking steps used for this analysis.

\subsection{Frequency maps}

\subsubsection{DIRBE}
The Diffuse InfraRed Explorer (DIRBE), whose main goal was the mapping of the
CIB, was part of the \COBE\ satellite \citep{boggess92, silverberg93}, and
measured the sky in ten frequency bands, from $1\,\mathrm{\mu m}$ to
$240\,\mathrm{\mu m}$. 

In this analysis, we started from the $\nside=512$ maps
created with the procedure described in \cite{CG02_01} from the DIRBE CIO
(Calibrated Individual Observations). The shortest-wavelength features of thermal
dust are the polycyclic aromatic hydrocarbon features, which are expected to be detectable in the the
spectrum down to a few microns \citep{Hensley2021}. We, therefore, include the eight
longest wavelength bands of DIRBE, ranging from $240\,\mu\mathrm m$ to $3.5\,\mu\mathrm m$.

\subsubsection{Planck HFI}
The \Planck\ High Frequency Instrument (HFI; \citealt{planck2016-l03}) observed
the sky in six frequency channels from 100 GHz to 857 GHz, with the primary
purpose of characterizing temperature fluctuations in the CMB to
unprecedentedly small scales. As it also was able to target the far-infrared
sky, it is able to shed some light on the dust population of the sky
\citep{planck2013-XVII,planck2014-a12,planck2016-l03}.

Since this work is primarily concerned with dust modeling, we attempt to avoid
contamination from (especially) the CMB signal, which dominates at the
lower \Planck\ frequencies. To that end, we use only the HFI data from
353 to 857\,GHz. All of the input frequency maps are taken from the \Planck\ PR4 data release
\citep{npipe}.

%The \Planck\ High Frequency Instrument (HFI; \citealt{planck2016-l03}) observed the sky in six channels from 100\,GHz to 857\,GHz from May 2009--2013, with angular resolution of 10\arcm--\,4\arcm. While the primary purpose of the \Planck\ mission was to characterize fluctuations in the CMB, a large part of its scientific legacy comes from its observations of the far-infrared sky, with robust characterization of the Milky Way \citep{planck2013-XVII,planck2014-a12,planck2016-l03} and of CIB fluctuations \citep{planck2014-a12,planck2013-XVII,lenz2019,mccarthy:2024}.
%
%In addition to its complementary observation strategy, \Planck's frequency coverage has a relatively lower expected amount of zodiacal emission, with a total expected amplitude of $\lesssim1\,\%$ before any subtraction \citep{maris2006c,planck2013-pip88}, as compared to almost 100\,\% of the signal in some DIRBE bands. All delivered \Planck\ maps have had an estimate of the zodiacal emission modeled using the 3D model derived by K98 with varying emissivities per component. While this technically is redundant information that could contaminate this joint analysis, the already low amplitude of zodiacal emission in the HFI maps limits the potential impact of using a technically incorrect zodiacal emission model. A full analysis fitting for zodiacal emission parameters using both HFI and DIRBE will be left for future work.
%
%At the same time, CIB fluctuations with a similar SED to the Milky Way have been detected with high significance in the HFI data, and are directly visible in 353--857\,GHz maps at high Galactic latitudes. Incorrectly modeled, this could bias the Galactic thermal dust model and lead to an incorrect model of the sky in the DIRBE range. In order to avoid this, we remove the GNILC \citep{planck2016-XLVIII} estimate of the CIB from the HFI maps before including them in our analysis.


\subsection{Thermal dust tracers}

In this work, we take from legacy and state-of-the-art data sets in order to find reliable maps for the tracers of thermal dust emission we are looking for. Below, we describe each of these data sets in turn, and in \cref{fig:tempfit_inputs}, we show the input maps used for the template fitting.

\subsubsection{Ionized carbon --- \COBE-FIRAS \CII}
The \COBE-FIRAS instrument is the only full-sky spectrometric survey in
the radio and infrared electromagnetic wavelengths, and as such, it provides
the best full-sky measurement of spectral line emission. The $157.7\,\mu\mathrm m$ \CII\
line emission has previously been recognized as a gas tracer; here, we
investigate its potential as a tracer of interstellar dust.

We use the \COBE-FIRAS \CII\ map, available on the LAMBDA website.\footnote{\url{lambda.gsfc.nasa.gov}} As this is the lowest-resolution map in the
analysis, we do not perform any preprocessing on the map before using it as a
template.

\subsubsection{Neutral hydrogen -- HI4PI}
The HI4PI HI survey \citep{HI4PI:2016} is a full-sky survey of the 21 cm
hydrogen line, performed by a combination of data from EBHIS and GASS. The 21
cm line has been used as a tracer for dust structures \citep{planck2011-7.12},
and is typically associated with low-intensity regions of the sky (as it
requires the hydrogen to not be ionized, and be in its lowest energy state).

\subsubsection{Nearby dust through extinction -- Gaia}
One source of important information about the 3-dimensional structure of dust
is starlight extinction. By comparing expected stellar spectra with observed
ones, \cite{edenhofer:2024} was able to reconstruct such dust structures
up to a distance of (TBD), represented by a series of maps for each distance
bin. Using the accompanying software, we integrated these maps out to their
furthest distance, which still represents relatively close dust structures,
galactically speaking, and we use this as a template for nearby dust
structures. The choice of the distance cut is driven more by the available
Edenhofer maps than any physical considerations; still, in this work, this
seems to be the best way of utilizing the available additional information
about the dust morphology that these maps represent.


\subsubsection{Ionized hydrogen -- WHAM}
The Wisconsin H-Alpha Mapper (WHAM) \citep{wham_north:2003, wham_south:2010}
surveyed the intensity of the H$\alpha$ line over the whole sky, with a
resolution of 1 degree, and a velocity resolution of 12 km/s. This line is
typically associated with higher energy regions, as it is most commonly emitted
from hydrogen atoms recombining after ionization. The WHAM H$\alpha$ map was
used in \Planck\ as a tracer of nearby free-free emission (morphologically, it
is clearly tracing ``globules'' of activity which, both by virtue of their size
on the sky and the fact that the galactic center is not discernible, must be
relatively close).


\subsubsection{Carbon monoxide -- Dame CO\,$J=1$-0}
\citet{dame:2001} surveyed about 45\% of the sky for the $J=1$-0 CO line within
30 degrees of the Galactic equator. As dust is expected to be located in
molecular clouds, the CO line would be a natural tracer for such dust
populations.

\subsection{Low-resolution preprocessing}
The template fitting procedure described in \cref{sec:tempfit}
requires all maps to be smoothed to a common resolution -- if not, the
extra degrees of freedom at smaller scales in some of the maps and
templates might bias the fit at larger scales. Out of all the data and
templates used in this analysis, the \COBE-FIRAS \CII\ map is the one
with the lowest native resolution. We therefore convolve all maps with
the \COBE-FIRAS beam, shown in \cref{fig:firasbeam}, taking into
account the drift along the Ecliptic ({\bf cite Odegaard or similar}),
and we repixelate all maps to a HEALPix resolution of $\nside=16$.


\subsection{Processing mask}
The FIRAS CII map in particular contains systematic effects, which we mask in this analysis. In addition, due to the complexity of the galactic center, as well as certain high-activity complexes (TBD), we mask these as well. 

\section{Results}
\label{sec:results}

\subsection{Template amplitude fits}
\label{sec:tempamp}

The fit template amplitudes per frequency band are shown in \cref{fig:tempamp}. As is evident, the spectra of all templates show a
characteristic bump, which is the expected behavior of modified blackbody SEDs
with temperatures in the typical thermal dust temperature range (see below for
actual temperature estimations of these spectra).

We also see that the H$\alpha$ component has a negative SED, showing that it
indeed functions as a dust extinction template.



\subsection{Model efficiency and goodness-of-fit}
\label{sec:residuals}
In \cref{fig:tempfit_map_vs_res1,fig:tempfit_map_vs_res2,fig:tempfit_map_vs_res3}, we show a
comparison between the original data (left columns) and the template cleaned
maps (right columns). Clearly, the template fitting performs well on all
frequencies, leaving residual signals that are associated with high-activity
regions of the Milky Way (especially evident in the 857\,GHz and 3000\,GHz
residuals), as well as unmodelled zodiacal emission from the Solar System
(evident in the two highest frequency maps). Overall, a by-eye survey seems to
indicate that the template fitting is an efficient procedure for accounting for
most of the dust.

In order to quantify the degree to which this impression bears out in practice,
in \cref{fig:temprms} we construct a measure of the amount of signal that
is ``explained'' by the template fit. It is constructed by taking the RMS of
the pixel values in the original map, as well as those of the residual map
after template subtraction, and subtracting these values from each other,
normalizing the value by the original RMS. We see that the fraction of
unexplained RMS left in the map ranges from about 2\% (for the lower
frequencies) to around 15\%, and we note that the only two maps that have more
than 5\% of unexplained RMS are the two maps where the imprint of remaining
zodiacal light is the strongest.




\subsection{Modified blackbody fits}
\label{sec:mbb}
In \cref{fig:Bnu_comparison}, we fit the following modified blackbody model
to the template fitting SEDs:

\begin{equation}
    s(\nu) = A \frac{\nu^{\beta + 3}}{\exp{(\frac{h\nu}{k_b T})} - 1}.
\end{equation}
Here, A is an arbitrary scaling factor, T is the blackbody temperature, and
$\beta$ is the spectral index of the modified blackbody. It is especially
interesting to compare the two strongest SEDs -- those associated with the CII
template and the HI template, respectively, as we see that they correspond to
significantly different temperatures. Thus, the dust that the CII template
traces morphologically (which we \emph{a priori} would assume to be associated
with high-energy regions because of the high transitional energy of this
line) also seems to exhibit a high intrinsic temperature.

Similarly, we may draw the opposite conclusions regarding the dust traced by
the HI template: As expected, this dust exhibits a much lower temperature,
consonant with the picture of it being located in low-intensity regions.

The GAIA extinction template exhibits an even lower temperature, but it is also
modified by a significantly steeper spectral index, indicating a higher degree
of absorption at the lower frequencies.

The Dame CO template is the coldest of all the components, while still
exhibiting a relatively steep spectral index, which might indicate that this
dust emission originates from cold star-forming regions.

The WHAM H$\alpha$ template exihibits the steepest spectrum of all the
components, consistent with it being correlated with regions of high activity.

\begin{figure}
  \centering
  \includegraphics[width=\columnwidth]{figures/tempfit_vs_Bnu_v2.pdf}
  \caption{Comparison between template fit amplitudes (solid lines, reproduced from \cref{fig:tempamp}) with best-fit modified blackbody models (dotted lines), in the thermal regime for each component. The best-fit MBB temperatures and $\beta$s for each curve are listed in the figure legend.}
  \label{fig:Bnu_comparison}
\end{figure}


\section{Conclusions}
\label{sec:conclusions}

In this paper, we have fit five templates based on pre-existing data sets which
might be expected to correlate to dust in some way to frequency maps from
\Planck\ HFI and DIRBE: A CII template based on FIRAS data; a HI template from
the HI4PI survey; a ``nearby dust'' template constructed from integrated GAIA
extinction maps made by Edenhofer et al., the Dame CO template; and a WHAM
H$\alpha$ template.

We fit these to data (downgraded to the FIRAS beam) from \Planck\ HFI and
DIRBE, from 353\,GHz to 12$\mu$ m, where thermal dust emission is one of the
main sources of foreground contamination. We show that this template fit is
both able to account for at least 95\% of the dust RMS for all but the two
highest frequency bands, and that the resulting SEDs follow modified blackbody
curves with clearly defined temperatures and spectral indices.

In particular, we find that the two most dominant components -- traced by the
CII and HI templates -- exhibit similar spectral indices, but very different
blackbody temperatures (24.7 and 15.7 K, respectively). Thus, these
templates trace dust populations that originate in significantly
different environments. The H$\alpha$ template traces dust that is mainly
making its presence known through extinction -- the steep spectral index for
this template is consistent with the assumption that this dust is mostly
located in nearby high-activity regions.

This analysis shows that it is possible to vastly reduce the complexity of the
challenge of thermal dust modelling: Instead of operating with models that have
several free parameters per pixel, this approach shows that it should be
possible to reduce the degrees of freedom quite extensively by assuming
a limited number of populations of dust extending across the sky with common
physical parameters, and that such populations can helpfully be informed by
utilizing data that is assumed to trace physical processes and environmental
conditions.

A natural question arising from this analysis is whether the method proposed
here also applies to polarized dust emission, and this question will be the
subject of a future analysis. If this turns out to be the case, it would have a
significant impact on the feasibility of modelling polarized dust emission for
future CMB experiments, which are dependent on accurate modelling of polarized
foregrounds to reach the sensitivities required for their science goals (in
particular, the detection of $r$).

\begin{acknowledgements}
  We thank \textbf{Richard Arendt}, Tony Banday, Johannes Eskilt, Dale Fixsen, Ken Ganga, Paul
  Goldsmith, Shuji Matsuura, Sven Wedemeyer, Janet Weiland and \textbf{Edward Wright} for
  useful suggestions and guidance.  The current work has received
  funding from the European Union’s Horizon 2020 research and
  innovation programme under grant agreement numbers 819478 (ERC;
  \textsc{Cosmoglobe}), 772253 (ERC; \textsc{bits2cosmology}),
  \textbf{101165647 (ERC, \textsc{Origins}), 101141621 (ERC,
  \textsc{Commander}), and 101007633 (MSCA; \textsc{CMBInflate}).
  This article reflects the views of the authors only. The funding
  body is not responsible for any use that may be made of the
  information contained therein. This research is also funded by the
  Research Council of Norway under grant agreement number 344934 (YRT;
  \textsc{CosmoglobeHD}).} Some of the results in this paper have been
  derived using healpy \citep{Zonca2019} and the HEALPix
  \citep{healpix} packages.  We acknowledge the use of the Legacy
  Archive for Microwave Background Data Analysis (LAMBDA), part of the
  High Energy Astrophysics Science Archive Center
  (HEASARC). HEASARC/LAMBDA is a service of the Astrophysics Science
  Division at the NASA Goddard Space Flight Center. This publication
  makes use of data products from the Wide-field Infrared Survey
  Explorer, which is a joint project of the University of California,
  Los Angeles, and the Jet Propulsion Laboratory/California Institute
  of Technology, funded by the National Aeronautics and Space
  Administration. This work has made use of data from the European
  Space Agency (ESA) mission {\it Gaia}
  (\url{https://www.cosmos.esa.int/gaia}), processed by the {\it Gaia}
  Data Processing and Analysis Consortium (DPAC,
  \url{https://www.cosmos.esa.int/web/gaia/dpac/consortium}). Funding
  for the DPAC has been provided by national institutions, in
  particular the institutions participating in the {\it Gaia}
  Multilateral Agreement.  We acknowledge the use of data provided by
  the Centre d'Analyse de Données Etendues (CADE), a service of
  IRAP-UPS/CNRS (http://cade.irap.omp.eu, \citealt{paradis:2012}).  
  This paper and related research have been conducted during and with the support of the Italian national inter-university PhD programme in Space Science and Technology. Work on this article was produced while attending the PhD program in PhD in Space Science and Technology at the University of Trento, Cycle XXXIX, with the support of a scholarship financed by the Ministerial Decree no. 118 of 2nd March 2023, based on the NRRP - funded by the European Union - NextGenerationEU - Mission 4 "Education and Research", Component 1 "Enhancement of the offer of educational services: from nurseries to universities” - Investment 4.1 “Extension of the number of research doctorates and innovative doctorates for public administration and cultural heritage” - CUP E66E23000110001.
\end{acknowledgements}


%-------------------------------------------------------------
%                                       Table with references 
%-------------------------------------------------------------
%

\bibliographystyle{aa}
\bibliography{../../common/CG_bibliography,references,../../common/Planck_bib}
\end{document}
%%%% End of aa.dem
