%                                                                 aa.dem
% AA vers. 9.1, LaTeX class for Astronomy & Astrophysics
% demonstration file
%                                                       (c) EDP Sciences
%-----------------------------------------------------------------------
%
% \documentclass[referee]{aa} % for a referee version
%\documentclass[onecolumn]{aa} % for a paper on 1 column  
%\documentclass[longauth]{aa} % for the long lists of affiliations 
%\documentclass[letter]{aa} % for the letters 
%\documentclass[bibyear]{aa} % if the references are not structured 
%                              according to the author-year natbib style

%

\documentclass{aa}  

%
\usepackage{graphicx}
\usepackage{amsmath,amsfonts,amssymb}
\usepackage{natbib}


%%%%%%%%%%%%%%%%%%%%%%%%%%%%%%%%%%%%%%%%
\usepackage{txfonts}
\usepackage{xcolor}

\usepackage{blindtext}
%%%%%%%%%%%%%%%%%%%%%%%%%%%%%%%%%%%%%%%%
% \usepackage[options]{hyperref}
% To add links in your PDF file, use the package "hyperref"
% with options according to your LaTeX or PDFLaTeX drivers.
\usepackage{float}
%\usepackage{stfloats}
\usepackage{dblfloatfix}
\usepackage{afterpage}
\usepackage{ifthen}
\usepackage[morefloats=12]{morefloats}

\usepackage{placeins}
\usepackage{multicol}
%\usepackage[breaklinks,colorlinks,citecolor=blue]{hyperref}
\bibpunct{(}{)}{;}{a}{}{,}
\usepackage[switch]{lineno}
\definecolor{linkcolor}{rgb}{0.6,0,0}
\definecolor{citecolor}{rgb}{0,0,0.75}
\definecolor{urlcolor}{rgb}{0.12,0.46,0.7}
\usepackage[breaklinks, colorlinks, urlcolor=urlcolor,
    linkcolor=linkcolor,citecolor=citecolor,pdfencoding=auto]{hyperref}
\hypersetup{linktocpage}
\usepackage{bold-extra}



\def\setsymbol#1#2{\expandafter\def\csname #1\endcsname{#2}}
\def\getsymbol#1{\csname #1\endcsname}

\def\Planck{\textit{Planck}}

\def\HeJT{$^4$He-JT}

\def\allearlypapers{\nocite{planck2011-1.1, planck2011-1.3, planck2011-1.4, planck2011-1.5, planck2011-1.6, planck2011-1.7, planck2011-1.10, planck2011-1.10sup, planck2011-5.1a, planck2011-5.1b, planck2011-5.2a, planck2011-5.2b, planck2011-5.2c, planck2011-6.1, planck2011-6.2, planck2011-6.3a, planck2011-6.4a, planck2011-6.4b, planck2011-6.6, planck2011-7.0, planck2011-7.2, planck2011-7.3, planck2011-7.7a, planck2011-7.7b, planck2011-7.12, planck2011-7.13}}

\def\alltwentythirteenresultspapers{\nocite{planck2013-p01, planck2013-p02, planck2013-p02a, planck2013-p02d, planck2013-p02b, planck2013-p03, planck2013-p03c, planck2013-p03f, planck2013-p03d, planck2013-p03e, planck2013-p01a, planck2013-p06, planck2013-p03a, planck2013-pip88, planck2013-p08, planck2013-p11, planck2013-p12, planck2013-p13, planck2013-p14, planck2013-p15, planck2013-p05b, planck2013-p17, planck2013-p09, planck2013-p09a, planck2013-p20, planck2013-p19, planck2013-pipaberration, planck2013-p05, planck2013-p05a, planck2013-pip56, planck2013-p06b, planck2013-p01a}}

\def\alltwentyfifteenresultspapers{\nocite{planck2014-a01, planck2014-a03, planck2014-a04, planck2014-a05, planck2014-a06, planck2014-a07, planck2014-a08, planck2014-a09, planck2014-a11, planck2014-a12, planck2014-a13, planck2014-a14, planck2014-a15, planck2014-a16, planck2014-a17, planck2014-a18, planck2014-a19, planck2014-a20, planck2014-a22, planck2014-a24, planck2014-a26, planck2014-a28, planck2014-a29, planck2014-a30, planck2014-a31, planck2014-a35, planck2014-a36, planck2014-a37, planck2014-ES}}

\newbox\tablebox    \newdimen\tablewidth
\def\leaderfil{\leaders\hbox to 5pt{\hss.\hss}\hfil}
\def\endPlancktable{\tablewidth=\columnwidth 
    $$\hss\copy\tablebox\hss$$
    \vskip-\lastskip\vskip -2pt}
\def\endPlancktablewide{\tablewidth=\textwidth 
    $$\hss\copy\tablebox\hss$$
    \vskip-\lastskip\vskip -2pt}
\def\tablenote#1 #2\par{\begingroup \parindent=0.8em
    \abovedisplayshortskip=0pt\belowdisplayshortskip=0pt
    \noindent
    $$\hss\vbox{\hsize\tablewidth \hangindent=\parindent \hangafter=1 \noindent
    \hbox to \parindent{$^#1$\hss}\strut#2\strut\par}\hss$$
    \endgroup}
\def\doubleline{\vskip 3pt\hrule \vskip 1.5pt \hrule \vskip 5pt}

\def\L2{\ifmmode L_2\else $L_2$\fi}
\def\dtt{\Delta T/T}
\def\DeltaT{\ifmmode \Delta T\else $\Delta T$\fi}
\def\deltat{\ifmmode \Delta t\else $\Delta t$\fi}
\def\fknee{\ifmmode f_{\rm knee}\else $f_{\rm knee}$\fi}
\def\Fmax{\ifmmode F_{\rm max}\else $F_{\rm max}$\fi}
\def\solar{\ifmmode{\rm M}_{\mathord\odot}\else${\rm M}_{\mathord\odot}$\fi}
\def\Msolar{\ifmmode{\rm M}_{\mathord\odot}\else${\rm M}_{\mathord\odot}$\fi}
\def\Lsolar{\ifmmode{\rm L}_{\mathord\odot}\else${\rm L}_{\mathord\odot}$\fi}
\def\inv{\ifmmode^{-1}\else$^{-1}$\fi}
\def\mo{\ifmmode^{-1}\else$^{-1}$\fi}
\def\sup#1{\ifmmode ^{\rm #1}\else $^{\rm #1}$\fi}
\def\expo#1{\ifmmode \times 10^{#1}\else $\times 10^{#1}$\fi}
\def\,{\thinspace}
\def\lsim{\mathrel{\raise .4ex\hbox{\rlap{$<$}\lower 1.2ex\hbox{$\sim$}}}}
\def\gsim{\mathrel{\raise .4ex\hbox{\rlap{$>$}\lower 1.2ex\hbox{$\sim$}}}}
\let\lea=\lsim
\let\gea=\gsim
\def\simprop{\mathrel{\raise .4ex\hbox{\rlap{$\propto$}\lower 1.2ex\hbox{$\sim$}}}}
\def\deg{\ifmmode^\circ\else$^\circ$\fi}
\def\pdeg{\ifmmode $\setbox0=\hbox{$^{\circ}$}\rlap{\hskip.11\wd0 .}$^{\circ}
          \else \setbox0=\hbox{$^{\circ}$}\rlap{\hskip.11\wd0 .}$^{\circ}$\fi}
\def\arcs{\ifmmode {^{\scriptstyle\prime\prime}}
          \else $^{\scriptstyle\prime\prime}$\fi}
\def\arcm{\ifmmode {^{\scriptstyle\prime}}
          \else $^{\scriptstyle\prime}$\fi}
\newdimen\sa  \newdimen\sb
\def\parcs{\sa=.07em \sb=.03em
     \ifmmode \hbox{\rlap{.}}^{\scriptstyle\prime\kern -\sb\prime}\hbox{\kern -\sa}
     \else \rlap{.}$^{\scriptstyle\prime\kern -\sb\prime}$\kern -\sa\fi}
\def\parcm{\sa=.08em \sb=.03em
     \ifmmode \hbox{\rlap{.}\kern\sa}^{\scriptstyle\prime}\hbox{\kern-\sb}
     \else \rlap{.}\kern\sa$^{\scriptstyle\prime}$\kern-\sb\fi}
\def\ra[#1 #2 #3.#4]{#1\sup{h}#2\sup{m}#3\sup{s}\llap.#4}
\def\dec[#1 #2 #3.#4]{#1\deg#2\arcm#3\arcs\llap.#4}
\def\deco[#1 #2 #3]{#1\deg#2\arcm#3\arcs}
\def\rra[#1 #2]{#1\sup{h}#2\sup{m}}
\def\page{\vfill\eject}
\def\dots{\relax\ifmmode \ldots\else $\ldots$\fi}
\def\WHzsr{\ifmmode $W\,Hz\mo\,sr\mo$\else W\,Hz\mo\,sr\mo\fi}
\def\mHz{\ifmmode $\,mHz$\else \,mHz\fi}
\def\GHz{\ifmmode $\,GHz$\else \,GHz\fi}
\def\mKs{\ifmmode $\,mK\,s$^{1/2}\else \,mK\,s$^{1/2}$\fi}
\def\muKs{\ifmmode \,\mu$K\,s$^{1/2}\else \,$\mu$K\,s$^{1/2}$\fi}
\def\muKRJs{\ifmmode \,\mu$K$_{\rm RJ}$\,s$^{1/2}\else \,$\mu$K$_{\rm RJ}$\,s$^{1/2}$\fi}
\def\muKHz{\ifmmode \,\mu$K\,Hz$^{-1/2}\else \,$\mu$K\,Hz$^{-1/2}$\fi}
\def\MJysr{\ifmmode \,$MJy\,sr\mo$\else \,MJy\,sr\mo\fi}
\def\MJysrmK{\ifmmode \,$MJy\,sr\mo$\,mK$_{\rm CMB}\mo\else \,MJy\,sr\mo\,mK$_{\rm CMB}\mo$\fi}
\def\microns{\ifmmode \,\mu$m$\else \,$\mu$m\fi}
\def\micron{\microns}
\def\muK{\ifmmode \,\mu$K$\else \,$\mu$\hbox{K}\fi}
\def\microK{\ifmmode \,\mu$K$\else \,$\mu$\hbox{K}\fi}
\def\muW{\ifmmode \,\mu$W$\else \,$\mu$\hbox{W}\fi}
\def\kms{\ifmmode $\,km\,s$^{-1}\else \,km\,s$^{-1}$\fi}
\def\kmsMpc{\ifmmode $\,\kms\,Mpc\mo$\else \,\kms\,Mpc\mo\fi}

\providecommand{\sorthelp}[1]{}


% Custom definitions
\newcommand{\mathsc}[1]{{\normalfont\textsc{#1}}}
\def\Cosmoglobe{\textsc{Cosmoglobe}}
\def\Planck{\textit{Planck}}
\def\WMAP{\textit{WMAP}}


\begin{document} 


   \title{\bfseries{\Cosmoglobe\ DR2. IV. Modelling starlight emission\\ in DIRBE with GAIA and WISE}}

   \author{Placeholder}

   \institute{Institute of Theoretical Astrophysics, University of Oslo, Blindern, Oslo, Norway}
  
   % Shortened title, author list for top of page 
   \titlerunning{Starlight emission in DIRBE}
   \authorrunning{Placeholder}

   \date{\today} 
   
   \abstract{We present a model of starlight emission in the Diffuse Infrared Background Explorer (DIRBE) data between 1.25 and 25$\,\mu$m based on GAIA and WISE measurements. For each star brighter than magnitude 8 in the WISE 3.4$\,\mu$m band, we extract estimates of the effective temperature, $T_{\mathrm{eff}}$, the gravitational acceleration, $\log g$, and the metallicity, $[M/H]$, from the GAIA DR2 database, and use those to identify a best-fit spectral energy density (SED) from the PHOENIX starlight database. This SED is convolved with the appropriate DIRBE bandpasses and beam response function, and only a single overall flux density is fitted per star. We find that the number of stars with a statistically significant flux density detected at Galactic latitudes $|b|>20^{\circ}$ at more than than $5\,\sigma$ is 313\,061 stars, for an average of 0.4~stars per DIRBE beam area. Furthermore, based on this model we find that total star emission accounts for 98\,\% of the observed flux density at 1.25\,$\mu$m; 83\,\% at 4.9$\,\mu$m; and 3\,\% at 25\,$\mu$m. As shown in companion papers, this new model is sufficiently accurate to allow for precise characterization of both extragalactic background (cosmic infrared and optical background; CIB and COB) fluctuations and Galactic (free-free and polycyclic aromatic hydrocarbon (PAH) dust) emission in the four highest DIRBE frequencies.      }

   \keywords{ISM: general - Zodiacal dust, Interplanetary medium - Cosmology: observations, diffuse radiation - Galaxy: general}

   \maketitle

\setcounter{tocdepth}{2}
\tableofcontents
   
% INTRODUCTION
%-------------------------------------------------------------------
\section{Introduction}
%\the\textwidth \the\columnwidth

Modelling the microwave sky in the COBE-Diffuse Infrared Background Explorer (DIRBE) data  \citep{DIRBE} from 1 to 100000 GHz requires a  understanding of all the various components that make it up. At the highest frequencies, the largest power contribution comes from stars in our galaxy, as can be seen in Fig. \ref{fig:sed}. The smallest wavelength/highest frequency bands 1-4 are dominated by star emissions, making accurate star modelling essential for using these data in combination with others in a comprehensive Bayesian model of the large-scale infrared sky. Additionally, stars are subdominant contributors in DIRBE bands 5 and 6, which if not handled correctly could heavily skew derived constraints on the zodiacal light and other components.

Many full-sky datasets exist which measure the emissions from stars in our galaxy. Staring with IRAS in 1983 \citep{iras}, there have been several ground and space missions to map the infrared sky and put constraints on star star formation and evolution, the Cosmic Infrared Background (CIB) and Active Galactic Nuclei (AGN), including the Infrared Space Observatory (ISO) \citep{iso}, Spitzer \citep{spitzer}, the 2-Micron All-Sky Survey (2MASS) \citep{2mass}, and the current-generation James Webb Space Telescope \citep{jwst}. For the purposes of compatibility with the DIRBE data, however, the most useful external datasets are the Wide-Field Infrared Survey Explorer (WISE) \citep{wise}, and GAIA, which produced high precision star observations at optical wavelengths \citep{gaia}. 

\begin{figure*}
  \centering
  \includegraphics[width=\textwidth]{figs/sed/all_fgs.pdf}
  \caption{Components of the infrared temperature sky from 0.1$\mu$m to 1m wavelengths. The contribution from stars is shown in blue, and the DIRBE and HFI band nominal frequencies are indicated by the vertical lines.}
  \label{fig:sed}
\end{figure*}

In this paper, part of the larger Cosmoglobe DR2 results paper suite, we discuss the modelling of stars as part of our larger full sky astrophysical model. Section \ref{sec:models} discusses star modelling in a larger Bayesian context. Section \ref{sec:impact} then shows the impact of stars on the DIRBE data. Section \ref{sec:data} discusses the data selection and cleaning used in our analysis, and then Section \ref{sec:results} shows the astrophysical results of the work by presenting a unified star model and its associated errors. Finally, we conclude in Section \ref{sec:conclusions}, and offer some avenues for future research.

\section{Star modelling in a Bayesian context}
\label{sec:models}


\begin{equation}
d = s_{stars}
\label{eq:datamodel}
\end{equation}


Equation \ref{eq:datamodel} shows the data model used in the analysis

\clearpage
\section{Impact of starlight on DIRBE observations}
\label{sec:impact}

%\begin{figure}
%  \centering
%  \includegraphics[width=\columnwidth]{figs/mask_proc_calib.pdf}
%  \caption{Processing masks use in the analysis.}
%  \label{fig:masks}
%\end{figure}

Diffuse stars vs. point sources? 



\clearpage
\section{Data selection and cleaning}
\label{sec:data}

The results presented in Section \ref{sec:results} are a part of the Cosmoglobe DR2 release. Here we will discuss only the data selection specific to star modelling, for a full overview of Cosmoglobe DR2, please refer to  \cite{CG02_01}.

\subsection{Catalogue Creation}

Star positions are drawn from a catalogue 

\subsection{Data Cleaning}

\clearpage
\section{Astrophysical results}
\label{sec:results}

Figure: Plot of star component

Figure: standard deviation of star amplitudes

Figure: all star fits as a function of frequency 

\clearpage
\section{Conclusions}
\label{sec:conclusions}

Stars are a critical component of the infrared sky, which must be correctly modelled in order to avoid contaminating other components. 

\subsection{Future Work}






\begin{acknowledgements}
 The current work has received funding from the European
  Union’s Horizon 2020 research and innovation programme under grant
  agreement numbers 819478 (ERC; \textsc{Cosmoglobe}) and 772253 (ERC;
  \textsc{bits2cosmology}). Some of the results in this paper have been derived using the HEALPix \citep{HEALPIX} package.
  We acknowledge the use of the Legacy Archive for Microwave Background Data
  Analysis (LAMBDA), part of the High Energy Astrophysics Science Archive Center
  (HEASARC). HEASARC/LAMBDA is a service of the Astrophysics Science Division at
  the NASA Goddard Space Flight Center.  
\end{acknowledgements}


%-------------------------------------------------------------
%                                       Table with references 
%-------------------------------------------------------------
%

\bibliographystyle{aa}
\bibliography{references, ../../common/CG_bibliography}
\end{document}
%%%% End of aa.dem
