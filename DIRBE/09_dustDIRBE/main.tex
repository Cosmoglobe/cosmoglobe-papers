%                                                                 aa.dem
% AA vers. 9.1, LaTeX class for Astronomy & Astrophysics
% demonstration file
%                                                       (c) EDP Sciences
%-----------------------------------------------------------------------
%
% \documentclass[referee]{aa} % for a referee version
%\documentclass[onecolumn]{aa} % for a paper on 1 column  
%\documentclass[longauth]{aa} % for the long lists of affiliations 
%\documentclass[letter]{aa} % for the letters 
%\documentclass[bibyear]{aa} % if the references are not structured 
%                              according to the author-year natbib style

%

\documentclass{aa}  

%
\usepackage{graphicx}
\usepackage{amsmath,amsfonts,amssymb}
\usepackage{natbib}


%%%%%%%%%%%%%%%%%%%%%%%%%%%%%%%%%%%%%%%%
\usepackage{txfonts}
\usepackage{xcolor}

\usepackage{blindtext}
%%%%%%%%%%%%%%%%%%%%%%%%%%%%%%%%%%%%%%%%
% \usepackage[options]{hyperref}
% To add links in your PDF file, use the package "hyperref"
% with options according to your LaTeX or PDFLaTeX drivers.
\usepackage{float}
%\usepackage{stfloats}
\usepackage{dblfloatfix}
\usepackage{afterpage}
\usepackage{ifthen}
\usepackage[morefloats=12]{morefloats}

\usepackage{placeins}
\usepackage{multicol}
%\usepackage[breaklinks,colorlinks,citecolor=blue]{hyperref}
\bibpunct{(}{)}{;}{a}{}{,}
\usepackage[switch]{lineno}
\definecolor{linkcolor}{rgb}{0.6,0,0}
\definecolor{citecolor}{rgb}{0,0,0.75}
\definecolor{urlcolor}{rgb}{0.12,0.46,0.7}
\usepackage[breaklinks, colorlinks, urlcolor=urlcolor,
    linkcolor=linkcolor,citecolor=citecolor,pdfencoding=auto]{hyperref}
\hypersetup{linktocpage}
\usepackage{bold-extra}
\usepackage{makecell}

\usepackage[capitalise]{cleveref}
\Crefname{section}{Sect.}{Sects.}
\Crefname{table}{Table}{Tables}
\Crefname{equation}{Eq.}{Eqs.}
\Crefname{appendix}{Appendix}{Appendices}
\makeatletter
\AddToHook{cmd/appendix/before}{\def\cref@section@alias{appendix}}
\makeatother



\def\setsymbol#1#2{\expandafter\def\csname #1\endcsname{#2}}
\def\getsymbol#1{\csname #1\endcsname}

\def\Planck{\textit{Planck}}

\def\HeJT{$^4$He-JT}

\def\allearlypapers{\nocite{planck2011-1.1, planck2011-1.3, planck2011-1.4, planck2011-1.5, planck2011-1.6, planck2011-1.7, planck2011-1.10, planck2011-1.10sup, planck2011-5.1a, planck2011-5.1b, planck2011-5.2a, planck2011-5.2b, planck2011-5.2c, planck2011-6.1, planck2011-6.2, planck2011-6.3a, planck2011-6.4a, planck2011-6.4b, planck2011-6.6, planck2011-7.0, planck2011-7.2, planck2011-7.3, planck2011-7.7a, planck2011-7.7b, planck2011-7.12, planck2011-7.13}}

\def\alltwentythirteenresultspapers{\nocite{planck2013-p01, planck2013-p02, planck2013-p02a, planck2013-p02d, planck2013-p02b, planck2013-p03, planck2013-p03c, planck2013-p03f, planck2013-p03d, planck2013-p03e, planck2013-p01a, planck2013-p06, planck2013-p03a, planck2013-pip88, planck2013-p08, planck2013-p11, planck2013-p12, planck2013-p13, planck2013-p14, planck2013-p15, planck2013-p05b, planck2013-p17, planck2013-p09, planck2013-p09a, planck2013-p20, planck2013-p19, planck2013-pipaberration, planck2013-p05, planck2013-p05a, planck2013-pip56, planck2013-p06b, planck2013-p01a}}

\def\alltwentyfifteenresultspapers{\nocite{planck2014-a01, planck2014-a03, planck2014-a04, planck2014-a05, planck2014-a06, planck2014-a07, planck2014-a08, planck2014-a09, planck2014-a11, planck2014-a12, planck2014-a13, planck2014-a14, planck2014-a15, planck2014-a16, planck2014-a17, planck2014-a18, planck2014-a19, planck2014-a20, planck2014-a22, planck2014-a24, planck2014-a26, planck2014-a28, planck2014-a29, planck2014-a30, planck2014-a31, planck2014-a35, planck2014-a36, planck2014-a37, planck2014-ES}}

\newbox\tablebox    \newdimen\tablewidth
\def\leaderfil{\leaders\hbox to 5pt{\hss.\hss}\hfil}
\def\endPlancktable{\tablewidth=\columnwidth 
    $$\hss\copy\tablebox\hss$$
    \vskip-\lastskip\vskip -2pt}
\def\endPlancktablewide{\tablewidth=\textwidth 
    $$\hss\copy\tablebox\hss$$
    \vskip-\lastskip\vskip -2pt}
\def\tablenote#1 #2\par{\begingroup \parindent=0.8em
    \abovedisplayshortskip=0pt\belowdisplayshortskip=0pt
    \noindent
    $$\hss\vbox{\hsize\tablewidth \hangindent=\parindent \hangafter=1 \noindent
    \hbox to \parindent{$^#1$\hss}\strut#2\strut\par}\hss$$
    \endgroup}
\def\doubleline{\vskip 3pt\hrule \vskip 1.5pt \hrule \vskip 5pt}

\def\L2{\ifmmode L_2\else $L_2$\fi}
\def\dtt{\Delta T/T}
\def\DeltaT{\ifmmode \Delta T\else $\Delta T$\fi}
\def\deltat{\ifmmode \Delta t\else $\Delta t$\fi}
\def\fknee{\ifmmode f_{\rm knee}\else $f_{\rm knee}$\fi}
\def\Fmax{\ifmmode F_{\rm max}\else $F_{\rm max}$\fi}
\def\solar{\ifmmode{\rm M}_{\mathord\odot}\else${\rm M}_{\mathord\odot}$\fi}
\def\Msolar{\ifmmode{\rm M}_{\mathord\odot}\else${\rm M}_{\mathord\odot}$\fi}
\def\Lsolar{\ifmmode{\rm L}_{\mathord\odot}\else${\rm L}_{\mathord\odot}$\fi}
\def\inv{\ifmmode^{-1}\else$^{-1}$\fi}
\def\mo{\ifmmode^{-1}\else$^{-1}$\fi}
\def\sup#1{\ifmmode ^{\rm #1}\else $^{\rm #1}$\fi}
\def\expo#1{\ifmmode \times 10^{#1}\else $\times 10^{#1}$\fi}
\def\,{\thinspace}
\def\lsim{\mathrel{\raise .4ex\hbox{\rlap{$<$}\lower 1.2ex\hbox{$\sim$}}}}
\def\gsim{\mathrel{\raise .4ex\hbox{\rlap{$>$}\lower 1.2ex\hbox{$\sim$}}}}
\let\lea=\lsim
\let\gea=\gsim
\def\simprop{\mathrel{\raise .4ex\hbox{\rlap{$\propto$}\lower 1.2ex\hbox{$\sim$}}}}
\def\deg{\ifmmode^\circ\else$^\circ$\fi}
\def\pdeg{\ifmmode $\setbox0=\hbox{$^{\circ}$}\rlap{\hskip.11\wd0 .}$^{\circ}
          \else \setbox0=\hbox{$^{\circ}$}\rlap{\hskip.11\wd0 .}$^{\circ}$\fi}
\def\arcs{\ifmmode {^{\scriptstyle\prime\prime}}
          \else $^{\scriptstyle\prime\prime}$\fi}
\def\arcm{\ifmmode {^{\scriptstyle\prime}}
          \else $^{\scriptstyle\prime}$\fi}
\newdimen\sa  \newdimen\sb
\def\parcs{\sa=.07em \sb=.03em
     \ifmmode \hbox{\rlap{.}}^{\scriptstyle\prime\kern -\sb\prime}\hbox{\kern -\sa}
     \else \rlap{.}$^{\scriptstyle\prime\kern -\sb\prime}$\kern -\sa\fi}
\def\parcm{\sa=.08em \sb=.03em
     \ifmmode \hbox{\rlap{.}\kern\sa}^{\scriptstyle\prime}\hbox{\kern-\sb}
     \else \rlap{.}\kern\sa$^{\scriptstyle\prime}$\kern-\sb\fi}
\def\ra[#1 #2 #3.#4]{#1\sup{h}#2\sup{m}#3\sup{s}\llap.#4}
\def\dec[#1 #2 #3.#4]{#1\deg#2\arcm#3\arcs\llap.#4}
\def\deco[#1 #2 #3]{#1\deg#2\arcm#3\arcs}
\def\rra[#1 #2]{#1\sup{h}#2\sup{m}}
\def\page{\vfill\eject}
\def\dots{\relax\ifmmode \ldots\else $\ldots$\fi}
\def\WHzsr{\ifmmode $W\,Hz\mo\,sr\mo$\else W\,Hz\mo\,sr\mo\fi}
\def\mHz{\ifmmode $\,mHz$\else \,mHz\fi}
\def\GHz{\ifmmode $\,GHz$\else \,GHz\fi}
\def\mKs{\ifmmode $\,mK\,s$^{1/2}\else \,mK\,s$^{1/2}$\fi}
\def\muKs{\ifmmode \,\mu$K\,s$^{1/2}\else \,$\mu$K\,s$^{1/2}$\fi}
\def\muKRJs{\ifmmode \,\mu$K$_{\rm RJ}$\,s$^{1/2}\else \,$\mu$K$_{\rm RJ}$\,s$^{1/2}$\fi}
\def\muKHz{\ifmmode \,\mu$K\,Hz$^{-1/2}\else \,$\mu$K\,Hz$^{-1/2}$\fi}
\def\MJysr{\ifmmode \,$MJy\,sr\mo$\else \,MJy\,sr\mo\fi}
\def\MJysrmK{\ifmmode \,$MJy\,sr\mo$\,mK$_{\rm CMB}\mo\else \,MJy\,sr\mo\,mK$_{\rm CMB}\mo$\fi}
\def\microns{\ifmmode \,\mu$m$\else \,$\mu$m\fi}
\def\micron{\microns}
\def\muK{\ifmmode \,\mu$K$\else \,$\mu$\hbox{K}\fi}
\def\microK{\ifmmode \,\mu$K$\else \,$\mu$\hbox{K}\fi}
\def\muW{\ifmmode \,\mu$W$\else \,$\mu$\hbox{W}\fi}
\def\kms{\ifmmode $\,km\,s$^{-1}\else \,km\,s$^{-1}$\fi}
\def\kmsMpc{\ifmmode $\,\kms\,Mpc\mo$\else \,\kms\,Mpc\mo\fi}

\providecommand{\sorthelp}[1]{}


% Custom definitions
\newcommand{\mathsc}[1]{{\normalfont\textsc{#1}}}
\newcommand{\dv}[0]{\vec{d}}
\newcommand{\s}[0]{\vec{s}}
\newcommand{\M}[0]{\tens{M}}
\renewcommand{\P}[0]{\tens{P}}
\newcommand{\G}[0]{\tens{G}}
\newcommand{\B}[0]{\tens{B}}
\renewcommand{\a}[0]{\vec{a}}
\newcommand{\n}[0]{\vec{n}}
\renewcommand{\t}[0]{\vec{t}}
\def\Cosmoglobe{\textsc{Cosmoglobe}}
\def\Planck{\textit{Planck}}
\def\WMAP{\textit{WMAP}}
\def\COBE{\textit{COBE}}
\def\GAIA{\textit{Gaia}}
\def\gaia{\textit{Gaia}}
\def\Gaia{\textit{Gaia}}
\def\WISE{WISE}
\def\AKARI{\textit{{AKARI}}}
\def\IRAS{\textit{{IRAS}}}
\def\nside{$N_{\mathrm{side}}$}
\newcommand{\cii}{\ensuremath{\mathsc{C\ ii}}}
\newcommand{\CII}{\ensuremath{\mathsc{C\ ii}}}
\newcommand{\hi}{\ensuremath{\mathsc{H\ i}}}
\newcommand{\HI}{\ensuremath{\mathsc{H\ i}}}
\def\Commander{\texttt{Commander} }
\def\commanderthree{\texttt{Commander3} }

\def\Tcmb{\ifmmode T_\mathrm{CMB}\else $T_{\mathrm{CMB}}$\fi}
\def\Tcold{\ifmmode T_\mathrm{c}\else $T_{\mathrm{c}}$\fi}
\def\Thot{\ifmmode T_\mathrm{h}\else $T_{\mathrm{h}}$\fi}
\def\Tnear{\ifmmode T_\mathrm{n}\else $T_{\mathrm{n}}$\fi}
\def\Thalpha{\ifmmode T_\mathrm{H\alpha}\else $T_{\mathrm{H\alpha}}$\fi}
\def\scmb{\ifmmode s_\mathrm{CMB}\else $s_{\mathrm{CMB}}$\fi}
\def\squad{\ifmmode s_\mathrm{quad}\else $s_{\mathrm{quad}}$\fi}
\def\ssynch{\ifmmode s_\mathrm{s}\else $s_\mathrm{s}$\fi}
\def\sdust{\ifmmode s_\mathrm{d}\else $s_{\mathrm{d}}$\fi}
\def\ssdust{\ifmmode s_\mathrm{sd}\else $s_{\mathrm{sd}}$\fi}
\def\same{\ifmmode s_\mathrm{AME}\else $s_{\mathrm{AME}}$\fi}
\def\ssrc{\ifmmode s_\mathrm{src}\else $s_{\mathrm{src}}$\fi}
\def\sco{\ifmmode s_\mathrm{CO}\else $s_{\mathrm{CO}}$\fi}
\def\sff{\ifmmode s_\mathrm{ff}\else $s_{\mathrm{ff}}$\fi}
\def\gff{\ifmmode g_\mathrm{ff}\else $g_{\mathrm{ff}}$\fi}
\def\fsynch{\ifmmode f_\mathrm{s}\else $f_{\mathrm{s}}$\fi}
\def\fsd{\ifmmode f_\mathrm{sd}\else $f_{\mathrm{sd}}$\fi}
\def\fame{\ifmmode f_\mathrm{AME}\else $f_{\mathrm{AME}}$\fi}
\def\alphasrc{\ifmmode \alpha_\mathrm{src}\else $\alpha_{\mathrm{src}}$\fi}
\def\bcold{\ifmmode \beta_\mathrm{c}\else $\beta_{\mathrm{c}}$\fi}
\def\bhot{\ifmmode \beta_\mathrm{h}\else $\beta_{\mathrm{h}}$\fi}
\def\bnear{\ifmmode \beta_\mathrm{n}\else $\beta_{\mathrm{n}}$\fi}
\def\bhalpha{\ifmmode \beta_\mathrm{H\alpha}\else $\beta_{\mathrm{H\alpha}}$\fi}
\def\bsynch{\ifmmode \beta_\mathrm{s}\else $\beta_{\mathrm{s}}$\fi} 
\def\bsun{\ifmmode \beta_\mathrm{sun}\else $\beta_{\mathrm{sun}}$\fi} 
\def\nuzeros{\ifmmode \nu_{0,\mathrm{s}}\else $\nu_{0,\mathrm{s}}$\fi} 
\def\nuzeroff{\ifmmode \nu_{0,\mathrm{ff}}\else $\nu_{0,\mathrm{ff}}$\fi} 
\def\nuzerocold{\ifmmode \nu_{0,\mathrm{c}}\else $\nu_{0,\mathrm{c}}$\fi}
\def\nuzerohot{\ifmmode \nu_{0,\mathrm{h}}\else $\nu_{0,\mathrm{h}}$\fi}
\def\nuzeronear{\ifmmode \nu_{0,\mathrm{n}}\else $\nu_{0,\mathrm{n}}$\fi} 
\def\nuzerohalpha{\ifmmode \nu_{0,\mathrm{H\alpha}}\else$\nu_{0,\mathrm{H\alpha}}$\fi} 
\def\nuzeroame{\ifmmode \nu_{0,\mathrm{AME}}\else $\nu_{0,\mathrm{AME}}$\fi} 
\def\nuzerosd{\ifmmode \nu_{0,\mathrm{}}\else $\nu_{0,\mathrm{sd}}$\fi} 
\def\nuzerosrc{\ifmmode \nu_{0,\mathrm{src}}\else $\nu_{0,\mathrm{src}}$\fi} 
\def\nup{\ifmmode \nu_{\mathrm{p}}\else $\nu_{\mathrm{p}}$\fi} 
\def\alphasd{\ifmmode \alpha_{\mathrm{sd}}\else $\alpha_{\mathrm{sd}}$\fi} 
\def\Te{\ifmmode T_{\mathrm{e}}\else $T_{\mathrm{e}}$\fi} 
\def\kB{\ifmmode k_\mathrm{B}\else $k_{\mathrm{B}}$\fi} 

\newcommand{\x}{\checkmark}


\begin{document} 


%\title{\bfseries{\Cosmoglobe\ DR2. VII. Multi-component spectral energy density model of large-scale thermal dust emission from 100\,GHz to 100\,THz}}
\title{\bfseries{\Cosmoglobe\ DR2. VII. Towards a concordance model of large-scale thermal dust emission for microwave and infrared frequencies}}
%\title{\bfseries{\Cosmoglobe\ DR2. VII. The mean thermal dust emission SED as seen by \Planck\ HFI and \COBE-DIRBE}}

   %This author list corresponds to \title{Author list for L04\_CMB\_Foregrounds\_Extraction}
%Prepared by M. Lopez-Caniego (Marcos.Lopez.Caniego@sciops.esa.int), ESAC/ESA
%This version is from Thu Jul 12 18:11:48 2018 CET
%\subtitle{There are 152 co-authors in this list}
\newcommand{\oslo}[0]{1}
%\newcommand{\MIT}[0]{2}
\newcommand{\milanoA}[0]{2}
\newcommand{\milanoB}[0]{3}
\newcommand{\milanoC}[0]{4}
\newcommand{\triesteB}[0]{5}
\newcommand{\planetek}[0]{6}
\newcommand{\princeton}[0]{7}
\newcommand{\jpl}[0]{8}
\newcommand{\helsinkiA}[0]{9}
\newcommand{\helsinkiB}[0]{10}
\newcommand{\nersc}[0]{11}
\newcommand{\haverford}[0]{12}
\newcommand{\mpa}[0]{13}
\newcommand{\triesteA}[0]{14}
\newcommand{\iia}[0]{2}

\author{\small
J.~R.~Eskilt\inst{\oslo}\thanks{Corresponding author: J.~R.~Eskilt; \url{j.r.eskilt@astro.uio.no}}
\and
K.~Lee\inst{\oslo}
\and
D.~J.~Watts\inst{\oslo}
\and
S.~Nerval\inst{\oslo}
\and
et al.
}
\institute{\small
        Institute of Theoretical Astrophysics, University of Oslo, Blindern, Oslo, Norway \goodbreak
}


   %\institute{Institute of Theoretical Astrophysics, University of Oslo, Blindern, Oslo, Norway}
  
   % Shortened title, author list for top of page 
   \titlerunning{Towards a concordance model for thermal dust emission}
   \authorrunning{Gjerløw et al.}

   \date{\today} 
   
   \abstract{
     We fit a four-component thermal dust model to \COBE-DIRBE data between 3.5 and 240\,$\mu$m within the global Bayesian end-to-end \Cosmoglobe\ DR2 reanalysis. Following a companion analysis of \Planck\ HFI, the four components of this model correspond to ``hot dust'', ``cold dust'', ``nearby dust'', and ``H$\alpha$ correlated dust'', respectively, and each component is modelled in terms of a fixed spatial template and a spatially isotropic spectral energy density (SED) defined by an overall free amplitude for each DIRBE channel. The H$\alpha$ dust is an extinction component and, as such, has a negative amplitude. Except for the cold dust amplitude, which is only robustly detected in the 240\,$\mu$m channel, we measure statistically significant template amplitudes for all components in all DIRBE channels between 3.5 and $240\,\mu\mathrm{m}$. However, the two highest frequency channels are too dominated by starlight emission to allow robust dust detections. The total number of DIRBE-specific degrees of freedom in this model is thus only 25. Despite this low dimensionality, the resulting total SED agrees well with recent astrodust predictions, and the overall model efficiency is high, although with a notable wavelength dependency. At both low and high frequencies, more than 95\,\% of the signal root mean squared is captured by the model, while at 60 and $100\,\mu\mathrm{m}$ about 70\,\% of the signal is successfully accounted for. The hot dust component, which was previously found to correlate strongly with \ion{C}{II} emission, has the highest absolute amplitude in all DIRBE frequency channels; at 3.5\,$\mu$m, which is known to be dominated by polycyclic aromatic hydrocarbon emission, this component accounts for at least 80\,\% of the total signal. This analysis represents an important step towards establishing a joint concordance model of thermal dust emission applicable to both the microwave and infrared regimes, and we conclude by outlining a roadmap to a future joint analysis of AKARI, DIRBE, IRAS, and \Planck.
   }
   \keywords{ISM: general - Zodiacal dust, Interplanetary medium - Cosmology: observations, diffuse radiation - Galaxy: general}

   \maketitle

\setcounter{tocdepth}{2}
\tableofcontents
   
% INTRODUCTION
%-------------------------------------------------------------------
\section{Introduction}

In a series of seven companion papers, within which this is the last,
we have reanalyzed the 30-year-old \COBE-DIRBE data using modern
end-to-end Bayesian statistical techniques as implemented in the
\Cosmoglobe\footnote{\url{http://cosmoglobe.uio.no}} framework. From this work, a set of 10
full-sky DIRBE frequency maps emerge, presented in \citet{CG02_01}, covering the infrared frequency range
between 1.25 and 240\,$\mu$m. These maps have both substantially lower
systematic errors and better error characterization compared to their
official counterparts \citep{hauser1998,CG02_01}, and, in particular,
they suffer far less from zodiacal light contamination
\citep{K98,CG02_02}. As a result they can be used for more detailed
astrophysics applications. A few examples of this are presented in the
current analysis, including improved estimates of the cosmic infrared
background spectrum \citep{CG02_03} and large-scale starlight emission
\citet{CG02_04}.

The main topic of the last three papers in the series (\citet{CG02_05, CG02_06};
and the present paper)  is modelling thermal dust radiation efficiently on
large angular scales in the microwave and infrared frequency regimes. This
issue has been the focus of intense scrutiny ever since the groundbreaking IRAS
measurements were published in 1982, and its scientific importance has only
increased through the release of a series of increasingly sensitive data sets,
such as \COBE-DIRBE and \Planck\ HFI. Today, detailed dust modelling plays a
key role in many of the most competitive fields of cosmology, from the search
for inflationary gravitational waves in cosmic microwave background (CMB)
$B$-mode polarization data \citep{bicep2021} to measurements of dark energy
using distant supernovae \citep{popovic:2025}.

As of today, \Planck\ HFI defines the state-of-the-art for full-sky thermal
dust mapping, both in terms of signal-to-noise ratio as well as systematic
control. Based on \Planck's nine frequency channels, the team
produced several exquisite dust models in both intensity and
polarization
\citep{planck2013-XVII,planck2013-p06b,planck2013-XIV,planck2014-XIX,planck2014-a12,planck2014-XXII,planck2016-l11A,planck2016-l11B},
and these now form the basis for much of the dust modelling efforts in
the field \citep[e.g.,][]{pysm2,pysm3}. However, at the same time, the
limited frequency range of \Planck, covering only 30-857\,GHz,
implies that the applicability of these models is currently quite
limited. Furthermore, the absolute calibration of the 857\,GHz \Planck\
channel, which is nominally the most sensitive \Planck\ dust channel,
is uncertain at the $\sim$10\,\% level \citep{planck2016-l03,npipe},
and this induces a large uncertainty on the dust spectral energy density (SED) parameters when
extrapolating to higher frequencies.

In the current paper, we address these issues by fitting the
multi-component dust model proposed by \citet{CG02_05} and
\citet{CG02_06} to the re-processed \COBE-DIRBE data within the
\Cosmoglobe\ DR2 analysis framework. This model consists of four
primary components: 1) cold dust, 2) hot
dust, 3) nearby dust, and 4) H$\alpha$-correlated dust (observed in
extinction). From the previous papers, this model is already known to fit the
\Planck\ HFI frequencies very well when coupled to simple modified
blackbody (MBB) SEDs with spatially
constant spectral parameters, and in this paper we show that the same
spatial morphologies also trace dust in the DIRBE frequencies very
well, although with more complicated SED behaviour. The combined
result is a global model that jointly describes both microwave and
infrared frequencies.

So far, this model has only been developed for and applied to
intensity measurements. However, polarized thermal dust emission also
plays a key role in modern astrophysics and cosmology. For instance,
massive resources are currently being spent on searching for and
constraining the amplitude of inflationary gravitational waves through
deep CMB $B$-mode polarization experiments \citep{litebird2022, SO2019}, and polarized thermal dust
emission represents a key challenge for these. Given the high
efficiency of the \Cosmoglobe\ DR2 four-component dust model for
intensity data, it is reasonable to expect a similar performance for
polarization observations. We therefore conclude this paper by
outlining one potential roadmap towards a future concordance model for
thermal dust emission in both intensity and polarization. 

\FloatBarrier
\begin{figure*}[htb]
  \centering
  \includegraphics[width=0.49\linewidth]{figures/CosmoglobeDR2_VII_545-1_dust_cold_v1.pdf}
  \includegraphics[width=0.49\linewidth]{figures/CosmoglobeDR2_VII_545-1_dust_hot_v1.pdf}\\
  \includegraphics[width=0.49\textwidth]{figures/CosmoglobeDR2_VII_545-1_dust_near_v1.pdf}
  \includegraphics[width=0.49\textwidth]{figures/CosmoglobeDR2_VII_545-1_dust_Ha_v1.pdf}
  \caption{Dust amplitude maps used in the \Cosmoglobe\ DR2 sky model, as evaluated for the \Planck\ HFI 545-1 bolometer channel. From left to right and top to bottom, the four panels show 1) the cold dust amplitude, $\a_{\mathrm{cold}}$; 2) the hot dust amplitude, $\a_{\mathrm{hot}}$; 3) the nearby dust amplitude, $\a_{\mathrm{near}}$; and 4) the (absolute value of the) H$_{\alpha}$-correlated dust extinction amplitude, $\a_{\mathrm{H}\alpha}$. All panels employ the \Planck\ non-linear high dynamic range color scheme, defined by $\log_{10}((\a + \sqrt{4+\a^2})/2)$, which results in a nearly linear behaviour for small values and exponential for large values.  }
  \label{fig:templates}
\end{figure*}


\section{Bayesian modelling of thermal dust emission in \Cosmoglobe\ DR2}
\label{sec:dr2}
This paper, being a part of the second \Cosmoglobe\ Data Release (DR2), is a global, Bayesian analysis mapping out
the joint posterior of all involved parameters, and thus intricately bound up with the
other papers in the release, both in terms of the data sets and data model
used. Hence, we will give here a recap of the data model used, as presented in
\citep{CG02_01}, after which we will devote a subsection to how the thermal
dust part of the global model is described.

\subsection{Data model and posterior distribution}
In the \Cosmoglobe\ framework, the standard modus operandi is to begin with an
explicit parametric model incorporating all known aspects of the dataset with
which we are working -- including both instrumental effects as well as a
physical model of what is being observed. The DIRBE time-ordered
data (TOD) used in this analysis have been parametrized thus:
\begin{align}
	\label{eq:model}
    \dv_{\mathrm{DIRBE}}(\nu)
    &=\G\P\left[\B\sum_{c=1}^{n_{\mathrm{comp}}}\M_c(\nu)\a_c+\s_{\mathrm{zodi}}(\nu) +
          \s_{\mathrm{static}}(\nu)\right] + \n \\
                         & \equiv \s^{\mathrm{tot}} + \n \nonumber,
\end{align}
where $\nu$ is the frequency at which we observe, $\dv$ is the stacked TODs,
$\G$ is an overall gain factor, $\P$ is the pointing matrix which projects the
pixelated sky onto a $n_{\mathrm{tod}}$-sized space, $\B$ is the instrumental
beam convolution operator, and $\n$ is the instrumental noise. The physical sky
is represented by three terms. Firstly, a sum over sky components that can be
modelled as constant at every point in time. We sum HEALPix\footnote{\url{http://healpix.sourceforge.net/}} \citep{healpix,Zonca2019} maps of
amplitudes for each sky component ($\a_c$) multiplied by a mixing matrix $\M_c$ which
extrapolates the given component to the frequencies observed. Secondly, there
is a term representing the zodiacal emission, which cannot be treated as a time
constant. Finally, there is a term representing a component that is static in
solar coordinates, which may either be related to the DIRBE sidelobes, or be
genuine excess radiation originating in the Solar System. These two last terms
are treated in \citet{CG02_02} and \citet{CG02_03}. In the
present work we focus on the first of the three terms, which will be expanded on in what
follows.


%As it stands, this is a ``complete'' model for the frequency domain we are interested in this paper, but we expect that it will have to expand as additional data is added and more discoveries are made.

\subsubsection{Bayesian end-to-end analysis}
In this analysis, we draw samples from the \emph{posterior}
distribution of the full set of parameters of our model
\citep{CG02_01} -- in formulaic terms, we are mapping out $P(\theta | \dv)$,
the probability distribution of the set of parameters $\theta$ given the
observed data $\dv$. Bayes' theorem allows us to write that
\begin{equation}
    \label{eq:bayestheorem}
    P(\theta|\dv) = \frac{P(\dv|\theta)P(\theta)}{P(\dv)},
\end{equation}
and, since $P(\dv)$ typically only enters as a normalizing term, as long as the
parameter space does not change, sampling from $P(\theta|\dv)$ is (modulo a
prior term) is equivalent to sampling from
$P(\dv|\theta)\equiv\mathcal{L}(\theta)$, the so-called \emph{likelihood}
function.

The number of parameters involved in our model (Eq. \ref{eq:model}) is of the
order of millions, making sampling from the likelihood function a non-trivial
task. The \Cosmoglobe\ framework is based on the \Commander software
\citep{eriksen:2004,seljebotn:2019,bp03}, which maps out the posterior
parameter distribution through a process called \emph{Gibbs sampling}
\citep[e.g.,][]{geman:1984}, a Monte-Carlo method based on sequentially
sampling each parameter (or a subset of parameters) from their respective
marginal distributions with respect to all other parameters. The theory of Gibbs sampling
then says that by combining these samples into a full set for all the
parameters involved, this set will represent a proper sample from the joint
distribution. For the Cosmoglobe DR2 analysis, the Gibbs chain looks as
follows:

%\begin{equation}
%\begin{alignat}{11}
%\G &\,\leftarrow P(\G&\,\mid &\,\dv,&\, &\,\phantom{\G} &\,\xi_n, &
%\,\beta_{\mathrm{sky}}& \,\a_{\mathrm{sky}}, &\,\zeta_{\mathrm{z}},
%&\,\a_{\mathrm{static}})\\ \nonumber
%\xi_{\mathrm{n}} &\,\leftarrow P(\xi_{\mathrm{n}}&\,\mid &\,\dv,&\, &\,\G, &\,\phantom{\xi_n} &
%\,\beta_{\mathrm{sky}}& \,\a_{\mathrm{sky}}, &\,\zeta_{\mathrm{z}},
%&\,\a_{\mathrm{static}})\\ \nonumber
%\beta_{\mathrm{sky}} &\,\leftarrow P(\beta_{\mathrm{sky}}&\,\mid &\,\dv,&\, &\,\G, &\,\xi_n, &
%\,\phantom{\beta_{\mathrm{sky}}}& \,\a_{\mathrm{sky}}, &\,\zeta_{\mathrm{z}}, &\,\a_{\mathrm{static}})\\ \nonumber
%\a_{\mathrm{sky}} &\,\leftarrow P(\a_{\mathrm{sky}}&\,\mid &\,\dv,&\, &\,\G, &\,\xi_n, &
%\,\beta_{\mathrm{sky}},& \,\phantom{\a_{\mathrm{sky}},}
%&\,\zeta_{\mathrm{z}}, &\,\a_{\mathrm{static}})\\ \nonumber
%\zeta_{\mathrm{z}} &\,\leftarrow P(\zeta_{\mathrm{z}}&\,\mid &\,\dv,&\, &\,\G, &\,\xi_n, &
%\,\beta_{\mathrm{sky}},& \,\a_{\mathrm{sky}},
%&\,\phantom{\zeta_{\mathrm{z}},} &\,\a_{\mathrm{static}})\\ \nonumber
%\a_{\mathrm{static}} &\,\leftarrow P(\a_{\mathrm{static}}&\,\mid &\,\dv,&\, &\,\G, &\,\xi_n, &
%\,\beta_{\mathrm{sky}},& \,\a_{\mathrm{sky}}, &\,\zeta_{\mathrm{z}} &\,\phantom{\a_{\mathrm{static}}})\label{eq:gibbs_static}.
%\end{alignat}
%\end{equation}

\begin{equation}
    \label{eq:gibbschain}
\begin{aligned}
\G &\,\leftarrow P(\G&\,\mid &\,\dv,&\, &\,\phantom{\G} &\,\xi_n, &
\,\beta_{\mathrm{sky}}& \,\a_{\mathrm{sky}}, &\,\zeta_{\mathrm{z}},
&\,\a_{\mathrm{static}})\\
\xi_{\mathrm{n}} &\,\leftarrow P(\xi_{\mathrm{n}}&\,\mid &\,\dv,&\, &\,\G, &\,\phantom{\xi_n} &
\,\beta_{\mathrm{sky}}& \,\a_{\mathrm{sky}}, &\,\zeta_{\mathrm{z}},
&\,\a_{\mathrm{static}})\\
\beta_{\mathrm{sky}} &\,\leftarrow P(\beta_{\mathrm{sky}}&\,\mid &\,\dv,&\, &\,\G, &\,\xi_n, &
\,\phantom{\beta_{\mathrm{sky}}}& \,\a_{\mathrm{sky}}, &\,\zeta_{\mathrm{z}}, &\,\a_{\mathrm{static}})\\
\a_{\mathrm{sky}} &\,\leftarrow P(\a_{\mathrm{sky}}&\,\mid &\,\dv,&\, &\,\G, &\,\xi_n, &
\,\beta_{\mathrm{sky}},& \,\phantom{\a_{\mathrm{sky}},}
&\,\zeta_{\mathrm{z}}, &\,\a_{\mathrm{static}})\\
\zeta_{\mathrm{z}} &\,\leftarrow P(\zeta_{\mathrm{z}}&\,\mid &\,\dv,&\, &\,\G, &\,\xi_n, &
\,\beta_{\mathrm{sky}},& \,\a_{\mathrm{sky}},
&\,\phantom{\zeta_{\mathrm{z}},} &\,\a_{\mathrm{static}})\\
\a_{\mathrm{static}} &\,\leftarrow P(\a_{\mathrm{static}}&\,\mid &\,\dv,&\, &\,\G, &\,\xi_n, &
\,\beta_{\mathrm{sky}},& \,\a_{\mathrm{sky}}, &\,\zeta_{\mathrm{z}} &\,\phantom{\a_{\mathrm{static}}})
\end{aligned}
\end{equation}
Here, the symbol $\leftarrow$ indicates the operation of drawing a sample from
the distribution on the right-hand side (see \citet{CG02_01} for a complete
overview of all the symbols in this equation) . After some burn-in period, the
resulting joint parameter sets will correspond to samples drawn from the true
underlying joint posterior.

Since every step of the Gibbs sampling process assumes that all other
parameters are ``given'', we can now treat a highly interconnected problem
(i.e. sampling from the joint posterior of all parameters involved in our data
model) as a highly modular one -- meaning that we can perform each
``sub-analysis'' without being concerned with the other parts of the problem.
Hence, in this paper, we mainly focus on the three first components of Eq.
(\ref{eq:skymodel}), leaving the treatment of stars to \citet{CG02_04},
monopoles to \citet{CG02_01}, zodiacal light to \citet{CG02_02}, and condition
on free-free emission as determined by \citet{planck2014-a12}.


\subsection{Multi-component thermal dust modelling}

%\section{Dust modelling}
%\subsection{Current status}
%\label{sec:current_dust}
Interstellar dust -- amorphous particles of silicate and carbonaceous materials
-- makes its presence known on practically all astrophysically relevant
wavelengths. The efforts to classify and describe this material are significant
in their own right, but knowing its properties is critical for better and more
precise astrophysical foreground removal in cases where interstellar dust
emission contaminate the other signals of interest\footnote{For a recent
review, see \citet{Hensley2021}.}.

Recently, the ``astrodust+PAH'' model \citep{Hensley2023} was introduced, wherein
the diffuse interstellar medium (ISM) is hypothesised to be made up of a single
composite material (the eponymous astrodust) for scales larger than
$\sim0.02~\mu$m, and a distinct variety of materials -- including so-called
polycyclic aromatic hydrocarbons (PAH) -- on scales smaller than this.

In the wavelength regime between $3000-100~\mu$m, this model is described well
 by an MBB SED \footnote{The actual
 astrodust model is made up of a composite MBB which has a transition between
353 and 217 GHz}, i.e., an SED that follows
\begin{equation}
s(\nu) \propto \nu^\beta B(\nu, T),
\label{eq:mbb}
\end{equation}
where $\nu$ is frequency, $B$ is the Planck law for a perfect blackbody at a
certain frequency $\nu$ and temperature $T$, and $\beta$ is the spectral index.
Typical temperatures of this blackbody in the diffuse ISM are around $\sim$ 20
K, meaning that the distribution typically peaks around 150\,$\mu$m ($\sim
2000$ GHz).

At lower wavelengths (2.5\,$\mu$m -- 12\,$\mu$m), the astrodust+PAH model is
mostly dominated by the nanoscale particle emission, and exhibits strong
emission lines at various wavelengths (see Fig.~10 in \citealt{Hensley2023}).

\subsubsection{The four-component dust model}
The astrodust model provides a general picture. Typically, in a given line-of-sight, the relative
contribution of various dust components will vary. At the same time, the degree
to which such variations can be detected and described is limited by the
resolution and signal-to-noise ratios of the available data at the wavelengths
involved. Thus, classifying populations of interstellar dust
with common spectral parameters has been of high importance.

It was demonstrated by \citet{CG02_05} that a natural and highly
effective classification of such populations can be achieved through the use of
templates derived from surveys of spectral line emission (\CII, H$\alpha$, CO
and \HI) and from inference of nearby dust structures via starlight extincion
\citep{edenhofer:2024}. In this paper, we showed that using the linear minimization of
five such templates we can explain more than 95\% of the signal variance at dust-dominated
frequencies in \Planck\ (353--857\,GHz) and DIRBE (240--60 $\mu$m), while still accounting for more than 80\% of the signal variance at the
starlight-dominated DIRBE bands (25 and 12 $\mu$m).

\citet{CG02_06} applied the main idea behind that
result to all \Planck\ HFI data using an extensive search of the full parameter space. It was shown
that, by only using the \GAIA\ extinction template, and by assuming three other
components with freely varying amplitudes per pixel, but with fixed global
temperatures and spectral indices per component, we arrived at a dust model
that could explain between 98.5\% and 99.9\% of the non-CMB signal in these
channels. Furthermore, the resulting dust amplitude maps exhibited morphologies that
turned out to be very similar to the templates used in \citet{CG02_05},
giving an independent confirmation of the appropriateness of assuming a
morphological correlation between those astrophysical templates and dust populations.

Encouraged by these results, the main \Cosmoglobe\ DR2 analysis employs a
similar four-component thermal dust model, in which four spatial templates are
used to model the morphology of the thermal dust.

One of these templates, used in both \citet{CG02_05} and \citet{CG02_06}, was
derived from the \citet{edenhofer:2024} extinction maps (see \citealt{CG02_05}
for more details on how this template was constructed), while the three others
were the best-fit solutions obtained in \citet{CG02_06}. We plot these
templates in Fig. \ref{fig:templates}.

The SEDs of these components are not directly modelled as
modified blackbodies, as they were in \citet{CG02_06}. Rather, we define a set
of SED bins, each of which is chosen to correspond to the width of a DIRBE
band, as shown in \cref{tab:bands}. Each dust component is then set to have a
constant amplitude within a given bin, meaning that there is one free parameter
per bin per component, and each amplitude scales one of the four component
templates. These amplitudes are then sampled over in the Gibbs chain.

With this, then, the total sky model (the third term of Eq. \ref{eq:model}) in
the frequency domain of DIRBE can be written,

\begin{equation}
\label{eq:skymodel}
\begin{aligned}
  \sum_{c=1}^{n_{\mathrm{comp}}} \M_c(\nu) \a_c  = 
    &[a]_{\mathrm{cold}}(\nu)\t_{\mathrm{cold}} && \textrm{(Cold dust)} \\
    + & [a]_{\mathrm{hot}}(\nu)\t_{\mathrm{hot}} && \textrm{(Hot dust)}\\
    + & [a]_{\mathrm{nearby}}(\nu)\t_{\mathrm{nearby}} && \textrm{(Nearby dust)} \\
    + & [a]_{\mathrm{H\alpha}}(\nu)\t_{H\alpha} && \textrm{(H$\alpha$ correlated dust)} \\
  + &\left(\frac{\nuzeroff}{\nu}\right)^2
  \frac{g_{\mathrm{ff}}(\nu;\Te) }{g_{\mathrm{ff}}(\nuzeroff;\Te)}
  \vec{t}_{\mathrm{ff}} && \textrm{(Free-free)} \\
  + &U_{\mathrm{mJy}} \sum_{j=1}^{n_{\mathrm{s}}}
  f_{\mathit{Gaia},j} a_{\mathrm{s},j}, &\quad&
  \textrm{(Bright stars)} \\
  + &U_{\mathrm{mJy}} f_{\mathit{Gaia},j} \a_{\mathrm{fs},j}, &\quad&
  \textrm{(Faint stars)} \\  
  + &m_{\nu} && \textrm{(Monopole)}. 
\end{aligned}
\end{equation}
In this equation, the bracketed amplitudes, $[a](\nu)$, indicate amplitudes
that are constant within each frequency range that corresponds to a DIRBE band
(i.e. the abovementioned bins).

The first four terms form the complete model of thermal dust, which is the main
focus of this work. The fifth term models the free-free emission, which is
expected to contribute moderately at all relevant frequencies without becoming
a dominant term. Then, there are two point source terms, all of which are dealt
with in \citet{CG02_04}. Finally, the last term represents the monopole at each
frequency, which is treated in \citet{CG02_03}.


\begin{table*}[htb]
    \centering
    \caption{Components enabled for each frequency band. The dust band widths represent the width of each dust band used in this analysis, not the instrumental bandwidths.}
    \begin{tabular}{c|c|c|c|c|c|c}
        \label{tab:bands}
        Band & \makecell{Dust Band\\Width (GHz)} & Hot dust & Cold dust & Nearby dust & H$\alpha$-correlated dust \\
        \hline
        DIRBE 240 $\mu\mathrm m$ & 617   &\x &\x &\x &\x \\
        DIRBE 140 $\mu\mathrm m$ & 873   &\x & &\x &\x \\
        DIRBE 100 $\mu\mathrm m$ & 1524  &\x & &\x &\x \\
        DIRBE 60  $\mu\mathrm m$ & 4936  &\x & &\x &\x \\
        DIRBE 25  $\mu\mathrm m$ & 9100  &\x & &\x &\x \\
        DIRBE 12  $\mu\mathrm m$ & 27400 &\x & &\x &\x \\
        DIRBE 4.9 $\mu\mathrm m$ & 24715 &\x & & & \\
        DIRBE 3.5 $\mu\mathrm m$ & 39275 &\x & & & \\
    \end{tabular} 
\end{table*}

\section{Data}
\label{sec:data}
There are two datasets used directly in the \Cosmoglobe\ DR2 analysis: low-level data
from \COBE-DIRBE, and starlight data from \WISE\ \citep{wright:2010} and
\GAIA\ \citep{gaia:2016,gaia:2018}. They are supplemented by our dust templates, which are drawn from the analysis performed in \citet{CG02_06}, and incorporate data from \Planck\ HFI and \Gaia. Below we give a succinct description of these
data sets and the preprocessing performed; for a more in-depth description,
see \citet{CG02_01} and \citet{CG02_04}.

\subsection{\COBE-DIRBE}
The Diffuse InfraRed Explorer (DIRBE), whose main goal was the mapping out of the cosmic infrared background
(CIB), was part of the \COBE\ satellite \citep{boggess92, silverberg93}, and
measured the sky in ten frequency bands from $1.25\,\mathrm{\mu m}$ to
$240\,\mathrm{\mu m}$. In this analysis, we have converted the original DIRBE
CIO (Calibrated Individual Observations), whose pointing information is given
in terms of Quadcube pixels with a resolution of 20\arcs\, into HEALPix
\nside=512 pixelation maps, giving an approximate resolution of 42\arcm. Following
the nomenclature of \cite{CG02_01}, we refer to the DIRBE CIOs as
``time-ordered data'' (TOD). Performing our analysis on DIRBE TOD directly
allows us to target the zodiacal light \citep{CG02_02}, which, as mentioned
above, must be treated as a time-dependent sky component, in contrast to other
astrophysical sources.

The use of DIRBE data also allows a fuller exploration of the scales relevant
for interstellar dust modelling; in particular, we are able to capture the peak
of the MBB's and their exponential fall-off at shorter wavelengths. Although
these data introduce additional modeling complexity, namely PAH emission, they can break key degeneracies between $T$ and $\beta$.

\subsection{\WISE\ and \GAIA}
At the higher DIRBE frequency bands (25--1.25 $\mu$m), starlight emission
becomes a significant source of emission, both as point sources and as a
diffuse component. Using the AllWise point source catalog \citep{CatWISE}, we
crossmatched the brighest stars in the \textit W1 band against \Gaia\ DR2, and used the
estimated physical parameters from that catalogue to model the bright stars. In
addition, we created a general faint source template, similar to that used in
\citet{hauser1998}, based on the stars
that were not part of the bright star component. Together, these two components
comprise the sixth and seventh component in \cref{eq:skymodel}.

\subsection{Data selection}
As noted in \citet{CG02_05}, not all thermal dust components are expected to
contribute to the DIRBE bands we are considering in this paper (i.e.
240--12$\mu$m). 
Following the same logic as in that paper, we restrict the dust
components to be active in the various DIRBE bands as indicated in Table
\ref{tab:bands}. In particular, the cold dust component peaks at 1 THz, and
falls off exponentially, making it vastly subdominant to all other components.
In contrast, the hot dust component is visible until the 3.5\,$\mathrm{\mu m}$
channel, and must be included to obtain a satisfactory fit. The nearby and
H$\alpha$-correlated dust components are of intermediate strength. Additionally,
these components are pure templates, and are susceptible to spurious
correlations in the shortest wavelength channels.

\subsection{Masks}
\label{sec:masks}
As discussed in \citet{CG02_01}, analysis masks are generated based on dust
residual amplitudes and zodiacal dust residuals. For the dust components,
regions that are poorly modeled are excluded depending on frequency and residual
amplitude. This is determined using a combination of smoothed residual maps and
$\chi^2$ maps per frequency channel. This ensures that only data that are poorly
modeled are masked, while retaining well-modeled high signal-to-noise pixels.
For the bands where zodiacal dust is brightest
(12--100\,$\mathrm{\mu m}$), we apply an additional mask along the ecliptic
plane. This mask was determined solely by the brightest zodiacal dust channel,
$25\,\mathrm{\mu m}$, and is largely due to asteroidal band uncertainty.

%\FloatBarrier
\section{Results}

The analysis consists of five Gibbs chains, each of which contains approximately
250 samples.  Out of these, the first 100 samples are considered part of the
burn-in period, since all chains have converged by that point. In the following,
we include trace plots with the burnin period included to illustrate the
convergence properties of the chains.

\subsection{Markov chains, correlations and convergence}
In \cref{fig:trace_colddust,fig:trace_hotdust,fig:trace_hacorr_dust,fig:trace_nearbydust}, we plot the Gibbs
chain traceplots for all the five Gibbs chains used in the analysis, for all
four dust components. After a period of reaching equilibrium -- which varies
from almost instantly (for the cold dust amplitude, for example) to around 100
samples at the most (particularly evident for the hot dust and
H$\alpha$-correlated dust 25 and 12 $\mu $m bins) -- the chains generally do not
exhibit any long correlations and seem to mix fairly well. The only exceptions
are the 240 and 120 $\mu$m H$\alpha$-correlated dust bins, where we see more
long-term trends that indicate a slower traversal through parameter space.

\begin{figure}[htb]
  \centering
  \includegraphics[width=\columnwidth]{figures/traceplots_dust_seds_cold_dust.pdf}
  \caption{Cold dust amplitude as a function of iteration for the 240 $\mu$m channel where it is included. The five lines correspond to the five independent sampling chains in the analysis. We see robust mixing in all chains.}
  \label{fig:trace_colddust}
\end{figure}

\begin{figure}[htb]
  \centering
  \includegraphics[width=\columnwidth]{figures/traceplots_dust_seds_hot_dust.pdf}
  \caption{Hot dust amplitudes as a function of iteration for the eight lowest frequency DIRBE channels, with all five sampling chains overplotted. We see that the 3.5 $\mu$m and 4.9 $\mu$m channels exhibit slower mixing than the others, but still manage to explore the full parameter space.}
  \label{fig:trace_hotdust}
\end{figure}
\begin{figure}[htb]
  \centering
  \includegraphics[width=\columnwidth]{figures/traceplots_dust_seds_nearby_dust.pdf}
  \caption{Nearby dust amplitudes as a function of the iteration for the six lowest frequency DIRBE channels, with all five sampling chains overplotted. We see robust mixing in all chains and for all frequency channels.}
  \label{fig:trace_nearbydust}
\end{figure}
\begin{figure}[htb]
  \centering
  \includegraphics[width=\columnwidth]{figures/traceplots_dust_seds_ha-correlated_dust.pdf}
  \caption{H$\alpha$ dust amplitudes as a function of the iteration for the six
	lowest frequency DIRBE channels, with all five sampling chains
	overplotted. The 25 and 12 $\mu$m channels exhibit slower mixing than the others, but still manage to explore the full parameter space.}
  \label{fig:trace_hacorr_dust}
\end{figure}

In Fig. \ref{fig:corrmat}, we show the correlations between the various SED
bins in the run, calculated over all five chains after discarding burn-in. The
strongest correlations are internally between the nearby dust bins. Most of the
bins are correlated with each other, but they are anticorrelated with the 12
$\mu$m amplitude. The correlation of the nearby dust component's amplitude
between different bins is due to indirect interactions of the parameters in the
model, as the morphology of the nearby dust is fixed, and the amplitudes are
sampled separately. Based on our understanding of the model, correlation
matrices, and residuals, we do not identify a physical mechanism for this
correlation. The most reasonable possibility is correlation with zodiacal dust,
although we have not been able to confirm this quantitatively.

Those same bins, and in particular the 100 $\mu$m bin, are also
anticorrelated with the 100 $\mu$m hot dust amplitude. 
This is not unexpected, as there is morphological overlap between the hot dust and nearby dust components, as displayed in Fig.~\ref{fig:templates}. Therefore, a negative correlation would be expected to model the same total amplitude in the DIRBE map itself. The same
phenomenon can be seen between hot and cold dust at 240 $\mu$m.

%\begin{figure}
%  \centering
%  \includegraphics[width=\columnwidth]{figures/traceplots.pdf}
%  \caption{Traceplots.}
%  \label{fig:trace}
%\end{figure}


\begin{figure}[htb]
  \centering
  \includegraphics[width=\columnwidth]{figures/corrmat.pdf}
  \caption{Correlations between the dust amplitudes at each frequency, as
	computed over the full sample set. They are largely
	uncorrelated, but there are some structures within some bands as the
	sampler trades off between components (e.g., 240 $\mu$m in the upper
	left).There is also some fainter structure in the nearby dust amplitudes across channels.}
  \label{fig:corrmat}
\end{figure}

\subsection{Multi-component thermal dust SED posteriors}

In Table \ref{tab:tempamp}, we summarize the thermal dust SED posteriors
resulting from the above analysis as the mean value and variance of the chain
samples (including all five chains of the analysis, discarding burn-in).
Similarly, we plot the posterior mean values per bin in Fig.
\ref{fig:total_sed}, where the thickness of the line indicates the standard
deviation of that amplitude.
In the same plot, we also show the posterior total SED, wherein all four components
are summed up, as well as the Astrodust+PAH model that is the best fit to
the mean total dust SED.\footnote{Specifically,
    we fit an overall amplitude $A$, as well as the
    $\log_{10}{U}$ parameter used in \citet{Hensley2023}, with the total model
to fit being $A\cdot\mathrm{EM}(\log_{10}{U})$, where $EM$ is their tabulated
function returning the emission as a function of $\log_{10}{U}$.}

\begin{table*}[htb]
\newdimen\tblskip \tblskip=5pt
\caption{Summary of dust template amplitude posterior constraints.}
\label{tab:tempamp}
\vskip -4mm
\footnotesize
\setbox\tablebox=\vbox{
 \newdimen\digitwidth
 \setbox0=\hbox{\rm 0}
 \digitwidth=\wd0
 \catcode`*=\active
 \def*{\kern\digitwidth}
%
  \newdimen\dpwidth
  \setbox0=\hbox{.}
  \dpwidth=\wd0
  \catcode`!=\active
  \def!{\kern\dpwidth}
%
  \halign{\hbox to 1.7cm{#\leaderfil}\tabskip 2em&
    \hfil$#$\hfil \tabskip 0.5em&
    \hfil$#$\hfil \tabskip 1em& 
    \hfil$#$\hfil \tabskip 1em& 
    \hfil$#$\hfil \tabskip 1em& 
    \hfil$#$\hfil \tabskip 1em& 
    \hfil$#$\hfil \tabskip 1em\cr
\noalign{\doubleline}
\omit& \multispan2\hfil\sc Band (THz) \hfil&\multispan4\hfil\sc Template amplitude (unit) \hfil\cr
\noalign{\vskip -3pt}
\omit& \multispan2\hrulefill& \multispan4\hrulefill\cr
\noalign{\vskip 3pt} 
\omit\sc $\lambda$ ($\mu\mathrm{m}$)\hfil& \nu_{\mathrm{min}} & \nu_{\mathrm{max}} & \a_{\mathrm{h}} & \a_{\mathrm{c}} & \a_{\mathrm{n}} & \a_{\mathrm{H}\alpha}\cr
\noalign{\vskip 3pt\hrule\vskip 5pt}
*3.5  & 70.2 & 109.5 & 0.05\pm0.00 & 0.00\pm0.01 & 0.01\pm0.00 & 0.00\pm0.01 \cr
*4.9  & 45.5 & *70.2 & 0.01\pm0.00 & 0.00\pm0.01 & 0.01\pm0.00 & 0.00\pm0.01 \cr
*12   & 18.1 & *45.5 & 0.30\pm0.00 & 0.00\pm0.01 & 0.19\pm0.00 & -0.32\pm0.01 \cr
*25   & *9.0 & *18.1 & 0.22\pm0.00 & 0.00\pm0.01 & 0.13\pm0.00 & -0.28\pm0.01 \cr
*60   & *4.1 & **9.0 & 2.45\pm0.00 & 0.00\pm0.01 & 0.31\pm0.00 & -2.41\pm0.03 \cr
100   & *2.5 & **4.1 & 10.92\pm0.03 & 0.00\pm0.01 & 2.61\pm0.00 & -12.40\pm0.02 \cr
140   & *1.7 & **2.5 & 24.58\pm0.01 & 0.00\pm0.01 & 10.04\pm0.01 & -23.40\pm0.08 \cr
240   & *1.1 & **1.7 & 11.33\pm0.01 & 2.60\pm0.01 & 8.09\pm0.01 & -8.94\pm0.04 \cr
\noalign{\vskip 5pt\hrule\vskip 5pt}}}
\endPlancktablewide
\par
\end{table*}

\begin{figure}[htb]
  \centering
  \includegraphics[width=\columnwidth]{figures/all_components_sed.pdf}
  \caption{The total dust SED as a function of frequency, as well as the four
      dust components. The best-fit astrodust model fit to the total SED for
  the frequencies shown is also plotted in brown.}
  \label{fig:total_sed}
\end{figure}

Both the hot dust and nearby dust components follow a typical thermal
dust modified blackbody curve, mirroring the results found in \citet{CG02_05}.
At higher frequencies, they exhibit the characteristic rise in
the SED, where models like Astrodust+PAH predicts emission lines from the
nanoscale particles. The figure also shows that fitting the total (i.e., the
sum of all the components) dust model to the Astrodust+PAH model yields a
curve that fits very well with the posterior mean SED
values.

We cannot draw any such conclusions about the the cold dust component, as it is
only active for the lowest SED bin. The amplitude of the cold dust being of
order ten times smaller than the hot and nearby dust at 1 THz is consistent with
the results of \citet{CG02_05}. In principle, the component could be fit as
well, but in practice the bin values are consistent with zero, and increase the
uncertainty and degeneracies of other higher signal-to-noise components. Future
work, in conjunction with higher sensitivity datasets, such as \IRAS\ and AKARI,
will be useful in further determing the SED of this component.

The H$\alpha$ component, as in the previous analyses, exhibits a negative SED,
and is thus responsible for dust extinction rather than emission, but it
exhibits the same blackbody spectrum with a hint of a rise towards the PAH
emission region. Again, this is consistent with the results in \citet{CG02_05}.
In addition, the existence of a rise towards shorter wavelengths is consistent
with the Astrodust+PAH model.

%\begin{figure}
%  \centering
%  \includegraphics[width=\columnwidth]{figures/cold_dust_sed.pdf}
%  \caption{Cold dust SED.}
%  \label{fig:cold_dust_sed}
%\end{figure}
%\begin{figure}
%  \centering
%  \includegraphics[width=\columnwidth]{figures/hot_dust_sed.pdf}
%  \caption{Hot dust SED.}
%  \label{fig:hot_dust_sed}
%\end{figure}
%\begin{figure}
%  \centering
%  \includegraphics[width=\columnwidth]{figures/nearby_dust_sed.pdf}
%  \caption{Nearby dust SED.}
%  \label{fig:nearby_dust_sed}
%\end{figure}
%
%\begin{figure}
%  \centering
%  \includegraphics[width=\columnwidth]{figures/nearby_dust_sed.pdf}
%  \caption{H$\alpha$ SED.}
%  \label{fig:wham_sed}
%\end{figure}

\subsection{Model efficiency and residuals}

In \cref{fig:dustmaps}, we show the dust frequency maps -- that is, the
frequency maps with starlight and free-free signal subtracted -- as well as the
residual maps after subtracting the four-component dust model used in this
analysis. The grey pixels indicate the masks used at each frequency, as
described in \cref{sec:masks}. In all the bands to some degree, we see in the
residuals a weak pattern of over-and-undersubtractions in the Galactic plane,
suggesting that there is room for further sophistication of the template model.
For example, the Galactic plane is exactly the region where we would expect true
spatial variations within a single dust component, which we have not allowed for
in this work. To quantify this, we estimate the reduction of the RMS of the
residuals with respect to the original map in \cref{fig:efficiency}.

The 240--60 $\,\mathrm{\mu m}$ channels have $\gtrsim90\,\%$ of their emission
explained by the four component model. The first two channels are
the cleanest, in part due to their relatively high noise. 
Between 140 and 60\,$\mathrm{\mu m}$, the model is clearly struggling to account for
emission from the Galactic center, with similar excesses and oversubtractions,
characteristic of temperature variations. This indicates a consistent mismodeled
component affecting the fit throughout this region. One possible contaminant not
explicitly modeled in this work is the $158\,\mathrm{\mu m}$ \ion{C}{ii} cooling
line as measured in, e.g., \citet{1999ApJ...526..207F}. This emission, brightest
in the Galactic plane, could lead to degeneracies with continuum emission
sources, biasing results. A full joint analysis with FIRAS data will be
necessary to properly account for this emission.

The 25 and 12 $\mathrm{\mu m}$ channels have only $\sim$60\,\% of their emission
explained by the dust model. This can partially be explained by incomplete
zodiacal dust subtraction, as evidenced by negative regions in both the signal
and residual maps. Much of the residual structure is uncorrelated with the
Galactic plane, indicating that dust modeling is not the primary cause of the
excess residuals.

At $4.9$ and $3.5\,\mathrm{\mu m}$, most of the residuals show clear imprints of
starlight emission. Here, we are limited by the variation of the starlight SEDs,
detailed knowledge of the starlight SEDs, incomplete extinction corrections,
and the resolution of DIRBE \citep{CG02_04}. Because this work is focused on
thermal dust, we leave these residuals for future work. In particular, the
SPHEREx satellite will be able to resolve and directly measure the SED of most
of the infreared-bright point sources close to the Galactic plane.

\begin{figure*}[htb]
  \centering
  \includegraphics[width=0.235\linewidth]{figures/paperVII_10_dustres_v1.pdf}
  \includegraphics[width=0.235\linewidth]{figures/paperVII_10_todres_v1.pdf}\hspace*{5mm}
  \includegraphics[width=0.235\linewidth]{figures/paperVII_06_dustres_v1.pdf}
  \includegraphics[width=0.235\linewidth]{figures/paperVII_06_todres_v1.pdf}\\
  \includegraphics[width=0.235\linewidth]{figures/paperVII_09_dustres_v1.pdf}
  \includegraphics[width=0.235\linewidth]{figures/paperVII_09_todres_v1.pdf}\hspace*{5mm}
  \includegraphics[width=0.235\linewidth]{figures/paperVII_05_dustres_v1.pdf}
  \includegraphics[width=0.235\linewidth]{figures/paperVII_05_todres_v1.pdf}\\
  \includegraphics[width=0.235\linewidth]{figures/paperVII_08_dustres_v1.pdf}
  \includegraphics[width=0.235\linewidth]{figures/paperVII_08_todres_v1.pdf}\hspace*{5mm}
  \includegraphics[width=0.235\linewidth]{figures/paperVII_04_dustres_v1.pdf}
  \includegraphics[width=0.235\linewidth]{figures/paperVII_04_todres_v1.pdf}\\
  \includegraphics[width=0.235\linewidth]{figures/paperVII_07_dustres_v1.pdf}
  \includegraphics[width=0.235\linewidth]{figures/paperVII_07_todres_v1.pdf}\hspace*{5mm}
  \includegraphics[width=0.235\linewidth]{figures/paperVII_03_dustres_v1.pdf}
  \includegraphics[width=0.235\linewidth]{figures/paperVII_03_todres_v1.pdf}
  \caption{Comparison between thermal dust frequency maps (i.e., stationary sky signal minus starlight and free-free; \emph{first and third columns}) and residual maps for each frequency channel (\emph{second and fourth column}). Gray pixels indicate the analysis masks used for each frequency channel.}
  \label{fig:dustmaps}
\end{figure*}

\begin{figure}[htb]
  \centering
  \includegraphics[width=\columnwidth]{figures/template_signal_rms_v1.pdf}
  \caption{Dust model efficiency as a function of frequency, as defined in terms of variance reduction. The red line shows results for DIRBE, as evaluated from the maps shown in Fig.~\ref{fig:dustmaps}, while the blue line shows results for \Planck\ HFI, as presented by \citet{CG02_06}. Vertical bands indicate the position and bandwidth of each DIRBE (orange) and HFI (cyan) frequency channel.}
  \label{fig:efficiency}
\end{figure}

\begin{figure}[htb]
  \centering
  \includegraphics[width=\linewidth]{figures/paperVII_07_todres_zoom_v1.pdf}\\
  \includegraphics[width=\linewidth]{figures/paperVII_08_todres_zoom_v1.pdf}
	\caption{Full-sky data-minus-model residual maps for the 60 (\emph{top}) and 100\,$\mu$m (\emph{bottom}) DIRBE channels. }
  \label{fig:res0708}
\end{figure}

%\FloatBarrier
\section{\Cosmoglobe\ DR2 sky model}

Up until now, we have focused on the infrared regime of sky modeling, largely
decoupled from the microwave region where \Cosmoglobe\ was primarily developed.
Here, we unite the microwave and infrared regimes to present an integrated sky
model that will be the foundation of future work.

\subsection{Summary of global sky model}

In Fig. \ref{fig:SED_overview} we show the best-fit \Cosmoglobe\ sky model,
derived from the first and second data releases,
including the four-component dust model presented in this paper. Although the
amplitudes chosen are relatively arbitrary (every line of sight in the sky will
potentially have different scalings), the spectral shape of the lines are
directly taken from the Gibbs chains as presented in \citet{watts2023_dr1} and
\citet{CG02_01}. This plot represents the current state of the Cosmoglobe
effort to create a common sky model derived with the same data pipeline.


\begin{figure*}[htb]
  \centering
  \includegraphics[width=\textwidth]{figures/new_all_fgs_bands.png}
	\caption{Overview of the large-scale microwave and infrared sky from 1 GHz to 1$\mu$m, based on the \Cosmoglobe\ DR1 and DR2 data. The four component dust model is shown, as well as the best fit astrodust model. The \Planck\ HFI and DIRBE central frequencies are indicated with  vertical lines.}
  \label{fig:SED_overview}
\end{figure*}


%Thus, we can robustly conclude that both for the \Planck\ HFI and
%DIRBE regimes, dust is ``naturally'' decomposable into a relatively low
%number of constituent components, and that those components tend to correlate
%with traces of various kinds of activity in the Milky Way.

Finally, in \cref{fig:comp_vs_freq}, we show the amplitude maps of all the
components involved in the \Cosmoglobe\ DR2 analysis, including zodiacal light,
dust, starlight, and free-free emission. These maps represent the Cosmoglobe
sky model as applied to the DIRBE bands, and in the final column we see the
residual maps at each band. At the two lowest bands, where the thermal dust is
the dominant sky component (see Fig. \ref{fig:SED_overview}), the model is
performing quite well (with the same caveats as for the pure dust residual
maps), especially given that we essentially treat the thermal dust as a model
with $\sim$20 free parameters, instead of fitting two spectral parameters per
pixel on the sky.

At higher frequencies, the model struggles more (as also seen in Figs.
\ref{fig:dustmaps} and \ref{fig:efficiency}). There is clearly residual
zodiacal light at the intermediate frequencies, whereas at the highest
frequencies we see there is residual starlight in the maps (evident also in
Fig. \ref{fig:efficiency}), where band 2, which is the one where the starlight
SED peaks, is where the sky model currently performs the worst.
       \begin{figure*}[htb]
         \centering
         \includegraphics[width=0.16\linewidth]{figures/compfreq_mapzodi_10_v02.pdf}
         \includegraphics[width=0.16\linewidth]{figures/compfreq_zodi_10_v02.pdf}
         \includegraphics[width=0.16\linewidth]{figures/compfreq_dusttot_10_v02.pdf}
         %\includegraphics[width=0.16\linewidth]{figures/compfreq_white_nobar.pdf}
         \includegraphics[width=0.16\linewidth]{figures/stars_label.pdf}
         \includegraphics[width=0.16\linewidth]{figures/compfreq_ff_10_v02.pdf}
         \includegraphics[width=0.16\linewidth]{figures/compfreq_todres_10_v02.pdf}\\
         \includegraphics[width=0.16\linewidth]{figures/compfreq_mapzodi_09_v02.pdf}
         \includegraphics[width=0.16\linewidth]{figures/compfreq_zodi_09_v02.pdf}
         \includegraphics[width=0.16\linewidth]{figures/compfreq_dusttot_09_v02.pdf}
         \includegraphics[width=0.16\linewidth]{figures/compfreq_white_nobar.pdf}
         \includegraphics[width=0.16\linewidth]{figures/compfreq_ff_09_v02.pdf}
         \includegraphics[width=0.16\linewidth]{figures/compfreq_todres_09_v02.pdf}\\
         \includegraphics[width=0.16\linewidth]{figures/compfreq_mapzodi_08_v02.pdf}
         \includegraphics[width=0.16\linewidth]{figures/compfreq_zodi_08_v02.pdf}
         \includegraphics[width=0.16\linewidth]{figures/compfreq_dusttot_08_v02.pdf}
         \includegraphics[width=0.16\linewidth]{figures/compfreq_white_nobar.pdf}
         \includegraphics[width=0.16\linewidth]{figures/compfreq_ff_08_v02.pdf}
         \includegraphics[width=0.16\linewidth]{figures/compfreq_todres_08_v02.pdf}\\
         \includegraphics[width=0.16\linewidth]{figures/compfreq_mapzodi_07_v02.pdf}
         \includegraphics[width=0.16\linewidth]{figures/compfreq_zodi_07_v02.pdf}
         \includegraphics[width=0.16\linewidth]{figures/compfreq_dusttot_07_v02.pdf}
         \includegraphics[width=0.16\linewidth]{figures/compfreq_white_nobar.pdf}
         \includegraphics[width=0.16\linewidth]{figures/compfreq_ff_07_v02.pdf}
         \includegraphics[width=0.16\linewidth]{figures/compfreq_todres_07_v02.pdf}\\
         \includegraphics[width=0.16\linewidth]{figures/compfreq_mapzodi_06_v02.pdf}
         \includegraphics[width=0.16\linewidth]{figures/compfreq_zodi_06_v02.pdf}
         \includegraphics[width=0.16\linewidth]{figures/compfreq_dusttot_06_v02.pdf}
         \includegraphics[width=0.16\linewidth]{figures/compfreq_stars_06_v02.pdf}
         \includegraphics[width=0.16\linewidth]{figures/compfreq_ff_06_v02.pdf}
         \includegraphics[width=0.16\linewidth]{figures/compfreq_todres_06_v02.pdf}\\  
         \includegraphics[width=0.16\linewidth]{figures/compfreq_mapzodi_05_v02.pdf}
         \includegraphics[width=0.16\linewidth]{figures/compfreq_zodi_05_v02.pdf}
         \includegraphics[width=0.16\linewidth]{figures/compfreq_dusttot_05_v02.pdf}
         \includegraphics[width=0.16\linewidth]{figures/compfreq_stars_05_v02.pdf}
         \includegraphics[width=0.16\linewidth]{figures/compfreq_ff_05_v02.pdf}
         \includegraphics[width=0.16\linewidth]{figures/compfreq_todres_05_v02.pdf}\\
         \includegraphics[width=0.16\linewidth]{figures/compfreq_mapzodi_04_v02.pdf}
         \includegraphics[width=0.16\linewidth]{figures/compfreq_zodi_04_v02.pdf}
         \includegraphics[width=0.16\linewidth]{figures/compfreq_dusttot_04_v02.pdf}
         \includegraphics[width=0.16\linewidth]{figures/compfreq_stars_04_v02.pdf}
         \includegraphics[width=0.16\linewidth]{figures/compfreq_ff_04_v02.pdf}
         \includegraphics[width=0.16\linewidth]{figures/compfreq_todres_04_v02.pdf}\\  
         \includegraphics[width=0.16\linewidth]{figures/compfreq_mapzodi_03_v02.pdf}
         \includegraphics[width=0.16\linewidth]{figures/compfreq_zodi_03_v02.pdf}
         \includegraphics[width=0.16\linewidth]{figures/compfreq_dusttot_03_v02.pdf}
         \includegraphics[width=0.16\linewidth]{figures/compfreq_stars_03_v02.pdf}
         \includegraphics[width=0.16\linewidth]{figures/compfreq_ff_03_v02.pdf}
         \includegraphics[width=0.16\linewidth]{figures/compfreq_todres_03_v02.pdf}\\  
         \includegraphics[width=0.16\linewidth]{figures/compfreq_mapzodi_02_v02.pdf}
         \includegraphics[width=0.16\linewidth]{figures/compfreq_zodi_02_v02.pdf}
         \includegraphics[width=0.16\linewidth]{figures/compfreq_white_nobar.pdf}
         \includegraphics[width=0.16\linewidth]{figures/compfreq_stars_02_v02.pdf}
         \includegraphics[width=0.16\linewidth]{figures/compfreq_ff_02_v02.pdf}
         \includegraphics[width=0.16\linewidth]{figures/compfreq_todres_02_v02.pdf}\\
         \includegraphics[width=0.16\linewidth]{figures/compfreq_mapzodi_01_v02.pdf}
         \includegraphics[width=0.16\linewidth]{figures/compfreq_zodi_01_v02.pdf}
         \includegraphics[width=0.16\linewidth]{figures/compfreq_white_nobar.pdf}
         \includegraphics[width=0.16\linewidth]{figures/compfreq_stars_01_v02.pdf}
         \includegraphics[width=0.16\linewidth]{figures/compfreq_ff_01_v02.pdf}
         \includegraphics[width=0.16\linewidth]{figures/compfreq_todres_01_v02.pdf}\\
         \includegraphics[width=0.50\linewidth]{figures/colourbar_MJysr.pdf}
         \caption{Comparison between the re-analyzed DIRBE data and the various fitted components for one single Gibbs sample. Columns show, from left to right, 1) the time-ordered DIRBE data co-added into pixelized maps; 2) zodiacal light emission; 3) thermal dust emission; 4) star emission; 5) free-free emission; and 6) data-minus-model residual emission. Rows show individual frequency channels. Missing entries corresponds to components that are forced to zero in the model. Note that all panels are plotted with the same color scale in units of MJy/sr, and can be directly compared. \textbf{MG: stars needs a label}}
         \label{fig:comp_vs_freq}
       \end{figure*}

\FloatBarrier
\section{Conclusions}
In this analysis, which is part of the second \Cosmoglobe\ data release, we
have described the incorporation of a four-component model, presented in
\citet{CG02_06}, as part of the modelling of thermal dust at the eight lowest
DIRBE frequency bands. Recent analyses \citep{CG02_05, CG02_06} demonstrated
the feasibility and efficiency of decomposing thermal dust into a relatively
low number of distinct populations with global spectral parameters.

The model uses a nearby dust template presented in \citet{CG02_05}, labeled
``nearby dust'', as well as the three template components derived in
\citet{CG02_06}, labeled ``Hot dust'', ``Cold dust'', and
``H$\alpha$-correlated dust'', respectively. The SED of each component was
defined as a constant bin over each DIRBE band.

With this model as part of the larger DR2 analysis, we showed that the
resulting total SED is an excellent match to the Astrodust+PAH model, and that
the nanoparticle emission at high frequencies is detected in these components.
As in previous analyses, the H$\alpha$-correlated component absorbs rather than
emits radiation, and thus has a negative SED.

Finally, we present the current state of the \Cosmoglobe\ sky model, ranging
from a few GHz to ~$10^3$ THz, and containing diverse components, such as the
CMB, synchrotron radiation, zodiacal light, starlight, the various thermal
dust components, and the CIB.

Although the results in this and the preceding analyses are very encouraging
for the prospect of efficient thermal dust modelling, it is clearly also
something that is in need of further development....... Also, this analysis
only utilizes one data set in addition to the starlight data. An obvious
extension would be to perform a joint analysis ranging across the entirety of
the thermal dust spectrum, with higher-resolution data both in the spatial
domain and in the frequency domain.

A promising combination of data, which is definitely within the realm of
possibility in a joint analysis framework such as \Cosmoglobe, could be as
follows: 1) \Planck\ low-level data for the low end of the thermal dust spectrum,
and correlation with various types of line emission; 2) DIRBE, as in the
present analysis, to constrain the upper part of the thermal dust spectrum,
including the PAH features; 3) AKARI \citep{murakami:2007}, observing in the
same range as DIRBE, but with higher resolution, thus shedding light on an
otherwise data-poor part of the electromagnetic spectrum; and 4) \IRAS, to provide
....



\begin{acknowledgements}
 The current work has received funding from the European
  Union’s Horizon 2020 research and innovation programme under grant
  agreement numbers 819478 (ERC; \textsc{Cosmoglobe}) and 772253 (ERC;
  \textsc{bits2cosmology}). Some of the results in this paper have been derived using the HEALPix \citep{healpix} package.
  We acknowledge the use of the Legacy Archive for Microwave Background Data
  Analysis (LAMBDA), part of the High Energy Astrophysics Science Archive Center
  (HEASARC). HEASARC/LAMBDA is a service of the Astrophysics Science Division at
  the NASA Goddard Space Flight Center.  
  This paper and related research have been conducted during and with the support of the Italian national inter-university PhD programme in Space Science and Technology. Work on this article was produced while attending the PhD program in PhD in Space Science and Technology at the University of Trento, Cycle XXXIX, with the support of a scholarship financed by the Ministerial Decree no. 118 of 2nd March 2023, based on the NRRP - funded by the European Union - NextGenerationEU - Mission 4 "Education and Research", Component 1 "Enhancement of the offer of educational services: from nurseries to universities” - Investment 4.1 “Extension of the number of research doctorates and innovative doctorates for public administration and cultural heritage” - CUP E66E23000110001.
\end{acknowledgements}


%-------------------------------------------------------------
%                                       Table with references 
%-------------------------------------------------------------
%

\bibliographystyle{aa}
\bibliography{../../common/CG_bibliography,references,../../common/Planck_bib}
\end{document}
%%%% End of aa.dem




\begin{table*}[htb]
    \centering
    \caption{Defintion of the SED bins used in this paper}
    \begin{tabular}{c|c|c|c}
        \label{tab:bins}
        Bin number & Frequency range (GHz) & Wavelength range ($\mu$m) & DIRBE band correspondence \\
        \hline
        1 & 1050-1667 & 286.5-179.8 & 10 \\
        2 & 1667-2540 & 179.8-118.0 & 9 \\
        3 & 2540-4064 & 118.0-73.8 & 8 \\
        4 & 4064-9000 & 73.8-33.3 & 7 \\
        5 & 9000-18100 & 33.3-16.6 & 6 \\
        6 & 18100-45500 & 16.6-6.6 & 5  \\
        7 & 45500-70215 & 6.6-4.3 & 4 \\
        8 & 70215-109490 & 4.3-2.7 & 3
    \end{tabular}
\end{table*}
