%                                                                 aa.dem
% AA vers. 9.1, LaTeX class for Astronomy & Astrophysics
% demonstration file
%                                                       (c) EDP Sciences
%-----------------------------------------------------------------------
%
% \documentclass[referee]{aa} % for a referee version
%\documentclass[onecolumn]{aa} % for a paper on 1 column  
%\documentclass[longauth]{aa} % for the long lists of affiliations 
%\documentclass[letter]{aa} % for the letters 
%\documentclass[bibyear]{aa} % if the references are not structured 
%                              according to the author-year natbib style

%

\documentclass{aa}  

%
\usepackage{graphicx}
\usepackage{amsmath,amsfonts,amssymb}
\usepackage{natbib}


%%%%%%%%%%%%%%%%%%%%%%%%%%%%%%%%%%%%%%%%
\usepackage{txfonts}
\usepackage{xcolor}

\usepackage{blindtext}
%%%%%%%%%%%%%%%%%%%%%%%%%%%%%%%%%%%%%%%%
% \usepackage[options]{hyperref}
% To add links in your PDF file, use the package "hyperref"
% with options according to your LaTeX or PDFLaTeX drivers.
\usepackage{float}
%\usepackage{stfloats}
\usepackage{dblfloatfix}
\usepackage{afterpage}
\usepackage{ifthen}
\usepackage[morefloats=12]{morefloats}

\usepackage{placeins}
\usepackage{multicol}
%\usepackage[breaklinks,colorlinks,citecolor=blue]{hyperref}
\bibpunct{(}{)}{;}{a}{}{,}
\usepackage[switch]{lineno}
\definecolor{linkcolor}{rgb}{0.6,0,0}
\definecolor{citecolor}{rgb}{0,0,0.75}
\definecolor{urlcolor}{rgb}{0.12,0.46,0.7}
\usepackage[breaklinks, colorlinks, urlcolor=urlcolor,
    linkcolor=linkcolor,citecolor=citecolor,pdfencoding=auto]{hyperref}
\hypersetup{linktocpage}
\usepackage{bold-extra}



\def\setsymbol#1#2{\expandafter\def\csname #1\endcsname{#2}}
\def\getsymbol#1{\csname #1\endcsname}

\def\Planck{\textit{Planck}}

\def\HeJT{$^4$He-JT}

\def\allearlypapers{\nocite{planck2011-1.1, planck2011-1.3, planck2011-1.4, planck2011-1.5, planck2011-1.6, planck2011-1.7, planck2011-1.10, planck2011-1.10sup, planck2011-5.1a, planck2011-5.1b, planck2011-5.2a, planck2011-5.2b, planck2011-5.2c, planck2011-6.1, planck2011-6.2, planck2011-6.3a, planck2011-6.4a, planck2011-6.4b, planck2011-6.6, planck2011-7.0, planck2011-7.2, planck2011-7.3, planck2011-7.7a, planck2011-7.7b, planck2011-7.12, planck2011-7.13}}

\def\alltwentythirteenresultspapers{\nocite{planck2013-p01, planck2013-p02, planck2013-p02a, planck2013-p02d, planck2013-p02b, planck2013-p03, planck2013-p03c, planck2013-p03f, planck2013-p03d, planck2013-p03e, planck2013-p01a, planck2013-p06, planck2013-p03a, planck2013-pip88, planck2013-p08, planck2013-p11, planck2013-p12, planck2013-p13, planck2013-p14, planck2013-p15, planck2013-p05b, planck2013-p17, planck2013-p09, planck2013-p09a, planck2013-p20, planck2013-p19, planck2013-pipaberration, planck2013-p05, planck2013-p05a, planck2013-pip56, planck2013-p06b, planck2013-p01a}}

\def\alltwentyfifteenresultspapers{\nocite{planck2014-a01, planck2014-a03, planck2014-a04, planck2014-a05, planck2014-a06, planck2014-a07, planck2014-a08, planck2014-a09, planck2014-a11, planck2014-a12, planck2014-a13, planck2014-a14, planck2014-a15, planck2014-a16, planck2014-a17, planck2014-a18, planck2014-a19, planck2014-a20, planck2014-a22, planck2014-a24, planck2014-a26, planck2014-a28, planck2014-a29, planck2014-a30, planck2014-a31, planck2014-a35, planck2014-a36, planck2014-a37, planck2014-ES}}

\newbox\tablebox    \newdimen\tablewidth
\def\leaderfil{\leaders\hbox to 5pt{\hss.\hss}\hfil}
\def\endPlancktable{\tablewidth=\columnwidth 
    $$\hss\copy\tablebox\hss$$
    \vskip-\lastskip\vskip -2pt}
\def\endPlancktablewide{\tablewidth=\textwidth 
    $$\hss\copy\tablebox\hss$$
    \vskip-\lastskip\vskip -2pt}
\def\tablenote#1 #2\par{\begingroup \parindent=0.8em
    \abovedisplayshortskip=0pt\belowdisplayshortskip=0pt
    \noindent
    $$\hss\vbox{\hsize\tablewidth \hangindent=\parindent \hangafter=1 \noindent
    \hbox to \parindent{$^#1$\hss}\strut#2\strut\par}\hss$$
    \endgroup}
\def\doubleline{\vskip 3pt\hrule \vskip 1.5pt \hrule \vskip 5pt}

\def\L2{\ifmmode L_2\else $L_2$\fi}
\def\dtt{\Delta T/T}
\def\DeltaT{\ifmmode \Delta T\else $\Delta T$\fi}
\def\deltat{\ifmmode \Delta t\else $\Delta t$\fi}
\def\fknee{\ifmmode f_{\rm knee}\else $f_{\rm knee}$\fi}
\def\Fmax{\ifmmode F_{\rm max}\else $F_{\rm max}$\fi}
\def\solar{\ifmmode{\rm M}_{\mathord\odot}\else${\rm M}_{\mathord\odot}$\fi}
\def\Msolar{\ifmmode{\rm M}_{\mathord\odot}\else${\rm M}_{\mathord\odot}$\fi}
\def\Lsolar{\ifmmode{\rm L}_{\mathord\odot}\else${\rm L}_{\mathord\odot}$\fi}
\def\inv{\ifmmode^{-1}\else$^{-1}$\fi}
\def\mo{\ifmmode^{-1}\else$^{-1}$\fi}
\def\sup#1{\ifmmode ^{\rm #1}\else $^{\rm #1}$\fi}
\def\expo#1{\ifmmode \times 10^{#1}\else $\times 10^{#1}$\fi}
\def\,{\thinspace}
\def\lsim{\mathrel{\raise .4ex\hbox{\rlap{$<$}\lower 1.2ex\hbox{$\sim$}}}}
\def\gsim{\mathrel{\raise .4ex\hbox{\rlap{$>$}\lower 1.2ex\hbox{$\sim$}}}}
\let\lea=\lsim
\let\gea=\gsim
\def\simprop{\mathrel{\raise .4ex\hbox{\rlap{$\propto$}\lower 1.2ex\hbox{$\sim$}}}}
\def\deg{\ifmmode^\circ\else$^\circ$\fi}
\def\pdeg{\ifmmode $\setbox0=\hbox{$^{\circ}$}\rlap{\hskip.11\wd0 .}$^{\circ}
          \else \setbox0=\hbox{$^{\circ}$}\rlap{\hskip.11\wd0 .}$^{\circ}$\fi}
\def\arcs{\ifmmode {^{\scriptstyle\prime\prime}}
          \else $^{\scriptstyle\prime\prime}$\fi}
\def\arcm{\ifmmode {^{\scriptstyle\prime}}
          \else $^{\scriptstyle\prime}$\fi}
\newdimen\sa  \newdimen\sb
\def\parcs{\sa=.07em \sb=.03em
     \ifmmode \hbox{\rlap{.}}^{\scriptstyle\prime\kern -\sb\prime}\hbox{\kern -\sa}
     \else \rlap{.}$^{\scriptstyle\prime\kern -\sb\prime}$\kern -\sa\fi}
\def\parcm{\sa=.08em \sb=.03em
     \ifmmode \hbox{\rlap{.}\kern\sa}^{\scriptstyle\prime}\hbox{\kern-\sb}
     \else \rlap{.}\kern\sa$^{\scriptstyle\prime}$\kern-\sb\fi}
\def\ra[#1 #2 #3.#4]{#1\sup{h}#2\sup{m}#3\sup{s}\llap.#4}
\def\dec[#1 #2 #3.#4]{#1\deg#2\arcm#3\arcs\llap.#4}
\def\deco[#1 #2 #3]{#1\deg#2\arcm#3\arcs}
\def\rra[#1 #2]{#1\sup{h}#2\sup{m}}
\def\page{\vfill\eject}
\def\dots{\relax\ifmmode \ldots\else $\ldots$\fi}
\def\WHzsr{\ifmmode $W\,Hz\mo\,sr\mo$\else W\,Hz\mo\,sr\mo\fi}
\def\mHz{\ifmmode $\,mHz$\else \,mHz\fi}
\def\GHz{\ifmmode $\,GHz$\else \,GHz\fi}
\def\mKs{\ifmmode $\,mK\,s$^{1/2}\else \,mK\,s$^{1/2}$\fi}
\def\muKs{\ifmmode \,\mu$K\,s$^{1/2}\else \,$\mu$K\,s$^{1/2}$\fi}
\def\muKRJs{\ifmmode \,\mu$K$_{\rm RJ}$\,s$^{1/2}\else \,$\mu$K$_{\rm RJ}$\,s$^{1/2}$\fi}
\def\muKHz{\ifmmode \,\mu$K\,Hz$^{-1/2}\else \,$\mu$K\,Hz$^{-1/2}$\fi}
\def\MJysr{\ifmmode \,$MJy\,sr\mo$\else \,MJy\,sr\mo\fi}
\def\MJysrmK{\ifmmode \,$MJy\,sr\mo$\,mK$_{\rm CMB}\mo\else \,MJy\,sr\mo\,mK$_{\rm CMB}\mo$\fi}
\def\microns{\ifmmode \,\mu$m$\else \,$\mu$m\fi}
\def\micron{\microns}
\def\muK{\ifmmode \,\mu$K$\else \,$\mu$\hbox{K}\fi}
\def\microK{\ifmmode \,\mu$K$\else \,$\mu$\hbox{K}\fi}
\def\muW{\ifmmode \,\mu$W$\else \,$\mu$\hbox{W}\fi}
\def\kms{\ifmmode $\,km\,s$^{-1}\else \,km\,s$^{-1}$\fi}
\def\kmsMpc{\ifmmode $\,\kms\,Mpc\mo$\else \,\kms\,Mpc\mo\fi}

\providecommand{\sorthelp}[1]{}


% Custom definitions
\newcommand{\mathsc}[1]{{\normalfont\textsc{#1}}}
\newcommand{\dv}[0]{\vec{d}}
\newcommand{\s}[0]{\vec{s}}
\newcommand{\M}[0]{\tens{M}}
\renewcommand{\P}[0]{\tens{P}}
\newcommand{\G}[0]{\tens{G}}
\newcommand{\B}[0]{\tens{B}}
\renewcommand{\a}[0]{\vec{a}}
\newcommand{\n}[0]{\vec{n}}
\renewcommand{\t}[0]{\vec{t}}
\def\Cosmoglobe{\textsc{Cosmoglobe}}
\def\Planck{\textit{Planck}}
\def\WMAP{\textit{WMAP}}
\def\COBE{\textit{COBE}}
\def\GAIA{\textit{Gaia}}
\def\gaia{\textit{Gaia}}
\def\Gaia{\textit{Gaia}}
\def\WISE{WISE}
\def\AKARI{\textit{{AKARI}}}
\def\IRAS{\textit{{IRAS}}}
\def\nside{$N_{\mathrm{side}}$}
\newcommand{\cii}{\ensuremath{\mathsc{C\ ii}}}
\newcommand{\CII}{\ensuremath{\mathsc{C\ ii}}}
\def\Commander{\texttt{Commander} }
\def\commanderthree{\texttt{Commander3} }

\def\Tcmb{\ifmmode T_\mathrm{CMB}\else $T_{\mathrm{CMB}}$\fi}
\def\Tcold{\ifmmode T_\mathrm{c}\else $T_{\mathrm{c}}$\fi}
\def\Thot{\ifmmode T_\mathrm{h}\else $T_{\mathrm{h}}$\fi}
\def\Tnear{\ifmmode T_\mathrm{n}\else $T_{\mathrm{n}}$\fi}
\def\Thalpha{\ifmmode T_\mathrm{H\alpha}\else $T_{\mathrm{H\alpha}}$\fi}
\def\scmb{\ifmmode s_\mathrm{CMB}\else $s_{\mathrm{CMB}}$\fi}
\def\squad{\ifmmode s_\mathrm{quad}\else $s_{\mathrm{quad}}$\fi}
\def\ssynch{\ifmmode s_\mathrm{s}\else $s_\mathrm{s}$\fi}
\def\sdust{\ifmmode s_\mathrm{d}\else $s_{\mathrm{d}}$\fi}
\def\ssdust{\ifmmode s_\mathrm{sd}\else $s_{\mathrm{sd}}$\fi}
\def\same{\ifmmode s_\mathrm{AME}\else $s_{\mathrm{AME}}$\fi}
\def\ssrc{\ifmmode s_\mathrm{src}\else $s_{\mathrm{src}}$\fi}
\def\sco{\ifmmode s_\mathrm{CO}\else $s_{\mathrm{CO}}$\fi}
\def\sff{\ifmmode s_\mathrm{ff}\else $s_{\mathrm{ff}}$\fi}
\def\gff{\ifmmode g_\mathrm{ff}\else $g_{\mathrm{ff}}$\fi}
\def\fsynch{\ifmmode f_\mathrm{s}\else $f_{\mathrm{s}}$\fi}
\def\fsd{\ifmmode f_\mathrm{sd}\else $f_{\mathrm{sd}}$\fi}
\def\fame{\ifmmode f_\mathrm{AME}\else $f_{\mathrm{AME}}$\fi}
\def\alphasrc{\ifmmode \alpha_\mathrm{src}\else $\alpha_{\mathrm{src}}$\fi}
\def\bcold{\ifmmode \beta_\mathrm{c}\else $\beta_{\mathrm{c}}$\fi}
\def\bhot{\ifmmode \beta_\mathrm{h}\else $\beta_{\mathrm{h}}$\fi}
\def\bnear{\ifmmode \beta_\mathrm{n}\else $\beta_{\mathrm{n}}$\fi}
\def\bhalpha{\ifmmode \beta_\mathrm{H\alpha}\else $\beta_{\mathrm{H\alpha}}$\fi}
\def\bsynch{\ifmmode \beta_\mathrm{s}\else $\beta_{\mathrm{s}}$\fi} 
\def\bsun{\ifmmode \beta_\mathrm{sun}\else $\beta_{\mathrm{sun}}$\fi} 
\def\nuzeros{\ifmmode \nu_{0,\mathrm{s}}\else $\nu_{0,\mathrm{s}}$\fi} 
\def\nuzeroff{\ifmmode \nu_{0,\mathrm{ff}}\else $\nu_{0,\mathrm{ff}}$\fi} 
\def\nuzerocold{\ifmmode \nu_{0,\mathrm{c}}\else $\nu_{0,\mathrm{c}}$\fi}
\def\nuzerohot{\ifmmode \nu_{0,\mathrm{h}}\else $\nu_{0,\mathrm{h}}$\fi}
\def\nuzeronear{\ifmmode \nu_{0,\mathrm{n}}\else $\nu_{0,\mathrm{n}}$\fi} 
\def\nuzerohalpha{\ifmmode \nu_{0,\mathrm{H\alpha}}\else$\nu_{0,\mathrm{H\alpha}}$\fi} 
\def\nuzeroame{\ifmmode \nu_{0,\mathrm{AME}}\else $\nu_{0,\mathrm{AME}}$\fi} 
\def\nuzerosd{\ifmmode \nu_{0,\mathrm{}}\else $\nu_{0,\mathrm{sd}}$\fi} 
\def\nuzerosrc{\ifmmode \nu_{0,\mathrm{src}}\else $\nu_{0,\mathrm{src}}$\fi} 
\def\nup{\ifmmode \nu_{\mathrm{p}}\else $\nu_{\mathrm{p}}$\fi} 
\def\alphasd{\ifmmode \alpha_{\mathrm{sd}}\else $\alpha_{\mathrm{sd}}$\fi} 
\def\Te{\ifmmode T_{\mathrm{e}}\else $T_{\mathrm{e}}$\fi} 
\def\kB{\ifmmode k_\mathrm{B}\else $k_{\mathrm{B}}$\fi} 


\begin{document} 


%\title{\bfseries{\Cosmoglobe\ DR2. VII. Multi-component spectral energy density model of large-scale thermal dust emission from 100\,GHz to 100\,THz}}
\title{\bfseries{\Cosmoglobe\ DR2. VII. Towards a concordance model of large-scale thermal dust emission for microwave and infrared frequencies}}
%\title{\bfseries{\Cosmoglobe\ DR2. VII. The mean thermal dust emission SED as seen by \Planck\ HFI and \COBE-DIRBE}}

   %This author list corresponds to \title{Author list for L04\_CMB\_Foregrounds\_Extraction}
%Prepared by M. Lopez-Caniego (Marcos.Lopez.Caniego@sciops.esa.int), ESAC/ESA
%This version is from Thu Jul 12 18:11:48 2018 CET
%\subtitle{There are 152 co-authors in this list}
\newcommand{\oslo}[0]{1}
%\newcommand{\MIT}[0]{2}
\newcommand{\milanoA}[0]{2}
\newcommand{\milanoB}[0]{3}
\newcommand{\milanoC}[0]{4}
\newcommand{\triesteB}[0]{5}
\newcommand{\planetek}[0]{6}
\newcommand{\princeton}[0]{7}
\newcommand{\jpl}[0]{8}
\newcommand{\helsinkiA}[0]{9}
\newcommand{\helsinkiB}[0]{10}
\newcommand{\nersc}[0]{11}
\newcommand{\haverford}[0]{12}
\newcommand{\mpa}[0]{13}
\newcommand{\triesteA}[0]{14}
\newcommand{\iia}[0]{2}

\author{\small
J.~R.~Eskilt\inst{\oslo}\thanks{Corresponding author: J.~R.~Eskilt; \url{j.r.eskilt@astro.uio.no}}
\and
K.~Lee\inst{\oslo}
\and
D.~J.~Watts\inst{\oslo}
\and
S.~Nerval\inst{\oslo}
\and
et al.
}
\institute{\small
        Institute of Theoretical Astrophysics, University of Oslo, Blindern, Oslo, Norway \goodbreak
}


   %\institute{Institute of Theoretical Astrophysics, University of Oslo, Blindern, Oslo, Norway}
  
   % Shortened title, author list for top of page 
   \titlerunning{Towards a concordance model for thermal dust emission}
   \authorrunning{Gjerløw et al.}

   \date{\today} 
   
   \abstract{
     We fit a four-component thermal dust model to \COBE-DIRBE data between 3.5 and 240\,$\mu$m within the global Bayesian end-to-end \Cosmoglobe\ DR2 reanalysis. Following a companion analysis of \Planck\ HFI, the four components of this model correspond to ``hot dust'', ``cold dust'', ``nearby dust'', and ``H$\alpha$ correlated dust'', respectively, and each component is modelled in terms of a fixed spatial template and a spatially isotropic spectral energy density (SED) defined by an overall free amplitude for each DIRBE channel. The H$\alpha$ dust is an extinction component and, as such, has a negative amplitude. Except for the cold dust amplitude, which is only robustly detected in the 240\,$\mu$m channel, we measure statistically significant template amplitudes for all components in all DIRBE channels between 3.5 and $240\,\mu\mathrm{m}$; the two highest frequency channels are too dominated by starlight emission to allow robust dust detections. The total number of DIRBE-specific degrees of freedom in this model is thus only 25. Despite this low dimensionality, the resulting total SED agrees well with recent Astrodust predictions, and the overall model efficiency is high although with a notable wavelength dependency; at both low and high frequencies, more than 95\,\% of the signal rms is captured by the model, while at 60 and $100\,\mu\mathrm{m}$ only about 70\,\% of the signal is successfully accounted for. The hot dust component, which was previously found to correlate strongly with \ion{C}{II} emission, has the highest absolute amplitude in all DIRBE frequency channels; at 3.5\,$\mu$m, which is known to be dominated by PAH emission, this component accounts for at least 80\,\% of the total signal. This analysis represents an important step towards establishing a joint concordance model of thermal dust emission applicable to both the microwave and infrared regimes, and we conclude by outlining a roadmap to a future joint analysis of AKARI, DIRBE, IRAS, and \Planck.
   }
   \keywords{ISM: general - Zodiacal dust, Interplanetary medium - Cosmology: observations, diffuse radiation - Galaxy: general}

   \maketitle

\setcounter{tocdepth}{2}
\tableofcontents
   
% INTRODUCTION
%-------------------------------------------------------------------
\section{Introduction}

(ADD REFERENCES)

In a series of seven companion papers, within which this is the last,
we have reanalyzed the 30-years old \COBE-DIRBE data using modern
end-to-end Bayesian statistical techniques as implemented in the
\Cosmoglobe\footnote{\url{http://cosmoglobe.uio.no}} framework. The
most important products emerging from this work is a set of 10
full-sky DIRBE frequency maps, covering the infrared frequency range
between 1.25 and 240\,$\mu$m. These maps have both substantially lower
systematic errors and better error characterization compared to their
official counterparts \citep{hauser1998,CG02_01}, and, in particular,
they suffer far less from zodiacal light contamination
\citep{K98,CG02_02}. As a result they can be used for more detailed
astrophysics applications; a few examples of this is presented in the
current analysis, including improved estimates of the cosmic infrared
background spectrum \citep{CG02_03} and large-scale starlight emission
\citet{CG02_04}.

The main topic of the last three papers in the series revolves around
how to model thermal dust radiation efficiently on large angular
scales in the microwave and infrared frequency regimes. This issue has
been the focus of intense scrutiny ever since the groundbreaking IRAS
measurements were published in 1982, and its scientific importance has
only increased through the release of a series of increasingly
sensitive data sets, such as \COBE-DIRBE and \Planck\ HFI. Today,
detailed dust modelling plays a key role in many of the most
competitive fields of cosmology, from the search for inflationary
gravitational waves in cosmic microwave background (CMB) $B$-mode
polarization data to measurements of dark energy using distant
supernovae.

As of today, \Planck\ HFI defines the state-of-the-art for thermal
dust mapping both in terms of signal-to-noise ratio and systematic
control. Based on its six frequency channels, the \Planck\ team
produced several exquisite dust models in both intensity and
polarization
\citep{planck2013-XVII,planck2013-p06b,planck2013-XIV,planck2014-XIX,planck2014-a12,planck2014-XXII,planck2016-l11A,planck2016-l11B},
and these now form the basis for much of the dust modelling efforts in
the field \citep[e.g.,][]{pysm2,pysm3}. However, at the same time the
limited frequency range of \Planck\ HFI, covering only 100-857\,GHz,
implies that the applicability of these models are currently quite
limited. Furthermore, the absolute calibration of the 857\,GHz \Planck\
channel, which nominally is the most sensitive \Planck\ dust channel,
is uncertain at the $\sim$10\,\% level \citep{planck2016-l03,npipe},
and this induces a large uncertainty on the dust SED parameters when
extrapolating to higher frequencies.

In the current paper, we address these issues by fitting the
multi-component dust model proposed by \citet{CG02_05} and
\citet{CG02_06} to the re-processed \COBE-DIRBE data within the
\Cosmoglobe\ DR2 analysis framework. This model consists of four
primary components, corresponding respectively to 1) cold dust; 2) hot
dust; 3) nearby dust; and 4) H$\alpha$-correlated dust (observed in
extinction). From the previous papers, this model is already known to fit the
\Planck\ HFI frequencies very well when coupled to simple modified
blackbody (MBB) spectral energy densities (SEDs) with spatially
constant spectral parameters, and in this paper we show that the same
spatial morphologies also trace dust in the DIRBE frequencies very
well, although with more complicated SED behaviour. The combined
result is an global model that jointly describes both microwave and
infrared frequencies.

So far, this model has only been developed for and applied to
intensity measurements. However, polarized thermal dust emission also
plays a key role in modern astrophysics and cosmology. For instance,
massive resources are currently being spent on searching for and
constraining the amplitude of inflationary gravitational waves through
deep CMB $B$-mode polarization experiments, and polarized thermal dust
emission represents a key challenge for these. Given the high
efficiency of the \Cosmoglobe\ DR2 four-component dust model for
intensity data, it is reasonable to expect a similar performance for
polarization observations. We therefore conclude this paper by
outlining one potential roadmap towards a future concordance model for
thermal dust emission in both intensity and polarization. 

\section{Bayesian modelling of thermal dust emission in \Cosmoglobe\ DR2}
\label{sec:dr2}
This paper, being a part of the second \Cosmoglobe\ Data Release (DR2), is by
the nature of that analysis -- that is, a global, Bayesian analysis mapping out
the joint posterior of all involved unknowns -- intricately bound up with the
other papers in the release, both in terms of the data sets and data model
used. Hence, here we will here recap of the data model used, as presented in
\citep{CG02_01}, after which we will devote a subsection to how the thermal
dust part of the global model is modelled.

\subsection{Data model and posterior distribution}
In the \Cosmoglobe\ framework, the standard modus operandi is to begin with an
explicit parametric model incorporating all known aspects of the dataset with
which we are working -- including both instrumental effects as well as a
physical model of what is being observed. For instance, the DIRBE time-ordered
data (TOD) used in this analysis have been parametrized thus:
\begin{align}
	\label{eq:model}
    \dv_{\mathrm{dirbe}} &=\G\P\left[\B\sum_{c=1}^{n_{\mathrm{comp}}}\M_c\a_c+\s_{\mathrm{zodi}} +
          \s_{\mathrm{static}}\right] + \n \\
                         & \equiv \s^{\mathrm{tot}} + \n \nonumber,
\end{align}
where $\dv$ is the stacked TODs, $\G$ is an overall gain factor, $\P$ is the
pointing matrix which projects the physical sky onto a $n_{\mathrm{tod}}$-sized
space, $\B$ is the instrumental beam convolution operator, and $\n$ is the
instrumental noise. The physical sky is represented by the next three terms:
first, a sum over sky components that can be modelled as constant at every
point in time (in galactic coordinates) -- the sum is over a HEALPix map of
amplitudes for each sky component ($\a_c$) multiplied by a mixing matrix which
extrapolates the given component to the frequencies observed; secondly, there
is a term representing the zodiacal emission, which cannot be treated as a time
constant; finally there is a term representing a component that is static in
solar coordinates, which may either be related to the DIRBE sidelobes, or be
geniune excess radiation originating in the Solar System. These two last terms
are treated in \citet{CG02_02} and \citet{CG02_03}, and what concerns us in the
present work is the first of the three terms, which we can write out more
explicitly as follows:

\begin{equation}
\label{eq:skymodel}
\begin{aligned}
  \sum_{c=1}^{n_{\mathrm{comp}}} \M_c \a_c  = \,
  &\M_{\mathrm{mbb}}(\bcold,\Tcold,\nuzerocold)\vec{a}_{\mathrm{cold}}
  && \textrm{(Cold dust)} \\
  + &\M_{\mathrm{mbb}}(\bhot,\Thot,\nuzerohot)
  \vec{a}_{\mathrm{hot}} && \textrm{(Hot dust)}\\
  + &\M_{\mathrm{mbb}}(\bnear,\Tnear,\nuzeronear) \t_{\mathrm{near}}
  a_{\nu} && \textrm{(Nearby dust)} \\
  + &\M_{\mathrm{mbb}}(\bhalpha,\Thalpha,\nuzerohalpha)
  a_{\nu} && \textrm{(H$\alpha$ correlated dust)} \\
  + &\left(\frac{\nuzeroff}{\nu}\right)^2
  \frac{g_{\mathrm{ff}}(\nu;\Te) }{g_{\mathrm{ff}}(\nuzeroff;\Te)}
  \vec{t}_{\mathrm{ff}} && \textrm{(Free-free)} \\
  + &U_{\mathrm{mJy}} \sum_{j=1}^{n_{\mathrm{s}}}
  f_{\mathit{Gaia},j} a_{\mathrm{s},j}, &\quad&
  \textrm{(Bright stars)} \\
  + &U_{\mathrm{mJy}} f_{\mathit{Gaia},j} \a_{\mathrm{fs},j}, &\quad&
  \textrm{(Faint stars)} \\  
  + &m_{\nu} && \textrm{(Monopole)}. 
\end{aligned}
\end{equation}

As it stands, this is as ``complete'' of a static model we currently can write down
for the frequency domain we are interested in this paper. The first four
terms is the complete model of thermal dust, which is the main focus of this
work, and will be elaborated on below. The fourth term models the free-free
dust component, which is expected to contribute moderately at all relevant
frequencies without becoming a dominant term. Then, there are two point source
terms, all of which are dealt with in \citet{CG02_04}. Finally, the final term
represents the monopole at each frequency, which is also treated in
\citet{CG02_03}.

\subsubsection{Bayesian end-to-end analysis}
Within the wherein we draw samples from the \emph{posterior}
distribution of the full set of parameters that enter the analysis
\citep{CG02_01} -- in formulaic terms, we are mapping out $P(\theta | \dv)$,
the probability distribution of the set of parameters $\theta$ given the
observed data $\dv$. Bayes' theorem allows us to write
\begin{equation}
    \label{eq:bayestheorem}
    P(\theta|\dv) = \frac{P(\dv|\theta)P(\theta)}{P(\dv)},
\end{equation}
and, since $P(\dv)$ typically only enters as a normalizing term as long as the
parameter space does not change, sampling from $P(\theta|\dv)$ is (modulo a
prior term) is equivalent to sampling from
$P(\dv|\theta)\equiv\mathcal{L}(\theta)$, the so-called \emph{likelihood}
function.

The number of parameters involved in our model (Eq. \ref{eq:model}) is of the
order of millions, making sampling from the likelihood function a non-trivial
task. The \Cosmoglobe\ framework is based on the \Commander software
\citep{eriksen:2004,seljebotn:2019,bp03}, which maps out the posterior
parameter distribution through a process called \emph{Gibbs sampling}
\citep[e.g.,][]{geman:1984}, a Monte-Carlo method based on sequentially
sampling each parameter (or a subset of parameters) from their respective
marginal distributions wrt. all other parameters. The theory of Gibbs sampling
then says that by combining these samples into a full set for all the
parameters involved, this set will represent a proper sample from the joint
distribution. Applied to Eq. \ref{eq:model}, we get a sequence of operations as
follows:

%\begin{equation}
%\begin{alignat}{11}
%\G &\,\leftarrow P(\G&\,\mid &\,\dv,&\, &\,\phantom{\G} &\,\xi_n, &
%\,\beta_{\mathrm{sky}}& \,\a_{\mathrm{sky}}, &\,\zeta_{\mathrm{z}},
%&\,\a_{\mathrm{static}})\\ \nonumber
%\xi_{\mathrm{n}} &\,\leftarrow P(\xi_{\mathrm{n}}&\,\mid &\,\dv,&\, &\,\G, &\,\phantom{\xi_n} &
%\,\beta_{\mathrm{sky}}& \,\a_{\mathrm{sky}}, &\,\zeta_{\mathrm{z}},
%&\,\a_{\mathrm{static}})\\ \nonumber
%\beta_{\mathrm{sky}} &\,\leftarrow P(\beta_{\mathrm{sky}}&\,\mid &\,\dv,&\, &\,\G, &\,\xi_n, &
%\,\phantom{\beta_{\mathrm{sky}}}& \,\a_{\mathrm{sky}}, &\,\zeta_{\mathrm{z}}, &\,\a_{\mathrm{static}})\\ \nonumber
%\a_{\mathrm{sky}} &\,\leftarrow P(\a_{\mathrm{sky}}&\,\mid &\,\dv,&\, &\,\G, &\,\xi_n, &
%\,\beta_{\mathrm{sky}},& \,\phantom{\a_{\mathrm{sky}},}
%&\,\zeta_{\mathrm{z}}, &\,\a_{\mathrm{static}})\\ \nonumber
%\zeta_{\mathrm{z}} &\,\leftarrow P(\zeta_{\mathrm{z}}&\,\mid &\,\dv,&\, &\,\G, &\,\xi_n, &
%\,\beta_{\mathrm{sky}},& \,\a_{\mathrm{sky}},
%&\,\phantom{\zeta_{\mathrm{z}},} &\,\a_{\mathrm{static}})\\ \nonumber
%\a_{\mathrm{static}} &\,\leftarrow P(\a_{\mathrm{static}}&\,\mid &\,\dv,&\, &\,\G, &\,\xi_n, &
%\,\beta_{\mathrm{sky}},& \,\a_{\mathrm{sky}}, &\,\zeta_{\mathrm{z}} &\,\phantom{\a_{\mathrm{static}}})\label{eq:gibbs_static}.
%\end{alignat}
%\end{equation}

\begin{equation}
    \label{eq:gibbschain}
\begin{aligned}
\G &\,\leftarrow P(\G&\,\mid &\,\dv,&\, &\,\phantom{\G} &\,\xi_n, &
\,\beta_{\mathrm{sky}}& \,\a_{\mathrm{sky}}, &\,\zeta_{\mathrm{z}},
&\,\a_{\mathrm{static}})\\
\xi_{\mathrm{n}} &\,\leftarrow P(\xi_{\mathrm{n}}&\,\mid &\,\dv,&\, &\,\G, &\,\phantom{\xi_n} &
\,\beta_{\mathrm{sky}}& \,\a_{\mathrm{sky}}, &\,\zeta_{\mathrm{z}},
&\,\a_{\mathrm{static}})\\
\beta_{\mathrm{sky}} &\,\leftarrow P(\beta_{\mathrm{sky}}&\,\mid &\,\dv,&\, &\,\G, &\,\xi_n, &
\,\phantom{\beta_{\mathrm{sky}}}& \,\a_{\mathrm{sky}}, &\,\zeta_{\mathrm{z}}, &\,\a_{\mathrm{static}})\\
\a_{\mathrm{sky}} &\,\leftarrow P(\a_{\mathrm{sky}}&\,\mid &\,\dv,&\, &\,\G, &\,\xi_n, &
\,\beta_{\mathrm{sky}},& \,\phantom{\a_{\mathrm{sky}},}
&\,\zeta_{\mathrm{z}}, &\,\a_{\mathrm{static}})\\
\zeta_{\mathrm{z}} &\,\leftarrow P(\zeta_{\mathrm{z}}&\,\mid &\,\dv,&\, &\,\G, &\,\xi_n, &
\,\beta_{\mathrm{sky}},& \,\a_{\mathrm{sky}},
&\,\phantom{\zeta_{\mathrm{z}},} &\,\a_{\mathrm{static}})\\
\a_{\mathrm{static}} &\,\leftarrow P(\a_{\mathrm{static}}&\,\mid &\,\dv,&\, &\,\G, &\,\xi_n, &
\,\beta_{\mathrm{sky}},& \,\a_{\mathrm{sky}}, &\,\zeta_{\mathrm{z}} &\,\phantom{\a_{\mathrm{static}}})
\end{aligned}
\end{equation}

Here, the symbol $\leftarrow$ indicates the operation of drawing a sample from
the distribution on the right-hand side. After some burn-in period, the
resulting joint parameter sets will correspond to samples drawn from the true
underlying joint posterior.

Since every step of the Gibbs sampling process assumes that all other
parameters are ``given'', we can now treat a highly interconnected problem
(i.e. sampling from the joint posterior of all parameters involved in our data
model) as a highly modular one -- meaning that we can perform each
``sub-analysis'' without being concerned with the other parts of the problem.
Hence, in this paper, we mainly focus on the three first components of Eq.
(\ref{eq:skymodel}), and leave the other components for other papers. (TODO
maybe actually cite those papers)


\subsection{Multi-component thermal dust modelling}

%\section{Dust modelling}
%\subsection{Current status}
%\label{sec:current_dust}
Interstellar dust -- amorphous particles of silicate and carbonaceous materials
-- makes its presence known on practically all astrophysically relevant
wavelengths. The efforts to classify and describe this material is significant
in its own right, but knowing its properties also allows for better and more
precise astrophysical foreground removal in cases where interstellar dust
emission contaminate the other signals of interest.\footnote{For a recent
review, see \cite{Hensley2021}}.

Recently, the "astrodust+PAH" model \citep{Hensley2023} was introduced, wherein
the diffuse interstellar medium is hypothesised to be made up of a single
composite material (the eponymous astrodust) for scales larger than
$\sim0.02~\mu$m, and a distinct variety of materials -- including so-called
polycyclic aromatic hydrocarbons (PAH) -- on scales smaller than this.

In the wavelength regime between $3000-100~\mu$ m, this model is described well
 by modified blackbody spectral energy density (SED)\footnote{The actual
 astrodust model is made up of a composite MBB which has a transition between
353 and 217 GHz}, that is a SED with the following behavior
\begin{equation}
s(\nu) \propto \nu^\beta B(\nu, T),
\label{eq:mbb}
\end{equation}

where $\nu$ is the frequency, $B$ is the Planck law for a perfect blackbody,
and $\beta$ is the spectral index. Typical temperatures of this blackbody in
the diffuse ISM is around $\sim$ 20 K, meaning that the distribution typically
peaks around 150\,$\mu$ m ($\sim 2000$ GHz).

At lower wavelengths (2.5\,$\mu$m - 12\,$\mu$m), the astrodust+PAH model is
mostly dominated by the nanoscale particle emission, and exhibits strong
emission lines at various wavelengths (see Fig.~10 in \cite{Hensley2023}).

\subsubsection{The four-component dust model}
This is a general picture -- typically, in a given line-of-sight the relative
contribution of various dust components will vary. At the same time, the degree
to which such variations can be detected and described is limited by the
resolution and signal-to-noise ratios of the available data at the wavelengths
involved. For example, .... Thus classifying populations of interstellar dust
with common spectral parameters has been of high importance.

Recently, it was demonstrated in \citet{CG02_05} that a natural and highly
effective classification of such populations can be achieved through the use of
templates derived from surveys of spectral line emission (CII, H$\alpha$, CO
and HI) and from inference of nearby dust structures via starlight extincion
\citep{edenhofer:2024}. In that paper, we showed that with linear minimization
five such templates, more than 95\% of the signal variance at dust-dominated
frequencies in \Planck\ (353-857\,GHz) and DIRBE (240 - 60 $\mu$m), and that it
still accounted for more than 80\% of the signal variance at the
starlight-dominated DIRBE bands (25 and 12 $\mu$ m).

In another recent paper \citep{CG02_06}, we applied the main idea behind that
result to all \Planck\ HFI data within the \Cosmoglobe\ framework. We showed
that by only using the GAIA extinction template, and by assuming three other
components with freely varying amplitudes per pixel, but with fixed global
temperatures and spectral indices per component, we arrived at a dust model
that could explain between 98.5\% and 99.9\% of the non-CMB signal in these
channels. Furthermore, the resulting amplitude maps exhibited morphologies that
turned out to be very similar to the templates used in \citet{CG02_05},
giving an independent confirmation of the appropriateness of assuming a
morphological correlation between those templates and dust populations.

Encouraged by these results, in the main \Cosmoglobe\ DR2 analysis, we employ a
jimilar four-component thermal dust model, which is given by the first four terms
of Eq. (\ref{eq:skymodel}), similar to the four
dust components used in \citet{CG02_06}. Three of the components (hot, cold and
H$\alpha$-correlated dust) are allowed to have a freely varying amplitude on
the sky, whereas the nearby dust component is constrained morphologically to
the nearby dust template used in \citet{CG02_05, CG02_06} which was derived
from the \citet{edenhofer:2024} extinction maps (see \citet{CG02_05} for
details on how this template was constructed).

The SEDs of these components are not directly modelled as
modified blackbodies, as they were in \citet{CG02_06}. Rather, we define a set
of SED bins, each of which is chosen to correspond to the width of the DIRBE
bands, as shown in table \ref{tab:bins}. Each dust component is then set to be
constant within a given bin, meaning that there is one free parameter per bin
per component, which is then sampled over in the Gibbs chain.

%One natural distinction that has been made is between ``hot'' and ``cold''
%dust. If we are able to divide the interstellar dust into components of high
%and low energy, such components could lend themselves to be traced by
%independent probes of interstellar matter. For the low-energy component, a
%natural tracer would be the H I column density as measured by experiments such
%as HI4PI. The hot dust component, similarly, could be traced by other measures
%of high-energy regions, with the traditional candidates being either the
%H$\alpha$ or C II spectral lines.
%
%In addition to this, we may also use known distance information to infer
%various dust populations. As an example, \cite{edenhofer:2024} constructed a
%series of high-resolution (\nside 256) extinction maps using data from GAIA and
%
%MORE HERE ON HALPHA

%\subsection{The \Cosmoglobe DR2 analysis}
%\subsubsection{Four-component dust model}
%As mentioned above, we here adopt a model for thermal dust that comprises three
%distinct components, represented by the first four terms of Eq.
%(\ref{eq:skymodel}). All four components are modelled as having a continuous
%modified blackbody behaviour between 100 and 1050~GHz, i.e. below the DIRBE
%frequencies, multiplied by an amplitude per pixel, which for two of the
%components is allowed to vary as a free parameter in the Gibbs chain, and for
%the last component is defined as full-sky template.
%
%The spectral parameters -- that is, $\beta$ and $T$ in Eq. (\ref{eq:mbb}) -- of
%this model are set to be constant across the sky for each of the three
%components. Thus, each of the component has a different spectral parameter pair
%than the others, but it is not allowed to vary spatially.
%
%For the frequencies covered by DIRBE bands 3-10, we extend this model in the
%following way: We define a set of SED bins, each of which is chosen to
%correspond to the width of the DIRBE bands, as shown in table \ref{tab:bins}.
%Each of these bins are associated with a free parameter for that dust
%component; namely, the amplitude of the SED in that bin. In other words, the
%SED for a given component for a given SED bin is a constant, but the exact
%value of this constant is allowed to vary in the Gibbs chain.

%Since the phenomenology of thermal dust gets more complicated above 1050~GHz
%(see sec. \ref{sec:current_dust}), instead of adopting a model based on
%underlying physics, we define a set of bins, each of which is chosen to roughly
%correspond to the width of DIRBE bands 3-10, respectively, as shown in Table
%\ref{tab:bins}. Each of the bins is assigned an amplitude for each dust
%component, and the SED of that component for the frequencies encompassed by a
%given bin is given by the amplitude assigned to that bin\footnote{Only one of
%    the three components are allowed to have nonzero amplitudes at DIRBE band 3
%and 4 -- see below}. We do not define such bins for DIRBE bands 1 and 2, and we
%force the total dust SED to be zero at these bands, since the thermal dust
%emission is expected to be negligible at these frequenies.

%The first component we call ``nearby dust''. We define this component in order
%to properly utilize the available knowledge about nearby dust structure --
%several works, such as \citet{Dharmawardena:2024} and \citet{edenhofer:2024}
%use extinction data from catalogues such as Gaia to infer the dust structure
%out to 1-3~kpc. We define a template (see Section \ref{sec:data} for a detailed
%explanation of this template) which traces these nearby dust structures. The
%complete SED of this component is thus defined as
%\begin{equation}
%    \label{eq:nearby_dust}
%    s(\nu) = \begin{cases} a\t\nu^{\beta} B(\nu, T) & \nu \in [100,
%        1050~\text{GHz}] \\
%        a^i_{\textrm{bin}}\t & \nu \in [\nu_{\textrm{bin, start}}^i,
%        \nu_{\textrm{bin, end}}^i]
%            \end{cases},
%\end{equation}
%where $a$ is a scaling parameter, $\t$ is the nearby dust template, and the
%remaining parameters on the first line are the same as in Eq.~(\ref{eq:mbb}),
%and $\nu_{\textrm{bin, start}}^i$ and $\nu_{\textrm{bin, end}}^i$ are the start
%and end frequencies of bin $i$, respectively.
%
%The two next components are not constrained spatially by templates; rather,
%their amplitudes are allowed to vary from pixel to pixel. We call these
%components ``cold'' and ``hot'' dust, respectively, and their SEDs can both be
%expressed as follows,
%\begin{equation}
%    \label{eq:coldhotdust}
%    s(\nu) = \begin{cases} \a\nu^{\beta} B(\nu, T) & \nu \in [100,
%        1050~\text{GHz}] \\
%        a^i_{\textrm{bin}} & \nu \in [\nu_{\textrm{bin, start}}^i,
%        \nu_{\textrm{bin, end}}^i]
%            \end{cases},
%\end{equation}
%where $\a$ is the vector of per-pixel amplitudes. The only difference between
%the two components is that SED amplitude of the hot dust component is allowed
%to be nonzero for bins 7 and 8 (DIRBE bands 3 and 4), while the cold dust
%(as well as the nearby dust) SEDs are only allowed to be nonzero for bins 1-6
%(DIRBE bands 10 through 5).
%
%With this model, we are effectively operating with roughly two degrees of
%freedom per pixel -- namely, the hot and cold dust amplitudes. The rest of the
%parameters are all spatially constant, and so adds just fractional degrees of
%freedom to the total dust model.
%
%We emphasize that this model is mostly phenomenological, and we are already
%planning several improvements to the model to be published in the future.
%However, already with this relatively crude model the results are of sufficient
%interest for the community that we think it worthwhile to report.
%
%
%\subsection{Gibbs sampling}


\section{Data}
\label{sec:data}
%This paper concerns itself with a subset of the total global analysis which is
%described more fully in \cite{CG02_01}; namely, the part of the analysis that
%models the three thermal dust components. Describing the data that goes into
%this part of the analysis is two-tiered, in the sense that there may be (and in
%this case, is) data that is used in the total global analysis which is not
%directly impacting the dust models, and there is data that does impact it. 
%
%directly used for this purpose. However, even the data that is not directly
%used to fit our dust models still affect the total analysis by
%affecting the overall data model fit (eq.~\ref{eq:skymodel}).
%
%Below, we first describe the data that directly impacts the dust fitting --
%namely, the DIRBE and \Planck\ HFI data -- before describing the ancillary data
%that is used in the global the analysis. All of the data used is described in
%more detail in \cite{CG02_01}, and we refer the interested reader to that
%paper.

There are two datasets used in the \Cosmoglobe\ DR2 analysis: Low-level data
from \COBE-DIRBE, and starlight data from \WISE \citep{wright:2010} and
\GAIA \citep{gaia:2016,gaia:2018}. Below we give a succinct description of these
data sets and the preprocessing performed; for a more in-depth description,
see \citet{CG02_01} and \citet{CG02_04}.

\subsection{\COBE-DIRBE}
The Diffuse InfraRed Explorer DIRBE, whose main goal was the mapping out of the
CIB, was part of the \COBE\ satellite \citep{boggess92, silverberg93}, and
measured the sky in ten frequency bands from $1\,\mathrm{\mu m}$ to
$240\,\mathrm{\mu m}$. In this analysis, we have converted the original DIRBE
CIO (Calibrated Individual Observations), whose pointing information is given
in terms of Quadcube pixels with a resolution of 20\arcs\ into a HEALPix
\nside=512 pixelation, giving an approximate resolution of 7\arcm. Following
the nomenclature of \cite{CG02_01}, we refer to the DIRBE CIOs as
``time-ordered data'' (TOD). Including DIRBE TOD in the analysis in this way
allows us to target the zodiacal light \citep{CG02_02}, which, as mentioned
above, must be analysed as a time-varying component on the sky, in contrast to
other astrophysical components.

The use of DIRBE data also allows a fuller exploration of the scales relevant
for interstellar dust modelling; in particular, we are able to capture the
``bump'' and falling off of the thermal dust SEDs. 

\subsection{\WISE\ and \GAIA}
At the higher DIRBE frequency bands (25-1.25 $\mu$m), starlight emission
becomes a significant source of emission, both as point sources and as a
diffuse component. Using the AllWise pointsource catalog \citep{CatWISE}, we
crossmatched the brighest stars in the W1 band against \Gaia\ DR2, and used the
estimated physical parameters from that catalogue to model the bright stars. In
addition, we created a general ``faint source'' template based on the stars
that were not part of the bright star component. Together, these two components
comprise the sixth and seventh component in eq. (\ref{eq:skymodel}).

\begin{table*}
    \centering
    \caption{Components enabled for each frequency band}
    \begin{tabular}{c|c|c|c|c}
        \label{tab:bands}
        Band & Hot dust & Cold dust & Nearby dust & H$\alpha$-correlated dust \\
        \hline
        DIRBE 240 $\mu m$ & x & x & x & x \\
        DIRBE 140 $\mu m$ & x & & x & x \\
        DIRBE 100 $\mu m$ & x & & x & x \\
        DIRBE 60 $\mu m$ & x & & x & x \\
        DIRBE 25 $\mu m$ & x & & x & x \\
        DIRBE 12 $\mu m$ & x & & x & x \\
        DIRBE 4.9 $\mu m$ & x & & & \\
        DIRBE 3.5 $\mu m$ & x & & & 
    \end{tabular} 
\end{table*}

\subsection{Data selection}
As noted in \citet{CG02_05}, not all thermal dust components are expected to
contribute to the DIRBE bands we are considering in this paper (i.e.
240-12$\mu$m). Following the same logic as in that paper, we restrict the dust
components to be active in the various DIRBE bands as indicated in Table
\ref{tab:bands}. NEED EXPLANATION HERE.

\subsection{Masks}
\label{sec:masks}
As discussed in \citet{CG02_01}...

\section{Results}

In what follows, we have discarded the first N samples of the Gibbs chain as
burnin.
\subsection{Markov chains, correlations and convergence}
In Figs. \ref{fig:trace_colddust}-\ref{fig:trace_hacorr_dust} we plot the Gibbs
chain traceplots for all the five Gibbs chains used in the analysis, for all
four dust components. Despite some residual burnin, the chains are generally
well mixed, with a slightly longer correlation length for the two first bins of
the H$\alpha$-correlated component.

In Fig. \ref{fig:corrmat}, we show the correlations between the various SED
bins (WAIT TO SEE WHAT IT LOOKS LIKE WITH LESS BURNIN).

%\begin{figure}
%  \centering
%  \includegraphics[width=\columnwidth]{figures/traceplots.pdf}
%  \caption{Traceplots.}
%  \label{fig:trace}
%\end{figure}
\begin{figure}
  \centering
  \includegraphics[width=\columnwidth]{figures/traceplots_dust_seds_cold_dust.pdf}
  \caption{Cold dust amplitude as a function of iteration for the 240$\mu$m channel where it is included. The five lines correspond to the five independent sampling chains in the analysis. We see robust mixing in all chains.}
  \label{fig:trace_colddust}
\end{figure}
\begin{figure}
  \centering
  \includegraphics[width=\columnwidth]{figures/traceplots_dust_seds_hot_dust.pdf}
  \caption{Hot dust amplitude as a function of iteration for the eight lowest frequency DIRBE channels, with all five sampling chains overplotted. We see that the 3.5$\mu$m and 4.9$\mu$m channels exhibit slower mixing that the others, but still manage to explore the full parameter space.}
  \label{fig:trace_hotdust}
\end{figure}
\begin{figure}
  \centering
  \includegraphics[width=\columnwidth]{figures/traceplots_dust_seds_nearby_dust.pdf}
  \caption{Same as Fig. \ref{fig:trace_hotdust}, but for the nearby dust amplitudes in the six channels where they are sampled.}
  \label{fig:trace_nearbydust}
\end{figure}
\begin{figure}
  \centering
  \includegraphics[width=\columnwidth]{figures/traceplots_dust_seds_ha-correlated_dust.pdf}
  \caption{Same as Fig. \ref{fig:trace_hotdust}, but for the H$\alpha$ dust amplitudes in the six channels where they are sampled.}
  \label{fig:trace_hacorr_dust}
\end{figure}


\begin{figure}
  \centering
  \includegraphics[width=\columnwidth]{figures/corrmat.pdf}
  \caption{Correlations between the dust amplitudes at each frequency, as computed over the full sample set. We see that they are largely uncorrelated, but we see some structures within some bands as the sampler trades off between components (eg. 240$\mu$m in the upper left). We also see some fainter structure in the nearby dust amplitudes across channels.}
  \label{fig:corrmat}
\end{figure}



\subsection{Multi-component thermal dust SED posteriors}

In Table \ref{tab:tempamp}, we summarize the thermal dust SED posteriors
resulting from the above analysis as the mean value and variance of the chain
samples (including all five chains of the analysis). Similarly, we plot the
posterior mean values per bin in Fig. \ref{fig:total_sed}, where the thickness
of the line indicates the standard deviation of that amplitude. 

Clearly, both the hot dust and nearby dust components follow a typical thermal
dust modified blackbody curve, mirroring the results found in \citet{CG02_05}.
At higher frequencies, we also see that they exhibit the characteristic rise in
the SED, where models like Astrodust+PAH predicts emission lines from the
nanoscale particles. The figure also shows that  fitting the total (i.e. the
sum of all the components) dust model to the Astrodust+PAH model yields a
version of the latter that seems to fit very well with the posterior mean SED
values.

We cannot draw any such conclusions about the the cold dust component, as it is
only active for the lowest SED bin; in any case, the hot dust component clearly
dominate this bin.

The H$\alpha$ component, as in the previous analyses, exhibits a negative SED,
and is thus responsible for dust extinction rather than emission.


\begin{table*}
\newdimen\tblskip \tblskip=5pt
\caption{Summary of dust template amplitude posterior constraints.}
\label{tab:tempamp}
\vskip -4mm
\footnotesize
\setbox\tablebox=\vbox{
 \newdimen\digitwidth
 \setbox0=\hbox{\rm 0}
 \digitwidth=\wd0
 \catcode`*=\active
 \def*{\kern\digitwidth}
%
  \newdimen\dpwidth
  \setbox0=\hbox{.}
  \dpwidth=\wd0
  \catcode`!=\active
  \def!{\kern\dpwidth}
%
  \halign{\hbox to 1.7cm{#\leaderfil}\tabskip 2em&
    \hfil$#$\hfil \tabskip 0.5em&
    \hfil$#$\hfil \tabskip 1em& 
    \hfil$#$\hfil \tabskip 1em& 
    \hfil$#$\hfil \tabskip 1em& 
    \hfil$#$\hfil \tabskip 1em& 
    \hfil$#$\hfil \tabskip 1em\cr
\noalign{\doubleline}
\omit& \multispan2\hfil\sc Band (THz) \hfil&\multispan4\hfil\sc Template amplitude (unit) \hfil\cr
\noalign{\vskip -3pt}
\omit& \multispan2\hrulefill& \multispan4\hrulefill\cr
\noalign{\vskip 3pt} 
\omit\sc $\lambda$ ($\mu\mathrm{m}$)\hfil& \nu_{\mathrm{min}} & \nu_{\mathrm{max}} & \a_{\mathrm{h}} & \a_{\mathrm{c}} & \a_{\mathrm{n}} & \a_{\mathrm{H}\alpha}\cr
\noalign{\vskip 3pt\hrule\vskip 5pt}
*3.5  & 70.2 & 109.5 & 0.05\pm0.00 & 0.00\pm0.01 & 0.01\pm0.00 & 0.00\pm0.01 \cr
*4.9  & 45.5 & *70.2 & 0.01\pm0.00 & 0.00\pm0.01 & 0.01\pm0.00 & 0.00\pm0.01 \cr
*12   & 18.1 & *45.5 & 0.30\pm0.00 & 0.00\pm0.01 & 0.19\pm0.00 & -0.32\pm0.01 \cr
*25   & *9.0 & *18.1 & 0.22\pm0.00 & 0.00\pm0.01 & 0.13\pm0.00 & -0.28\pm0.01 \cr
*60   & *4.1 & **9.0 & 2.45\pm0.00 & 0.00\pm0.01 & 0.31\pm0.00 & -2.41\pm0.03 \cr
100   & *2.5 & **4.1 & 10.92\pm0.03 & 0.00\pm0.01 & 2.61\pm0.00 & -12.40\pm0.02 \cr
140   & *1.7 & **2.5 & 24.58\pm0.01 & 0.00\pm0.01 & 10.04\pm0.01 & -23.40\pm0.08 \cr
240   & *1.1 & **1.7 & 11.33\pm0.01 & 2.60\pm0.01 & 8.09\pm0.01 & -8.94\pm0.04 \cr
\noalign{\vskip 5pt\hrule\vskip 5pt}}}
\endPlancktablewide
\par
\end{table*}
\begin{figure}
  \centering
  \includegraphics[width=\columnwidth]{figures/all_components_sed.pdf}
  \caption{The total dust SED as a function of frequency, as well as the three positive dust components. The best-fit astrodust model fit to the total SED for the frequencies shown is also plotted in brown. \textbf{MG: we need to either cut h alpha from the legend or plot |h alpha|}}
  \label{fig:total_sed}
\end{figure}

%\begin{figure}
%  \centering
%  \includegraphics[width=\columnwidth]{figures/cold_dust_sed.pdf}
%  \caption{Cold dust SED.}
%  \label{fig:cold_dust_sed}
%\end{figure}
%\begin{figure}
%  \centering
%  \includegraphics[width=\columnwidth]{figures/hot_dust_sed.pdf}
%  \caption{Hot dust SED.}
%  \label{fig:hot_dust_sed}
%\end{figure}
%\begin{figure}
%  \centering
%  \includegraphics[width=\columnwidth]{figures/nearby_dust_sed.pdf}
%  \caption{Nearby dust SED.}
%  \label{fig:nearby_dust_sed}
%\end{figure}
%
%\begin{figure}
%  \centering
%  \includegraphics[width=\columnwidth]{figures/nearby_dust_sed.pdf}
%  \caption{H$\alpha$ SED.}
%  \label{fig:wham_sed}
%\end{figure}

\subsection{Model efficiency and residuals}

In Fig. \ref{fig:dustmap}, we show the raw dust maps -- that is, the frequency
maps with starlight and free-free signal subtracted -- with the residual
frequency maps after subtraction of the four-component dust model used in this
analysis. The grey pixels indicate the masks used at each frequency, as
described in Section \ref{sec:masks}.

In all the bands to some degree, we see in the residuals a weak pattern of
over-and-undersubtractions in the galactic plane, suggesting there might be
smaller regions here driving the fit which should be modelled separately. At
the higher frequencies, the model is clearly struggling to account for emission
from the galactic center.

IS THE EFFICIENCY FIGURE ACCURATE?

\begin{figure*}
  \centering
  \includegraphics[width=0.235\linewidth]{figures/paperVII_10_dustres_v1.pdf}
  \includegraphics[width=0.235\linewidth]{figures/paperVII_10_todres_v1.pdf}\hspace*{5mm}
  \includegraphics[width=0.235\linewidth]{figures/paperVII_06_dustres_v1.pdf}
  \includegraphics[width=0.235\linewidth]{figures/paperVII_06_todres_v1.pdf}\\
  \includegraphics[width=0.235\linewidth]{figures/paperVII_09_dustres_v1.pdf}
  \includegraphics[width=0.235\linewidth]{figures/paperVII_09_todres_v1.pdf}\hspace*{5mm}
  \includegraphics[width=0.235\linewidth]{figures/paperVII_05_dustres_v1.pdf}
  \includegraphics[width=0.235\linewidth]{figures/paperVII_05_todres_v1.pdf}\\
  \includegraphics[width=0.235\linewidth]{figures/paperVII_08_dustres_v1.pdf}
  \includegraphics[width=0.235\linewidth]{figures/paperVII_08_todres_v1.pdf}\hspace*{5mm}
  \includegraphics[width=0.235\linewidth]{figures/paperVII_04_dustres_v1.pdf}
  \includegraphics[width=0.235\linewidth]{figures/paperVII_04_todres_v1.pdf}\\
  \includegraphics[width=0.235\linewidth]{figures/paperVII_07_dustres_v1.pdf}
  \includegraphics[width=0.235\linewidth]{figures/paperVII_07_todres_v1.pdf}\hspace*{5mm}
  \includegraphics[width=0.235\linewidth]{figures/paperVII_03_dustres_v1.pdf}
  \includegraphics[width=0.235\linewidth]{figures/paperVII_03_todres_v1.pdf}
  \caption{Comparison between thermal dust frequency maps (i.e., stationary sky signal minus starlight and free-free; \emph{first and third columns}) and residual maps for each frequency channel (\emph{second and fourth column}). Gray pixels indicate the analysis masks used for each frequency channel.}
  \label{fig:dustmaps}
\end{figure*}


\begin{figure}
  \centering
  \includegraphics[width=\columnwidth]{figures/template_signal_rms_v1.pdf}
  \caption{Dust model efficiency as a function of frequency, as defined in terms of variance reduction. The red line shows results for DIRBE, as evaluated from the maps shown in Fig.~\ref{fig:dustmaps}, while the blue line shows results for \Planck\ HFI, as presented by \citet{CG02_06}. Vertical bands indicate the position and bandwidth of each DIRBE (orange) and HFI (cyan) frequency channel.}
  \label{fig:efficiency}
\end{figure}

\begin{figure}
  \centering
  \includegraphics[width=\linewidth]{figures/paperVII_07_todres_zoom_v1.pdf}\\
  \includegraphics[width=\linewidth]{figures/paperVII_08_todres_zoom_v1.pdf}
  \caption{Full-sky data-minus-model residual maps for the 60 (\emph{top}) and 100\,$\mu$m (\emph{bottom}) DIRBE channels. \textbf{MG: not sure what this is in aid of as it is not referenced in the text}}
  \label{fig:res0708}
\end{figure}




\clearpage
\section{\Cosmoglobe\ DR2 sky model}

\subsection{Summary of global sky model}

In Fig. \ref{fig:SED_overview} we show the best-fit \Cosmoglobe\ sky model,
derived from the first and second data releases, including the four-component
dust model presented in this paper.


\begin{figure*}
  \centering
  \includegraphics[width=\textwidth]{figures/new_all_fgs_bands.png}
  \caption{Overview of the large-scale microwave and infrared sky from 1 GHz to 1$\mu$m, based on the \Cosmoglobe\ DR1 and DR2 data. The four component dust model is shown, as well as the best fit astrodust model. The \Planck\ HFI and DIRBE central frequencies are indicated with  vertical lines. \textbf{MG: I stole this off mattermost, someone should add a final version}}
  \label{fig:SED_overview}
\end{figure*}


\begin{figure*}
  \centering
  \includegraphics[width=0.49\linewidth]{figures/CosmoglobeDR2_VII_545-1_dust_cold_v1.pdf}
  \includegraphics[width=0.49\linewidth]{figures/CosmoglobeDR2_VII_545-1_dust_hot_v1.pdf}\\
  \includegraphics[width=0.49\textwidth]{figures/CosmoglobeDR2_VII_545-1_dust_near_v1.pdf}
  \includegraphics[width=0.49\textwidth]{figures/CosmoglobeDR2_VII_545-1_dust_Ha_v1.pdf}
  \caption{Dust amplitude maps used in the \Cosmoglobe\ DR2 sky model, as evaluated for the \Planck\ HFI 545-1 bolometer channel. From top to bottom and left to right, the four panels show 1) the cold dust amplitude, $\a_{\mathrm{c}}$; 2) the hot dust amplitude, $\a_{\mathrm{h}}$; 3) the nearby dust amplitude, $\a_{\mathrm{n}}$; and 4) the (absolute value of the) H$_{\alpha}$-correlated dust extinction amplitude, $\a_{\mathrm{H}\alpha}$. All panels employ the \Planck\ non-linear high dynamic range color scheme, defined by $\log_{10}((\a + \sqrt{4+\a^2})/2)$, which results in a nearly linear behaviour for small values and exponential for large values.  }
  \label{fig:templates}
\end{figure*}


\subsection{Component-wise decomposition}
The final posterior fits for the dust components are shown in Fig.
\ref{fig:templates}.  As is evident, the maps are morphologically distinct, and
similarly to what was concluded in \citet{CG02_06}, the morphologies correlate
with those of CII and HI emission maps (DON'T KNOW IF WE WANT THIS). The
H$\alpha$ component is also clearly morphologically similar to the WHAM
H$\alpha$ maps, and likely originates in nearby bubbles of star formation.

Thus, it seems to be a robust conclusion that both for the \Planck\ HFI and
DIRBE regimes, the dust is ``naturally'' decomposable into a relatively low
number of constituent components, and that those components tend to correlate
with traces of various kinds of activity in the Milky Way.

Finally, in \ref{fig:comp_vs_freq}, we show the amplitude maps of all the
components involved in the \Cosmoglobe\ DR2 analysis, including zodiacal light,
dust, starlight, and free-free emission.
       \begin{figure*}
         \centering
         \includegraphics[width=0.16\linewidth]{figures/compfreq_mapzodi_10_v02.pdf}
         \includegraphics[width=0.16\linewidth]{figures/compfreq_zodi_10_v02.pdf}
         \includegraphics[width=0.16\linewidth]{figures/compfreq_dusttot_10_v02.pdf}
         \includegraphics[width=0.16\linewidth]{figures/compfreq_white_nobar.pdf}
         \includegraphics[width=0.16\linewidth]{figures/compfreq_ff_10_v02.pdf}
         \includegraphics[width=0.16\linewidth]{figures/compfreq_todres_10_v02.pdf}\\
         \includegraphics[width=0.16\linewidth]{figures/compfreq_mapzodi_09_v02.pdf}
         \includegraphics[width=0.16\linewidth]{figures/compfreq_zodi_09_v02.pdf}
         \includegraphics[width=0.16\linewidth]{figures/compfreq_dusttot_09_v02.pdf}
         \includegraphics[width=0.16\linewidth]{figures/compfreq_white_nobar.pdf}
         \includegraphics[width=0.16\linewidth]{figures/compfreq_ff_09_v02.pdf}
         \includegraphics[width=0.16\linewidth]{figures/compfreq_todres_09_v02.pdf}\\
         \includegraphics[width=0.16\linewidth]{figures/compfreq_mapzodi_08_v02.pdf}
         \includegraphics[width=0.16\linewidth]{figures/compfreq_zodi_08_v02.pdf}
         \includegraphics[width=0.16\linewidth]{figures/compfreq_dusttot_08_v02.pdf}
         \includegraphics[width=0.16\linewidth]{figures/compfreq_white_nobar.pdf}
         \includegraphics[width=0.16\linewidth]{figures/compfreq_ff_08_v02.pdf}
         \includegraphics[width=0.16\linewidth]{figures/compfreq_todres_08_v02.pdf}\\
         \includegraphics[width=0.16\linewidth]{figures/compfreq_mapzodi_07_v02.pdf}
         \includegraphics[width=0.16\linewidth]{figures/compfreq_zodi_07_v02.pdf}
         \includegraphics[width=0.16\linewidth]{figures/compfreq_dusttot_07_v02.pdf}
         \includegraphics[width=0.16\linewidth]{figures/compfreq_white_nobar.pdf}
         \includegraphics[width=0.16\linewidth]{figures/compfreq_ff_07_v02.pdf}
         \includegraphics[width=0.16\linewidth]{figures/compfreq_todres_07_v02.pdf}\\
         \includegraphics[width=0.16\linewidth]{figures/compfreq_mapzodi_06_v02.pdf}
         \includegraphics[width=0.16\linewidth]{figures/compfreq_zodi_06_v02.pdf}
         \includegraphics[width=0.16\linewidth]{figures/compfreq_dusttot_06_v02.pdf}
         \includegraphics[width=0.16\linewidth]{figures/compfreq_stars_06_v02.pdf}
         \includegraphics[width=0.16\linewidth]{figures/compfreq_ff_06_v02.pdf}
         \includegraphics[width=0.16\linewidth]{figures/compfreq_todres_06_v02.pdf}\\  
         \includegraphics[width=0.16\linewidth]{figures/compfreq_mapzodi_05_v02.pdf}
         \includegraphics[width=0.16\linewidth]{figures/compfreq_zodi_05_v02.pdf}
         \includegraphics[width=0.16\linewidth]{figures/compfreq_dusttot_05_v02.pdf}
         \includegraphics[width=0.16\linewidth]{figures/compfreq_stars_05_v02.pdf}
         \includegraphics[width=0.16\linewidth]{figures/compfreq_ff_05_v02.pdf}
         \includegraphics[width=0.16\linewidth]{figures/compfreq_todres_05_v02.pdf}\\
         \includegraphics[width=0.16\linewidth]{figures/compfreq_mapzodi_04_v02.pdf}
         \includegraphics[width=0.16\linewidth]{figures/compfreq_zodi_04_v02.pdf}
         \includegraphics[width=0.16\linewidth]{figures/compfreq_dusttot_04_v02.pdf}
         \includegraphics[width=0.16\linewidth]{figures/compfreq_stars_04_v02.pdf}
         \includegraphics[width=0.16\linewidth]{figures/compfreq_ff_04_v02.pdf}
         \includegraphics[width=0.16\linewidth]{figures/compfreq_todres_04_v02.pdf}\\  
         \includegraphics[width=0.16\linewidth]{figures/compfreq_mapzodi_03_v02.pdf}
         \includegraphics[width=0.16\linewidth]{figures/compfreq_zodi_03_v02.pdf}
         \includegraphics[width=0.16\linewidth]{figures/compfreq_dusttot_03_v02.pdf}
         \includegraphics[width=0.16\linewidth]{figures/compfreq_stars_03_v02.pdf}
         \includegraphics[width=0.16\linewidth]{figures/compfreq_ff_03_v02.pdf}
         \includegraphics[width=0.16\linewidth]{figures/compfreq_todres_03_v02.pdf}\\  
         \includegraphics[width=0.16\linewidth]{figures/compfreq_mapzodi_02_v02.pdf}
         \includegraphics[width=0.16\linewidth]{figures/compfreq_zodi_02_v02.pdf}
         \includegraphics[width=0.16\linewidth]{figures/compfreq_white_nobar.pdf}
         \includegraphics[width=0.16\linewidth]{figures/compfreq_stars_02_v02.pdf}
         \includegraphics[width=0.16\linewidth]{figures/compfreq_ff_02_v02.pdf}
         \includegraphics[width=0.16\linewidth]{figures/compfreq_todres_02_v02.pdf}\\
         \includegraphics[width=0.16\linewidth]{figures/compfreq_mapzodi_01_v02.pdf}
         \includegraphics[width=0.16\linewidth]{figures/compfreq_zodi_01_v02.pdf}
         \includegraphics[width=0.16\linewidth]{figures/compfreq_white_nobar.pdf}
         \includegraphics[width=0.16\linewidth]{figures/compfreq_stars_01_v02.pdf}
         \includegraphics[width=0.16\linewidth]{figures/compfreq_ff_01_v02.pdf}
         \includegraphics[width=0.16\linewidth]{figures/compfreq_todres_01_v02.pdf}\\
         \includegraphics[width=0.50\linewidth]{figures/colourbar_MJysr.pdf}
         \caption{Comparison between the raw DIRBE data and the various fitted components for one single Gibbs sample. Columns show, from left to right, 1) the time-ordered DIRBE data co-added into pixelized maps; 2) zodiacal light emission; 3) thermal dust emission; 4) star emission; 5) free-free emission; and 6) data-minus-model residual emission. Rows show individual frequency channels. Missing entries corresponds to components that are forced to zero in the model. Note that all panels are plotted with the same color scale in units of MJy/sr, and can be directly compared. \textbf{MG: stars needs a label}}
         \label{fig:comp_vs_freq}
       \end{figure*}


       


\clearpage
\section{Conclusions}
In this analysis, which is part of the second \Cosmoglobe\ data release, we
have described the incorporation of a four-component model as part of the
modelling of thermal dust at the eight lowest DIRBE frequency bands. This
model was inspired by recent analyses which demonstrated the feasibility and
efficiency of decomposing thermal dust into a relatively low number of distinct
populations with global spectral parameters.

Applying this model to DIRBE data, and inspired by the abovementioned
analyses, we defined the four components as ``hot dust'', ``cold dust'',
``nearby dust'' and ``H$\alpha$-correlated dust'', respectively, and 
the SED of each component was defined as a constant bin over each DIRBE band.

With this model as part of the larger DR2 analysis, we showed that the
resulting total SED is an excellent match to the Astrodust+PAH model, and that
the nanoparticle emission at high frequencies is detected in these components.
Furthermore, we show that the morphologies of the resulting component maps are
both similar to those found in a previous HFI-based analysis with the same
methodology, and that the hot and cold dust components are morphologically
similar to tracers of CII and HI emission, respectively. As in previous
analyses, the H$\alpha$-correlated component absorbs rather than emits
radiation, and thus has a negative SED. It is clearly also correlated with
H$\alpha$.

Finally, we present the current state of the \Cosmoglobe\ sky model, ranging
from a few GHz to ~$10^3$ THz, and containing such diverse components as the
CMB, synchrotron radiation, zodiacal light, starlight, the various thermal
dust components, and the CIB.

Although the results in this and the preceding analyses are very encouraging
for the prospect of efficient thermal dust modelling, it is clearly also
something that is in need of further development....... Also, this analysis
only utilizes one data set in addition to the starlight data. An obvious
extension would be to perform a joint analysis ranging across the entirety of
the thermal dust spectrum, with higher-resolution data both in the spatial
domain and in the frequency domain.

A promising combination of data, which is definitely within the realm of
possibility in a joint analysis framework such as \Cosmoglobe, could be as
follows: 1) \Planck\ low-level data for the low end of the thermal dust spectrum,
and correlation with various types of line emission; 2) DIRBE, as in the
present analysis, to constrain the upper part of the thermal dust spectrum,
including the PAH features; 3) AKARI \citep{murakami:2007}, observing in the
same range as DIRBE, but with higher resolution, thus shedding light on an
otherwise data-poor part of the electromagnetic spectrum; and 4) IRAS, to provide
....



\begin{acknowledgements}
 The current work has received funding from the European
  Union’s Horizon 2020 research and innovation programme under grant
  agreement numbers 819478 (ERC; \textsc{Cosmoglobe}) and 772253 (ERC;
  \textsc{bits2cosmology}). Some of the results in this paper have been derived using the HEALPix \citep{HEALPIX} package.
  We acknowledge the use of the Legacy Archive for Microwave Background Data
  Analysis (LAMBDA), part of the High Energy Astrophysics Science Archive Center
  (HEASARC). HEASARC/LAMBDA is a service of the Astrophysics Science Division at
  the NASA Goddard Space Flight Center.  
  This paper and related research have been conducted during and with the support of the Italian national inter-university PhD programme in Space Science and Technology. Work on this article was produced while attending the PhD program in PhD in Space Science and Technology at the University of Trento, Cycle XXXIX, with the support of a scholarship financed by the Ministerial Decree no. 118 of 2nd March 2023, based on the NRRP - funded by the European Union - NextGenerationEU - Mission 4 "Education and Research", Component 1 "Enhancement of the offer of educational services: from nurseries to universities” - Investment 4.1 “Extension of the number of research doctorates and innovative doctorates for public administration and cultural heritage” - CUP E66E23000110001.
\end{acknowledgements}


%-------------------------------------------------------------
%                                       Table with references 
%-------------------------------------------------------------
%

\bibliographystyle{aa}
\bibliography{../../common/CG_bibliography,references,../../common/Planck_bib}
\end{document}
%%%% End of aa.dem




\begin{table*}
    \centering
    \caption{Defintion of the SED bins used in this paper}
    \begin{tabular}{c|c|c|c}
        \label{tab:bins}
        Bin number & Frequency range (GHz) & Wavelength range ($\mu$m) & DIRBE band correspondence \\
        \hline
        1 & 1050-1667 & 286.5-179.8 & 10 \\
        2 & 1667-2540 & 179.8-118.0 & 9 \\
        3 & 2540-4064 & 118.0-73.8 & 8 \\
        4 & 4064-9000 & 73.8-33.3 & 7 \\
        5 & 9000-18100 & 33.3-16.6 & 6 \\
        6 & 18100-45500 & 16.6-6.6 & 5  \\
        7 & 45500-70215 & 6.6-4.3 & 4 \\
        8 & 70215-109490 & 4.3-2.7 & 3
    \end{tabular}
\end{table*}
