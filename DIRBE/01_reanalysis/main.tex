%                                                                 aa.dem
% AA vers. 9.1, LaTeX class for Astronomy & Astrophysics
% demonstration file
%                                                       (c) EDP Sciences
%-----------------------------------------------------------------------
%
% \documentclass[referee]{aa} % for a referee version
%\documentclass[onecolumn]{aa} % for a paper on 1 column  
%\documentclass[longauth]{aa} % for the long lists of affiliations 
%\documentclass[letter]{aa} % for the letters 
%\documentclass[bibyear]{aa} % if the references are not structured 
%                              according to the author-year natbib style

%

\documentclass{aa}  

%
\usepackage{graphicx}
\usepackage{amsmath,amsfonts,amssymb}
\usepackage{natbib}


%%%%%%%%%%%%%%%%%%%%%%%%%%%%%%%%%%%%%%%%
\usepackage{txfonts}
\usepackage{xcolor}

\usepackage{blindtext}
%%%%%%%%%%%%%%%%%%%%%%%%%%%%%%%%%%%%%%%%
% \usepackage[options]{hyperref}
% To add links in your PDF file, use the package "hyperref"
% with options according to your LaTeX or PDFLaTeX drivers.
\usepackage{float}
%\usepackage{stfloats}
\usepackage{dblfloatfix}
\usepackage{afterpage}
\usepackage{ifthen}
\usepackage[morefloats=12]{morefloats}

\usepackage{placeins}
\usepackage{multicol}
%\usepackage[breaklinks,colorlinks,citecolor=blue]{hyperref}
\bibpunct{(}{)}{;}{a}{}{,}
\usepackage[switch]{lineno}
\definecolor{linkcolor}{rgb}{0.6,0,0}
\definecolor{citecolor}{rgb}{0,0,0.75}
\definecolor{urlcolor}{rgb}{0.12,0.46,0.7}
\usepackage[breaklinks, colorlinks, urlcolor=urlcolor,
    linkcolor=linkcolor,citecolor=citecolor,pdfencoding=auto]{hyperref}
\hypersetup{linktocpage}
\usepackage{bold-extra}



\def\setsymbol#1#2{\expandafter\def\csname #1\endcsname{#2}}
\def\getsymbol#1{\csname #1\endcsname}

\def\Planck{\textit{Planck}}

\def\HeJT{$^4$He-JT}

\def\allearlypapers{\nocite{planck2011-1.1, planck2011-1.3, planck2011-1.4, planck2011-1.5, planck2011-1.6, planck2011-1.7, planck2011-1.10, planck2011-1.10sup, planck2011-5.1a, planck2011-5.1b, planck2011-5.2a, planck2011-5.2b, planck2011-5.2c, planck2011-6.1, planck2011-6.2, planck2011-6.3a, planck2011-6.4a, planck2011-6.4b, planck2011-6.6, planck2011-7.0, planck2011-7.2, planck2011-7.3, planck2011-7.7a, planck2011-7.7b, planck2011-7.12, planck2011-7.13}}

\def\alltwentythirteenresultspapers{\nocite{planck2013-p01, planck2013-p02, planck2013-p02a, planck2013-p02d, planck2013-p02b, planck2013-p03, planck2013-p03c, planck2013-p03f, planck2013-p03d, planck2013-p03e, planck2013-p01a, planck2013-p06, planck2013-p03a, planck2013-pip88, planck2013-p08, planck2013-p11, planck2013-p12, planck2013-p13, planck2013-p14, planck2013-p15, planck2013-p05b, planck2013-p17, planck2013-p09, planck2013-p09a, planck2013-p20, planck2013-p19, planck2013-pipaberration, planck2013-p05, planck2013-p05a, planck2013-pip56, planck2013-p06b, planck2013-p01a}}

\def\alltwentyfifteenresultspapers{\nocite{planck2014-a01, planck2014-a03, planck2014-a04, planck2014-a05, planck2014-a06, planck2014-a07, planck2014-a08, planck2014-a09, planck2014-a11, planck2014-a12, planck2014-a13, planck2014-a14, planck2014-a15, planck2014-a16, planck2014-a17, planck2014-a18, planck2014-a19, planck2014-a20, planck2014-a22, planck2014-a24, planck2014-a26, planck2014-a28, planck2014-a29, planck2014-a30, planck2014-a31, planck2014-a35, planck2014-a36, planck2014-a37, planck2014-ES}}

\newbox\tablebox    \newdimen\tablewidth
\def\leaderfil{\leaders\hbox to 5pt{\hss.\hss}\hfil}
\def\endPlancktable{\tablewidth=\columnwidth 
    $$\hss\copy\tablebox\hss$$
    \vskip-\lastskip\vskip -2pt}
\def\endPlancktablewide{\tablewidth=\textwidth 
    $$\hss\copy\tablebox\hss$$
    \vskip-\lastskip\vskip -2pt}
\def\tablenote#1 #2\par{\begingroup \parindent=0.8em
    \abovedisplayshortskip=0pt\belowdisplayshortskip=0pt
    \noindent
    $$\hss\vbox{\hsize\tablewidth \hangindent=\parindent \hangafter=1 \noindent
    \hbox to \parindent{$^#1$\hss}\strut#2\strut\par}\hss$$
    \endgroup}
\def\doubleline{\vskip 3pt\hrule \vskip 1.5pt \hrule \vskip 5pt}

\def\L2{\ifmmode L_2\else $L_2$\fi}
\def\dtt{\Delta T/T}
\def\DeltaT{\ifmmode \Delta T\else $\Delta T$\fi}
\def\deltat{\ifmmode \Delta t\else $\Delta t$\fi}
\def\fknee{\ifmmode f_{\rm knee}\else $f_{\rm knee}$\fi}
\def\Fmax{\ifmmode F_{\rm max}\else $F_{\rm max}$\fi}
\def\solar{\ifmmode{\rm M}_{\mathord\odot}\else${\rm M}_{\mathord\odot}$\fi}
\def\Msolar{\ifmmode{\rm M}_{\mathord\odot}\else${\rm M}_{\mathord\odot}$\fi}
\def\Lsolar{\ifmmode{\rm L}_{\mathord\odot}\else${\rm L}_{\mathord\odot}$\fi}
\def\inv{\ifmmode^{-1}\else$^{-1}$\fi}
\def\mo{\ifmmode^{-1}\else$^{-1}$\fi}
\def\sup#1{\ifmmode ^{\rm #1}\else $^{\rm #1}$\fi}
\def\expo#1{\ifmmode \times 10^{#1}\else $\times 10^{#1}$\fi}
\def\,{\thinspace}
\def\lsim{\mathrel{\raise .4ex\hbox{\rlap{$<$}\lower 1.2ex\hbox{$\sim$}}}}
\def\gsim{\mathrel{\raise .4ex\hbox{\rlap{$>$}\lower 1.2ex\hbox{$\sim$}}}}
\let\lea=\lsim
\let\gea=\gsim
\def\simprop{\mathrel{\raise .4ex\hbox{\rlap{$\propto$}\lower 1.2ex\hbox{$\sim$}}}}
\def\deg{\ifmmode^\circ\else$^\circ$\fi}
\def\pdeg{\ifmmode $\setbox0=\hbox{$^{\circ}$}\rlap{\hskip.11\wd0 .}$^{\circ}
          \else \setbox0=\hbox{$^{\circ}$}\rlap{\hskip.11\wd0 .}$^{\circ}$\fi}
\def\arcs{\ifmmode {^{\scriptstyle\prime\prime}}
          \else $^{\scriptstyle\prime\prime}$\fi}
\def\arcm{\ifmmode {^{\scriptstyle\prime}}
          \else $^{\scriptstyle\prime}$\fi}
\newdimen\sa  \newdimen\sb
\def\parcs{\sa=.07em \sb=.03em
     \ifmmode \hbox{\rlap{.}}^{\scriptstyle\prime\kern -\sb\prime}\hbox{\kern -\sa}
     \else \rlap{.}$^{\scriptstyle\prime\kern -\sb\prime}$\kern -\sa\fi}
\def\parcm{\sa=.08em \sb=.03em
     \ifmmode \hbox{\rlap{.}\kern\sa}^{\scriptstyle\prime}\hbox{\kern-\sb}
     \else \rlap{.}\kern\sa$^{\scriptstyle\prime}$\kern-\sb\fi}
\def\ra[#1 #2 #3.#4]{#1\sup{h}#2\sup{m}#3\sup{s}\llap.#4}
\def\dec[#1 #2 #3.#4]{#1\deg#2\arcm#3\arcs\llap.#4}
\def\deco[#1 #2 #3]{#1\deg#2\arcm#3\arcs}
\def\rra[#1 #2]{#1\sup{h}#2\sup{m}}
\def\page{\vfill\eject}
\def\dots{\relax\ifmmode \ldots\else $\ldots$\fi}
\def\WHzsr{\ifmmode $W\,Hz\mo\,sr\mo$\else W\,Hz\mo\,sr\mo\fi}
\def\mHz{\ifmmode $\,mHz$\else \,mHz\fi}
\def\GHz{\ifmmode $\,GHz$\else \,GHz\fi}
\def\mKs{\ifmmode $\,mK\,s$^{1/2}\else \,mK\,s$^{1/2}$\fi}
\def\muKs{\ifmmode \,\mu$K\,s$^{1/2}\else \,$\mu$K\,s$^{1/2}$\fi}
\def\muKRJs{\ifmmode \,\mu$K$_{\rm RJ}$\,s$^{1/2}\else \,$\mu$K$_{\rm RJ}$\,s$^{1/2}$\fi}
\def\muKHz{\ifmmode \,\mu$K\,Hz$^{-1/2}\else \,$\mu$K\,Hz$^{-1/2}$\fi}
\def\MJysr{\ifmmode \,$MJy\,sr\mo$\else \,MJy\,sr\mo\fi}
\def\MJysrmK{\ifmmode \,$MJy\,sr\mo$\,mK$_{\rm CMB}\mo\else \,MJy\,sr\mo\,mK$_{\rm CMB}\mo$\fi}
\def\microns{\ifmmode \,\mu$m$\else \,$\mu$m\fi}
\def\micron{\microns}
\def\muK{\ifmmode \,\mu$K$\else \,$\mu$\hbox{K}\fi}
\def\microK{\ifmmode \,\mu$K$\else \,$\mu$\hbox{K}\fi}
\def\muW{\ifmmode \,\mu$W$\else \,$\mu$\hbox{W}\fi}
\def\kms{\ifmmode $\,km\,s$^{-1}\else \,km\,s$^{-1}$\fi}
\def\kmsMpc{\ifmmode $\,\kms\,Mpc\mo$\else \,\kms\,Mpc\mo\fi}

\providecommand{\sorthelp}[1]{}


% Custom definitions
%\newcommand{\mathsc}[1]{{\normalfont\textsc{#1}}}
\def\Cosmoglobe{\textsc{Cosmoglobe}}
\def\commanderthree{\texttt{Commander3}}
\def\commander{\texttt{Commander}}
\def\Planck{\textit{Planck}}
\def\WMAP{\textit{WMAP}}
\def\COBE{\textit{COBE}}
\def\GAIA{\textit{Gaia}}
\def\gaia{\textit{Gaia}}
\def\Gaia{\textit{Gaia}}
\def\WISE{WISE}

\newcommand{\CII}{\ensuremath{\textsc{C\,ii}}}

\newcommand{\phm}{\phantom{-}}
\newcommand{\dv}[0]{\vec{d}}
\renewcommand{\t}[0]{\vec{t}}
\newcommand{\A}[0]{\tens{A}}
\newcommand{\B}[0]{\tens{B}}
\newcommand{\Y}[0]{\tens{Y}}
\newcommand{\G}[0]{\tens{G}}
\newcommand{\n}[0]{\vec{n}}
\newcommand{\red}[0]{\color{red}}
\newcommand{\green}[0]{\color{green}}
\newcommand{\s}[0]{\vec{s}}
\renewcommand{\a}[0]{\vec{a}}
\newcommand{\m}[0]{\vec{m}}
\newcommand{\bv}[0]{\vec{b}}
\newcommand{\f}[0]{\vec{f}}
\newcommand{\F}[0]{\tens{F}}
\newcommand{\T}[0]{\tens{T}}
\newcommand{\Cp}[0]{\tens{C}}
\renewcommand{\L}[0]{\tens{L}}
\newcommand{\g}[0]{\vec{g}}
\newcommand{\N}[0]{\tens{N}}
\newcommand{\M}[0]{\tens{M}}
\newcommand{\iN}[0]{\tens{N}^{-1}}
\newcommand{\iM}[0]{\tens{M}^{-1}}
\newcommand{\w}[0]{\vec{w}}
\renewcommand{\S}[0]{\tens{S}}
\renewcommand{\r}[0]{\vec{r}}
\renewcommand{\u}[0]{\vec{u}}
\newcommand{\q}[0]{\vec{q}}
\renewcommand{\v}[0]{\vec{v}}
\renewcommand{\P}[0]{\tens{P}}
\newcommand{\dt}[0]{d_t}
\newcommand{\di}[0]{d_i}
\newcommand{\nt}[0]{n_t}
\newcommand{\st}[0]{s_t}
\newcommand{\mt}[0]{m_t}
\newcommand{\ft}[0]{f_t}
\newcommand{\Te}[0]{T_{\rm e}}
\newcommand{\EM}[0]{\rm EM}
\newcommand{\mathsc}[1]{{\normalfont\textsc{#1}}}
\newcommand{\hi}{\ensuremath{\mathsc {H\,i}}}
\newcommand{\bpbold}{\bfseries{\scshape{BeyondPlanck}}}
\newcommand{\BP}{\textsc{BeyondPlanck}}
\newcommand{\bp}{\textsc{BeyondPlanck}}
\newcommand{\cosmoglobe}{\textsc{Cosmoglobe}}
%\newcommand{\Cosmoglobe}{\textsc{Cosmoglobe}}
\newcommand{\lfi}[0]{LFI}
\newcommand{\hfi}[0]{HFI}
\newcommand{\npipe}[0]{\texttt{NPIPE}}
\newcommand{\K}[0]{\textit K}
\newcommand{\Ka}[0]{\textit{Ka}}
\newcommand{\Q}[0]{\textit Q}
\newcommand{\V}[0]{\textit V}
\newcommand{\W}[0]{\textit W}
\newcommand{\e}{\mathrm e}
\newcommand{\cvar}{\ensuremath{c(\vartheta, \varphi, \psi)}}


\def\Tcmb{\ifmmode T_\mathrm{CMB}\else $T_{\mathrm{CMB}}$\fi}
\def\Tcold{\ifmmode T_\mathrm{c}\else $T_{\mathrm{c}}$\fi}
\def\Thot{\ifmmode T_\mathrm{h}\else $T_{\mathrm{h}}$\fi}
\def\Tnear{\ifmmode T_\mathrm{n}\else $T_{\mathrm{n}}$\fi}
\def\scmb{\ifmmode s_\mathrm{CMB}\else $s_{\mathrm{CMB}}$\fi}
\def\squad{\ifmmode s_\mathrm{quad}\else $s_{\mathrm{quad}}$\fi}
\def\ssynch{\ifmmode s_\mathrm{s}\else $s_\mathrm{s}$\fi}
\def\sdust{\ifmmode s_\mathrm{d}\else $s_{\mathrm{d}}$\fi}
\def\ssdust{\ifmmode s_\mathrm{sd}\else $s_{\mathrm{sd}}$\fi}
\def\same{\ifmmode s_\mathrm{AME}\else $s_{\mathrm{AME}}$\fi}
\def\ssrc{\ifmmode s_\mathrm{src}\else $s_{\mathrm{src}}$\fi}
\def\sco{\ifmmode s_\mathrm{CO}\else $s_{\mathrm{CO}}$\fi}
\def\sff{\ifmmode s_\mathrm{ff}\else $s_{\mathrm{ff}}$\fi}
\def\gff{\ifmmode g_\mathrm{ff}\else $g_{\mathrm{ff}}$\fi}
\def\fsynch{\ifmmode f_\mathrm{s}\else $f_{\mathrm{s}}$\fi}
\def\fsd{\ifmmode f_\mathrm{sd}\else $f_{\mathrm{sd}}$\fi}
\def\fame{\ifmmode f_\mathrm{AME}\else $f_{\mathrm{AME}}$\fi}
\def\alphasrc{\ifmmode \alpha_\mathrm{src}\else $\alpha_{\mathrm{src}}$\fi}
\def\bcold{\ifmmode \beta_\mathrm{c}\else $\beta_{\mathrm{c}}$\fi}
\def\bhot{\ifmmode \beta_\mathrm{h}\else $\beta_{\mathrm{h}}$\fi}
\def\bnear{\ifmmode \beta_\mathrm{n}\else $\beta_{\mathrm{n}}$\fi}
\def\bsynch{\ifmmode \beta_\mathrm{s}\else $\beta_{\mathrm{s}}$\fi} 
\def\bsun{\ifmmode \beta_\mathrm{sun}\else $\beta_{\mathrm{sun}}$\fi} 
\def\nuzeros{\ifmmode \nu_{0,\mathrm{s}}\else $\nu_{0,\mathrm{s}}$\fi} 
\def\nuzeroff{\ifmmode \nu_{0,\mathrm{ff}}\else $\nu_{0,\mathrm{ff}}$\fi} 
\def\nuzerocold{\ifmmode \nu_{0,\mathrm{c}}\else $\nu_{0,\mathrm{c}}$\fi}
\def\nuzerohot{\ifmmode \nu_{0,\mathrm{h}}\else $\nu_{0,\mathrm{h}}$\fi}
\def\nuzeronear{\ifmmode \nu_{0,\mathrm{n}}\else $\nu_{0,\mathrm{n}}$\fi} 
\def\nuzeroame{\ifmmode \nu_{0,\mathrm{AME}}\else $\nu_{0,\mathrm{AME}}$\fi} 
\def\nuzerosd{\ifmmode \nu_{0,\mathrm{}}\else $\nu_{0,\mathrm{sd}}$\fi} 
\def\nuzerosrc{\ifmmode \nu_{0,\mathrm{src}}\else $\nu_{0,\mathrm{src}}$\fi} 
\def\nup{\ifmmode \nu_{\mathrm{p}}\else $\nu_{\mathrm{p}}$\fi} 
\def\alphasd{\ifmmode \alpha_{\mathrm{sd}}\else $\alpha_{\mathrm{sd}}$\fi} 
\def\Te{\ifmmode T_{\mathrm{e}}\else $T_{\mathrm{e}}$\fi} 
\def\kB{\ifmmode k_\mathrm{B}\else $k_{\mathrm{B}}$\fi} 


\begin{document} 


   \title{\bfseries{\Cosmoglobe\ DR2. I. Global Bayesian analysis of \COBE-DIRBE }}

   %This author list corresponds to \title{Author list for L04\_CMB\_Foregrounds\_Extraction}
%Prepared by M. Lopez-Caniego (Marcos.Lopez.Caniego@sciops.esa.int), ESAC/ESA
%This version is from Thu Jul 12 18:11:48 2018 CET
%\subtitle{There are 152 co-authors in this list}
\newcommand{\oslo}[0]{1}
%\newcommand{\MIT}[0]{2}
\newcommand{\milanoA}[0]{2}
\newcommand{\milanoB}[0]{3}
\newcommand{\milanoC}[0]{4}
\newcommand{\triesteB}[0]{5}
\newcommand{\planetek}[0]{6}
\newcommand{\princeton}[0]{7}
\newcommand{\jpl}[0]{8}
\newcommand{\helsinkiA}[0]{9}
\newcommand{\helsinkiB}[0]{10}
\newcommand{\nersc}[0]{11}
\newcommand{\haverford}[0]{12}
\newcommand{\mpa}[0]{13}
\newcommand{\triesteA}[0]{14}
\newcommand{\iia}[0]{2}

\author{\small
J.~R.~Eskilt\inst{\oslo}\thanks{Corresponding author: J.~R.~Eskilt; \url{j.r.eskilt@astro.uio.no}}
\and
K.~Lee\inst{\oslo}
\and
D.~J.~Watts\inst{\oslo}
\and
S.~Nerval\inst{\oslo}
\and
et al.
}
\institute{\small
        Institute of Theoretical Astrophysics, University of Oslo, Blindern, Oslo, Norway \goodbreak
}


   %\institute{Institute of Theoretical Astrophysics, University of Oslo, Blindern, Oslo, Norway}
  
   % Shortened title, author list for top of page 
   \titlerunning{\Cosmoglobe: DIRBE reanalysis}
   \authorrunning{D.~Watts et al.}

   \date{\today} 
   
   \abstract{We present the first global Bayesian analysis of the time-ordered Diffuse Infrared Background Experiment (DIRBE) data within the \Cosmoglobe\ framework, building on the same methodology that has previously been successfully applied to \Planck\ LFI and \WMAP. These data are analysed jointly with \COBE-FIRAS, \GAIA, \Planck\ HFI, and WISE observations, which allows for a more accurate instrumental and astrophysical characterization than possible through single-experiment analysis only. This paper provides an overview of the analysis pipeline and main results, and we present and characterize a new set of zodiacal light subtracted mission average (ZSMA) DIRBE maps spanning the wavelength range between 1.25 and 240\,$\mu$m. A key novel aspect of this processing is the characterization and removal of excess radiation between 4.9 and 100$\,\mu$m that appears static in solar-centric coordinates; whether this radiation is due to real small-scale structures in the zodiacal light emission, or it is due to straylight in the DIRBE optics, should be addressed through future analysis. The new DR2 ZSMA maps have several notable advantages with respect to the previously available maps, including 1) lower zodiacal light (and possibly straylight) residuals; 2) better determined zero-levels; 3) natively HEALPix tesselated maps with a $7\arcm$ pixel size; 4) nearly white noise at pixel scales; and 5) a more complete and accurate noise characterization established through the combination of Markov Chain Monte Carlo samples and half-mission maps. In addition, because the model has been simultaneously fitted with both DIRBE and HFI data, this is the first consistent unification of the infrared and CMB wavelength ranges into one global sky model covering 100\,GHz to 1\,$\mu$m. However, we do note that even though the new maps are a massive improvement with respect to the official maps, and should be preferred for most future analysis that require DIRBE sky maps, they still exhibit non-negligible zodiacal light residuals between 12 and 60$\,\mu$m. Further improvements should be made through joint analysis with complementry infrared experiments such IRAS, AKARI, WISE and SPHEREx, and thereby release the full combined potential of all these powerful infrared observatories. 
   }

   \keywords{ISM: general - Zodiacal dust, Interplanetary medium - Cosmology: observations, diffuse radiation - Galaxy: general}

   \maketitle

\setcounter{tocdepth}{2}
\tableofcontents
   
% INTRODUCTION
%-------------------------------------------------------------------
\section{Introduction}
%\the\textwidth \the\columnwidth


The astrophysical sky contains a wealth of information about our own Solar System, the Milky Way, and the high-frequency universe in the infrared wavelength regime from roughly 1 to 1000\,$\mu$m \citep[e.g.,][]{johnson:1966,soifer:1987,gardner:2006}. These wavelengths have therefore been the target of many ground-breaking experiments during the last five decades, most of which have been satellite-based due to the high opacity of the Earth's atmosphere. The first transformational observations were made by the NASA-led Infrared Astronomical Satellite (IRAS, \citealt{neugebauer:1984}), which observed the sky for ten months in 1983, covering four wavelength bands from 12 to 100$\,\mu$m. IRAS revealed for the first time the intricate nature of thermal dust emission both in the Solar System and the Milky Way.

IRAS was quickly followed by another NASA-led satellite experiment called Cosmic Background Explorer (\COBE, \citealt{boggess92}), which launched in 1989 and carried three instruments. One of these was the Diffuse Infrared Background Experiment (DIRBE; \citealp{hauser1998}), which observed the sky in ten wavelength bands between 1.25 to 240 microns, with the primary aim to characterize the statistical properties of the Cosmic Infrared Background (CIB; \citealp{partridge1967}). The CIB is thermal infrared radiation from both dust particles in distant galaxies and their redshifted starlight, and contains a large fraction of the total energy released in the Universe since the formation of galaxies. After an extended period of detailed analysis, clear CIB signatures were finally discovered in the DIRBE data ({\bf ADD REFS}), but confusion from both zodiacal light from the Solar system and thermal dust emission from the Milky Way made it difficult to fully reach DIRBE's original goal \citep{arendt1998,hauser1998,kelsall1998}. However, the fact that these emission processes are so bright also have ensured that the DIRBE data have had a far-reaching legacy value, and it remains one of the most important data sets for understanding zodiacal light emission to this date. The main goal of the work presented in this paper, and in its companion papers, is to resolve the most important and long-standing problems regarding the DIRBE data, and thereby finally release the full potential of these invaluable measurements.

Following DIRBE, almost a dozen other satellite experiments have targeted the same wavelengths with different angular resolution, sensitivity, and observation strategies, and today there exists a wealth of complementary and ancillary information that was not available between 1990 and 1994, when the official DIRBE analysis was completed. Two examples of such experiments are AKARI \citep{murakami:2007}, which covered six bands from 9 to 180\,$\mu$m, and WISE \citep{wright:2010}, which covered four bands from 3.4 to 22\,$\mu$m. Another important example of a recent and highly complementary experiment is the optical \GAIA\ mission \citep{gaia:2016}, which recently completed a deep survey of stars in the Milky Way \citep{gaia:2018}.

Not only has great observational progress been made in terms of detailed measurements in the infrared regime during the last decades, but major breakthroughs have also been achieved both in terms of understanding the detailed structure of the Milky Way, and in how to analyse complex datasets optimally. One particularly striking example of this is provided by the cosmic microwave background (CMB) community, which through a long series of transformational experiments has revolutionized our understanding of the early universe; only a few examples include ACT, BICEP/Keck, \COBE \citep{mather:1994}, SPT, and \WMAP\ \citep{bennett2012} ({\bf ADD REFS and more experiments}). The current state-of-the-art in terms of full-sky CMB sensitivity is defined by ESA's \Planck\ satellite experiment. However, precisely because of its exquisite signal-to-noise ratio, a long series of key data analysis challenges had to be overcome before its full cosmological potential could be released. Indeed, \Planck\ was the first full-sky CMB experiment for which instrumental and astrophysical uncertainties dominated the total error budget, as opposed to white noise. As such, \Planck\ faced many of the same types of problems that DIRBE had experienced two decades earlier, and massive amounts of algorithm development efforts were spent by hundreds of scientists on resolving these.

One of the main lessons learned from that work was that, in order to properly mitigate the dominant systematic effects, it was no longer possible to consider each systematic uncertainty in isolation. Rather, it was necessary to perform a global integrated analysis in which all parameters are optimized simultaneously, whether they happen to be of instrumental or astrophysical origin. Two pioneering efforts in this direction were the SROLL\ \citep{sroll2} and NPIPE \citep{npipe} data analysis pipelines, both of which were developed within the official \Planck\ consortium, and eventually formed the algorithmic basis for the \Planck\ PR3 and PR4 data releases, respectively. In particular, both SROLL and NPIPE integrated knowledge about the astrophysical sky directly in their instrument calibration and mapmaking steps, even though neither actually fitted the corresponding astrophysical parameters themselves  during the low-level processing.

The first pipeline to perform true integrated global analysis of \Planck\ data was implemented in a computer code called \commanderthree\ \citep{bp03} by the \textsc{BeyondPlanck} collaboration \citep{bp01}. This extended earlier work on Bayesian component separation that was performed within the \Planck\ collaboration \citep{planck2014-a12}, and it was implemented in terms of an end-to-end Bayesian Monte Carlo Gibbs sampler in which an explicit parametric data model was fitted to raw uncalibrated time-ordered data (TOD). As a result of this integrated analysis, a number of long-standing problems regarding the \Planck\ LFI data were resolved, in particular with respect to gain calibration, and the full LFI data set was now for the first time finally available for cosmological analysis.

Recognizing the importance of global Bayesian analysis for all high-sensitivity experiments, a separate effort called \Cosmoglobe\ was started in 2019. The basic idea underlying \Cosmoglobe\ was very simple: All radio, microwave and infrared experiments measure fundamentally the same sky. However, due to technical limitations, each experiment only measures a relatively small part of the electromagnetic spectrum, and with limited angular resolution and sensitivity. At the same time, the field as a whole is currently at a stage where astrophysical uncertainties play a dominating role in understanding the systematic properties of each experiment. It is therefore natural to expect that better results may be obtained by analyzing multiple complementary experiments together, as opposed to each separately, and in effect use information from one experiment to break the basic degeneracies in another. The long-term goal of the \Cosmoglobe\ effort is thererfore to establish one single state-of-the-art model of the astrophysical sky that covers the entire electromagnetic spectrum, using all available experiments at the same time. Obviously, this is a monumental task, and it will require the combined effort of the entire astrophysical community in order to be successful. However, because of the large investments made by the \Planck\ and BeyondPlanck collaborations, the algorithmic basis needed to do this now actually does exist, and is ready to be gradually expanded with new astrophysical effects and experiments.

The first explicit demonstration of this simple idea was presented by \citet{watts2023_dr1}, who generalized the \commanderthree\ computer code to perform joint end-to-end Bayesian analysis of both the \WMAP\ and \Planck\ LFI data at the same time. This turned out to be very effective, and the introduction of LFI measurements effectively resolved a number of long-standing calibration issues in the \WMAP\ data that never could be resolved with \WMAP\ data alone. The products from this analysis were released in March 2022 as ``\Cosmoglobe\ Data Release 1 (DR1)'', and defines today the state-of-the-art in terms of both \Planck\ LFI and \WMAP\ sky maps.

This is the first of a series of papers in which we perform a similar analysis for the \COBE-DIRBE data. This represents a major step forward in the \Cosmoglobe\ program by expanding the modelled frequency range by three orders of magnitude, and it is a first step towards merging the microwave and infrared fields into one joint effort. The reasons for considering \COBE-DIRBE in this first step, as opposed to AKARI, IRAS, or WISE, are two-fold. First and foremost, DIRBE has excellent systematics properties, both in terms of absolute calibration and zero-level determination, thermal stability, and in terms of a highly inter-connected scanning strategy. At the same time, both its data volume and angular resolution are relatively modest, and this makes the computational load and debug cycle very manageable. Overall, DIRBE is an ideal dataset for generalizing the previous CMB-oriented model and computer code into the infrared regime.

At the same time, the fundamental challenges faced by DIRBE are very similar to those of faced by any other infrared experiment. In particular, the single most challenging aspect is the zodiacal emission (ZE). This is thermal emission and scattered sunlight from interplanetary dust (IPD) grains. The main difficulty when dealing with zodiacal emission contamination in infrared data is that as the observed emission is highly dependent on the position of the observer, and as such, it cannot be modeled like a static foreground, as for instance Galactic foregrounds are treated in the CMB community. Rather, the state-of-the-art method to remove zodiacal emission from timestreams today is to use a three-dimensional interplanetary dust model which describes the distribution of interplanetary dust within the solar system, and perform line-of-sight integration for every single time step. The IPD model most widely used today is the so-called K98 model \citep{kelsall1998} produced by the DIRBE team, or variants thereof. In one of the companion papers in the current suite, we present a major step forward in this modelling, by exploiting the fact that some of the IPD components do indeed appear stationary on the sky as seen from the Earth (namely the so-called circumsolar ring and Earth-trailing feature), and these may therefore be modelled as stationary 2D signals per sky pixel rather than as highly constrained 3D objects. This, together with joint diffuse component separation algorithms, use of ancillary datasets, and better parameter fitting algorithms lead to a greatly improved ZE model that should be use to the entire infrared community.

These improvements also lead to better science exploitation overall. Most notably, we are in the current analysis for the first time able to characterize fluctuations in the extra-galactic optical background (COB) at 3.5 and 4.9$\,\mu$m. We also obtain robust CIB measurements, both in terms of fluctuations and zero-levels at all wavelengths between 60 and 240$\,\mu$m, and these measurements are highly complementary to those obtained from \Planck\ and other experiments, allowing for new constraints on cosmic structure formation. Similar improvements are obtained for the Galactic science targets, and we are for instance now able to constrain diffuse polycyclic aromatic hydrocarbon (PAH) emission between 3.5 and 12$\,\mu$m at high confidence. Finally, we also present a novel three-component thermal dust emission model without spatial spectral variations that is able to describe thermal dust emission from 100\,GHz to 1.25$\,\mu$m with only two free parameters per pixel; a novel WISE-plus-\GAIA\ based starlight model for frequencies between 1.25 and 25$\,\mu$m; and a new map of ionized carbon (\CII) with an angular resolution of $1^{\circ}$ FWHM. Overall, we argue that these new products come close to fulfilling the original science potential of the DIRBE experiment, even though there are still a number of issues that should be resolved through further joint analysis with other complementary datasets, such as AKARI and WISE. All products and computer codes are made publicly available,\footnote{\url{https://cosmoglobe.uio.no}} and we will in the following refer to the current data release as ``\Cosmoglobe\ Data Release 2''(or CGDR2 for short), or simply DR2.

The rest of the paper is organized as follows. In Sect.~\ref{sec:global_modelling} we review the historical background and algorithmic foundation of the \Cosmoglobe\ effort. In Sect.~\ref{sec:data} we briefly describe the DIRBE data as provided by the \COBE\ team, the pre-processing steps that we apply the them, the fitted data model, and the new \commanderthree\ extenstions that are needed to fit this model. Next, in Sect.~\ref{sec:posteriors} we present the low-level and frequency map posterior distribution derived from the DIRBE data,  before considering the main astrophysical results in Sect.~\ref{sec:astrophysics}; full details are provided in the respective companion papers. Finally, we conclude and discuss avenues for future work in Sect.~\ref{sec:conclusions}.


\section{Global Bayesian modelling of the infrared sky}
\label{sec:global_modelling}

The use of Bayesian sampling methods have become widespread in the CMB
community during the last few decades for at least two important
reasons. First, for any analysis task that may be phrased in terms of
a classical parameter estimation problem with measured data $\dv$ and a
model with some set of unknown parameters $\omega$, the posterior
distribution $P(\omega\mid\dv)$ represents a complete summary of the
information about $\omega$ contained in the current data, both in
terms of best-fit point estimates and corresponding
uncertainties. Second, both due to the innovation of a wide range of
efficient Monte Carlo sampling methods and the exponential growth of
computing power took place until very recently, far more complex
models can be mapped out today than was possible only one or two
decades ago. As a particularly relevant case in point for the current
paper is \commander, which is a Gibbs sampler designed to perform
end-to-end analysis with time-ordered data. While the primary
motivation for developing this machinery until today has been
CMB-oriented applications, we show in the following that the same
framework is also very well suited for analysis of observations in the
infrared regime, and, indeed, that it may be used to construct one
global model that includes both microwave and infrared wavelengths.

\subsection{Data model and posterior distribution}
\label{sec:datamodel}

The first step in any parametric Bayesian analysis is simply to write
down a model for the data in question, and the quality of the final
results depends sensitively on the accuracy and completeness of this
model, which must be monitored through detailed goodness-of-fit
statistics, typically in the form of residual and $\chi^2$
measures. In practice, an initial model is typically established based
on a pre-existing knowledge about both the astrophysical sky and
instrument in question, and the model is then gradually refined until
the residuals are consistent with instrumental noise. The model
described in this section represents the product of such a process
that has involved hundreds of trial runs, starting from a model very
similar to that described by the official DIRBE and \Planck\ teams,
but then gradually generalized with new parameters.

As described in Sect.~\ref{sec:data}, we will in the current analysis
focus on the so-called DIRBE Calibrated Individual Observations
(CIOs), as opposed to raw uncalibrated TOD, primarily because these
are the only ones that are publicly available at the current time, but
also because they are easier to work with than the raw uncalibrated
TOD. However, we note that this immediate implies that there are
important degrees of freedom, in particular with respect to gain and
zero-level determination, that relies directly on the official
analysis, and that may need to be revisited at a later stage. We will
in the following refer to the DIRBE CIOs simply as ``TOD''.

\subsubsection{TOD model}

We adopt the following high-level parametric data model for the DIRBE
TOD,
\begin{align}
	\label{eq:model}
	\dv &=\G\P\left[\B\sum_{c=1}^{n_{\mathrm{comp}}}\M_c\a_c+\s_{\mathrm{zodi}} +
          \s_{\mathrm{static}}\right] + \n^\mathrm{corr} + \n^\mathrm w,\\
        &\equiv \s^{\mathrm{tot}} + \n^{\mathrm{w}}
\end{align}
where $\dv$ denotes a stacked vector of all DIRBE TOD for all
frequency bands; $\G$ represents an $n_{\mathrm{tod}}\times
n_{\mathrm{tod}}$ diagonal matrix with an overall constant gain
calibration factor per frequency channel; $\P$ denotes a satellite
pointing matrix, which we define in Galactic coordinates; $\B$ denotes
an instrumental beam (or point spread function) convolution operator;
the sum runs over $\n_{\mathrm{comp}}$ astrophysical components, each
with a free amplitude $\a_c$ at some reference frequency and a mixing
matrix $\M_c$ which defines the effective scaling from the reference
frequency to an observed frequency for each component, taking into the
bandpass of each detector; $\s_{\mathrm{zodi}}$ represents zodiacal
light emission from components that appear time-variable as seen from
Earth (e.g., the zodiacal cloud and asteroidal bands);
$\s^{\mathrm{static}}$ represents any signal that appears stationary
with respect to the Earth-Sun system (e.g., the so-called zodiacal
circumsolar ring and Earth-trailing feature, but also potential far
sidelobe contamination from the Sun); $\n^{\mathrm{corr}}$ represents
correlated instrumental noise (which for now is only fitted for the
lowest DIRBE frequency channel); and $\n^{\mathrm{w}}$ denotes white
instrumental noise. We also define $\s^{\mathrm{tot}}$ to be the sum
of all terms in data model except the white noise.

For the gain, $\G$, we only fit a single overall multiplicative value for each
frequency band, and only at a few channels for which we have a good calibrator;
in practice, that is only channels between 100 and 240\,$\mu$m, which have
previously been noted to be in moderate tension with FIRAS \citep{fixsen1997}.
For all other channels, we rely on the DIRBE team's original calibration.

Both the pointing $\P$ and the beam operator $\B$ are provided by the
DIRBE team, and we do not account for any uncertainties or free
parameters in these. However, we do note that the DIRBE beams have an
intrinsically square shape, while our current beam convolution
implementation only supports azimuthally symmetric beams. This will
necessary lead to a residual that should ideally be accounted for
through full integration, for instance using a Conviqt-style
algorithm \citep{prezeau2010,keihanen2012}; this is left for future work. Similar remarks apply to
bandpass definitions as well; for now, we neglect the uncertainty in the bandpass
profiles provided by the DIRBE team. For
further details regarding the pointing, beam and bandpasses, see
Sect.~\ref{sec:data}.

The sky model is described in detail in the next section. However, we
will refer to the set of all linear sky component amplitude parameters
as $\a_{\mathrm{sky}}$ and the set of all spectral parameters as
$\beta_{\mathrm{sky}}$ in the following.

For the dynamic component of the zodiacal light emission, we fit a
limited number of shape parameters per interplanetary dust component,
in addition to linear emissivity and albedo parameters. In total,
there are 69 free parameters in this model, and these are collectively
denoted $\zeta_{\mathrm{z}}$. In addition, the static components are
described by $\s_{\mathrm{static}}$, and modelled in terms of one sky
map per frequency. We denote these simply as $\a_{\mathrm{static}}$,
such that $\s_{\mathrm{static}} = \P\a_{\mathrm{static}}$.

We assume that the instrumental noise is piecewise stationary, and we
model it with an uncorrelated zero-mean Gaussian distribution with a
free standard deviation per sample, $\sigma_{\mathrm{n}}$ for all
channels except 240\,$\mu$m. The stationarity period is assumed to be
24~hours, and the data are correspondingly processed in segments. For
the 240\,$\mu$m channel alone, we include a correlated noise term, and
we assume that the time-domain noise power spectrum may be described
by a standard $1/f$ profile on the form $P(f) = \sigma_{\mathrm{n}}^2
(1+(f/f_{\mathrm{knee}})^{\alpha})$, where the slope $\alpha$ and knee
frequency $f_{\mathrm{knee}}$ are fitted independently in each data
segment. Ideally, we would like to include this component in all
frequencies. However, we find that the current sky model does not yet
represent a sufficiently good fit at any channel with a wavelength
shorter than or equal to 100\,$\mu$m. In principle, we could have
included it for the 140\,$\mu$m channel, which has similar excellent
goodness-of-fit as 240\,$\mu$m, but in this case we fit a
\CII\ component per pixel, and $\n_{\mathrm{corr}}$ then introduces a
strong degeneracy with respect to this component. In total, we denote
the sum of all noise parameters by $\xi_{\mathrm{n}}$.


\subsubsection{Sky model}

For the sky signal defined implicitly by the sum in
Eq.~\eqref{eq:model}, we adopt the following model in units of
brightness temperature and frequency,\footnote{Due to its CMB-oriented
origin, \commander\ uses brightness temperature and frequency units
for internal calculations, rather than flux density and wavelength
units which would be more natural for DIRBE. This has, however, no
actual effect on the final results, but only requires appropriate unit
conversions to be applied during input and output operations. }
\begin{alignat}{4}
  \sum_{c=1}^{n_{\mathrm{comp}}} \M_c \a_c  = \,
  &\M_{\mathrm{mbb}}(\bcold,\Tcold,\nuzerocold)\vec{a}_{\mathrm{cold}}
  && \textrm{(Cold dust)}\label{eq:skymodel}\\
  + &\M_{\mathrm{mbb}}(\bhot,\Thot,\nuzerohot)
  \vec{a}_{\mathrm{hot}} && \textrm{(Hot dust)}\nonumber \\
  + &\M_{\mathrm{mbb}}(\bnear,\Tnear,\nuzeronear) \t_{\mathrm{near}}
  a_{\nu} && \textrm{(Nearby dust)} \nonumber \\
  + &\left(\frac{\nuzeroff}{\nu}\right)^2
  \frac{g_{\mathrm{ff}}(\nu;\Te) }{g_{\mathrm{ff}}(\nuzeroff;\Te)}
  \vec{t}_{\mathrm{ff}} && \textrm{(Free-free)} \nonumber\\
  + &\delta(\nu-\nu_{0,\mathrm{CO}}^i) \t_{\mathrm{CO}}
  h^{\mathrm{CO}}_{\nu,i} && \textrm{(CO)}\nonumber\\
	+ &\delta(\nu-\nu_{0,\CII}) \a_{\CII}
  h^{\CII}_{\nu} && \CII\nonumber \\
  + &U_{\mathrm{mJy}} \sum_{j=1}^{n_{\mathrm{s}}}
  f_{\GAIA,j} a_{\mathrm{s},j}, &\quad&
  \textrm{(Stars)} \nonumber\\
    + &U_{\mathrm{mJy}} \sum_{j=1}^{n_{\mathrm{e}}}
  M_{\mathrm{mbb}}(\beta_{\mathrm{e},j},
  T_{\mathrm{e},j}, \nu_{0,\mathrm{e}})
  a_{\mathrm{e},j} && \textrm{(FIR sources)}\nonumber\\
  + &m_{\nu} && \textrm{(Monopole)}. \nonumber
\end{alignat}
In this expression, we have defined a special function of the form
\begin{equation}
  M_{\mathrm{mbb}}(\beta, T, q_i;\nu_0, \{\Delta\nu_i\}) =
    \begin{cases}
      q_i & \nu \in \Delta\nu_i\\
      \left(\frac{\nu}{\nu_0}\right)^{\beta+1}
  \frac{\e^{h\nu_0/\kB T}-1}{\e^{h\nu/\kB T}-1} & \nu \notin \Delta\nu_i
    \end{cases}       
\end{equation}
which represents a generalized modified blackbody function. In
addition to the usual emissivity index and temperature, $\beta$ and
$T$, this function takes a set of constant values, $q_i$, and
corresponding frequency ranges $\Delta\nu_i$. If the requested
frequency happens to lie in any one of $\Delta\nu_i$, then $q_i$ is
returned; otherwise the default is to return the standard modified
blackbody spectrum. Also, if a given amplitude is denoted by $\a$, it
is fitted freely to the current data, while if it is denoted by $\t$,
it is fixed to an external template.

As indicated by Eq.~\eqref{eq:skymodel}, we fit a three-component
generalized modified blackbody model to account for thermal dust
emission across the combined DIRBE and \Planck\ HFI frequency
range. The three components correspond to cold dust, hot dust, and
nearby dust, respectively. All three are modeled with spatially
constant spectral indices, but only the cold and hot component
amplitudes are fitted pixel-by-pixel; the amplitude of the nearby
component is fixed to a \GAIA-based template covering distances up to
1.25\,kpc produced by \citep{edenhofer:2024}. 
This model only has two degrees of freedom per pixel, in addition
to less than 30 spatially constant SED parameters. This is an
extremely economical model of thermal dust emission, considering the
fact that it describes the entire combined frequency range covered by
both \Planck\ HFI and DIRBE, from 100\,GHz to 1\,$\mu$m.

To account for free-free emission, we adopt the model presented by
\citet{planck2014-a11}, both in terms of spatial distribution and
spectrum. The SED is essentially defined by the Gaunt factor,
$g_{\mathrm{ff}}(\nu; T_e)$ \citep{dickinson:2003,draine:2011}, which corresponds to a shift in the
spectral index of about $-0.14$ in the CMB frequency range; however,
at the very high DIRBE frequencies of up to 300\,THz, it takes on
significantly more extreme values, and this should at least provide
greater sensitivity to the electron temperature, $T_e$. For now,
however, we adopt the $T_e$ distribution presented by
\citet{planck2014-a11} as given, and aim to revisit this in future work.

The next component corresponds CO line emission, produced by
transitions between two quantized angular momentum eigenstates in the
CO molecule. The resulting emission forms effectively a ladder in
frequency space in multiples of 115.27\,GHz, and \COBE-FIRAS identified
emission all the way up to 922\,GHz, but with low sensitivity and
angular resolution. In contrast, \Planck\ produced high-resolution
maps with high sensitivity of the $J$=1$\leftarrow$0, 2$\leftarrow$1,
and 3$\leftarrow$2 transitions, which contributed to the HFI\,100,
217, and 353\,GHz frequency maps, but was unable to identify CO
emission at higher frequencies due to strong thermal dust emission. In
the current work, we adopt the \citet{dame:2001} CO
$J$=1$\leftarrow$0 map as a fixed tracer for all variations of CO
emission, and only fit a single amplitude $h_{\nu,i}$ to each band
which has a CO contribution, including \Planck\ 545 and 857\,GHz.

Similarly, the fourth component corresponds to \CII\ line emission,
which has a rest frequency of 1900\,GHz. As such, it only affects the
DIRBE 140\,$\mu$m map in our dataset, in addition to selected FIRAS
bands. In this case, we fit for a free amplitude per pixel with DIRBE
140\,$\mu$m, using the general sky model determined by near-by
channels to remove thermal dust emission, and then exploit the near-by
FIRAS channels to monitor the overall reconstruction
quality. The result is a novel \CII\ map with an angular resolution of
about $1^{\circ}$ FWHM.

The fifth component correspond to starlight emission, which is
relevant in the frequency range between 1 and 25\,$\mu$m. In this
case, we first extract a baseline star catalog by thresholding the
AllWISE 3.5\,$\mu$m catalog at magnitude 8, resulting in a set of about
770\,000 sources.\footnote{We have tried different thresholds, and
found that magnitude 8 provides a good compromise; magnitude 6 results
in obviously missing sources, while magnitude 10 leads to too many
unconstrained sources.} For each of these, we search the \GAIA\ DR2
catalog, and if this returns a fit within 20~arcsec, we identify the
object as a star, and store the best-fit temperature $T_\mathrm{s}$,
surface gravity $g$, and metallicity $[\mathrm{M}/\mathrm{H}]$ as
determined by \GAIA. These are then used to estimate the best-fit SED
using the PHOENIX spectrum grid \citep{husser:2013}, which is convolved with the
bandpass and beam profile of each DIRBE channel. The resulting
bandpass- and beam-convolved SED is denoted $f_{\GAIA,j}$,
which is unique for each star. We then fit one overall amplitude for
each star to the four highest DIRBE frequency bands between 1.25 and
4.9\,$\mu$m; we also account for star emission in the 12 and
25\,$\mu$m bands, but these bands are not used for the actual fit. A
total of 717\,000 stars are included in the fit.

\begin{figure*}
  \centering
   	\includegraphics[width=0.95\textwidth]{figs/tod.pdf}
  	\caption{Time-ordered data segment of each DIRBE band (1-10) from top to bottom.}
	\label{fig: cios}
\end{figure*}

The remaining 53\,000 WISE sources that do not have a \GAIA\
counterpart are assumed to be of extra-galactic origin, and we fit
these with a modified blackbody spectrum across the same frequency
range, as described in the eigth line in
Eq.~\eqref{eq:skymodel}. Algorithmically speaking, this component is
otherwise identical to the star component.

The ninth and final component is simply a monopole per frequency. For
DIRBE and FIRAS, this should ideally describe the CIB spectrum, but it
is also sensitive to zodiacal light and Galactic residuals. For
\Planck, the monopoles account for the arbitrary zero-levels present
in the \Planck\ PR4 maps.

Finally, it is worth noting that there is no CMB component present in
the current sky model, even though it applies to the \Planck\ HFI as
well as DIRBE. Since the main focus in the current work is DIRBE, we
have chosen to pre-subtract the CMB component (including the solar CMB
dipole and relativistic quadrupole corrections) from each frequency
map. Similarly, we neglect the impact of synchrotron emission,
anomalous microwave emission, the Sunyaev-Zeldovich effect, and other
smaller contributions; these will instead be included in a future
analysis that also has HFI as a main target.

\subsubsection{Posterior distribution}

In principle, all quantities in Eq.~\ref{eq:model} (except $\dv$),
whether they are of instrumental or astrophysical origin, are
associated with free parameters and uncertainties that must be
estimated from the data themselves. We collectively call the full set
of free parameters by $\omega =
\{\G,\xi_{\mathrm{n}},
\beta_{\mathrm{sky}},\a_{\mathrm{sky}},\zeta_{\mathrm{z}},\a_{\mathrm{static}}\}$,
and our goal is now to derive an explicit expression for the global
posterior distribution, $P(\omega\mid\dv)$. This is most easily done
through Bayes' theorem,
\begin{equation}
P(\omega\mid\dv) = \frac{P(\dv\mid\omega) P(\omega)}{P(\dv)} \propto
\mathcal{L}(\omega) P(\omega).
\end{equation}
In this expression, $\mathcal{L}(\omega) \equiv  P(\dv\mid\omega)$ is
called the likelihood, $P(\omega)$ is called the prior; $P(\dv)$ is a
normalization constant that does not depend on $\omega$, and we
therefore neglect in the following.

Under the common assumption that the white noise component is Gaussian
distributed with zero mean and some covariance matrix,
$\N_{\mathrm{w}}$, we can write the log-likelihood in the usual
explicit form,
\begin{equation}
-2\ln\mathcal{L}(\omega) = (\dv-\s^{\mathrm{tot}}(\omega))^t
  \N_{\mathrm{w}}^{-1}(\dv-\s^{\mathrm{tot}}(\omega)) \equiv \chi^2(\omega),
\end{equation}
once again up to an irrelevant normalization constant, and we have for
notational compactness suppressed the fact that also $\N_{\mathrm{w}}$ has free
parameters. 

Regarding $P(\omega)$, we will in this analysis operate primarily with
three types of priors. First, for zodiacal light parameters we adopt
uniform priors between pre-defined limits, to avoid the algorithms to
diverge into obviously pathological solutions. Second, for
astrophysical spectral parameters, such as temperature and spectral
indices, we adopt products of uniform priors with broad limits and
Gaussian priors with spectral parameters informed by \Planck\ where
applicable. Finally, for a few select astrophysical components, for
instance free-free and carbon monoxide line emission, we adopt
existing spatial templates as delta function priors on the spatial
morphology, and only fit overall free amplitudes in the current
analysis. For full details regarding the use of priors for a given
component, see the respective specialized paper.

\subsection{Gibbs sampling with \commanderthree}

As described in Sect.~\ref{sec:datamodel}, the current data model
contains millions of strongly correlated parameters, ranging from
affecting individual time samples (such as the correlated noise,
$\n_{\mathrm{}}$) to describing the astrophysical signal in a few
pixels (such as the cold dust amplitude $\a_{\mathrm{c}}$ or star
amplitude $\a_{\mathrm{s}}$), to simultaneously affecting the
essentially every single data point, such as the zodiacal light shape
parameters. Mapping out this distribution is therefore highly
non-trivial.

So far, the only algorithm that has been demonstrated in practice to
work well on such complex end-to-end analysis problems is Gibbs
sampling, which is a special case of the Metropolis-Hastings Monte
Carlo sampling algorithm. The defining feature of this algorithm is
that it loops over all free parameters (which may be divided into
groups), and draws a sample from each conditional
distribution. Returning to the defining data model in
Eqs.~\eqref{eq:model}--\eqref{eq:skymodel}, and noting that the set
of free parameters is $\omega = \{\G,\xi_{\mathrm{n}},
\beta_{\mathrm{sky}},\a_{\mathrm{sky}},\zeta_{\mathrm{z}},\a_{\mathrm{static}}\}$,
we can immediately write down a corresponding Gibbs sampling chain of
the following form:
\begin{alignat}{11}
\G &\,\leftarrow P(\G&\,\mid &\,\dv,&\, &\,\phantom{\G} &\,\xi_n, &
\,\beta_{\mathrm{sky}}& \,\a_{\mathrm{sky}}, &\,\zeta_{\mathrm{z}},
&\,\a_{\mathrm{static}}&)\label{eq:gibbs_G}\\
\xi_{\mathrm{n}} &\,\leftarrow P(\xi_{\mathrm{n}}&\,\mid &\,\dv,&\, &\,\G, &\,\phantom{\xi_n} &
\,\beta_{\mathrm{sky}}& \,\a_{\mathrm{sky}}, &\,\zeta_{\mathrm{z}},
&\,\a_{\mathrm{static}}&)\\
\beta_{\mathrm{sky}} &\,\leftarrow P(\beta_{\mathrm{sky}}&\,\mid &\,\dv,&\, &\,\G, &\,\xi_n, &
\,\phantom{\beta_{\mathrm{sky}}}& \,\a_{\mathrm{sky}}, &\,\zeta_{\mathrm{z}}, &\,\a_{\mathrm{static}}&)\\
\a_{\mathrm{sky}} &\,\leftarrow P(\a_{\mathrm{sky}}&\,\mid &\,\dv,&\, &\,\G, &\,\xi_n, &
\,\beta_{\mathrm{sky}},& \,\phantom{\a_{\mathrm{sky}},}
&\,\zeta_{\mathrm{z}}, &\,\a_{\mathrm{static}}&)\\
\zeta_{\mathrm{z}} &\,\leftarrow P(\zeta_{\mathrm{z}}&\,\mid &\,\dv,&\, &\,\G, &\,\xi_n, &
\,\beta_{\mathrm{sky}},& \,\a_{\mathrm{sky}},
&\,\phantom{\zeta_{\mathrm{z}},} &\,\a_{\mathrm{static}}&)\label{eq:gibbs_zodi}\\
\a_{\mathrm{static}} &\,\leftarrow P(\a_{\mathrm{static}}&\,\mid &\,\dv,&\, &\,\G, &\,\xi_n, &
\,\beta_{\mathrm{sky}},& \,\a_{\mathrm{sky}}, &\,\zeta_{\mathrm{z}} &\,\phantom{\a_{\mathrm{static}}}&).\label{eq:gibbs_static}
\end{alignat}
Here, the symbol $\leftarrow$ indicates drawing a sample from the
conditional distribution on the right-hand side. However, we note that
our codes are also designed to perform maximum-posterior (or
likelihood) analysis, in which case we maximize the probability
distribution instead of drawing a sample from it.

The current state-of-the-art in terms of CMB Gibbs sampling
implementation is \commander\ \citep{Eriksen:2004ss}, which was used
extensively for the \Planck\ analysis. However, during the
\Planck\ analysis this code only supported high-level component
separation operations, and the low-level time-domain support was added
after the official end of \Planck. The first incarnation of this
end-to-end framework is called \commanderthree\ \citep{bp03}, which
was applied to the \Planck\ LFI data by the \BP\ collaboration
\citep{bp01}. Shortly after, a slightly extended version was applied
to the combination of \Planck\ LFI and \WMAP\ by
\citet{watts2023_dr1}, and the results from this analysis formed the
basis for \Cosmoglobe\ DR1. 

The existing \commanderthree\ implementation used for \BP\ and
\Cosmoglobe\ DR1 already provides sampling steps for most of the above
conditional distributions, and these can be reused with minimial
modifications. In particular, \citet{bp07} describe how to sample
instrumental gain; \citet{bp06} describe how to estimate instrumental
noise parameters, and \citet{bp02} discuss how to make optimal maps
with full noise propagation efficiently with Gibbs sampling; and
\citet{bp13} describe how to sample from intensity foregrounds
posteriors. 

While by far most of the code infrastructure required to process the
DIRBE TOD already exists, several of these steps and models discussed
above require slight modifications in order to work efficiently in a
production environment. In particular, efficient diffuse foreground
sampling for DIRBE -- and the closely connected gain sampler -- are
described by \citet{CG02_05}; the novel starlight model and sampler
are described by \citet{CG02_04}; and the \CII\ sampler is described
by \citet{CG02_06}. 

The two last sampling steps summarized in Eqs.~\eqref{eq:gibbs_zodi}
and \eqref{eq:gibbs_static}, however, did until the current work not
have support in the existing \commander\ implementation, and had to be
developed from scratch. An early step towards goal this was described
by \citet{san:2022}, who reimplemented the default DIRBE zodiacal
light model (K98; \citealp{kelsall1998}) in Python. This served as the
basis for the code developed here, which now is a set of native
\commander\ modules written in Fortran. The full details of the new
framework, including a dramatically improved best-fit model with
respect to K98, is presented by \citet{CG02_02}.

\section{Diffuse Infrared Background Experiment}
\label{sec:dirbe}

\begin{figure*}
	\centering
	\includegraphics[width=\linewidth]{figs/DIRBE_optics.png}
	\caption{DIRBE optics module. The optical design includes two field stops, one of which is square, to reduce straylight contamination. Reproduced from \cite{magner87}.
	}
	\label{fig:optics_model}
\end{figure*}


\begin{figure}
  \centering
  \includegraphics[width=\linewidth]{figs/DIRBE_beam_theta.pdf}\\
  \includegraphics[width=\linewidth]{figs/DIRBE_beam_ell.pdf}
  \caption{Symmetrized beam response functions for each DIRBE channel, both in real space (\emph{top}) and in harmonic space (\emph{bottom}).}
  \label{fig:beams}
\end{figure}

\begin{figure}
  \centering
  \includegraphics[width=\linewidth]{figs/DIRBE_bp.pdf}
  \caption{Bandpass response functions for each DIRBE channel, plotted as a function of frequency.}
  \label{fig:bandpass}
\end{figure}


\subsection{The DIRBE instrument}

The Diffuse InfraRed Background Explorer (DIRBE) was one of three experiments on the Cosmic Background Explorer (\COBE) satellite \citep{boggess92}. DIRBE was designed to characterize the infrared sky from $1\,\mathrm{\mu m}$ to $240\,\mathrm{\mu m}$, with the sensitivity required to characterize thermal dust emission, zodiacal emission, and to detect the Cosmic Infrared Background (CIB). Many of the design considerations were specifically made for achieving this scientific goal, as emphasized by \citet{silverberg93}. The DIRBE experiment was limited by its cryogenic requirement, using 600 L of superfluid \element[ ][4]{He}, cooling the instrument to 1.6 K.
The DIRBE central bandpasses, given in terms of central frequency and central wavelength, are given in Table \ref{tab:summary}, and the full bandpass responses as reported in \citet{cobe_exsupp} are shown in Fig.~\ref{fig:bandpass}.


The DIRBE optical design included several design solutions for calibration and stray light reduction. Straylight reduction was prioritized in the design, largely because of the difficulty in distinguishing this systematic effect from a true diffuse background. In particular, there are several straylight stops to reduce sidelobe contamination, mainly in the form of a square beam, as can be seen in Fig.~\ref{fig:optics_model}. In addition to the straylight reduction, DIRBE alternates observations between the sky and an internal calibration source that chops between the two light sources at a rate of 32 Hz.
All bands observe the same $0\fdg7\times0\fdg7$  field simultaneously, with small adjustments of the beam centroids depending on the location of the detectors. The light is divided using beam splitters to split the light into various detector assemblies. Detectors 1--3 were polarization-sensitive, with light parallel and perpendicular to the scan direction being detected. Because of the DIRBE scan strategy, there is poor polarization angle coverage across the sky.

%\begin{figure}
%	\centering
%	\includegraphics[width=\linewidth]{figs/cobeslide04.jpg}
%	\caption{DIRBE optical design, courtesy of NASA/\COBE\ Science Team.
%	}
%	\label{fig:optics}
%\end{figure}


The different detector technology for each band accounts for the different performances and systematics found in each of the bands. \citet{silverberg93} in particular highlights the Ge:Ga photoconductors' response to ionizing radiation in the South Atlantic Anomaly (SAA) being worse than other detectors, requiring long time for the detectors to return to normal. Similarly, the use of a composite Si bolometer for bands 9 and 10 partially explain the over an order of magnitude increase in noise when compared to adjacent bands.

The symmetrized beam, shown in the top panel of Fig.~\ref{fig:beams}, resembles a tophat with a slow falloff. The harmonic-space representation of the beam is therefore reminiscent of a sinc function, with oscillations about zero above $\ell\gtrsim500$. For a band-limited signal to be represented fully in map space, the 42\arcm\ DIRBE beams must be represented with a 21\arcm\ or smaller pixelization. However, in order to be fully represented in harmonic space, a requirement for the \commanderthree\ multi-resolution component separation, the pixelization scheme must have support up to $\ell\lesssim1500$. The original DIRBE maps have pixel size of 21\arcm, insufficient for harmonic space analysis.
Both the original maps in Quadcube\footnote{
	Quadrilateralized Spherical Cube \url{https://lambda.gsfc.nasa.gov/product/cobe/skymap\_info\_new.html}
}
pixelization (resolution 9, pixel size 21\arcm) and the CADE\footnote{Centre d'Analyse de Données Etendues, \url{http://cade.irap.omp.eu/dokuwiki/doku.php?id=dirbe}} reprojection into HEALPix ($N_\mathrm{side}=256$, pixel size 13\farcm7) do not have the required support over the full multipole range. 


%Some of the optical design choices cause issues for modern reprocessing. Of particular note is the square straylight reduction, which has an instantaneous square beam. The symmetrized beam, shown in the top panel of Fig.~\ref{fig:beams}, resembles a tophat with a slow falloff. The harmonic-space representation of the beam is therefore reminiscent of a sinc function, with oscillations about zero above $\ell\gtrsim500$. For a band-limited signal to be represented fully in map space, the 42\arcm\ DIRBE beams must be represented with a 21\arcm\ or smaller pixelization. However, in order to be fully represented in harmonic space, the pixelization scheme must have support up to $\ell\lesssim1500$. Both the original QUADCube maps (res 5, pixel size bla) and the CADE reprojection into HEALPix ($N_\mathrm{side}=256$, pixel size bla), do not have the required support over the full multipole range. In order to represent the harmonic space information fully in HEALPix, an $N_\mathrm{side}=512$ map is the lowest possible resolution. Put another way, the maximum resolution spherical harmonic space depends not just on the size of the beam itself, but the size of the beam's features that must be resolved.

% \input{detector_table.tex}


%See also \citet{cobe_exsupp}.

%Instead of using the raw time-ordered data, the DIRBE team recommends using the CIOs, which can be though of as a user friendly version of the TODs. The CIOs consist of 1/8th second calibrated observations ordered after pixel numbers. The CIOs were created pre-HEALPix era, and all three experiments aboard \COBE used the Quadrilateralized Spherical Cube pixelization (QUADCUBE) scheme which is an approximately equal-area projection where each pixel lie on one of four cube faces. The DIRBE team did provide us with a script which converts these QUADCUBE pixels into ecliptic longitue and latitude coordinates.

\subsection{Pointing, beam and bandpass response}




\subsection{Data selection and masking}
%\label{sec:tod}

In order to produce maps with full multipole support, we analyze the CIOs directly and convert the pointing into 6\farcm9 $N_\mathrm{side}=512$ pixels from the native resolution 15 Quadcube pixels with 20\arcs\ in the delivered CIOs.\footnote{\url{https://lambda.gsfc.nasa.gov/product/cobe/dirbe\_cio\_data\_get.html}} The conversion from pixel CIO pixel indices to Galactic longitude and latitude is detailed in \citet{cobe_exsupp}, and is reimplemented in a Python preprocessing script.\footnote{\url{https://github.com/MetinSa/quadcube}}

The delivered CIOs are organized into 285 single-day files, with the datapoints ordered by Quadcube pixel index. The primary processing step was converting the pointing into $N_\mathrm{side}=512$ pixels using the \texttt{quadcube} package. The data are sampled at 8\,Hz and labeled by time index in seconds since January 1, 1981 00:00 UTC. Because the data are pre-calibrated and bad data are already removed, there are some gaps in the data, which we fill manually with an appropriate flag. Additional flags, such as excess noise, orbit and attitude errors, and presence of the South Atlantic Anomaly, are additionally extracted. In total, the data are placed in one hdf5 file per band, following the format enumerated in \citet{bp03}.
The planet flags are not present in the CIOs, and are regenerated beforehand. Using the radii as defined in Table~\ref{tab:planet_flags}, we mark data points within the pointing of each pixel. Note that this is not strictly optimal due to the non-circular beam shape, and can be optimized in future analyses.

\begin{table}[t]
  \begingroup
  \newdimen\tblskip \tblskip=5pt
	\caption{List of celestial body flags.}
  \label{tab:planet_flags}
  \nointerlineskip
  \vskip -3mm
  \footnotesize
  \setbox\tablebox=\vbox{
    \newdimen\digitwidth
    \setbox0=\hbox{\rm 0}
    \digitwidth=\wd0
    \catcode`*=\active
    \def*{\kern\digitwidth}
    %
    \newdimen\signwidth
    \setbox0=\hbox{-}
    \signwidth=\wd0
    \catcode`!=\active
    \def!{\kern\signwidth}
    %
 \halign{
      \hbox to 1.5cm{#\leaderfil}\tabskip 1em&
      \hfil#\hfil\tabskip 1em&
      \hfil#\hfil\tabskip 1em&
      \hfil#\hfil\tabskip 1em&
      \hfil#\hfil\tabskip 1em&
      \hfil#\hfil\tabskip 1em&
      \hfil#\hfil\tabskip 1em&
      \hfil#\hfil\tabskip 1em&
      \hfil#\hfil\tabskip 1em&
      \hfil#\hfil\tabskip 1em&
      \hfil#\hfil\tabskip 1em&
      #\tabskip 0em\hfil\cr
    \noalign{\doubleline}
      \omit Object\hfil&
      \omit\hfil Radius ($^\circ$) \hfil\cr
      \noalign{\vskip 4pt\hrule\vskip 4pt}
      Moon****** & 10 \cr
      Mercury*** & *1 \cr
      Venus***** & *2 \cr
      Mars****** & *2 \cr
      Jupiter*** & *2 \cr
      Saturn**** & *1 \cr
      Uranus**** & *1 \cr
      Neptune*** & *1 \cr
      \noalign{\vskip 4pt\hrule\vskip 5pt} } }
  \endPlancktablewide \endgroup
\end{table}


%\subsubsection{DIRBE Calibrated Individual Observations (CIOs)}
%\subsubsection{Quadrilateralized Spherical Cube to HEALPix conversion}
%\subsubsection{Time reordering}
%\subsubsection{Gap-filling}
%\subsubsection{Regenerating planet flags}
%\subsection{Publicly available DIRBE products}



%\subsection{Data selection and masking}



\section{Ancillary data sets}

As demonstrated by the success of the \bp\ and \cosmoglobe\ projects, the use of complementary datasets with different angular resolution, frequency coverage, and observation strategies, can greatly improve the quality of low-level data processing. In this work, we use \Planck\ High Frequency Instrument, \WISE, \GAIA, and \COBE-FIRAS to better constrain our sky model and characterize the DIRBE data.


\subsection{\Planck\ HFI}


The \Planck\ High Frequency Instrument (HFI; \citealt{planck2016-l03}) observed the sky in six channels from 100\,GHz to 857\,GHz from May 2009--2013, with angular resolution of 10\arcm--\,4\arcm. While the primary purpose of the \Planck\ mission was to characterize fluctuations in the CMB, a large part of its scientific legacy comes from its observations of the far-infrared sky, with robust characterization the Milky Way \citep{planck2013-XVII,planck2014-a12,planck2016-l03} and of CIB fluctuations \citep{planck2014-a12,planck2013-XVII,lenz2019,mccarthy:2024}.

In addition to its complementary observation strategy, \Planck's frequency coverage has a relatively lower expected amount of zodiacal emission, with a total expected amplitude of $\lesssim1\,\%$ before any subtraction \citep{maris2006c,planck2013-pip88}, as compared to almost 100\,\% of the signal in some DIRBE bands. All delivered \Planck\ maps have had an estimate of the zodiacal emission modeled using the 3D model derived by K98 with varying emissivities per component. While this technically is redundant information that could contaminate this joint analysis, the already low amplitude of zodiacal emission in the HFI maps limits the potential impact of using a technically incorrect zodiacal emission model. A full analysis fitting for zodiacal emission parameters using both HFI and DIRBE will be left for future work.


At the same time, CIB fluctuations with a similar SED to the Milky Way have been detected with high significance in the HFI data, and are directly visible in 353--857\,GHz maps at high Galactic latitudes. Incorrectly modeled, this could bias the Galactic thermal dust model and lead to an incorrect model of the sky in the DIRBE range. In order to avoid this, we remove the GNILC estimate of the CIB from the HFI maps before including them in our analysis.

Since this work is primarily concerned with the DIRBE dataset, the modeling of CMB temperature fluctuations gives an unnecessary degree of freedom to be marginalized over. Therefore, we subtract the \commanderthree\ PR3 CMB temperature estimate from the \Planck\ HFI maps, effectively conditioning the entire Gibbs chain on this CMB estimate.
We use single \Planck\ detector maps to avoid the complication of subtle bandpass mismatches between nearby detectors. In total, we use the 100-1, 217-1, and 353-1 temperature maps and the total 545\,GHz and 857\,GHz maps, all from the PR4 release \citep{npipe}.

\subsection{\GAIA\ and \WISE}

In the near-infrared, most of the sky observed by DIRBE consists of stars and point sources. We therefore use catalogs derived from external datasets as fixed locations of each source, while fitting for the amplitudes in the Gibbs chain. As of this publication, the most complete catalog of stars in the  Milky Way comes from the \Gaia\ mission \cite{gaia:2016,gaia:2018}. In particular, physics models of 100\,million stars are provided in \Gaia\ DR2. Despite \Gaia\ operating between 330--1050\,nm, these models can be used as informative priors for the amplitude of stars at the DIRBE near-infrared bands. With a limiting magnitude of $G\sim20$ across the entire sky,

Conversely, the \WISE\ satellite \citep{wright:2010} has mapped the sky at 3.4, 4.6, 12, and 22 $\mathrm{\mu m}$, with resolutions of 6\farcs11, 6\farcs4, 6\farcs5, and 12\farcs0. This gives a direct estimate of point source brightness and location, and allows for direct cross-matching with the \GAIA\ DR2 catalog. In order to use the leverage the \gaia\ data properly, we extract the SED for point sources in both \gaia\ and \WISE\ with $<8$ mag at $3.4\,\mathrm{\mu m}$. These SEDs are then scaled directly per star in the Gibbs chain to correct for absolute calibration differences between the \gaia+\WISE\ catalog and DIRBE. There are a total of 717\,454 stars in both catalogs. There are an additional 3122 extragalactic sources that exist in \WISE\ but not in \gaia. These are fit as modified blackbodies per source.


Within the Galactic plane, there are many stars per single DIRBE pixel. In order to avoid degeneracies between individual point sources, we create a map at $N_\mathrm{side}=512$ that includes all \WISE\ point sources that are $>8$ mag at $3.4\,\mathrm{\mu m}$ but have not been identified within the \gaia\ catalog. The SED's as derived by \gaia\ are then averaged over and used to scale the entire template. Within the Gibbs chain, this is sampled a total scaling parameter, with a fixed map and relative amplitudes between different frequencies. For a full description of the star model, see the work in companion paper \citet{CG02_04}.



\subsection{\COBE-FIRAS}

The \COBE-FIRAS experiment was an absolutely calibrated differential Michelson Fourier transform interferometer that observed the full sky from 68\,GHz--2911\,GHz with 13.6\,GHz frequency resolution \citep{fixsen:1994,mather:1999}. Due to the observation strategy, there are non-negligible correlations between nearby frequencies, and smearing along the scanning direction, corresponding to an effectively nonsymmetric beam.

The dense frequency spacing of the FIRAS data makes it ideal for determining the continuum behavior of sky emission and allows for identification of emission lines \citep{bennett:1994}.


%\section{Data selection and goodness-of-fit}
%\label{sec:data_selection}

%\subsection{Baseline data selection}

\subsection{Mask definitions}

\begin{figure}
  \centering
  \includegraphics[width=\columnwidth]{figs/mask_01.pdf}
  \includegraphics[width=\columnwidth]{figs/mask_06.pdf}
  \includegraphics[width=\columnwidth]{figs/mask_10.pdf}
  \caption{Processing masks use in the analysis. Green regions correspond to the general TOD processing masks, and the blue regions correspond to the zodiacal emission masks.}
  \label{fig:masks}
\end{figure}



%\subsubsection{Galactic masks}

%\subsubsection{Solar-centered residual masks}

%\subsection{Goodness-of-fit statistics}

%\subsection{Summary of included data}

\section{Excess radiation model}

\begin{figure*}
  \centering
  \includegraphics[width=0.40\linewidth]{figs/solarmap_01_v1.pdf}\hspace*{5mm}
  \includegraphics[width=0.40\linewidth]{figs/solarmap_02_v1.pdf}\\
  \includegraphics[width=0.40\linewidth]{figs/solarmap_03_v1.pdf}\hspace*{5mm}
  \includegraphics[width=0.40\linewidth]{figs/solarmap_04_v1.pdf}\\
  \includegraphics[width=0.40\linewidth]{figs/solarmap_05_v1.pdf}\hspace*{5mm}
  \includegraphics[width=0.40\linewidth]{figs/solarmap_06_v1.pdf}\\
  \includegraphics[width=0.40\linewidth]{figs/solarmap_07_v1.pdf}\hspace*{5mm}
  \includegraphics[width=0.40\linewidth]{figs/solarmap_08_v1.pdf}\\
  \includegraphics[width=0.40\linewidth]{figs/solarmap_09_v1.pdf}\hspace*{5mm}
  \includegraphics[width=0.40\linewidth]{figs/solarmap_10_v1.pdf}
  \caption{Solar-centric residual maps, derived by co-adding the residual TOD, $\r = \dv - \s_{\mathrm{fg}} - \s_{\mathrm{zodi}} - \n_{\mathrm{corr}}$, into solar-centric coordinates. The Sun is located in the center of each panel, and the equator is aligned with the Ecliptic plane. The gray boundaries indicate the solar-centric exclusion masks used for each channel; no masks are applied for 140 and 240\,$\mu$m. }
  \label{fig:solarmaps}
\end{figure*}



%\begin{figure*}
%  \centering
%  \includegraphics[width=0.40\linewidth]{figs/mean_solar_04.pdf}\hspace*{5mm}
%  \includegraphics[width=0.40\linewidth]{figs/mean_solar_05.pdf}\\
%  \includegraphics[width=0.40\linewidth]{figs/mean_solar_06.pdf}\hspace*{5mm}
%  \includegraphics[width=0.40\linewidth]{figs/mean_solar_07.pdf}\\
%  \includegraphics[width=0.40\linewidth]{figs/mean_solar_08.pdf}\hspace*{5mm}
%  \phantom{\includegraphics[width=0.40\linewidth]{figs/mean_solar_05.pdf}}
%  \caption{Mean solar-centric maps}
%  \label{fig:solarmaps_mean}
%\end{figure*}

\begin{figure}
  \centering
  \includegraphics[width=0.83\linewidth]{figs/rms_solar_04.pdf}\\\vspace*{-2mm}
  \includegraphics[width=0.83\linewidth]{figs/rms_solar_05.pdf}\\\vspace*{-2mm}
  \includegraphics[width=0.83\linewidth]{figs/rms_solar_06.pdf}\\\vspace*{-2mm}
  \includegraphics[width=0.83\linewidth]{figs/rms_solar_07.pdf}\\\vspace*{-2mm}
  \includegraphics[width=0.83\linewidth]{figs/rms_solar_08.pdf}
  \caption{RMS solar-centric maps}
  \label{fig:solarmaps_rms}
\end{figure}

\clearpage
\section{Markov chains, burn-in and convergence}
\label{sec:chains}

\begin{figure*}
    \centering
    \includegraphics[width=0.975\textwidth]{figs/total_trace.pdf}
    \caption{Trace plots. {\bf (Needs to be updated!)}}
    \label{fig:trace}
\end{figure*}

\begin{figure*}
    \centering
    \includegraphics[width=1\textwidth]{figs/correlation_matrix.pdf}
    \caption{Correlations between a selection of all parameters. {\bf (Needs to be updated!)}}
    \label{fig:correlations}
\end{figure*}


\section{Noise estimation and goodness-of-fit}

\subsection{Instrumental noise}


\begin{figure}
	\centering
	\includegraphics[width=\columnwidth]{figs/sigma0_bands.pdf}
	\caption{Instrumental noise $\sigma_0$ for each band, averaged over all Gibbs chains. Data not used in the Gibbs chain are not displayed.}
	\label{fig:sigma0}
\end{figure}

\begin{figure*}
  \centering
  \includegraphics[width=0.39\linewidth]{figs/rms_maps/rms_10.pdf}       
  \includegraphics[width=0.39\linewidth]{figs/rms_maps/rms_09.pdf}\\
  \includegraphics[width=0.39\linewidth]{figs/rms_maps/rms_08.pdf}       
  \includegraphics[width=0.39\linewidth]{figs/rms_maps/rms_07.pdf}\\
  \includegraphics[width=0.39\linewidth]{figs/rms_maps/rms_06.pdf}       
  \includegraphics[width=0.39\linewidth]{figs/rms_maps/rms_05.pdf}\\       
  \includegraphics[width=0.39\linewidth]{figs/rms_maps/rms_04.pdf}         
  \includegraphics[width=0.39\linewidth]{figs/rms_maps/rms_03.pdf}\\
  \includegraphics[width=0.39\linewidth]{figs/rms_maps/rms_02.pdf}
  \includegraphics[width=0.39\linewidth]{figs/rms_maps/rms_01.pdf}       
	\caption{White noise rms maps for each DIRBE channel. All maps are in units of $\mathrm{MJy\,sr^{-1}}$.}
  \label{fig:sigma0_map}
\end{figure*}


\begin{figure}
  % script to make these are in /mn/stornext/u3/metins/dirbe/DR2/plot_maps.py (cosmoglobe_analysis_plotter.plot_ncorr_maps(sample=19))
	\centering
  
	\includegraphics[width=\columnwidth]{figs/ncorr_240a.pdf}

	\caption{Correlated noise sample for the first half-mission of the 240$\,\mu$m channel.}
	\label{fig:ncorr}
\end{figure}

\begin{figure}
	\centering
	\includegraphics[width=\columnwidth]{figs/ncorr_powerspec.pdf}
	\caption{Correlated noise angular power spectrum for the 240$\,\mu$m channel.
	\textbf{Needs to be updated}}
	\label{fig:ncorrpowspec}
\end{figure}


\subsection{Goodness-of-fit}



       
       \begin{figure*}
       	\centering
       	\includegraphics[width=0.39\linewidth]{figs/dirbe_10_todres_3deg_v1.pdf}       
       	\includegraphics[width=0.39\linewidth]{figs/dirbe_09_todres_3deg_v1.pdf}\\
       	\includegraphics[width=0.39\linewidth]{figs/dirbe_08_todres_v1.pdf}       
       	\includegraphics[width=0.39\linewidth]{figs/dirbe_07_todres_v1.pdf}\\
       	\includegraphics[width=0.39\linewidth]{figs/dirbe_06_todres_v1.pdf}       
       	\includegraphics[width=0.39\linewidth]{figs/dirbe_05_todres_v1.pdf}\\       
        \includegraphics[width=0.39\linewidth]{figs/dirbe_04_todres_v1.pdf}         
        \includegraphics[width=0.39\linewidth]{figs/dirbe_03_todres_v1.pdf}\\
       	\includegraphics[width=0.39\linewidth]{figs/dirbe_02_todres_v1.pdf}
       	\includegraphics[width=0.39\linewidth]{figs/dirbe_01_todres_v1.pdf}       
       	\caption{Data-minus-model residual maps for each band. The 140 and 240\,$\mu$m channels have been smoothed to an angular resolution of $3^{\circ}$, while all others are shown at their native resolution. }
       	\label{fig:res}
       \end{figure*}

       \begin{figure*}
       	\centering
       	\includegraphics[width=0.39\linewidth]{figs/dirbe_10_hmhd_3deg_v1.pdf}       
       	\includegraphics[width=0.39\linewidth]{figs/dirbe_09_hmhd_3deg_v1.pdf}\\
       	\includegraphics[width=0.39\linewidth]{figs/dirbe_08_hmhd_v1.pdf}       
       	\includegraphics[width=0.39\linewidth]{figs/dirbe_07_hmhd_v1.pdf}\\
       	\includegraphics[width=0.39\linewidth]{figs/dirbe_06_hmhd_v1.pdf}       
       	\includegraphics[width=0.39\linewidth]{figs/dirbe_05_hmhd_v1.pdf}\\       
        \includegraphics[width=0.39\linewidth]{figs/dirbe_04_hmhd_v1.pdf}         
        \includegraphics[width=0.39\linewidth]{figs/dirbe_03_hmhd_v1.pdf}\\
       	\includegraphics[width=0.39\linewidth]{figs/dirbe_02_hmhd_v1.pdf}
       	\includegraphics[width=0.39\linewidth]{figs/dirbe_01_hmhd_v1.pdf}       
       	\caption{Half-mission half-difference maps for each channel. The 140 and 240\,$\mu$m channels have been smoothed to an angular resolution of $3^{\circ}$, while all others are shown at their native resolution. }
       	\label{fig:hmhd}
       \end{figure*}
       

       \begin{figure}
       	\centering
       	\includegraphics[width=\linewidth]{figs/chisq_CG02_c1.pdf}
       	\caption{Pixel-space reduced normalized $\chi^2$ in units of $\sigma$. The number of degrees-of-freedom per pixel is here assumed to be 103, which is the sum of the number of observations per $N_{\mathrm{side}}=512$ pixel minus the number of freely fitted Galactic components. }
       	\label{fig:chisq}
       \end{figure}

       \begin{figure}
	 \centering
	 \includegraphics[width=\columnwidth]{figs/cls_DR2_hmhd_v1.pdf}
	 \caption{Half-mission half-difference angular power spectra (blue curves) for each DIRBE channel computed with the CIB monopole masks defined by \citet{CG02_03}. The red curves show the power spectrum computed from one white noise realization with the same mask.}
	 \label{fig:hmhd_powspec}
       \end{figure}



%\begin{figure*}
%	\centering
%	\includegraphics[width=\linewidth]{figs/diff_grid.png}
	% Figure to make this script is currently in /mn/stornext/d5/data/duncanwa/DIRBE/plots/diff_grid.py
%	\caption{Cosmoglobe, DIRBE ZSMA official maps, their difference, and the internal halfmission splits for Cosmoglobe.}
%	\label{fig:diff_grid}
%\end{figure*}


%\begin{figure}
%	\centering
%	\includegraphics[width=\columnwidth]{figs/sigma0_trace.pdf}
%	\caption{Sigma0 for scan 100}
%	\label{fig:sigma0_band8}
%\end{figure}


%\begin{figure*}

%  \centering
%    \includegraphics[width=0.8\columnwidth]{figs/freq_maps/freq_01_c0001_000022.pdf}\includegraphics[width=0.8\columnwidth]{figs/freq_maps/freq_06_c0001_000022.pdf}
    % \vspace*{-0.5cm}

%    \includegraphics[width=0.4\columnwidth]{figs/freq_maps/freq_cbar_01_c0001_000022.pdf}\hspace{3.6cm}\includegraphics[width=0.4\columnwidth]{figs/freq_maps/freq_cbar_06_c0001_000022.pdf}

%    \includegraphics[width=0.8\columnwidth]{figs/freq_maps/freq_02_c0001_000022.pdf}\includegraphics[width=0.8\columnwidth]{figs/freq_maps/freq_07_c0001_000022.pdf}
    % \vspace*{-0.5cm}

%    \includegraphics[width=0.4\columnwidth]{figs/freq_maps/freq_cbar_02_c0001_000022.pdf}\hspace{3.6cm}\includegraphics[width=0.4\columnwidth]{figs/freq_maps/freq_cbar_07_c0001_000022.pdf}

  
%    \includegraphics[width=0.8\columnwidth]{figs/freq_maps/freq_03_c0001_000022.pdf}\includegraphics[width=0.8\columnwidth]{figs/freq_maps/freq_08_c0001_000022.pdf}
    % \vspace*{-0.5cm}

%    \includegraphics[width=0.4\columnwidth]{figs/freq_maps/freq_cbar_03_c0001_000022.pdf}\hspace{3.6cm}\includegraphics[width=0.4\columnwidth]{figs/freq_maps/freq_cbar_08_c0001_000022.pdf}

  
%    \includegraphics[width=0.8\columnwidth]{figs/freq_maps/freq_04_c0001_000022.pdf}\includegraphics[width=0.8\columnwidth]{figs/freq_maps/freq_09_c0001_000022.pdf}
    % \vspace*{-0.5cm}

%    \includegraphics[width=0.4\columnwidth]{figs/freq_maps/freq_cbar_04_c0001_000022.pdf}\hspace{3.6cm}\includegraphics[width=0.4\columnwidth]{figs/freq_maps/freq_cbar_09_c0001_000022.pdf}

  
%    \includegraphics[width=0.8\columnwidth]{figs/freq_maps/freq_05_c0001_000022.pdf}\includegraphics[width=0.8\columnwidth]{figs/freq_maps/freq_10_c0001_000022.pdf}
    % \vspace*{-0.5cm}

%    \includegraphics[width=0.4\columnwidth]{figs/freq_maps/freq_cbar_05_c0001_000022.pdf}\hspace{3.6cm}\includegraphics[width=0.4\columnwidth]{figs/freq_maps/freq_cbar_10_c0001_000022.pdf}

  
  
%    \caption{Frequency maps}
%    \label{fig:freq_maps}
%\end{figure*}

\clearpage
\section{Frequency maps}
\label{sec:maps}

\subsection{ZSMA frequency maps}

%\begin{figure*}
%	\centering
%	\includegraphics[width=0.43\linewidth]{figs/map_01.pdf}
%	\includegraphics[width=0.43\linewidth]{figs/map_02.pdf}\\
%        \includegraphics[width=0.43\linewidth]{figs/map_03.pdf}
%        \includegraphics[width=0.43\linewidth]{figs/map_04.pdf}\\
%        \includegraphics[width=0.43\linewidth]{figs/map_05.pdf}
%        \includegraphics[width=0.43\linewidth]{figs/map_06.pdf}\\
%        \includegraphics[width=0.43\linewidth]{figs/map_07.pdf}
%        \includegraphics[width=0.43\linewidth]{figs/map_08.pdf}\\
%        \includegraphics[width=0.43\linewidth]{figs/map_09.pdf}
%        \includegraphics[width=0.43\linewidth]{figs/map_10.pdf}
%	\caption{\cosmoglobe\ DR2 ZSMA frequency maps. Missing pixels have been replaced with the median of values within a $2^\circ$ radius.}
%	\label{fig:freqmaps}
%\end{figure*}


\begin{figure*}
	\centering
	\includegraphics[width=\linewidth]{figs/map_01.pdf}\\
	\includegraphics[width=\linewidth]{figs/map_02.pdf}
	\caption{\cosmoglobe\ DR2 ZSMA maps at 1.25 (\emph{top}) and
          2.2$\,\mu$m (\emph{bottom}). Missing pixels have been replaced with
          the median of values within a $2^\circ$ radius.}
	\label{fig:freqmaps1_2}
\end{figure*}

\begin{figure*}
	\centering
	\includegraphics[width=\linewidth]{figs/map_03.pdf}\\
	\includegraphics[width=\linewidth]{figs/map_04.pdf}
	\caption{\cosmoglobe\ DR2 ZSMA maps at 3.5 (\emph{top}) and
          4.9$\,\mu$m (\emph{bottom}). Missing pixels have been replaced with
          the median of values within a $2^\circ$ radius.}
	\label{fig:freqmaps3_4}
\end{figure*}

\begin{figure*}
	\centering
	\includegraphics[width=\linewidth]{figs/map_05.pdf}\\
	\includegraphics[width=\linewidth]{figs/map_06.pdf}
	\caption{\cosmoglobe\ DR2 ZSMA maps at 12 (\emph{top}) and
          25$\,\mu$m (\emph{bottom}). Missing pixels have been replaced with
          the median of values within a $2^\circ$ radius.}
	\label{fig:freqmaps5_6}
\end{figure*}

\begin{figure*}
	\centering
	\includegraphics[width=\linewidth]{figs/map_07.pdf}\\
	\includegraphics[width=\linewidth]{figs/map_08.pdf}
	\caption{\cosmoglobe\ DR2 ZSMA maps at 60 (\emph{top}) and
          100$\,\mu$m (\emph{bottom}). Missing pixels have been replaced with
          the median of values within a $2^\circ$ radius.}
	\label{fig:freqmaps7_8}
\end{figure*}

\begin{figure*}
	\centering
	\includegraphics[width=\linewidth]{figs/map_09.pdf}\\
	\includegraphics[width=\linewidth]{figs/map_10.pdf}
	\caption{\cosmoglobe\ DR2 ZSMA maps at 140 (\emph{top}) and
          240$\,\mu$m (\emph{bottom}). Missing pixels have been replaced with
          the median of values within a $2^\circ$ radius.}
	\label{fig:freqmaps9_10}
\end{figure*}


%\begin{figure*}
%	\centering
%	\includegraphics[width=0.58\linewidth]{figs/map_01.pdf}
%	\includegraphics[width=0.58\linewidth]{figs/map_02.pdf}
%	\includegraphics[width=0.58\linewidth]{figs/map_03.pdf}
%	\includegraphics[width=0.58\linewidth]{figs/map_04.pdf}
%	\caption{DIRBE bands 1--4. Missing pixels have been replaced with the median of values within a $2^\circ$ radius.}
%	\label{fig:freqmaps1_4}
%\end{figure*}

%\begin{figure*}
%	\centering
%	\includegraphics[width=0.73\linewidth]{figs/map_05.pdf}
%	\includegraphics[width=0.73\linewidth]{figs/map_06.pdf}
%	\includegraphics[width=0.73\linewidth]{figs/map_07.pdf}
%	\caption{DIRBE bands 5--7. Missing pixels have been replaced with the median of values within a $2^\circ$ radius.}
%	\label{fig:freqmaps5_7}
%\end{figure*}

%\begin{figure*}
%	\centering
%	\includegraphics[width=0.73\linewidth]{figs/map_08.pdf}
%	\includegraphics[width=0.73\linewidth]{figs/map_09.pdf}
%	\includegraphics[width=0.73\linewidth]{figs/map_10.pdf}
%	\caption{DIRBE bands 8--10. Missing pixels have been replaced with the median of values within a $2^\circ$ radius.}
%	\label{fig:freqmaps8_10}
%\end{figure*}

\begin{figure*}
  \centering
  \includegraphics[width=0.39\linewidth]{figs/std_10.pdf}       
  \includegraphics[width=0.39\linewidth]{figs/std_09.pdf}\\
  \includegraphics[width=0.39\linewidth]{figs/std_08.pdf}       
  \includegraphics[width=0.39\linewidth]{figs/std_07.pdf}\\
  \includegraphics[width=0.39\linewidth]{figs/std_06.pdf}       
  \includegraphics[width=0.39\linewidth]{figs/std_05.pdf}\\       
  \includegraphics[width=0.39\linewidth]{figs/std_04.pdf}         
  \includegraphics[width=0.39\linewidth]{figs/std_03.pdf}\\
  \includegraphics[width=0.39\linewidth]{figs/std_02.pdf}
  \includegraphics[width=0.39\linewidth]{figs/std_01.pdf}       
  \caption{Posterior rms maps for each DIRBE channel.  Posterior standard deviation maps for each DIRBE channel. All maps are in units of $\mathrm{MJy\,sr^{-1}}$.}
  \label{fig:rms}
\end{figure*}






\subsection{Comparison with previous products}

\begin{table*}[t]
  \begingroup
  \newdimen\tblskip \tblskip=5pt
  \caption{}
  \label{tab:summary}
  \nointerlineskip
  \vskip -3mm
  \footnotesize
  \setbox\tablebox=\vbox{
    \newdimen\digitwidth
    \setbox0=\hbox{\rm 0}
    \digitwidth=\wd0
    \catcode`*=\active
    \def*{\kern\digitwidth}
    %
    \newdimen\signwidth
    \setbox0=\hbox{-}
    \signwidth=\wd0
    \catcode`!=\active
    \def!{\kern\signwidth}
    %
 \halign{
      \hbox to 1.5cm{#\leaderfil}\tabskip 1em&
      \hfil#\hfil\tabskip 1em&
      \hfil#\hfil\tabskip 1em&
      \hfil#\hfil\tabskip 1em&
      \hfil#\hfil\tabskip 1em&
      \hfil#\hfil\tabskip 1em&
      \hfil#\hfil\tabskip 1em&
      \hfil#\hfil\tabskip 0em\cr
      \noalign{\doubleline}
      \omit&\multispan3\hfil\sc Bandpass \hfil&\multispan2\hfil\sc Relative calibration, $\alpha$ \hfil&\multispan2\hfil\sc Noise rms (kJy/sr)) \hfil\cr
\noalign{\vskip -3pt}
\omit&\multispan3\hrulefill&\multispan2\hrulefill&\multispan2\hrulefill\cr
\noalign{\vskip 3pt} 
\omit\sc Channel ID\hfil& $\nu_{\mathrm{c}}$ (THz) & $\lambda_{\mathrm{c}}$ ($\mu$m) & $\Delta\nu/\nu\hfil$ & High lat & Full sky & DIRBE & DR2 \cr
\noalign{\vskip 3pt\hrule\vskip 5pt}
      10 & 1.25 & 240  & 0.40 &  0.00 & 0.00 & *860 & *730 \cr
      *9 & 2.14 & 140  & 0.28 &  0.00 & 0.00 & 1530 & 1240 \cr
      *8 & 3.00 & 100  & 0.32 &  0.00 & 0.00 & **17 & **23 \cr
      *7 & 5.00 & *60  & 0.46 &  0.00 & 0.00 & **18 & **21 \cr
      *6 & 12.0 & *25  & 0.34 &  0.00 & 0.00 & ***7.5 & **15 \cr
      *5 & 25.0 & *12  & 0.53 &  0.00 & 0.00 & ***3.6 & **10 \cr
      *4 & 61.2 & 4.9  & 0.13 &  0.00 & 0.00 & ***1.3 & ***1.7 \cr
      *3 & 85.7 & 3.5  & 0.26 &  0.00 & 0.00 & ***1.1 & ***1.3 \cr
      *2 & 136  & 2.2  & 0.16 &  0.00 & 0.00 & ***1.2 & ***1.6 \cr
      *1 & 240  & 1.25 & 0.25 &  0.00 & 0.00 & ***1.0 & ***1.3 \cr
      \noalign{\vskip 4pt\hrule\vskip 5pt} } }
  \endPlancktablewide \endgroup
\end{table*}



\begin{figure*}
	\centering
	\includegraphics[width=0.33\linewidth]{figs/CG_DIRBE_01_I_n0256_DR2.pdf}
        \includegraphics[width=0.33\linewidth]{figs/DIRBE_ZSMA_01_1_256.pdf}
        \includegraphics[width=0.33\linewidth]{figs/diff_CG_DIRBE_ZSMA_01_n256.pdf}\\
	\includegraphics[width=0.33\linewidth]{figs/CG_DIRBE_02_I_n0256_DR2.pdf}
        \includegraphics[width=0.33\linewidth]{figs/DIRBE_ZSMA_02_1_256.pdf}
        \includegraphics[width=0.33\linewidth]{figs/diff_CG_DIRBE_ZSMA_02_n256.pdf}\\
	\includegraphics[width=0.33\linewidth]{figs/CG_DIRBE_03_I_n0256_DR2.pdf}
        \includegraphics[width=0.33\linewidth]{figs/DIRBE_ZSMA_03_1_256.pdf}
        \includegraphics[width=0.33\linewidth]{figs/diff_CG_DIRBE_ZSMA_03_n256.pdf}\\
	\includegraphics[width=0.33\linewidth]{figs/CG_DIRBE_04_I_n0256_DR2.pdf}
        \includegraphics[width=0.33\linewidth]{figs/DIRBE_ZSMA_04_1_256.pdf}
        \includegraphics[width=0.33\linewidth]{figs/diff_CG_DIRBE_ZSMA_04_n256.pdf}\\
	\includegraphics[width=0.33\linewidth]{figs/CG_DIRBE_05_I_n0256_DR2.pdf}
        \includegraphics[width=0.33\linewidth]{figs/DIRBE_ZSMA_05_1_256.pdf}
        \includegraphics[width=0.33\linewidth]{figs/diff_CG_DIRBE_ZSMA_05_n256.pdf}
	\caption{Comparison of \Cosmoglobe\ DR2 (\emph{left column}) and K98 (\emph{middle column}) zodiacal light subtracted mission average maps for the 1.25--12$\,\mu$m channels. Difference maps are shown in the rightmost column. Full maps are plotted with a non-linear color scale, while difference maps are plotted with a linear and symmetric color range.}
	\label{fig:freqmaps_cg_vs_dirbe1}
\end{figure*}


\begin{figure*}
	\centering
	\includegraphics[width=0.33\linewidth]{figs/CG_DIRBE_06_I_n0256_DR2.pdf}
        \includegraphics[width=0.33\linewidth]{figs/DIRBE_ZSMA_06_1_256.pdf}
        \includegraphics[width=0.33\linewidth]{figs/diff_CG_DIRBE_ZSMA_06_n256.pdf}\\
	\includegraphics[width=0.33\linewidth]{figs/CG_DIRBE_07_I_n0256_DR2.pdf}
        \includegraphics[width=0.33\linewidth]{figs/DIRBE_ZSMA_07_1_256.pdf}
        \includegraphics[width=0.33\linewidth]{figs/diff_CG_DIRBE_ZSMA_07_n256.pdf}\\
	\includegraphics[width=0.33\linewidth]{figs/CG_DIRBE_08_I_n0256_DR2.pdf}
        \includegraphics[width=0.33\linewidth]{figs/DIRBE_ZSMA_08_1_256.pdf}
        \includegraphics[width=0.33\linewidth]{figs/diff_CG_DIRBE_ZSMA_08_n256.pdf}\\
        \includegraphics[width=0.33\linewidth]{figs/CG_DIRBE_09_I_n0256_DR2.pdf}
        \includegraphics[width=0.33\linewidth]{figs/DIRBE_ZSMA_09_1_256.pdf}
        \includegraphics[width=0.33\linewidth]{figs/diff_CG_DIRBE_ZSMA_09_n256.pdf}\\
        \includegraphics[width=0.33\linewidth]{figs/CG_DIRBE_10_I_n0256_DR2.pdf}
        \includegraphics[width=0.33\linewidth]{figs/DIRBE_ZSMA_10_1_256.pdf}
        \includegraphics[width=0.33\linewidth]{figs/diff_CG_DIRBE_ZSMA_10_n256.pdf}\\
	\caption{Same as Fig.~\ref{fig:freqmaps_cg_vs_dirbe1}, but for the 25--240$\,\mu$m channels.}
	\label{fig:freqmaps_cg_vs_dirbe2}
\end{figure*}

\subsection{Angular power spectra}

\begin{figure}
	\centering
	\includegraphics[width=\columnwidth]{figs/cls_DR2_vs_DIRBE_v1.pdf}
	\caption{Comparison of angular power spectra computed from the DIRBE (red) and DR2 (blue) ZSMA maps.}
	\label{fig:powspec}
\end{figure}




%       \clearpage
%       \section{\Cosmoglobe\ Data Release 2}
%       \label{sec:astrophysics}
%       
%       \subsection{Frequency maps}
%       
%       \subsection{Zodiacal light emission}
%       
%       \clearpage
%       \subsection{Galactic emission}
%       
%       \begin{figure*}
%       	\centering
%       	\includegraphics[width=0.49\linewidth]{figs/CG_dust_v05_I_MEAN_w12_n2048_cb_c-sunburst.pdf}
%               \includegraphics[width=0.49\linewidth]{figs/CG_CO_tot_v05_I_MEAN_w12_n1024_cb_c-afmhot.pdf}\\
%               \includegraphics[width=0.49\linewidth]{figs/CG_freefree_v05_I_MEAN_w12_n1024_cb_c-freeze.pdf}
%               \includegraphics[width=0.49\linewidth]{figs/CG_stars_n0512_v06_I_MEAN_w12_n512_cb_c-afmhot.pdf}\\
%       	\caption{}
%       	\label{fig:fg_amp}
%       \end{figure*}
%       
%       \begin{figure*}
%         \centering
%         \includegraphics[width=0.19\linewidth]{figs/compfreq_mapzodi_10a_v01.pdf}
%         \includegraphics[width=0.19\linewidth]{figs/compfreq_zodi_10a_v01.pdf}
%         \includegraphics[width=0.19\linewidth]{figs/compfreq_dusttot_10a_v01.pdf}
%         \includegraphics[width=0.19\linewidth]{figs/compfreq_white_nobar.pdf}  
%         \includegraphics[width=0.19\linewidth]{figs/compfreq_todres_10a_v01.pdf}\\
%         \includegraphics[width=0.19\linewidth]{figs/compfreq_mapzodi_09a_v01.pdf}
%         \includegraphics[width=0.19\linewidth]{figs/compfreq_zodi_09a_v01.pdf}
%         \includegraphics[width=0.19\linewidth]{figs/compfreq_dusttot_09a_v01.pdf}
%         \includegraphics[width=0.19\linewidth]{figs/compfreq_white_nobar.pdf}  
%         \includegraphics[width=0.19\linewidth]{figs/compfreq_todres_09a_v01.pdf}\\
%         \includegraphics[width=0.19\linewidth]{figs/compfreq_mapzodi_08a_v01.pdf}
%         \includegraphics[width=0.19\linewidth]{figs/compfreq_zodi_08a_v01.pdf}
%         \includegraphics[width=0.19\linewidth]{figs/compfreq_dusttot_08a_v01.pdf}
%         \includegraphics[width=0.19\linewidth]{figs/compfreq_white_nobar.pdf}  
%         \includegraphics[width=0.19\linewidth]{figs/compfreq_todres_08a_v01.pdf}\\
%         \includegraphics[width=0.19\linewidth]{figs/compfreq_mapzodi_07a_v01.pdf}
%         \includegraphics[width=0.19\linewidth]{figs/compfreq_zodi_07a_v01.pdf}
%         \includegraphics[width=0.19\linewidth]{figs/compfreq_dusttot_07a_v01.pdf}
%         \includegraphics[width=0.19\linewidth]{figs/compfreq_white_nobar.pdf}  
%         \includegraphics[width=0.19\linewidth]{figs/compfreq_todres_07a_v01.pdf}\\
%         \includegraphics[width=0.19\linewidth]{figs/compfreq_mapzodi_06a_v01.pdf}
%         \includegraphics[width=0.19\linewidth]{figs/compfreq_zodi_06a_v01.pdf}
%         \includegraphics[width=0.19\linewidth]{figs/compfreq_dusttot_06a_v01.pdf}
%         \includegraphics[width=0.19\linewidth]{figs/compfreq_stars_06a_v01.pdf}  
%         \includegraphics[width=0.19\linewidth]{figs/compfreq_todres_06a_v01.pdf}\\  
%         \includegraphics[width=0.19\linewidth]{figs/compfreq_mapzodi_05a_v01.pdf}
%         \includegraphics[width=0.19\linewidth]{figs/compfreq_zodi_05a_v01.pdf}
%         \includegraphics[width=0.19\linewidth]{figs/compfreq_dusttot_05a_v01.pdf}
%         \includegraphics[width=0.19\linewidth]{figs/compfreq_stars_05a_v01.pdf}  
%         \includegraphics[width=0.19\linewidth]{figs/compfreq_todres_05a_v01.pdf}\\
%         \includegraphics[width=0.19\linewidth]{figs/compfreq_mapzodi_04a_v01.pdf}
%         \includegraphics[width=0.19\linewidth]{figs/compfreq_zodi_04a_v01.pdf}
%         \includegraphics[width=0.19\linewidth]{figs/compfreq_white_nobar.pdf}
%         \includegraphics[width=0.19\linewidth]{figs/compfreq_stars_04a_v01.pdf}  
%         \includegraphics[width=0.19\linewidth]{figs/compfreq_todres_04a_v01.pdf}\\  
%         \includegraphics[width=0.19\linewidth]{figs/compfreq_mapzodi_03a_v01.pdf}
%         \includegraphics[width=0.19\linewidth]{figs/compfreq_zodi_03a_v01.pdf}
%         \includegraphics[width=0.19\linewidth]{figs/compfreq_white_nobar.pdf}
%         \includegraphics[width=0.19\linewidth]{figs/compfreq_stars_03a_v01.pdf}  
%         \includegraphics[width=0.19\linewidth]{figs/compfreq_todres_03a_v01.pdf}\\  
%         \includegraphics[width=0.19\linewidth]{figs/compfreq_mapzodi_02a_v01.pdf}
%         \includegraphics[width=0.19\linewidth]{figs/compfreq_zodi_02a_v01.pdf}
%         \includegraphics[width=0.19\linewidth]{figs/compfreq_white_nobar.pdf}
%         \includegraphics[width=0.19\linewidth]{figs/compfreq_stars_02a_v01.pdf}  
%         \includegraphics[width=0.19\linewidth]{figs/compfreq_todres_02a_v01.pdf}\\
%         \includegraphics[width=0.19\linewidth]{figs/compfreq_mapzodi_01a_v01.pdf}
%         \includegraphics[width=0.19\linewidth]{figs/compfreq_zodi_01a_v01.pdf}
%         \includegraphics[width=0.19\linewidth]{figs/compfreq_white_nobar.pdf}
%         \includegraphics[width=0.19\linewidth]{figs/compfreq_stars_01a_v01.pdf}
%         \includegraphics[width=0.19\linewidth]{figs/compfreq_todres_01a_v01.pdf}\\
%         \includegraphics[width=0.50\linewidth]{figs/colourbar_MJysr.pdf}
%         \caption{Comparison between the raw DIRBE data and the various fitted components for one single Gibbs sample. Columns show, from left to right, 1) the time-ordered DIRBE data co-added into pixelized maps; 2) zodiacal light emission; 3) thermal dust emission; 4) star emission; and 5) data-minus-model residual emission. Rows show individual frequency channels. Missing entries corresponds to components that are forced to zero in the model. Note that all panels are plotted with the same color scale in units of MJy/sr, and can be directly compared.}
%         \label{fig:comp_vs_freq}
%       \end{figure*}
%       
%       \begin{figure*}
%       	\centering
%       	\includegraphics[width=\textwidth]{figs/all_fgs.pdf}
%       	\caption{Summary of all foregrounds.}
%       	\label{fig:SED_overview}
%       \end{figure*}
%       
%       
%       \clearpage
%       %\subsection{Goodness-of-fit and residual maps}
%       
%       
%       
%       \clearpage
%       \subsection{Preliminary CIB constraints}
%       
%       
%       
%       \begin{figure}
%       	\centering
%       	\includegraphics[width=\linewidth]{figs/CIB_THz.pdf}
%       	\caption{CIB monopole}
%       	\label{fig:CIB_mono}
%       \end{figure}
%       
%       \begin{figure}
%       	\centering
%       	\includegraphics[width=\linewidth]{figs/cross_8.png}
%       	\caption{Power spectrum}
%       	\label{fig:CIB_spectrum}
%       \end{figure}
%       
%       
%       
%       %\clearpage
%       %\section{Future directions}
%       %Most of the data from the DIRBE experiment is publicly available on the LAMBDA page along with a explanetory supplement which describes how to use it.
%       
%       
%       \section{The infrared \Cosmoglobe\ sky model}
%       \label{sec:maps}
%       
%       \subsection{Astrophysical component decomposition}
%       
%       \subsection{Outstanding issues and future directions}


\clearpage
\section{Conclusions}
\label{sec:conclusions}




\blindtext





\begin{acknowledgements}
 The current work has received funding from the European
  Union’s Horizon 2020 research and innovation programme under grant
  agreement numbers 819478 (ERC; \textsc{Cosmoglobe}) and 772253 (ERC;
  \textsc{bits2cosmology}). Some of the results in this paper have been derived using the HEALPix \citep{healpix} package.
  We acknowledge the use of the Legacy Archive for Microwave Background Data
  Analysis (LAMBDA), part of the High Energy Astrophysics Science Archive Center
  (HEASARC). HEASARC/LAMBDA is a service of the Astrophysics Science Division at
  the NASA Goddard Space Flight Center.  
This publication makes use of data products from the Wide-field Infrared Survey
Explorer, which is a joint project of the University of California, Los
Angeles, and the Jet Propulsion Laboratory/California Institute of Technology,
funded by the National Aeronautics and Space Administration.
This work has made use of data from the European Space Agency (ESA) mission
{\it Gaia} (\url{https://www.cosmos.esa.int/gaia}), processed by the {\it Gaia}
Data Processing and Analysis Consortium (DPAC,
\url{https://www.cosmos.esa.int/web/gaia/dpac/consortium}). Funding for the DPAC
has been provided by national institutions, in particular the institutions
participating in the {\it Gaia} Multilateral Agreement.
	We acknowledge the use of data provided by the Centre d'Analyse de Données Etendues (CADE), a service of IRAP-UPS/CNRS (http://cade.irap.omp.eu, \citealt{paradis:2012}). 
\end{acknowledgements}


%-------------------------------------------------------------
%                                       Table with references 
%-------------------------------------------------------------
%

\bibliographystyle{aa}
\bibliography{../../common/CG_bibliography,../../common/Planck_bib}
\end{document}
%%%% End of aa.dem




In this analysis, we describe how the DIRBE Calibrated Individual Observations (CIO), which are publicly available on NASA LAMBDA, was integrated into the \Cosmoglobe framework and used them to extend the \Cosmoglobe sky model from 857 GHz all the way to the optical at 1.25 microns. We may also have made the first ever maps of the CIB, made a three-dimensional dust and PAH model, detected Free-free emission at near terahertz frequencies, made maps of the stars, and more cool stuff.
