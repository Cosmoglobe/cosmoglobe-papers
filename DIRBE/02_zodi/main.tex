%                                                                 aa.dem
% AA vers. 9.1, LaTeX class for Astronomy & Astrophysics
% demonstration file
%                                                       (c) EDP Sciences
%-----------------------------------------------------------------------
%
% \documentclass[referee]{aa} % for a referee version
%\documentclass[onecolumn]{aa} % for a paper on 1 column  
%\documentclass[longauth]{aa} % for the long lists of affiliations 
%\documentclass[letter]{aa} % for the letters 
%\documentclass[bibyear]{aa} % if the references are not structured 
%                              according to the author-year natbib style

%

\documentclass{aa}  

%
\usepackage{graphicx}
\usepackage{amsmath,amsfonts,amssymb}
\usepackage{natbib}
\usepackage{tabularx}
\usepackage{collcell}
\usepackage{array}
\usepackage{booktabs}

%%%%%%%%%%%%%%%%%%%%%%%%%%%%%%%%%%%%%%%%
\usepackage{txfonts}
\usepackage{xcolor}
\usepackage{blindtext}
%%%%%%%%%%%%%%%%%%%%%%%%%%%%%%%%%%%%%%%%
% \usepackage[options]{hyperref}
% To add links in your PDF file, use the package "hyperref"
% with options according to your LaTeX or PDFLaTeX drivers.
\usepackage{float}
%\usepackage{stfloats}
\usepackage{dblfloatfix}
\usepackage{afterpage}
\usepackage{ifthen}
\usepackage[morefloats=12]{morefloats}

\usepackage{placeins}
\usepackage{multicol}
%\usepackage[breaklinks,colorlinks,citecolor=blue]{hyperref}
\bibpunct{(}{)}{;}{a}{}{,}
\usepackage[switch]{lineno}
\definecolor{linkcolor}{rgb}{0.6,0,0}
\definecolor{citecolor}{rgb}{0,0,0.75}
\definecolor{urlcolor}{rgb}{0.12,0.46,0.7}
\usepackage[breaklinks, colorlinks, urlcolor=urlcolor,
linkcolor=linkcolor,citecolor=citecolor,pdfencoding=auto]{hyperref}
\hypersetup{linktocpage}
\usepackage{bold-extra}



\def\setsymbol#1#2{\expandafter\def\csname #1\endcsname{#2}}
\def\getsymbol#1{\csname #1\endcsname}

\def\Planck{\textit{Planck}}

\def\HeJT{$^4$He-JT}

\def\allearlypapers{\nocite{planck2011-1.1, planck2011-1.3, planck2011-1.4, planck2011-1.5, planck2011-1.6, planck2011-1.7, planck2011-1.10, planck2011-1.10sup, planck2011-5.1a, planck2011-5.1b, planck2011-5.2a, planck2011-5.2b, planck2011-5.2c, planck2011-6.1, planck2011-6.2, planck2011-6.3a, planck2011-6.4a, planck2011-6.4b, planck2011-6.6, planck2011-7.0, planck2011-7.2, planck2011-7.3, planck2011-7.7a, planck2011-7.7b, planck2011-7.12, planck2011-7.13}}

\def\alltwentythirteenresultspapers{\nocite{planck2013-p01, planck2013-p02, planck2013-p02a, planck2013-p02d, planck2013-p02b, planck2013-p03, planck2013-p03c, planck2013-p03f, planck2013-p03d, planck2013-p03e, planck2013-p01a, planck2013-p06, planck2013-p03a, planck2013-pip88, planck2013-p08, planck2013-p11, planck2013-p12, planck2013-p13, planck2013-p14, planck2013-p15, planck2013-p05b, planck2013-p17, planck2013-p09, planck2013-p09a, planck2013-p20, planck2013-p19, planck2013-pipaberration, planck2013-p05, planck2013-p05a, planck2013-pip56, planck2013-p06b, planck2013-p01a}}

\def\alltwentyfifteenresultspapers{\nocite{planck2014-a01, planck2014-a03, planck2014-a04, planck2014-a05, planck2014-a06, planck2014-a07, planck2014-a08, planck2014-a09, planck2014-a11, planck2014-a12, planck2014-a13, planck2014-a14, planck2014-a15, planck2014-a16, planck2014-a17, planck2014-a18, planck2014-a19, planck2014-a20, planck2014-a22, planck2014-a24, planck2014-a26, planck2014-a28, planck2014-a29, planck2014-a30, planck2014-a31, planck2014-a35, planck2014-a36, planck2014-a37, planck2014-ES}}

\newbox\tablebox    \newdimen\tablewidth
\def\leaderfil{\leaders\hbox to 5pt{\hss.\hss}\hfil}
\def\endPlancktable{\tablewidth=\columnwidth 
    $$\hss\copy\tablebox\hss$$
    \vskip-\lastskip\vskip -2pt}
\def\endPlancktablewide{\tablewidth=\textwidth 
    $$\hss\copy\tablebox\hss$$
    \vskip-\lastskip\vskip -2pt}
\def\tablenote#1 #2\par{\begingroup \parindent=0.8em
    \abovedisplayshortskip=0pt\belowdisplayshortskip=0pt
    \noindent
    $$\hss\vbox{\hsize\tablewidth \hangindent=\parindent \hangafter=1 \noindent
    \hbox to \parindent{$^#1$\hss}\strut#2\strut\par}\hss$$
    \endgroup}
\def\doubleline{\vskip 3pt\hrule \vskip 1.5pt \hrule \vskip 5pt}

\def\L2{\ifmmode L_2\else $L_2$\fi}
\def\dtt{\Delta T/T}
\def\DeltaT{\ifmmode \Delta T\else $\Delta T$\fi}
\def\deltat{\ifmmode \Delta t\else $\Delta t$\fi}
\def\fknee{\ifmmode f_{\rm knee}\else $f_{\rm knee}$\fi}
\def\Fmax{\ifmmode F_{\rm max}\else $F_{\rm max}$\fi}
\def\solar{\ifmmode{\rm M}_{\mathord\odot}\else${\rm M}_{\mathord\odot}$\fi}
\def\Msolar{\ifmmode{\rm M}_{\mathord\odot}\else${\rm M}_{\mathord\odot}$\fi}
\def\Lsolar{\ifmmode{\rm L}_{\mathord\odot}\else${\rm L}_{\mathord\odot}$\fi}
\def\inv{\ifmmode^{-1}\else$^{-1}$\fi}
\def\mo{\ifmmode^{-1}\else$^{-1}$\fi}
\def\sup#1{\ifmmode ^{\rm #1}\else $^{\rm #1}$\fi}
\def\expo#1{\ifmmode \times 10^{#1}\else $\times 10^{#1}$\fi}
\def\,{\thinspace}
\def\lsim{\mathrel{\raise .4ex\hbox{\rlap{$<$}\lower 1.2ex\hbox{$\sim$}}}}
\def\gsim{\mathrel{\raise .4ex\hbox{\rlap{$>$}\lower 1.2ex\hbox{$\sim$}}}}
\let\lea=\lsim
\let\gea=\gsim
\def\simprop{\mathrel{\raise .4ex\hbox{\rlap{$\propto$}\lower 1.2ex\hbox{$\sim$}}}}
\def\deg{\ifmmode^\circ\else$^\circ$\fi}
\def\pdeg{\ifmmode $\setbox0=\hbox{$^{\circ}$}\rlap{\hskip.11\wd0 .}$^{\circ}
          \else \setbox0=\hbox{$^{\circ}$}\rlap{\hskip.11\wd0 .}$^{\circ}$\fi}
\def\arcs{\ifmmode {^{\scriptstyle\prime\prime}}
          \else $^{\scriptstyle\prime\prime}$\fi}
\def\arcm{\ifmmode {^{\scriptstyle\prime}}
          \else $^{\scriptstyle\prime}$\fi}
\newdimen\sa  \newdimen\sb
\def\parcs{\sa=.07em \sb=.03em
     \ifmmode \hbox{\rlap{.}}^{\scriptstyle\prime\kern -\sb\prime}\hbox{\kern -\sa}
     \else \rlap{.}$^{\scriptstyle\prime\kern -\sb\prime}$\kern -\sa\fi}
\def\parcm{\sa=.08em \sb=.03em
     \ifmmode \hbox{\rlap{.}\kern\sa}^{\scriptstyle\prime}\hbox{\kern-\sb}
     \else \rlap{.}\kern\sa$^{\scriptstyle\prime}$\kern-\sb\fi}
\def\ra[#1 #2 #3.#4]{#1\sup{h}#2\sup{m}#3\sup{s}\llap.#4}
\def\dec[#1 #2 #3.#4]{#1\deg#2\arcm#3\arcs\llap.#4}
\def\deco[#1 #2 #3]{#1\deg#2\arcm#3\arcs}
\def\rra[#1 #2]{#1\sup{h}#2\sup{m}}
\def\page{\vfill\eject}
\def\dots{\relax\ifmmode \ldots\else $\ldots$\fi}
\def\WHzsr{\ifmmode $W\,Hz\mo\,sr\mo$\else W\,Hz\mo\,sr\mo\fi}
\def\mHz{\ifmmode $\,mHz$\else \,mHz\fi}
\def\GHz{\ifmmode $\,GHz$\else \,GHz\fi}
\def\mKs{\ifmmode $\,mK\,s$^{1/2}\else \,mK\,s$^{1/2}$\fi}
\def\muKs{\ifmmode \,\mu$K\,s$^{1/2}\else \,$\mu$K\,s$^{1/2}$\fi}
\def\muKRJs{\ifmmode \,\mu$K$_{\rm RJ}$\,s$^{1/2}\else \,$\mu$K$_{\rm RJ}$\,s$^{1/2}$\fi}
\def\muKHz{\ifmmode \,\mu$K\,Hz$^{-1/2}\else \,$\mu$K\,Hz$^{-1/2}$\fi}
\def\MJysr{\ifmmode \,$MJy\,sr\mo$\else \,MJy\,sr\mo\fi}
\def\MJysrmK{\ifmmode \,$MJy\,sr\mo$\,mK$_{\rm CMB}\mo\else \,MJy\,sr\mo\,mK$_{\rm CMB}\mo$\fi}
\def\microns{\ifmmode \,\mu$m$\else \,$\mu$m\fi}
\def\micron{\microns}
\def\muK{\ifmmode \,\mu$K$\else \,$\mu$\hbox{K}\fi}
\def\microK{\ifmmode \,\mu$K$\else \,$\mu$\hbox{K}\fi}
\def\muW{\ifmmode \,\mu$W$\else \,$\mu$\hbox{W}\fi}
\def\kms{\ifmmode $\,km\,s$^{-1}\else \,km\,s$^{-1}$\fi}
\def\kmsMpc{\ifmmode $\,\kms\,Mpc\mo$\else \,\kms\,Mpc\mo\fi}

\providecommand{\sorthelp}[1]{}


% Custom definitions
\def\Cosmoglobe{\textsc{Cosmoglobe}}
\def\cosmoglobe{\textsc{Cosmoglobe}}
\def\Planck{\textit{Planck}}


% \renewcommand{\topfraction}{1.0}	% max fraction of floats at top
%     \renewcommand{\bottomfraction}{1.0}	% max fraction of floats at bottom
%     %   Parameters for TEXT pages (not float pages):
%     \setcounter{topnumber}{2}
%     \setcounter{bottomnumber}{2}
%     \setcounter{totalnumber}{4}     % 2 may work better
%     \setcounter{dbltopnumber}{2}    % for 2-column pages
%     \renewcommand{\dbltopfraction}{0.9}	% fit big float above 2-col. text
%     \renewcommand{\textfraction}{0.04}	% allow minimal text w. figs
%     %   Parameters for FLOAT pages (not text pages):
%     \renewcommand{\floatpagefraction}{0.9}	% require fuller float pages
% 	% N.B.: floatpagefraction MUST be less than topfraction !!
%     \renewcommand{\dblfloatpagefraction}{0.9}	% require fuller float pages



\begin{document} 

   \title{\bfseries{\Cosmoglobe\ DR2. II. Bayesian global modelling of zodiacal light\\ with a first application to \textit{COBE}-DIRBE}}

   %This author list corresponds to \title{Author list for L04\_CMB\_Foregrounds\_Extraction}
%Prepared by M. Lopez-Caniego (Marcos.Lopez.Caniego@sciops.esa.int), ESAC/ESA
%This version is from Thu Jul 12 18:11:48 2018 CET
%\subtitle{There are 152 co-authors in this list}
\newcommand{\oslo}[0]{1}
%\newcommand{\MIT}[0]{2}
\newcommand{\milanoA}[0]{2}
\newcommand{\milanoB}[0]{3}
\newcommand{\milanoC}[0]{4}
\newcommand{\triesteB}[0]{5}
\newcommand{\planetek}[0]{6}
\newcommand{\princeton}[0]{7}
\newcommand{\jpl}[0]{8}
\newcommand{\helsinkiA}[0]{9}
\newcommand{\helsinkiB}[0]{10}
\newcommand{\nersc}[0]{11}
\newcommand{\haverford}[0]{12}
\newcommand{\mpa}[0]{13}
\newcommand{\triesteA}[0]{14}
\newcommand{\iia}[0]{2}

\author{\small
J.~R.~Eskilt\inst{\oslo}\thanks{Corresponding author: J.~R.~Eskilt; \url{j.r.eskilt@astro.uio.no}}
\and
K.~Lee\inst{\oslo}
\and
D.~J.~Watts\inst{\oslo}
\and
S.~Nerval\inst{\oslo}
\and
et al.
}
\institute{\small
        Institute of Theoretical Astrophysics, University of Oslo, Blindern, Oslo, Norway \goodbreak
}


   \institute{Institute of Theoretical Astrophysics, University of Oslo, Blindern, Oslo, Norway}
  
   % Shortened title, author list for top of page 
   \titlerunning{\Cosmoglobe: Interplanetary dust}
   \authorrunning{M.~San et al.}

   \date{\today}
   

% write an abstract 

\abstract{We present the first Bayesian framework for global modelling of zodiacal light in the time domain and its application to the Diffuse Infrared Background Experiment (DIRBE) time-ordered data (TOD). The framework uses a modified version of the COBE/DIRBE zodiacal light model to evaluate the zodiacal light brightness along the observed line-of-sights in the time domain. We obtain a new state-of-the-art zodiacal light model by re-estimating the free parameters in the COBE/DIRBE model and produce DIRBE zodiacal light subtracted mission average (ZSMA) maps with much smaller zodiacal light residuals. We argue that the improved zodiacal light model fit becomes possible through global Bayesian end-to-end analysis of the DIRBE TODs with COBE-FIRAS, GAIA, Planck HFI, and WISE observations, giving us a more accurate instrumental and astophysical characterization of the infrared sky.
However, we do note that even though the zodiacal light model represent an improvement with respect to original DIRBE model when it comes to removing ZL from the DIRBE data, this is still a preliminary model. We illustrate the potential of this framework for building zodiacal models by conducting a small study where we further extend and modifcations to the zodiacal light model by adding more asteroidal bands, interstellar dust and using a more physically motivated parameterization of the interplanetary medium as described in \cite{RRM}. A more modern and physical parametrization of the interplanetary medium in combination with joint analysis with complementry infrared experiments such IRAS, AKARI, WISE and SPHEREx will 

}

%   \abstract{We present a new and improved interplanetary dust model. The interplanetary dust model is a re-estimation of the parameters in the Kelsall et al. (1998) model in addition to an interstellar dust component inspired by Robinson and May (200?). In addition, other small improvements such as using modern solar irradiance models are included. The model parameters are re-estimated using Commander, where we have added zodiacal parameters as an additional gibbs step. The 180 total parameters in the model are estimated using Gibbs sampling. We demonstrate the use of the new interplanetary model on the binned DIRBE CIOs along with the \Cosmoglobe\ sky model to produce the cleanest to date DIRBE sky maps. The Cosmoglobe model which is valid between 1.25 $\mu m$ and 240 $\mu m$ is added added as the new default interplanetary dust model in ZodiPy.}

   \keywords{Zodiacal dust, Interplanetary medium, Cosmology: cosmic background radiation}

   \maketitle

\setcounter{tocdepth}{3}
%\tableofcontents
   
\section{Introduction}
Zodiacal light (ZL, sometimes zodiacal emission or interplanetary dust emission) is the primary source of diffuse radiation observed in the infrared sky between 1-100 $\mu$m (\cite{leinert1998} and references therein). The light comes from scattering and re-emission of sunlight from interplanetary dust (IPD) grains. ZL is usually the most difficult source of foreground contamination to remove in studies of the Extragalactic Background Light (EBL) due to its seasonal nature. As such, a big part of the infrared sky has been inaccessible to cosmological analysis, due to difficulties with ZL modeling. Improving our models of the interplanetary medium is on of the most crucial steps for large scale detections of the EBL.

Existing ZL model typically consists of a three-dimensional model for the IPD distribution in the Solar system, along with a description of the ZL brightness through the radiative and scattering properties of the IPD. The ZL observed by an instrument is then computed by evaluating the brightness along each observed line-of-sight. The state-of-the-art ZL model is the field of cosmology is the Kelsall et al. 1998\citep{K98} (K98), which was built by the DIRBE team for the purpose of removing ZL from their data in the late 1900s. Since then, our understanding of the infrared sky has improved drastically with experiments such as Planck HFI, Gaia, and WISE, giving us access to far superior sky models. These new models, along with computational advances in Bayesian cosmological analysis \citep{BP2023, Galloway2023, Watts2023}, complimentary data in the infrared, and more compute power allows us to extract much more information about the true nature of the interplanetary medium from the DIRBE data then what was possible for the DIRBE team at the time.

In this paper we present the ZL modelling efforts in the \Cosmoglobe\ DR2 release. In Sect. 2, we give a brief overview of the ZL model and its implementation in Commander. In Sect. 3 we discuss sampling of ZL models and present new novel techniques. In Sect. 4 we present the new state-of-the-art ZL model obtained, which is a re-stimation of the the parameters in the K98 model with some additional modification, along with preliminary results from more comprehensive modifications to the ZL models achieved by generalizing the K98 model and incorporating various features from the Rowan-Robinson and May 2013 (RRM) model. Finally, we discuss the implications of these results and the future of ZL modeling in the \Cosmoglobe\ framework, with suggestions for better ZL modelling and which data sets to include in future studies.

\section{Zodiacal light modelling}
Generalizing the CMB-oriented Commander3 software into the infrared regime required the implementation of a ZL simulation module. A significant part of the algorithmical development in Commander3 between Cosmoglobe DR1 and DR2 invovled the implementation of the ZL module which is very similar to the implementation of ZodiPy\citep{San2022}. ZodiPy is an Astropy-affiliated Python package for zodiacal light simulations, which we developed and tested against the DIRBE CIOs\citep{San2024} as part of our transition into the infrared. 

The parametric ZL model used in this analysis is a modified version of the K98 model. In the following, we briefly introduce the model and the various zodiacal components that.

\subsection{Parameterization of interplanetary dust}
The IPD distribution in the zodiacal cloud is both smooth and stable and most of the dust can be accounted for by a diffuse cloud-like component. However, fine structures are produced within the otherwise smooth cloud, from processes such as asteroidal collisions and mass-shedding, cometary trails and outgassing, and gravitational resonance in the orbit of the planets. The full IPD distribution is therefore a sum of several zodiacal components, denoted by $c$, each described by a heliocentric ecliptic number density $n_c(x,y,z)$. Zodiacal components are allowed to have their own origins with respect to the heliocenter $(x_{0,c}, y_{0,c}, z_{0,c})$, so that the component-centric coordinates are
\begin{equation}    
    \begin{aligned}
        x_c&= x - x_{0,c}\\
        y_c&= y - y_{0,c}\\
        z_c&= z - z_{0,c}.
    \end{aligned}
\end{equation}
Similarly, the tilt and orientation of the respective symmetry planes are described by their inclination $i_c$ and ascending node $\Omega_c$. In the these coordinates, assuming that the component is symmetrically distributed in the plane, the density is fully described by the radial distance $r_c$ from the origin, and the height above the symmetry plane $Z_c$
\begin{align}
    r_c &= \sqrt{x_c^2 + y_c^2 + z_c^2},\\
    Z_c &= x_c\sin{\Omega_c}\sin{i_c} - y_c \cos{\Omega_c}\sin{i_c} + z_c \cos{i_c}.
\end{align}


\subsection{Zodiacal components}
\subsubsection{Smooth cloud}
The smooth cloud component represents the smooth IPD distribution in the zodiacal cloud and is parametrized as
\begin{equation}
    n_\mathrm{C}(x,y,z)=n_{0, \mathrm{C}}r_\mathrm{C}^{-\alpha}f(\zeta_\mathrm{C}),
\end{equation}
where $n_{0, \mathrm{C}}$ is the density at 1 AU, $\alpha$ is a power-law index, $f(\zeta_\mathrm{C})$ is the fan-like vertical distribution given as $f(\zeta_\mathrm{C}) = \exp {\left[-\beta g^\gamma \right]}$, where $\zeta_\mathrm{c} = |Z_\mathrm{c}|/r_\mathrm{c}$ is the radial height above the symmetry plane, 
\begin{equation}
    g = \begin{cases}
        \zeta^2/2\mu & \mathrm{for}\; \zeta < \mu,\\
        \zeta - \mu/2 & \mathrm{for}\; \zeta \geq \mu,
    \end{cases}
\end{equation}
and  $\beta$, $\gamma$ and $\mu$ are all shape parameters.

\subsubsection{Asteroidal dust bands}
The majority of the small scale structure in the model is represented by three asteroidal dust bands associated with asteroid families and collisions within the asteroid belt. These were parametrized as 
\begin{align}
    n_{\mathrm{B}_i}(x,y,z) &= \frac{3 n_{0, \mathrm{B}_i}}{r_{\mathrm{B}_i}} \exp \left[-\left(\frac{\zeta_{\mathrm{B}_i}}{\delta_{\zeta_{\mathrm{B}_i}}}\right)^{6}\right]\left[1 + \left(\frac{\zeta_{\mathrm{B}_i}}{\delta_{\zeta_{\mathrm{B}_i}}}\right)^{p}v^{-1}\right] \\
    &\times\left\{1-\exp \left[-\left(\frac{r_{\mathrm{B}_i}}{\delta_{r_{\mathrm{B}_i}}}\right)^{20}\right]\right\},
\end{align}
where $n_{0, \mathrm{B}_i}$ is the density of band $\mathrm{B}_i$ at 3 AU, $\delta_{r_{\mathrm{B}_i}}$ is the inner radial cut-off, and $p$, $v$ and $\delta_{\zeta_{\mathrm{B}_i}}$ are shape parameters.

\subsubsection{Circum-solar ring and Earth-trailing feature}
Finally, a circum-solar ring component was included, representing dust accumulated in Earth's orbit with an enhancment in Earth's wake, known as the trailing feature. This is parameterized as
\begin{align}
    n_\mathrm{R}(x, y, z, \theta)&=n_{0, \mathrm{SR}} \exp \left[-\frac{\left(r_\mathrm{R}-r_{0, \mathrm{SR}}\right)^2}{\sigma_{R,\mathrm{SR}} ^2}-\frac{\left| Z_\mathrm{R} \right|}{\sigma_{Z, \mathrm{SR}}}\right],\\
   &+ n_{0, \mathrm{TF}} \exp \left[-\frac{\left(r_\mathrm{R}-r_{0, \mathrm{TF}}\right)^{2}}{\sigma_{R, \mathrm{TF}}^{2}}-\frac{\left|Z_\mathrm{F}\right|}{\sigma_{Z, \mathrm{TF}}}-\frac{\left(\theta-\theta_{0, \mathrm{TF}}\right)^{2}}{\sigma_{\theta,\mathrm{TF}}^{2}}\right],
\end{align}
where $\theta$ is the heliocentric longitude of the Earth, and the radial locations $r_{0, \mathrm{SR}}$, $r_{0, \mathrm{TF}}$ specifies the distances to the peak densities $n_{0, \mathrm{SR}}$, $n_{0, \mathrm{TF}}$. The $\sigma$ parameters are length scales for the $r$, $Z$ and $\theta$ parameters, respectively. This is the only zodiacal component that is not perfectly symmetric in the plane due to the Earth-trailing feature.

\subsection{Radiative and scattering properties}
IPD grains are assumed to emit thermal emission at a wavelength $\lambda$ on the form of a blackbody modified by an emissivity factor $E_{c, \lambda}$
\begin{equation}
    I^\mathrm{Thermal}_{c,\lambda} = E_{c,\lambda} B_\lambda(T).
\end{equation}
where $B_\lambda$ is the Planck function. The temperature $T$ of the IPD is assumed to fall off with radial distance from the Sun $r$ as is moddeled as
\begin{equation}
    T(r) = T_0 R^{-\delta},
\end{equation}
where $\delta$ is the power law index, which is $0.5$ for theoretical gray dust.

At wavelengths where $\lambda \sim$ the grain size, sunlight will scatter off the IPD grains. The scattered light is modelled as 
\begin{equation}\label{eq: scat_term}
    I^\mathrm{Scattering}_{c, \lambda} = A_{c, \lambda} F_\lambda^\odot(r) \Phi_\lambda(\Theta).
\end{equation}
where, $A_{c, \lambda}$ is the albedo,  $F_\lambda^\odot(r)$ the solar flux at a heliocentric distance from the Sun $r$, and $\Phi_\lambda(\Theta)$ the phase function for scattering angles $\Theta$.

The total intensity from IPD grains is then
\begin{align}\label{eq:I_tot}
    I^\mathrm{Total}_{c, \lambda} &= I^\mathrm{Scattering}_{c,\lambda} + I^\mathrm{Thermal}_{c,\lambda}\\
    &= A_{c, \lambda} F_\lambda^\odot(R) \Phi_\lambda(\Theta) + E_{c,\lambda} B_\lambda(T(R)).
\end{align}

\subsection{Line-of-sight brightness integrals}
To evaluate the zodiacal light model we solve the brightness integral
\begin{equation}\label{eq:los}
    I_{p,t} = \sum_c \int n_c \left[  A_{c, \lambda} F_\lambda^\odot \Phi_\lambda + \left( 1 - A_{c, \lambda} \right) E_{c,\lambda} B_\lambda \right]\,\mathrm {ds},
\end{equation}
where $p$ is a point on the sky, $\lambda$ is the observational wavelength, and $t$ the time of observation and ds a small distance along the time dependant line-of-sight. This amounts to solving line-of-sight integrals from the position of the observer through the IPD distribution for each unique observation. Note that there is a factor (1 - $A_{c, \lambda}$) difference between equation \eqref{eq:los} and \eqref{eq:I_tot} which represents the extinction of the thermal emission in the line-of-sight by scattering. 

\subsection{Geocentric stationary zodiacal light}
The circumsolar-ring and Earth-trailing feature are by definition distributed with respect to the Earth. In an Earth-centric, or geocentric, reference frame, the signal from such components will be stationary on the sky, if these are infact perfectly following the Earth around in the orbit. 
We can therefore make a geocentric map of the full survey, where we have subtracted away the sky model in addition to the smooth cloud and the asteroidal dust bands. What we are left with is the static zodiacal emission coming from the circumsolar ring and the Earth-trailing feature and other potential earth centric interplanetary dust. 
This map can then be used as a lookup map for the combined emission from the circumsolar ring and the Earth-trailing feature components. The projection of the signal from this map to the timestream at a DIRBE frequency $\nu$, at a pixel $p$ in galactic coordinates, and at at a time $t$ is then a matter of a simple lookup
\begin{equation}
    S_{\nu, t, p}^{\mathrm{Ring + Feature}} = M^{\bigoplus}_{\nu, p'},
\end{equation}
where $M^{\bigoplus}_{\nu, p'}$ is the geocentric lookup map and $p'$ is the corresponding pixel index in the geocentric reference frame. An example of such a map is shown in figure \ref{fig: geomap}.
\begin{figure}
    \centering
    \includegraphics[width=\columnwidth]{figs/zodi_static.pdf}
    \caption{Geocentric map $M^{\bigoplus}_\nu$ of the full survey, where we have subtracted away the sky model in addition to the smooth cloud and the asteroidal dust bands. What we are left with is the stationary emission from the circumsolar ring and the Earth-trailing and possible other contamination from sidelobes.}
    \label{fig: geomap}
\end{figure}

\section{Bayesian zodiacal light parameter estimation}
\subsection{Data selection and masks}
\subsection{Time-ordered sampling vs week maps}
\subsection{Zodiacal light parameter atlas}
\subsection{Gibbs sampling of zodiacal light parameters}

\section{Improved zodiacal light models}
\subsection{Re-estimated DIRBE/K98 model}
\subsection{Exploring alternative zodiacal light models}
\subsubsection{Freeing up frozen K98 parameters}
\subsubsection{Inclusion of more asteroidal dust bands}
\subsubsection{The Rowan-Robinson and May parametrization}


%\subsection{A time-varying foreground}

% HKE: Commented out for now, since it's already shown in the Zodipy paper
%\begin{figure}
%    \centering
%      \includegraphics[width=0.7\linewidth]{figs/illustration.pdf}
%      \caption{Illustration showing that the integrated IPD along a line of sight toward a point on the celestial sphere as seen from Earth (blue circles) changes as Earth orbits the Sun (yellow circle).}
%      \label{fig: illustration}
%  \end{figure}

\subsubsection{Survey of zodiacal light parameter variations}

\begin{figure*}
    \centering
    \includegraphics[width=0.22\textwidth]{figs/zodi/zodi_10_tot.pdf}\includegraphics[width=0.22\textwidth]{figs/zodi/zodi_10_a.pdf}\includegraphics[width=0.22\textwidth]{figs/zodi/zodi_01_b.pdf}\includegraphics[width=0.22\textwidth]{figs/zodi/zodi_10_a-b.pdf} 
    \vspace{-0.3cm}

    \includegraphics[width=0.22\textwidth]{figs/zodi/zodi_09_tot.pdf}\includegraphics[width=0.22\textwidth]{figs/zodi/zodi_09_a.pdf}\includegraphics[width=0.22\textwidth]{figs/zodi/zodi_02_b.pdf}\includegraphics[width=0.22\textwidth]{figs/zodi/zodi_09_a-b.pdf}
    \vspace{-0.3cm}

    \includegraphics[width=0.22\textwidth]{figs/zodi/zodi_08_tot.pdf}\includegraphics[width=0.22\textwidth]{figs/zodi/zodi_08_a.pdf}\includegraphics[width=0.22\textwidth]{figs/zodi/zodi_03_b.pdf}\includegraphics[width=0.22\textwidth]{figs/zodi/zodi_08_a-b.pdf}
    \vspace{-0.3cm}

    \includegraphics[width=0.22\textwidth]{figs/zodi/zodi_07_tot.pdf}\includegraphics[width=0.22\textwidth]{figs/zodi/zodi_07_a.pdf}\includegraphics[width=0.22\textwidth]{figs/zodi/zodi_04_b.pdf}\includegraphics[width=0.22\textwidth]{figs/zodi/zodi_07_a-b.pdf}
    \vspace{-0.3cm}

    \includegraphics[width=0.22\textwidth]{figs/zodi/zodi_06_tot.pdf}\includegraphics[width=0.22\textwidth]{figs/zodi/zodi_06_a.pdf}\includegraphics[width=0.22\textwidth]{figs/zodi/zodi_05_b.pdf}\includegraphics[width=0.22\textwidth]{figs/zodi/zodi_06_a-b.pdf}
    \vspace{-0.3cm}

    \includegraphics[width=0.22\textwidth]{figs/zodi/zodi_05_tot.pdf}\includegraphics[width=0.22\textwidth]{figs/zodi/zodi_05_a.pdf}\includegraphics[width=0.22\textwidth]{figs/zodi/zodi_06_b.pdf}\includegraphics[width=0.22\textwidth]{figs/zodi/zodi_05_a-b.pdf}
    \vspace{-0.3cm}

    \includegraphics[width=0.22\textwidth]{figs/zodi/zodi_04_tot.pdf}\includegraphics[width=0.22\textwidth]{figs/zodi/zodi_04_a.pdf}\includegraphics[width=0.22\textwidth]{figs/zodi/zodi_07_b.pdf}\includegraphics[width=0.22\textwidth]{figs/zodi/zodi_04_a-b.pdf}
    \vspace{-0.3cm}

    \includegraphics[width=0.22\textwidth]{figs/zodi/zodi_03_tot.pdf}\includegraphics[width=0.22\textwidth]{figs/zodi/zodi_03_a.pdf}\includegraphics[width=0.22\textwidth]{figs/zodi/zodi_08_b.pdf}\includegraphics[width=0.22\textwidth]{figs/zodi/zodi_03_a-b.pdf}
    \vspace{-0.3cm}

    \includegraphics[width=0.22\textwidth]{figs/zodi/zodi_02_tot.pdf}\includegraphics[width=0.22\textwidth]{figs/zodi/zodi_02_a.pdf}\includegraphics[width=0.22\textwidth]{figs/zodi/zodi_09_b.pdf}\includegraphics[width=0.22\textwidth]{figs/zodi/zodi_02_a-b.pdf}
    \vspace{-0.3cm}

    \includegraphics[width=0.22\textwidth]{figs/zodi/zodi_01_tot.pdf}\includegraphics[width=0.22\textwidth]{figs/zodi/zodi_01_a.pdf}\includegraphics[width=0.22\textwidth]{figs/zodi/zodi_10_b.pdf}\includegraphics[width=0.22\textwidth]{figs/zodi/zodi_01_a-b.pdf}

    \caption{Zodi frequency maps}
    \label{fig:zodi_freq}
  \end{figure*}

  \begin{figure*}
    \centering
    \includegraphics[width=0.88\columnwidth]{figs/zodi_comps/zodi_06_cloud_week.pdf}\includegraphics[width=0.88\columnwidth]{figs/zodi_comps/zodi_06_cloud_full.pdf}

    \vspace{-0.6cm}

    \includegraphics[width=0.88\columnwidth]{figs/zodi_comps/zodi_06_band1_week.pdf}\includegraphics[width=0.88\columnwidth]{figs/zodi_comps/zodi_06_band1_full.pdf}

    \vspace{-0.6cm}

    \includegraphics[width=0.88\columnwidth]{figs/zodi_comps/zodi_06_band2_week.pdf}\includegraphics[width=0.88\columnwidth]{figs/zodi_comps/zodi_06_band2_full.pdf}

    \vspace{-0.6cm}

    \includegraphics[width=0.88\columnwidth]{figs/zodi_comps/zodi_06_band3_week.pdf}\includegraphics[width=0.88\columnwidth]{figs/zodi_comps/zodi_06_band3_full.pdf}

    \vspace{-0.6cm}

    \includegraphics[width=0.88\columnwidth]{figs/zodi_comps/zodi_06_ring_week.pdf}\includegraphics[width=0.88\columnwidth]{figs/zodi_comps/zodi_06_ring_full.pdf}

    \vspace{-0.6cm}

    \includegraphics[width=0.88\columnwidth]{figs/zodi_comps/zodi_06_feature_week.pdf}\includegraphics[width=0.88\columnwidth]{figs/zodi_comps/zodi_06_feature_full.pdf}

    \caption{week map of comps*}
    \label{fig: comp week}
  \end{figure*}

\begin{figure*}
\centering
\includegraphics[width=0.88\columnwidth]{figs/zodi_comps/zodi_sky_98_week.pdf}\includegraphics[width=0.88\columnwidth]{figs/zodi_comps/zodi_cloud_98_week.pdf}
\vspace{-0.6cm}

\includegraphics[width=0.88\columnwidth]{figs/zodi_comps/zodi_zodi_98_week.pdf}\includegraphics[width=0.88\columnwidth]{figs/zodi_comps/zodi_bands_98_week.pdf}
\vspace{-0.6cm}

\includegraphics[width=0.88\columnwidth]{figs/zodi_comps/zodi_res_98_week.pdf}\includegraphics[width=0.88\columnwidth]{figs/zodi_comps/zodi_ring+feature_98_week.pdf}
\caption{week map of comps*}
\label{fig: K98 week comparison}
\end{figure*}


\begin{table*}
    \centering
    \newcolumntype{C}{ @{}>{${}}r<{{}$}@{} }
    \begin{tabular}{l l *2{rCl}}
    \hline
    \hline
     Parameter & Description & \multicolumn{3}{c}{DIRBE} & \multicolumn{3}{c}{Cosmoglobe DR2} \\ 
     \hline
     \multicolumn{8}{c}{Smooth Cloud}\\
     \hline
     $n_{0, \mathrm{C}}$ [$10^{-7}$ AU$^{-1}$] & Number density & 1.13 &\pm& 0.0064 & 1.13 &\pm& 0.0064\\
     $\alpha$ & Radial power-law exponent \quad& 1.34 &\pm& 0.022 & 1.13 &\pm& 0.0064\\
     $\beta$ & Vertical shape parameter & 4.14 &\pm& 0.067 & 1.13 &\pm& 0.0064\\
     $\gamma$ & Vertical power-law exponent & 0.942 &\pm& 0.025 & 1.13 &\pm& 0.0064\\
     $\mu$ & Widening parameter & 0.189 &\pm& 0.014 & 1.13 &\pm& 0.0064\\
     $i$ [deg] & Inclination & 2.03 &\pm& 0.017 & 1.13 &\pm& 0.0064\\
     $\Omega$ [deg] & Ascending node & 77.7 &\pm& 0.6 & 1.13 &\pm& 0.0064\\
     $X_0$ [$10^{-3}$ AU] & x-offset from the Sun  & 11.9 &\pm& 1.1 & 1.13 &\pm& 0.0064\\
     $Y_0$ [$10^{-3}$ AU] & y-offset from the Sun & 5.48 &\pm& 0.77 & 1.13 &\pm& 0.0064\\
     $Z_0$ [$10^{-3}$ AU] & z-offset from the Sun & -2.15 &\pm& 0.43 & 1.13 &\pm& 0.0064\\
     \hline
     \multicolumn{8}{c}{Dust band 1}\\
     \hline
     $n_{0, \mathrm{B}_1}$ [$10^{-10}$ AU$^{-1}$] & Number density & 5.59 &\pm& 0.72 & 1.13 &\pm& 0.0064\\
     $\delta_{\zeta_{\mathrm{B}_1}}$ [deg] & Shape parameter & 8.78 && Fixed & 1.13 &\pm& 0.0064\\
     $v_{\mathrm{B}_1}$ & Shape parameter & 0.10 && Fixed & 1.13 &\pm& 0.0064\\
     $p_{\mathrm{B}_1}$ & Shape parameter & 4 && Fixed & 0.10 &\pm& 0.0064\\
     $i_{\mathrm{B}_1}$ [deg] & Inclination & 0.56 && Fixed & 0.10 &\pm& 0.0064\\
     $\Omega_{\mathrm{B}_1}$ [deg] & Ascending node & 80 && Fixed & 0.10 &\pm& 0.0064\\
     $\delta_{R_{\mathrm{B}_1}}$ [AU] & Inner radial cutoff & 1.5 && Fixed & 0.10 &\pm& 0.0064\\
     \hline
     \multicolumn{8}{c}{Dust band 2}\\
     \hline
     $n_{0, \mathrm{B}_2}$ [$10^{-9}$ AU$^{-1}$] & Number density & 1.99 &\pm& 0.128 & 1.13 &\pm& 0.0064\\
     $\delta_{\zeta_{\mathrm{B}_2}}$ [deg] & Shape parameter & 1.99 && Fixed & 1.13 &\pm& 0.0064\\
     $v_{\mathrm{B}_2}$ & Shape parameter & 0.9 && Fixed & 1.13 &\pm& 0.0064\\
     $p_{\mathrm{B}_2}$ & Shape parameter & 4 && Fixed & 0.10 &\pm& 0.0064\\
     $i_{\mathrm{B}_2}$ [deg] & Inclination & 1.2 && Fixed & 0.10 &\pm& 0.0064\\
     $\Omega_{\mathrm{B}_2}$ [deg] & Ascending node & 30.3 && Fixed & 0.10 &\pm& 0.0064\\
     $\delta_{R_{\mathrm{B}_2}}$ [AU] & Inner radial cutoff & 0.94 &\pm& 0.025 & 0.10 &\pm& 0.0064\\
     \hline
     \multicolumn{8}{c}{Dust band 3}\\
     \hline
     $n_{0, \mathrm{B}_3}$ [$10^{-10}$ AU$^{-1}$] & Number density & 1.44 &\pm& 0.234 & 1.13 &\pm& 0.0064\\
     $\delta_{\zeta_{\mathrm{B}_3}}$ [deg] & Shape parameter & 15 && Fixed & 1.13 &\pm& 0.0064\\
     $v_{\mathrm{B}_3}$ & Shape parameter & 0.05 && Fixed & 1.13 &\pm& 0.0064\\
     $p_{\mathrm{B}_3}$ & Shape parameter & 4 && Fixed & 0.10 &\pm& 0.0064\\
     $i_{\mathrm{B}_3}$ [deg] & Inclination & 0.8 && Fixed & 0.10 &\pm& 0.0064\\
     $\Omega_{\mathrm{B}_3}$ [deg] & Ascending node & 80 && Fixed & 0.10 &\pm& 0.0064\\
     $\delta_{R_{\mathrm{B}_3}}$ [AU] & Inner radial cutoff & 1.5 && Fixed & 0.10 &\pm& 0.0064\\
     \hline
     \multicolumn{8}{c}{Circum-solar ring}\\
     \hline
     $n_{0, \mathrm{SR}}$ [$10^{-8}$ AU$^{-1}$] & Number density & 1.83 &\pm& 0.127 & 1.13 &\pm& 0.0064\\
     $R_\mathrm{SR}$ [AU] & Radius of peak number density & 1.03 &\pm& 0.00064 & 1.13 &\pm& 0.0064\\
     $\sigma_{r,\mathrm{SR}}$ [AU] & Radial dispersion & 0.025 && Fixed & 1.13 &\pm& 0.0064\\
     $\sigma_{z,\mathrm{SR}}$ [AU] & Vertical dispersion & 0.054 &\pm& 0.0066 & 1.13 &\pm& 0.0064\\
     $i_{\mathrm{SR}}$ [deg] & Inclination & 0.49 &\pm& 0.063 & 0.10 &\pm& 0.0064\\
     $\Omega_{\mathrm{SR}}$ [deg] & Ascending node & 22.3 &\pm& 0.0014 & 0.10 &\pm& 0.0064\\
     \hline
     \multicolumn{8}{c}{Earth-trailing feature}\\
     \hline
     $n_{0, \mathrm{TB}}$ [$10^{-8}$ AU$^{-1}$] & Number density & 1.9 &\pm& 0.142 & 1.13 &\pm& 0.0064\\
     $R_\mathrm{TB}$ [AU] & Radius of peak number density & 1.06 &\pm& 0.011 & 1.13 &\pm& 0.0064\\
     $\sigma_{r,\mathrm{TB}}$ [AU] & Radial dispersion & 0.10 &\pm& 0.0097 & 1.13 &\pm& 0.0064\\
     $\sigma_{z,\mathrm{TB}}$ [AU] & Vertical dispersion & 0.091 &\pm& 0.013 & 1.13 &\pm& 0.0064\\
     $\theta_{\mathrm{TB}}$ [deg] & Longitude with respect to Earth  & -10 && Fixed & 0.10 &\pm& 0.0064\\
     $\sigma_{\theta,\mathrm{TB}}$ [deg] & Longitude dispersion & 12.1 &\pm& 3.4 & 0.10 &\pm& 0.0064\\
     \hline
    \end{tabular}
    \caption{Comparison between best-fit number density and geometrical interplanetary dust parameters in the DIRBE model and our model.}
    \label{table:zodi-params-geo}
    \end{table*}
    
    \begin{table*}
    \centering
    \newcolumntype{C}{ @{}>{${}}r<{{}$}@{} }
    \begin{tabular}{l l *2{rCl}}
    \hline
    \hline
     Parameter & Description & \multicolumn{3}{c}{DIRBE} & \multicolumn{3}{c}{Cosmoglobe DR2} \\ 
     \hline
     \multicolumn{8}{c}{Smooth Cloud}\\
     \hline
     $A_1$ & Albedo at 1.25$\mu $m & 0.204 &\pm& 0.0013 & 1.13 &\pm& 0.0064\\
     $A_2$ & Albedo at 2.2$\mu $m & 0.255 &\pm& 0.0017 & 1.13 &\pm& 0.0064\\
     $A_3$ & Albedo at 3.5$\mu $m & 0.210 &\pm& 0.019 & 1.13 &\pm& 0.0064\\
     $A_4$ & Albedo at 4.9$\mu $m  & 0 && Fixed & 1.13 &\pm& 0.0064\\
     $E_3$ & Emissivity at 3.5$\mu $m  & 1.66 &\pm& 0.088 & 1.13 &\pm& 0.0064\\
     $E_4$ & Emissivity at 4.9$\mu $m  & 0.997 &\pm& 0.0036 & 1.13 &\pm& 0.0064\\
     $E_5$ & Emissivity at 12$\mu $m  & 0.958 &\pm& 0.0026 & 1.13 &\pm& 0.0064\\
     $E_6$ & Emissivity at 25$\mu $m  & 1 && Fixed & 1.13 &\pm& 0.0064\\
     $E_7$ & Emissivity at 60$\mu $m  & 0.733 &\pm& 0.0055 & 1.13 &\pm& 0.0064\\
     $E_8$ & Emissivity at 100$\mu $m  & 0.647 &\pm& 0.012 & 1.13 &\pm& 0.0064\\
     $E_9$ & Emissivity at 140$\mu $m  & 0.677 && ? & 1.13 &\pm& 0.0064\\
     $E_10$ & Emissivity at 240$\mu$m    & 0.519 && ? & 1.13 &\pm& 0.0064\\
     \hline
     \multicolumn{8}{c}{Dust bands}\\
     \hline
     \hline
     $A_1$ & Albedo at 1.25$\mu $m & 0.204 && Fixed to smooth cloud & 1.13 &\pm& 0.0064\\
     $A_2$ & Albedo at 2.2$\mu $m & 0.255 && Fixed to smooth cloud & 1.13 &\pm& 0.0064\\
     $A_3$ & Albedo at 3.5$\mu $m & 0.210 && Fixed to smooth cloud & 1.13 &\pm& 0.0064\\
     $A_4$ & Albedo at 4.9$\mu $m  & 0 && Fixed & 1.13 &\pm& 0.0064\\
     $E_3$ & Emissivity at 3.5$\mu $m  & 1.66 && Fixed to smooth cloud & 1.13 &\pm& 0.0064\\
     $E_4$ & Emissivity at 4.9$\mu $m  & 0.359 &\pm& 0.054 & 1.13 &\pm& 0.0064\\
     $E_5$ & Emissivity at 12$\mu $m  & 1.01 &\pm& 0.15 & 1.13 &\pm& 0.0064\\
     $E_6$ & Emissivity at 25$\mu $m  & 1 && Fixed & 1.13 &\pm& 0.0064\\
     $E_7$ & Emissivity at 60$\mu $m  & 1.25 &\pm& 0.3 & 1.13 &\pm& 0.0064\\
     $E_8$ & Emissivity at 100$\mu $m  & 1.52 &\pm& 0.65 & 1.13 &\pm& 0.0064\\
     $E_9$ & Emissivity at 140$\mu $m  & 1.13 && ? & 1.13 &\pm& 0.0064\\
     $E_10$ & Emissivity at 240$\mu $m  & 1.40 && ? & 1.13 &\pm& 0.0064\\
     \hline
     \multicolumn{8}{c}{Circum-solar ring and Earth-trailing feature}\\
     \hline
     \hline
     $A_1$ & Albedo at 1.25$\mu $m & 0.204 && Fixed to smooth cloud & 1.13 &\pm& 0.0064\\
     $A_2$ & Albedo at 2.2$\mu $m & 0.255 && Fixed to smooth cloud & 1.13 &\pm& 0.0064\\
     $A_3$ & Albedo at 3.5$\mu $m & 0.210 && Fixed to smooth cloud & 1.13 &\pm& 0.0064\\
     $A_4$ & Albedo at 4.9$\mu $m  & 0 && Fixed & 1.13 &\pm& 0.0064\\
     $E_3$ & Emissivity at 3.5$\mu $m  & 1.66 && Fixed to smooth cloud & 1.13 &\pm& 0.0064\\
     $E_4$ & Emissivity at 4.9$\mu $m  & 1.06 &\pm& 0.0089 & 1.13 &\pm& 0.0064\\
     $E_5$ & Emissivity at 12$\mu $m  & 1.06 &\pm& 0.00078 & 1.13 &\pm& 0.0064\\
     $E_6$ & Emissivity at 25$\mu $m  & 1 && Fixed & 1.13 &\pm& 0.0064\\
     $E_7$ & Emissivity at 60$\mu $m  & 0.873 &\pm& 0.0042 & 1.13 &\pm& 0.0064\\
     $E_8$ & Emissivity at 100$\mu $m  & 1.1 &\pm& 0.000075 & 1.13 &\pm& 0.0064\\
     $E_9$ & Emissivity at 140$\mu $m  & 1.15 && ? & 1.13 &\pm& 0.0064\\
     $E_10$ & Emissivity at 240$\mu $m  & 0.858 && ? & 1.13 &\pm& 0.0064\\
     \hline
    \end{tabular}
    \caption{Comparison between best-fit spectral parameters in the DIRBE model and our model.}
    \label{table:zodi-params-spectral}
    \end{table*}
    

% % INTRODUCTION
% %-------------------------------------------------------------------
% \section{Introduction}
% Zodiacal light (ZL, sometimes zodiacal emission or interplanetary dust emission) is the primary source of diffuse radiation observed in the infrared sky between 1-100 $\mu$m (\cite{leinert1998} and references therein). The light comes from scattering and re-emission of sunlight from interplanetary dust grains. ZL has been the most problematic source of foreground contamination in studies of the Extragalactic Background Light (EBL) the infrared sky due, mainly due to its time-varying nature. Improving our understanding of the interplanetary medium and building better ZL models is arguably the most critical step in making a large part of the infrared sky accessible to cosmological analysis. 

% The current state-of-the-art ZL model is the Kelsall et al. 1998 (K98), which was developed by the DIRBE team for the purpose of removing ZL from their data. It consists of a parametric three-dimensional model of the interplanetary dust distribution with several distinguishable components such as 
% These components were assumed to emit like modified blackbodies and could be evaluated through line-of-sight integrations to simulate the observed zodiacal light. 


% Ever since it was first understood in the 17th century \citep{cassini}, zodiacal light has been a driving force for the exploration of the interplanetary medium. The Diffuse Infrared Background Experiment (DIRBE) instrument, onboard the Cosmic Background Explorer (COBE), found that the zodiacal light could be effectively characterized in the infrared \citep{mather:1994, hauser:1998}. The DIRBE team developed a geometric model that represented the interplanetary medium and its identifiable components.
% These components were assumed to emit like modified blackbodies and could be evaluated through line-of-sight integrations to simulate the observed zodiacal light. This model, detailed in \cite{K98}, (here-after K98), has demonstrated its effectiveness in describing zodiacal light in the infrared and sub-millimeter domains and has been the default modeling used in the cosmology community for the past twenty years. The Planck Collaboration \citep{PLANCK2013, PLANCK2015, PLANCK2018} recently utilized the DIRBE model in their analysis of the High Frequency Instrument (HFI) data. They adapted the model to be applicable at sub-terahertz frequencies by evaluating the K98 model with the HFI scanning strategy and fitting an overall amplitude to model components.

% The ZL is considered a local foreground in CMB studies with the emission originating from the near vicinity of the observer. This is a contrast to more common foregrounds such as the CMB and galactic thermal dust emission, which we assume to be stationary in the sky. Galactic and extragalactic foregrounds can be modeled with a single template describing the structure of the component at some reference frequency. The template can then be scaled to arbitrary frequencies given a description of the component's spectral energy density (SED). Simple models like this does not apply to the zodiacal light, which highly depends on the position and time of observation.
% It is impossible to describe and model the zodiacal light foreground through a single template, applicable to all experiments. Instead, the zodiacal light must be dynamically modeled on a per-experiment basis taking into account the position of the observer within the solar system and the scanning strategy. 

% The main highlight of this work is that we are able to produce a better zodiacal light model with much smaller residuals in the frequency maps using only the same zodiacal light-contaminated data used by the DIRBE team in their original analysis. We attribute most of this success to the Cosmoglobe effort of joint global Bayesian analysis of the time-ordered DIRBE data along with maps and point source catalogs from HFI, FIRAS, and WISE. This results in a much better constrained sky model than what was possible at the time of the original DIRBE analysis, making it easier to distinguish the zodiacal light from other signal sources. Additionally, when fitting the zodiacal light model parameters, the DIRBE team used week maps differenced by the full survey sky map. While this removes all static signals from the sky, it also kills much of the effective signal-to-noise ratio for both zodiacal light parameters and the zero levels. In the Cosmoglobe approach, we fit all zodiacal light parameters directly to the timestreams, making it easier to resolve the degeneracies of the geometric interplanetary dust parameters. 

% This goes to show that even when only using archival data, it is possible to create more robust descriptions of the interplanetary medium and the observed zodiacal light. In coming Cosmoglobe data releases we will include more zodiacal light-contaminated time-ordered data from experiments such as AKARI and IRAS, which will help break many of the geometric parameter degeneracies in the interplanetary dust model due to the complimentary scanning strategies of the respective experiments.

% In this paper, we will detail the zodiacal light modeling approach used in the Commander framework during the production of Cosmoglobe DR2. Furthermore, the perhaps greatest result from this works comes from the fitting of the model parameters from a Bayesian perspective, utilizing all DIRBE bands jointly along with HFI, FIRAS and WISE data to produce the best zodiacal light model of the infrared sky. 
% In Sect. 2 we describe zodiacal light model in terms of the interplanetary dust models, source functions, and line-of-sight integrations. 
% Additionally, introduce and interpret the zodiacal light in the DIRBE time-ordered data. Finally we discuss the sampling techniques used to fit the many zodiacal light model parameters from a Bayesian perspective. In Sect. 3 we reanalyze the DIRBE data using the K98 model as derived by the DIRBE team, and see that we recover their results. 
% In Sect. 4 we explore extensions to the K98 model by lifting some of the constraints set on the model parameter by the DIRBE team, and including some of the more physical descriptions of the zodiacal components from \cite{RRM}. These extended models are referred to as model A and B, where model A is the relaxed K98 model, while model B is a more complex model with additional modifications.



% \subsection{Rowan-Robinson and May (RRM) model}
% In the RRM model, the components are described in terms of the radial distance from the component center $r_c$ and the the latitude above the symmetry-plane $\beta_0$. The transformation between the two angular and cartesian representations are given by $z_c = r_c \sin{\beta_0}$, 
% \begin{equation}
%     \beta_0 = \arctan\left(\frac{z_c}{r_c}\right).
% \end{equation}

% \subsubsection{The fan}
% The fan is the RRM equivalent of the diffuse cloud in the K98 model. The density of the fan is described in a separable form similar to the diffuse cloud
% \begin{equation}
%     n_{\mathrm{Fan}}\left(r_\mathrm{Fan}, \beta_0\right)=n_0 r_\mathrm{Fan}^{-\gamma} f\left(\beta_0\right),
% \end{equation}
% but the vertical distribution is rather given as 
% \begin{equation}
% f\left(\beta_0\right)=\left(\cos \beta_0\right)^Q \exp \left(-P \sin \left|\beta_0\right|^{\xi}\right),
% \end{equation}
% where
% \begin{equation}\ 
% \xi= \begin{cases}
%     2 - |z_{\mathrm{Fan}}/z_{\mathrm{Fan}, 0}| \mid & \text { for }|z_\mathrm{Fan}|<z_{\mathrm{Fan}, 0} \\     1 & \text { otherwise },
% \end{cases}
% \end{equation}
% Add more components if we use them.
    
% \begin{figure}
%     \centering
%          \includegraphics[width=\linewidth]{figs/zodi_obs_diff.pdf}
%         \caption{Difference in simulated zodiacal light between an observed at the center of Earth and an observer moved 900km in the positive z-direction from the center of Earth.}
%     \label{fig: z}
% \end{figure}

% \begin{figure}
%     \centering
%         \includegraphics[width=\columnwidth]{figs/mask_zodi_fitting.pdf}
%         \caption{Masks applied when fitting zodiacal light parameters for the $2.2\mathrm{\mu m}$, $25\mathrm{\mu m}$ and $240\mathrm{\mu m}$ bands.}
%     \label{fig:masks}
% \end{figure}

% \begin{figure*}
%     \centering
%     \includegraphics[width=0.22\textwidth]{figs/zodi/zodi_10_tot.pdf}\includegraphics[width=0.22\textwidth]{figs/zodi/zodi_10_a.pdf}\includegraphics[width=0.22\textwidth]{figs/zodi/zodi_01_b.pdf}\includegraphics[width=0.22\textwidth]{figs/zodi/zodi_10_a-b.pdf} 
%     \vspace{-0.3cm}

%     \includegraphics[width=0.22\textwidth]{figs/zodi/zodi_09_tot.pdf}\includegraphics[width=0.22\textwidth]{figs/zodi/zodi_09_a.pdf}\includegraphics[width=0.22\textwidth]{figs/zodi/zodi_02_b.pdf}\includegraphics[width=0.22\textwidth]{figs/zodi/zodi_09_a-b.pdf}
%     \vspace{-0.3cm}

%     \includegraphics[width=0.22\textwidth]{figs/zodi/zodi_08_tot.pdf}\includegraphics[width=0.22\textwidth]{figs/zodi/zodi_08_a.pdf}\includegraphics[width=0.22\textwidth]{figs/zodi/zodi_03_b.pdf}\includegraphics[width=0.22\textwidth]{figs/zodi/zodi_08_a-b.pdf}
%     \vspace{-0.3cm}

%     \includegraphics[width=0.22\textwidth]{figs/zodi/zodi_07_tot.pdf}\includegraphics[width=0.22\textwidth]{figs/zodi/zodi_07_a.pdf}\includegraphics[width=0.22\textwidth]{figs/zodi/zodi_04_b.pdf}\includegraphics[width=0.22\textwidth]{figs/zodi/zodi_07_a-b.pdf}
%     \vspace{-0.3cm}

%     \includegraphics[width=0.22\textwidth]{figs/zodi/zodi_06_tot.pdf}\includegraphics[width=0.22\textwidth]{figs/zodi/zodi_06_a.pdf}\includegraphics[width=0.22\textwidth]{figs/zodi/zodi_05_b.pdf}\includegraphics[width=0.22\textwidth]{figs/zodi/zodi_06_a-b.pdf}
%     \vspace{-0.3cm}

%     \includegraphics[width=0.22\textwidth]{figs/zodi/zodi_05_tot.pdf}\includegraphics[width=0.22\textwidth]{figs/zodi/zodi_05_a.pdf}\includegraphics[width=0.22\textwidth]{figs/zodi/zodi_06_b.pdf}\includegraphics[width=0.22\textwidth]{figs/zodi/zodi_05_a-b.pdf}
%     \vspace{-0.3cm}

%     \includegraphics[width=0.22\textwidth]{figs/zodi/zodi_04_tot.pdf}\includegraphics[width=0.22\textwidth]{figs/zodi/zodi_04_a.pdf}\includegraphics[width=0.22\textwidth]{figs/zodi/zodi_07_b.pdf}\includegraphics[width=0.22\textwidth]{figs/zodi/zodi_04_a-b.pdf}
%     \vspace{-0.3cm}

%     \includegraphics[width=0.22\textwidth]{figs/zodi/zodi_03_tot.pdf}\includegraphics[width=0.22\textwidth]{figs/zodi/zodi_03_a.pdf}\includegraphics[width=0.22\textwidth]{figs/zodi/zodi_08_b.pdf}\includegraphics[width=0.22\textwidth]{figs/zodi/zodi_03_a-b.pdf}
%     \vspace{-0.3cm}

%     \includegraphics[width=0.22\textwidth]{figs/zodi/zodi_02_tot.pdf}\includegraphics[width=0.22\textwidth]{figs/zodi/zodi_02_a.pdf}\includegraphics[width=0.22\textwidth]{figs/zodi/zodi_09_b.pdf}\includegraphics[width=0.22\textwidth]{figs/zodi/zodi_02_a-b.pdf}
%     \vspace{-0.3cm}

%     \includegraphics[width=0.22\textwidth]{figs/zodi/zodi_01_tot.pdf}\includegraphics[width=0.22\textwidth]{figs/zodi/zodi_01_a.pdf}\includegraphics[width=0.22\textwidth]{figs/zodi/zodi_10_b.pdf}\includegraphics[width=0.22\textwidth]{figs/zodi/zodi_01_a-b.pdf}

%     \caption{Zodi frequency maps}
%     \label{fig:zodi_freq}
%   \end{figure*}

%   \begin{figure*}
%     \centering
%     \includegraphics[width=0.9\columnwidth]{figs/zodi_comps/zodi_06_cloud_week.pdf}\includegraphics[width=0.9\columnwidth]{figs/zodi_comps/zodi_06_cloud_full.pdf}

%     \vspace{-0.6cm}

%     \includegraphics[width=0.9\columnwidth]{figs/zodi_comps/zodi_06_band1_week.pdf}\includegraphics[width=0.9\columnwidth]{figs/zodi_comps/zodi_06_band1_full.pdf}

%     \vspace{-0.6cm}

%     \includegraphics[width=0.9\columnwidth]{figs/zodi_comps/zodi_06_band2_week.pdf}\includegraphics[width=0.9\columnwidth]{figs/zodi_comps/zodi_06_band2_full.pdf}

%     \vspace{-0.6cm}

%     \includegraphics[width=0.9\columnwidth]{figs/zodi_comps/zodi_06_band3_week.pdf}\includegraphics[width=0.9\columnwidth]{figs/zodi_comps/zodi_06_band3_full.pdf}

%     \vspace{-0.6cm}

%     \includegraphics[width=0.9\columnwidth]{figs/zodi_comps/zodi_06_ring_week.pdf}\includegraphics[width=0.9\columnwidth]{figs/zodi_comps/zodi_06_ring_full.pdf}

%     \vspace{-0.6cm}

%     \includegraphics[width=0.9\columnwidth]{figs/zodi_comps/zodi_06_feature_week.pdf}\includegraphics[width=0.9\columnwidth]{figs/zodi_comps/zodi_06_feature_full.pdf}

%     \caption{week map of comps*}
%     \label{fig: comp week}
%   \end{figure*}

% \begin{figure*}
% \centering
% \includegraphics[width=0.9\columnwidth]{figs/zodi_comps/zodi_sky_98_week.pdf}\includegraphics[width=0.9\columnwidth]{figs/zodi_comps/zodi_cloud_98_week.pdf}
% \vspace{-0.6cm}

% \includegraphics[width=0.9\columnwidth]{figs/zodi_comps/zodi_zodi_98_week.pdf}\includegraphics[width=0.9\columnwidth]{figs/zodi_comps/zodi_bands_98_week.pdf}
% \vspace{-0.6cm}

% \includegraphics[width=0.9\columnwidth]{figs/zodi_comps/zodi_res_98_week.pdf}\includegraphics[width=0.9\columnwidth]{figs/zodi_comps/zodi_ring+feature_98_week.pdf}
% \caption{week map of comps*}
% \label{fig: K98 week comparison}
% \end{figure*}


% \begin{figure*}
%   \centering
%    	\includegraphics[width=0.8\linewidth]{figs/atlas_1_v2.pdf}
%   	\caption{Atlas 1}
% 	\label{fig: atlas1}
% \end{figure*}

% \begin{figure*}
%     \centering
%          \includegraphics[width=0.8\linewidth]{figs/atlas_2_v2.pdf}
%         \caption{Atlas 2}
%       \label{fig: atlas2}
%   \end{figure*}


% \subsection{DIRBE data}
% Discuss data used for sampling of the model. Talk about time-order processing, downsampling, thinning, etc.

% %\subsection{Difficulties with sampling and degeneracies in the interplanetary dust model parameters}
% See atlases in figure \ref{fig: atlas1} and \ref{fig: atlas2} for degeneracies in the interplanetary dust model parameters.

% \subsection{Sampling techniques}

% \begin{figure*}
%     \centering
%     \includegraphics{figs/zodi_params_new.pdf}
%     \caption{A subset of the estimated zodiacal light parameters fit in this work.}
%     \label{fig: zodi_trace}

% \end{figure*}

% \begin{figure*}
%     \centering
%     \includegraphics[width=0.49\linewidth]{figs/maptot_06a_week_minus_full.pdf}
%     \includegraphics[width=0.49\linewidth]{figs/maptot_06a_week.pdf}\\
%     \includegraphics[width=0.49\linewidth]{figs/mapzodi_06a_week_minus_full.pdf}
%     \includegraphics[width=0.49\linewidth]{figs/mapzodi_06a_week.pdf}
%     \includegraphics[width=0.49\linewidth]{figs/map_06a_week_minus_full.pdf}
%     \includegraphics[width=0.49\linewidth]{figs/map_06a_week.pdf}
%     \caption{Illustration of the basic sky maps involved in the zodiacal light fitting algorithms adopted by the DIRBE (\emph{left column}) and \Cosmoglobe\ (\emph{right column}) pipelines for one week of $25\,\mu\mathrm{m}$ observations and adopting the K98 model. The DIRBE pipeline used exclusively differences between weekly and full-season maps, both for the observed signal, $\Delta I_{\nu} \equiv I_{\nu}-\left<I_{\nu}\right>$ (\emph{top left}), and the zodiacal light model, $\Delta Z_{\nu} = Z_{\nu}-\left<Z_{\nu}\right>$ (\emph{middle left}), where brackets indicate full-survey averages. Correspondingly, the final $\chi^2$ is defined through $\Delta I_{\nu} - \Delta Z_{\nu}$ (\emph{bottom left}), and is by constrution only sensitive to time-variable signals. 
%     In contrast, the basic data element in \Cosmoglobe\ is the full sky signal, $I_{\nu}$ (\emph{top right}), which is fitted with the full zodiacal light model, $Z_{\nu}$ (\emph{middle right}), both modelled in time-domain. The $\chi^2$ the minimizes minimize the total signal-minus-model residual, $I_{\nu}-Z_{\nu}$ (\emph{bottom right}). The main advantage of the DIRBE approach is insensitivity to stationary sky signals, in particular thermal dust and CIB, while the main advantage of the \Cosmoglobe\ approach is a much higher effective signal-to-noise ratio, both to zodiacal light parameters and zero-levels.}
%     \label{fig:week_vs_full}
%   \end{figure*}


% The approach we will use Gibbs sample each of the model parameters. This means that we propose a change to one model parameter, estimate the zodiacal emission over the full timestream of each ten DIRBE bands, compute a chi-squared and accept/reject. We perform N such proposals before we move on the the next parameter. Since the zodiacal emission is mostly very smooth on the sky, we can afford to downsample the TOD timestream before evaluating the zodiacal emission. For the diffuse cloud and the dust bands we downsample the TODS by X. For the circum-solar ring and the Earth-trailing feature, which are less smooth, and harder to constrain, we downsample the TODS by X. The downsampled TODS are subtracted by the \Cosmoglobe\ skymodel, and recomputed after after each parameter evaluation. 








% \section{Reanalysis of the K98 zodiacal light model}



% \begin{figure*}
% 	\centering
% 	\includegraphics[width=0.8\linewidth]{figs/zodi_diff.pdf}
% 	\caption{Half-mission model. Columns 1 and 2 are the prediction of the zodiacal emission for each band for the first and second half of the DIRBE mission. Column 3 is the difference between these two columns, and column 4 is the difference between the two zodi-subtracted half-mission maps}
% 	\label{fig: zodi_HM}
% \end{figure*}

% \begin{figure*}
%     \centering
% 	 \includegraphics[width=0.8\linewidth]{figs/tod_zodi_residuals.pdf}
% 	\caption{Data-minus-model residual for \cosmoglobe\ results (CG, black) and the official \cite{K98} model (K98, orange), as a function of ecliptic latitude, Galactic latitude, and Solar elongation. The offset for \cosmoglobe\ and the K98 model are listed on the left and right axes of each row, respectively. The \cosmoglobe\ and K98 values are horizontally offset left and right respectively for clarity.}
%       \label{fig: zodi_timestream}
%   \end{figure*}

% \begin{table*}
%     % \renewcommand{\arraystretch}{1.1} % Default value: 1
%     \begin{center}
%     \small
%     \caption{Best fit geometrical interplanetary dust parameters as fit by K98 and us.}
%     \label{table:zodi parameters}
%     \begin{tabular}{
%         l 
%         l 
%         >{\collectcell{}}r<{\endcollectcell}
%         @{${}\pm{}$}
%         >{\collectcell{}}l<{\endcollectcell}
%         >{\collectcell{}}r<{\endcollectcell}
%         @{${}\pm{}$}
%         >{\collectcell{}}l<{\endcollectcell}
%         >{\collectcell{}}r<{\endcollectcell}
%         @{${}\pm{}$}
%         >{\collectcell{}}l<{\endcollectcell}
%     }
%     \hline \hline
%     Parameter & Description & \multicolumn{2}{c}{K98} & \multicolumn{2}{c}{Model A} & \multicolumn{2}{c}{Model B} \\
%     \hline
%     \multicolumn{8}{c}{All zodiacal components}\\
%     \hline
%     $T_0$ [K]     & Temperature at 1 AU    & \multicolumn{2}{c}{286} & \multicolumn{2}{c}{286} & \multicolumn{2}{c}{286}\\
%     $\delta$      & Temperature power-law exponent    & \multicolumn{2}{c}{0.467} & \multicolumn{2}{c}{0.467} & \multicolumn{2}{c}{0.467}\\
%     \hline
%     \multicolumn{8}{c}{Diffuse cloud}\\
%     \hline
%     $n_0$ [$10^{-7}$ AU$^{-1}$]     & Density at 1 AU               & 1.13 & 0.0064           & 1.13 & 0.0064         & 1.13 & 0.0064\\
%     $\alpha$                        & Radial power-law exponent     & 1.34 & 0.022            & 1.34 & 0.022          & 1.34 & 0.022\\
%     $\beta$                         & Vertical shape parameter      & 4.14 & 0.067            & 4.14 & 0.067          & 4.14 & 0.067\\
%     $\gamma$                        & Vertical power-law exponent   & 0.942 & 0.025           & 0.942 & 0.025         & 0.942 & 0.025\\
%     $\mu$                           & Widening parameter            & 0.189 & 0.014           & 0.189 & 0.014         & 0.189 & 0.014\\
%     $i$ [deg]                       & Inclination                   & 2.03 & 0.017            & 2.03 & 0.017          & 2.03 & 0.017\\
%     $\Omega$ [deg]                  & Ascending node                & 77.7 & 0.6              & 77.7 & 0.6            & 77.7 & 0.6\\
%     $X_0$ [$10^{-3}$ AU]            & x offset from Sun             & 11.9 & 1.1 & 11.9 & 11.9 & 1.1 & 11.9 \\ 
%     $Y_0$ [$10^{-3}$ AU]            & y offset from Sun             & 5.48 & 0.77 & 5.48 & 5.48 & 0.77 & 5.48 \\ 
%     $Z_0$ [$10^{-3}$ AU]            & z offset from Sun             & 2.15 & 0.43 & 2.15 & 2.15 & 0.43 & 2.15 \\ 
%     \hline
%     \multicolumn{8}{c}{Asteroidal dust band 1}\\
%     \hline
%     $n_0$ [$10^{-10}$ AU$^{-1}$]  & Density at 3 AU               & 5.59 & 0.72                 & 5.59 & 0.72               & 5.59 & 0.72\\
%     $\delta_{\zeta_{B}}$ [deg]    & Shape parameter               & \multicolumn{2}{c}{8.78}    & \multicolumn{2}{c}{8.78}  & \multicolumn{2}{c}{8.78}\\
%     $v_{B}$                       & Shape parameter               & \multicolumn{2}{c}{0.1}     & \multicolumn{2}{c}{0.1}   & \multicolumn{2}{c}{0.1}\\
%     $p_{B}$                       & Shape parameter               & \multicolumn{2}{c}{4}       & \multicolumn{2}{c}{4}     & \multicolumn{2}{c}{4}\\
%     $i_{B}$ [deg]                 & Inclination                   & \multicolumn{2}{c}{0.56}    & \multicolumn{2}{c}{0.56}  & \multicolumn{2}{c}{0.56}\\
%     $\Omega_{B}$ [deg]            & Ascending node                & \multicolumn{2}{c}{80}      & \multicolumn{2}{c}{80}    & \multicolumn{2}{c}{80}\\
%     $\delta_{R_{B}}$ [AU]         & Inner radial cutoff           & \multicolumn{2}{c}{1.5}     & \multicolumn{2}{c}{1.5}   & \multicolumn{2}{c}{1.5}\\
%     \hline
%     \multicolumn{8}{c}{Asteroidal dust band 2}\\
%     \hline
%     $n_0$ [$10^{-9}$ AU$^{-1}$]   & Density at 3 AU               & 1.99 & 0.128                & 1.99 & 0.128              & 1.99 & 0.128\\
%     $\delta_{\zeta_{B}}$ [deg]    & Shape parameter               & \multicolumn{2}{c}{8.78}    & \multicolumn{2}{c}{8.78}  & \multicolumn{2}{c}{8.78}\\
%     $v_{B}$                       & Shape parameter               & \multicolumn{2}{c}{0.9}     & \multicolumn{2}{c}{0.9}   & \multicolumn{2}{c}{0.9}\\
%     $p_{B}$                       & Shape parameter               & \multicolumn{2}{c}{4}       & \multicolumn{2}{c}{4}     & \multicolumn{2}{c}{4}\\
%     $i_{B}$ [deg]                 & Inclination                   & \multicolumn{2}{c}{1.2}     & \multicolumn{2}{c}{1.2}   & \multicolumn{2}{c}{1.2}\\
%     $\Omega_{B}$ [deg]            & Ascending node                & \multicolumn{2}{c}{30.3}    & \multicolumn{2}{c}{30.3}  & \multicolumn{2}{c}{30.3}\\
%     $\delta_{R_{B}}$ [AU]         & Inner radial cutoff           & \multicolumn{2}{c}{0.94}    & \multicolumn{2}{c}{0.94}  & \multicolumn{2}{c}{0.94}\\
%     \hline
%     \multicolumn{8}{c}{Asteroidal dust band 3}\\
%     \hline
%     $n_0$ [$10^{-10}$ AU$^{-1}$]  & Density at 3 AU               & 1.44 & 0.234                & 1.44 & 0.234              & 1.44 & 0.234  \\
%     $\delta_{\zeta_{B}}$ [deg]    & Shape parameter               & \multicolumn{2}{c}{15}      & \multicolumn{2}{c}{15}    & \multicolumn{2}{c}{15}\\
%     $v_{B}$                       & Shape parameter               & \multicolumn{2}{c}{0.05}    & \multicolumn{2}{c}{0.05}  & \multicolumn{2}{c}{0.05}\\
%     $p_{B}$                       & Shape parameter               & \multicolumn{2}{c}{4}       & \multicolumn{2}{c}{4}     & \multicolumn{2}{c}{4}\\
%     $i_{B}$ [deg]                 & Inclination                   & \multicolumn{2}{c}{0.8}     & \multicolumn{2}{c}{0.8}   & \multicolumn{2}{c}{0.8}\\
%     $\Omega_{B}$ [deg]            & Ascending node                & \multicolumn{2}{c}{80}      & \multicolumn{2}{c}{80}    & \multicolumn{2}{c}{80}\\
%     $\delta_{R_{B}}$ [AU]         & Inner radial cutoff           & \multicolumn{2}{c}{1.5}     & \multicolumn{2}{c}{1.5}   & \multicolumn{2}{c}{1.5}\\
%     \hline
%     \multicolumn{8}{c}{Circumsolar Ring}\\
%     \hline
%     $n_\mathrm{SR}$ [$10^{-8}$ AU$^{-1}$]   & Density at 1 AU           & 1.83 & 0.127              & 1.83 & 0.127              & 1.83 & 0.127\\
%     $R_\mathrm{SR}$ [AU]                    & Radius of peak density    & 1.03 & 0.00016            & 1.03 & 0.00016            & 1.03 & 0.00016\\
%     $\sigma_\mathrm{rSR}$ [AU]              & Radial dispersion         & \multicolumn{2}{c}{0.025} & \multicolumn{2}{c}{0.025} & \multicolumn{2}{c}{0.025}\\
%     $\sigma_\mathrm{zSR}$ [AU]              & Vertical dispersion       & 0.054 & 0.0066            & 0.054 & 0.0066            & 0.054 & 0.0066\\
%     $i_\mathrm{RB}$ [deg]                   & Inclination               & 0.49  & 0.063             & 0.49  & 0.063             & 0.49  & 0.063\\
%     $\Omega_\mathrm{RB}$ [deg]              & Ascending node            & 22.3  & 0.0014            & 22.3 & 0.0014             & 22.3 & 0.0014\\
%     \hline
%     \multicolumn{8}{c}{Trailing Feature}\\
%     \hline
%     $n_\mathrm{TB}$ [$10^{-8}$ AU$^{-1}$]   & Density at 1 AU                   & 1.9 & 0.142 & 1.9 & 0.142 & 1.9 & 0.142\\
%     $R_\mathrm{TB}$ [AU]                    & Radius of peak density            & 1.06 & 0.011 & 1.06 & 0.011 & 1.06 & 0.011\\
%     $\sigma_\mathrm{rTB}$ [AU]              & Radial dispersion                 &  0.10 & 0.0097 & 0.10 & 0.0097 & 0.10 & 0.0097\\
%     $\sigma_\mathrm{zTB}$ [AU]              & Vertical dispersion               & 0.091 &  0.013 & 0.091 &  0.013 & 0.091 &  0.013\\
%     $\theta_\mathrm{TB}$ [deg]              & Longitude with respect to Earth   & \multicolumn{2}{c}{-10} & \multicolumn{2}{c}{-10} & \multicolumn{2}{c}{-10}\\
%     $\sigma_\mathrm{\theta TB}$ [deg]       & Longitude dispersion              & 12.1 & 3.4 & 12.1 & 3.4 & 12.1 & 3.4\\
%     \hline    
%     \end{tabular}
%     \end{center}
% \end{table*}


% \begin{table*}
%     % \renewcommand{\arraystretch}{1.1} % Default value: 1
%     \begin{center}
%     \small
%     \caption{Best fit zodiacal light source parameters from this analysis and the K98 model.}
%     \label{table:zodi parameters}
%     \begin{tabular}{
%         l 
%         >{\collectcell\Num}r<{\endcollectcell}
%         @{${}\pm{}$}
%         >{\collectcell\Num}l<{\endcollectcell}
%         >{\collectcell\Num}r<{\endcollectcell}
%         @{${}\pm{}$}
%         >{\collectcell\Num}l<{\endcollectcell}
%         >{\collectcell\Num}r<{\endcollectcell}
%         @{${}\pm{}$}
%         >{\collectcell\Num}l<{\endcollectcell}
%         >{\collectcell\Num}r<{\endcollectcell}
%         @{${}\pm{}$}
%         >{\collectcell\Num}l<{\endcollectcell}
%         >{\collectcell\Num}r<{\endcollectcell}
%         @{${}\pm{}$}
%         >{\collectcell\Num}l<{\endcollectcell}
%         >{\collectcell\Num}r<{\endcollectcell}
%         @{${}\pm{}$}
%         >{\collectcell\Num}l<{\endcollectcell}
%     }
%     \hline \hline
%     Channel [$\mu$m] & \multicolumn{2}{c}{Diffuse Cloud} & \multicolumn{2}{c}{Dust band 1} & \multicolumn{2}{c}{Dust band 2} & \multicolumn{2}{c}{Dust band 3} & \multicolumn{2}{c}{Circumsolar ring} & \multicolumn{2}{c}{Trailing feature} \\
%     \hline
%     \multicolumn{13}{c}{Emissivity}\\
%     \hline
%     3.5  & 1.66 & 0.088 & \multicolumn{2}{c}{1} & \multicolumn{2}{c}{1} & \multicolumn{2}{c}{1} & \multicolumn{2}{c}{1} & \multicolumn{2}{c}{1} \\
%     4.9  & 0.997 & 0.0036 & \multicolumn{2}{c}{1} & \multicolumn{2}{c}{1} & \multicolumn{2}{c}{1} & \multicolumn{2}{c}{1} & \multicolumn{2}{c}{1} \\
%     12  & 0.958 & 0.002 & \multicolumn{2}{c}{1} & \multicolumn{2}{c}{1} & \multicolumn{2}{c}{1} & \multicolumn{2}{c}{1} & \multicolumn{2}{c}{1} \\
%     25  & \multicolumn{2}{c}{1} & \multicolumn{2}{c}{1} & \multicolumn{2}{c}{1} & \multicolumn{2}{c}{1} & \multicolumn{2}{c}{1} & \multicolumn{2}{c}{1} \\
%     60  & 0.733 & 0.0055 & \multicolumn{2}{c}{1} & \multicolumn{2}{c}{1} & \multicolumn{2}{c}{1} & \multicolumn{2}{c}{1} & \multicolumn{2}{c}{1} \\
%     100  & 0.647 & 0.012 & \multicolumn{2}{c}{1} & \multicolumn{2}{c}{1} & \multicolumn{2}{c}{1} & \multicolumn{2}{c}{1} & \multicolumn{2}{c}{1} \\
%     140  & \multicolumn{2}{c}{1} & \multicolumn{2}{c}{1} & \multicolumn{2}{c}{1} & \multicolumn{2}{c}{1} & \multicolumn{2}{c}{1} & \multicolumn{2}{c}{1} \\
%     240  & \multicolumn{2}{c}{1} & \multicolumn{2}{c}{1} & \multicolumn{2}{c}{1} & \multicolumn{2}{c}{1} & \multicolumn{2}{c}{1} & \multicolumn{2}{c}{1} \\
%     \hline
%     \multicolumn{13}{c}{Albedo}\\
%     \hline
%     1.25  & \multicolumn{2}{c}{1} & \multicolumn{2}{c}{1} & \multicolumn{2}{c}{1} & \multicolumn{2}{c}{1} & \multicolumn{2}{c}{1} & \multicolumn{2}{c}{1} \\
%     2.2  & \multicolumn{2}{c}{1} & \multicolumn{2}{c}{1} & \multicolumn{2}{c}{1} & \multicolumn{2}{c}{1} & \multicolumn{2}{c}{1} & \multicolumn{2}{c}{1} \\
%     3.5  & \multicolumn{2}{c}{1} & \multicolumn{2}{c}{1} & \multicolumn{2}{c}{1} & \multicolumn{2}{c}{1} & \multicolumn{2}{c}{1} & \multicolumn{2}{c}{1} \\
%     \end{tabular}
%     \end{center}
% \end{table*}


% \clearpage
% \section{Extended zodiacal light models}

% \subsection{Generalized K98 modelling}

% \subsection{RRM modelling}

% %\subsection{Interplanetary dust model}
% Describe the model components used and tabulate all parameters fit.

% %\subsection{Spectral parameters (emissivities / albedos)}

% %\subsection{Zodi subtracted DIRBE maps and timestreams}

\section{Conclusions}


\begin{acknowledgements}
 The current work has received funding from the European
  Union’s Horizon 2020 research and innovation programme under grant
  agreement numbers 819478 (ERC; \textsc{Cosmoglobe}) and 772253 (ERC;
  \textsc{bits2cosmology}). Some of the results in this paper have been derived using the HEALPix \citep{HEALPIX} package.
  We acknowledge the use of the Legacy Archive for Microwave Background Data
  Analysis (LAMBDA), part of the High Energy Astrophysics Science Archive Center
  (HEASARC). HEASARC/LAMBDA is a service of the Astrophysics Science Division at
  the NASA Goddard Space Flight Center.  
\end{acknowledgements}


%-------------------------------------------------------------
%                                       Table with references 
%-------------------------------------------------------------
%

\bibliographystyle{aa}
\bibliography{references}
\end{document}
%%%% End of aa.dem
