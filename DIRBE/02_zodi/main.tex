%                                                                 aa.dem
% AA vers. 9.1, LaTeX class for Astronomy & Astrophysics
% demonstration file
%                                                       (c) EDP Sciences
%-----------------------------------------------------------------------
%
% \documentclass[referee]{aa} % for a referee version
%\documentclass[onecolumn]{aa} % for a paper on 1 column  
%\documentclass[longauth]{aa} % for the long lists of affiliations 
%\documentclass[letter]{aa} % for the letters 
%\documentclass[bibyear]{aa} % if the references are not structured 
%                              according to the author-year natbib style

%

\documentclass[twocolumn]{aa}  

%
\usepackage{graphicx}
\usepackage{amsmath,amsfonts,amssymb}
\usepackage{natbib}
\usepackage{tabularx}
\usepackage{collcell}
\usepackage{array}
\usepackage{booktabs}
\usepackage{subfigure}
%%%%%%%%%%%%%%%%%%%%%%%%%%%%%%%%%%%%%%%%
\usepackage{txfonts}
\usepackage{xcolor}
\usepackage{blindtext}
%%%%%%%%%%%%%%%%%%%%%%%%%%%%%%%%%%%%%%%%
% \usepackage[options]{hyperref}
% To add links in your PDF file, use the package "hyperref"
% with options according to your LaTeX or PDFLaTeX drivers.
\usepackage{float}
%\usepackage{stfloats}
\usepackage{dblfloatfix}
\usepackage{afterpage}
\usepackage{ifthen}
\usepackage[morefloats=12]{morefloats}
\usepackage{tabularx}
\usepackage{placeins}
\usepackage{multicol}
%\usepackage[breaklinks,colorlinks,citecolor=blue]{hyperref}
\bibpunct{(}{)}{;}{a}{}{,}
\usepackage[switch]{lineno}
\definecolor{linkcolor}{rgb}{0.6,0,0}
\definecolor{citecolor}{rgb}{0,0,0.75}
\definecolor{urlcolor}{rgb}{0.12,0.46,0.7}
\usepackage[breaklinks, colorlinks, urlcolor=urlcolor,
linkcolor=linkcolor,citecolor=citecolor,pdfencoding=auto]{hyperref}
\hypersetup{linktocpage}
\usepackage{bold-extra}
\usepackage{ifthen}


\def\setsymbol#1#2{\expandafter\def\csname #1\endcsname{#2}}
\def\getsymbol#1{\csname #1\endcsname}

\def\Planck{\textit{Planck}}

\def\HeJT{$^4$He-JT}

\def\allearlypapers{\nocite{planck2011-1.1, planck2011-1.3, planck2011-1.4, planck2011-1.5, planck2011-1.6, planck2011-1.7, planck2011-1.10, planck2011-1.10sup, planck2011-5.1a, planck2011-5.1b, planck2011-5.2a, planck2011-5.2b, planck2011-5.2c, planck2011-6.1, planck2011-6.2, planck2011-6.3a, planck2011-6.4a, planck2011-6.4b, planck2011-6.6, planck2011-7.0, planck2011-7.2, planck2011-7.3, planck2011-7.7a, planck2011-7.7b, planck2011-7.12, planck2011-7.13}}

\def\alltwentythirteenresultspapers{\nocite{planck2013-p01, planck2013-p02, planck2013-p02a, planck2013-p02d, planck2013-p02b, planck2013-p03, planck2013-p03c, planck2013-p03f, planck2013-p03d, planck2013-p03e, planck2013-p01a, planck2013-p06, planck2013-p03a, planck2013-pip88, planck2013-p08, planck2013-p11, planck2013-p12, planck2013-p13, planck2013-p14, planck2013-p15, planck2013-p05b, planck2013-p17, planck2013-p09, planck2013-p09a, planck2013-p20, planck2013-p19, planck2013-pipaberration, planck2013-p05, planck2013-p05a, planck2013-pip56, planck2013-p06b, planck2013-p01a}}

\def\alltwentyfifteenresultspapers{\nocite{planck2014-a01, planck2014-a03, planck2014-a04, planck2014-a05, planck2014-a06, planck2014-a07, planck2014-a08, planck2014-a09, planck2014-a11, planck2014-a12, planck2014-a13, planck2014-a14, planck2014-a15, planck2014-a16, planck2014-a17, planck2014-a18, planck2014-a19, planck2014-a20, planck2014-a22, planck2014-a24, planck2014-a26, planck2014-a28, planck2014-a29, planck2014-a30, planck2014-a31, planck2014-a35, planck2014-a36, planck2014-a37, planck2014-ES}}

\newbox\tablebox    \newdimen\tablewidth
\def\leaderfil{\leaders\hbox to 5pt{\hss.\hss}\hfil}
\def\endPlancktable{\tablewidth=\columnwidth 
    $$\hss\copy\tablebox\hss$$
    \vskip-\lastskip\vskip -2pt}
\def\endPlancktablewide{\tablewidth=\textwidth 
    $$\hss\copy\tablebox\hss$$
    \vskip-\lastskip\vskip -2pt}
\def\tablenote#1 #2\par{\begingroup \parindent=0.8em
    \abovedisplayshortskip=0pt\belowdisplayshortskip=0pt
    \noindent
    $$\hss\vbox{\hsize\tablewidth \hangindent=\parindent \hangafter=1 \noindent
    \hbox to \parindent{$^#1$\hss}\strut#2\strut\par}\hss$$
    \endgroup}
\def\doubleline{\vskip 3pt\hrule \vskip 1.5pt \hrule \vskip 5pt}

\def\L2{\ifmmode L_2\else $L_2$\fi}
\def\dtt{\Delta T/T}
\def\DeltaT{\ifmmode \Delta T\else $\Delta T$\fi}
\def\deltat{\ifmmode \Delta t\else $\Delta t$\fi}
\def\fknee{\ifmmode f_{\rm knee}\else $f_{\rm knee}$\fi}
\def\Fmax{\ifmmode F_{\rm max}\else $F_{\rm max}$\fi}
\def\solar{\ifmmode{\rm M}_{\mathord\odot}\else${\rm M}_{\mathord\odot}$\fi}
\def\Msolar{\ifmmode{\rm M}_{\mathord\odot}\else${\rm M}_{\mathord\odot}$\fi}
\def\Lsolar{\ifmmode{\rm L}_{\mathord\odot}\else${\rm L}_{\mathord\odot}$\fi}
\def\inv{\ifmmode^{-1}\else$^{-1}$\fi}
\def\mo{\ifmmode^{-1}\else$^{-1}$\fi}
\def\sup#1{\ifmmode ^{\rm #1}\else $^{\rm #1}$\fi}
\def\expo#1{\ifmmode \times 10^{#1}\else $\times 10^{#1}$\fi}
\def\,{\thinspace}
\def\lsim{\mathrel{\raise .4ex\hbox{\rlap{$<$}\lower 1.2ex\hbox{$\sim$}}}}
\def\gsim{\mathrel{\raise .4ex\hbox{\rlap{$>$}\lower 1.2ex\hbox{$\sim$}}}}
\let\lea=\lsim
\let\gea=\gsim
\def\simprop{\mathrel{\raise .4ex\hbox{\rlap{$\propto$}\lower 1.2ex\hbox{$\sim$}}}}
\def\deg{\ifmmode^\circ\else$^\circ$\fi}
\def\pdeg{\ifmmode $\setbox0=\hbox{$^{\circ}$}\rlap{\hskip.11\wd0 .}$^{\circ}
          \else \setbox0=\hbox{$^{\circ}$}\rlap{\hskip.11\wd0 .}$^{\circ}$\fi}
\def\arcs{\ifmmode {^{\scriptstyle\prime\prime}}
          \else $^{\scriptstyle\prime\prime}$\fi}
\def\arcm{\ifmmode {^{\scriptstyle\prime}}
          \else $^{\scriptstyle\prime}$\fi}
\newdimen\sa  \newdimen\sb
\def\parcs{\sa=.07em \sb=.03em
     \ifmmode \hbox{\rlap{.}}^{\scriptstyle\prime\kern -\sb\prime}\hbox{\kern -\sa}
     \else \rlap{.}$^{\scriptstyle\prime\kern -\sb\prime}$\kern -\sa\fi}
\def\parcm{\sa=.08em \sb=.03em
     \ifmmode \hbox{\rlap{.}\kern\sa}^{\scriptstyle\prime}\hbox{\kern-\sb}
     \else \rlap{.}\kern\sa$^{\scriptstyle\prime}$\kern-\sb\fi}
\def\ra[#1 #2 #3.#4]{#1\sup{h}#2\sup{m}#3\sup{s}\llap.#4}
\def\dec[#1 #2 #3.#4]{#1\deg#2\arcm#3\arcs\llap.#4}
\def\deco[#1 #2 #3]{#1\deg#2\arcm#3\arcs}
\def\rra[#1 #2]{#1\sup{h}#2\sup{m}}
\def\page{\vfill\eject}
\def\dots{\relax\ifmmode \ldots\else $\ldots$\fi}
\def\WHzsr{\ifmmode $W\,Hz\mo\,sr\mo$\else W\,Hz\mo\,sr\mo\fi}
\def\mHz{\ifmmode $\,mHz$\else \,mHz\fi}
\def\GHz{\ifmmode $\,GHz$\else \,GHz\fi}
\def\mKs{\ifmmode $\,mK\,s$^{1/2}\else \,mK\,s$^{1/2}$\fi}
\def\muKs{\ifmmode \,\mu$K\,s$^{1/2}\else \,$\mu$K\,s$^{1/2}$\fi}
\def\muKRJs{\ifmmode \,\mu$K$_{\rm RJ}$\,s$^{1/2}\else \,$\mu$K$_{\rm RJ}$\,s$^{1/2}$\fi}
\def\muKHz{\ifmmode \,\mu$K\,Hz$^{-1/2}\else \,$\mu$K\,Hz$^{-1/2}$\fi}
\def\MJysr{\ifmmode \,$MJy\,sr\mo$\else \,MJy\,sr\mo\fi}
\def\MJysrmK{\ifmmode \,$MJy\,sr\mo$\,mK$_{\rm CMB}\mo\else \,MJy\,sr\mo\,mK$_{\rm CMB}\mo$\fi}
\def\microns{\ifmmode \,\mu$m$\else \,$\mu$m\fi}
\def\micron{\microns}
\def\muK{\ifmmode \,\mu$K$\else \,$\mu$\hbox{K}\fi}
\def\microK{\ifmmode \,\mu$K$\else \,$\mu$\hbox{K}\fi}
\def\muW{\ifmmode \,\mu$W$\else \,$\mu$\hbox{W}\fi}
\def\kms{\ifmmode $\,km\,s$^{-1}\else \,km\,s$^{-1}$\fi}
\def\kmsMpc{\ifmmode $\,\kms\,Mpc\mo$\else \,\kms\,Mpc\mo\fi}

\providecommand{\sorthelp}[1]{}


% Custom definitions
\def\Cosmoglobe{\textsc{Cosmoglobe}}
\def\commander{\texttt{Commander}}
\def\commanderthree{\texttt{Commander3}}
\def\Commander{\texttt{Commander}}
\def\Planck{\textit{Planck}}
\def\WMAP{\textit{WMAP}}
\def\Spitzer{\textit{Spitzer}}
\def\Gaia{\textit{Gaia}}

\newcommand{\cii}{\ensuremath{\mathsc {C\ ii}}}

\newcommand{\nWmsr}{\mathrm{nW}\,\mathrm{m}^{-2}\,\mathrm{sr}^{-1}}
\newcommand{\um}{$\,\mu\mathrm{m}$}
\newcommand{\phm}{\phantom{-}}
\newcommand{\dv}[0]{\vec{d}}
\renewcommand{\t}[0]{\vec{t}}
\newcommand{\A}[0]{\tens{A}}
\newcommand{\B}[0]{\tens{B}}
\newcommand{\Y}[0]{\tens{Y}}
\newcommand{\n}[0]{\vec{n}}
\newcommand{\red}[0]{\color{red}}
\newcommand{\green}[0]{\color{green}}
\newcommand{\s}[0]{\vec{s}}
\renewcommand{\a}[0]{\vec{a}}
\newcommand{\m}[0]{\vec{m}}
\newcommand{\bv}[0]{\vec{b}}
\newcommand{\f}[0]{\vec{f}}
\newcommand{\F}[0]{\tens{F}}
\newcommand{\T}[0]{\tens{T}}
\newcommand{\Cp}[0]{\tens{C}}
\renewcommand{\L}[0]{\tens{L}}
\newcommand{\g}[0]{\vec{g}}
\newcommand{\N}[0]{\tens{N}}
\newcommand{\M}[0]{\tens{M}}
\newcommand{\iN}[0]{\tens{N}^{-1}}
\newcommand{\iM}[0]{\tens{M}^{-1}}
\newcommand{\w}[0]{\vec{w}}
\renewcommand{\S}[0]{\tens{S}}
\renewcommand{\r}[0]{\vec{r}}
\renewcommand{\u}[0]{\vec{u}}
\newcommand{\q}[0]{\vec{q}}
\renewcommand{\v}[0]{\vec{v}}
\renewcommand{\P}[0]{\tens{P}}
\newcommand{\dt}[0]{d_t}
\newcommand{\di}[0]{d_i}
\newcommand{\nt}[0]{n_t}
\newcommand{\st}[0]{s_t}
\newcommand{\mt}[0]{m_t}
\newcommand{\ft}[0]{f_t}
\newcommand{\Te}[0]{T_{\rm e}}
\newcommand{\EM}[0]{\rm EM}
\newcommand{\mathsc}[1]{{\normalfont\textsc{#1}}}
\newcommand{\hi}{\ensuremath{\mathsc {H\ i}}}
\newcommand{\bpbold}{\bfseries{\scshape{BeyondPlanck}}}
\newcommand{\BP}{\textsc{BeyondPlanck}}
\newcommand{\bp}{\textsc{BeyondPlanck}}
\newcommand{\cosmoglobe}{\textsc{Cosmoglobe}}
%\newcommand{\Cosmoglobe}{\textsc{Cosmoglobe}}
\newcommand{\lfi}[0]{LFI}
\newcommand{\hfi}[0]{HFI}
\newcommand{\npipe}[0]{\texttt{NPIPE}}
\newcommand{\K}[0]{\textit K}
\newcommand{\Ka}[0]{\textit{Ka}}
\newcommand{\Q}[0]{\textit Q}
\newcommand{\V}[0]{\textit V}
\newcommand{\W}[0]{\textit W}
\newcommand{\e}{\mathrm e}
\newcommand{\cvar}{\ensuremath{c(\vartheta, \varphi, \psi)}}


\def\Tcmb{\ifmmode T_\mathrm{CMB}\else $T_{\mathrm{CMB}}$\fi}
\def\Tcold{\ifmmode T_\mathrm{c}\else $T_{\mathrm{c}}$\fi}
\def\Thot{\ifmmode T_\mathrm{h}\else $T_{\mathrm{h}}$\fi}
\def\Tnear{\ifmmode T_\mathrm{n}\else $T_{\mathrm{n}}$\fi}
\def\scmb{\ifmmode s_\mathrm{CMB}\else $s_{\mathrm{CMB}}$\fi}
\def\squad{\ifmmode s_\mathrm{quad}\else $s_{\mathrm{quad}}$\fi}
\def\ssynch{\ifmmode s_\mathrm{s}\else $s_\mathrm{s}$\fi}
\def\sdust{\ifmmode s_\mathrm{d}\else $s_{\mathrm{d}}$\fi}
\def\ssdust{\ifmmode s_\mathrm{sd}\else $s_{\mathrm{sd}}$\fi}
\def\same{\ifmmode s_\mathrm{AME}\else $s_{\mathrm{AME}}$\fi}
\def\ssrc{\ifmmode s_\mathrm{src}\else $s_{\mathrm{src}}$\fi}
\def\sco{\ifmmode s_\mathrm{CO}\else $s_{\mathrm{CO}}$\fi}
\def\sff{\ifmmode s_\mathrm{ff}\else $s_{\mathrm{ff}}$\fi}
\def\gff{\ifmmode g_\mathrm{ff}\else $g_{\mathrm{ff}}$\fi}
\def\fsynch{\ifmmode f_\mathrm{s}\else $f_{\mathrm{s}}$\fi}
\def\fsd{\ifmmode f_\mathrm{sd}\else $f_{\mathrm{sd}}$\fi}
\def\fame{\ifmmode f_\mathrm{AME}\else $f_{\mathrm{AME}}$\fi}
\def\alphasrc{\ifmmode \alpha_\mathrm{src}\else $\alpha_{\mathrm{src}}$\fi}
\def\bcold{\ifmmode \beta_\mathrm{c}\else $\beta_{\mathrm{c}}$\fi}
\def\bhot{\ifmmode \beta_\mathrm{h}\else $\beta_{\mathrm{h}}$\fi}
\def\bnear{\ifmmode \beta_\mathrm{n}\else $\beta_{\mathrm{n}}$\fi}
\def\bsynch{\ifmmode \beta_\mathrm{s}\else $\beta_{\mathrm{s}}$\fi} 
\def\bsun{\ifmmode \beta_\mathrm{sun}\else $\beta_{\mathrm{sun}}$\fi} 
\def\nuzeros{\ifmmode \nu_{0,\mathrm{s}}\else $\nu_{0,\mathrm{s}}$\fi} 
\def\nuzeroff{\ifmmode \nu_{0,\mathrm{ff}}\else $\nu_{0,\mathrm{ff}}$\fi} 
\def\nuzerocold{\ifmmode \nu_{0,\mathrm{c}}\else $\nu_{0,\mathrm{c}}$\fi}
\def\nuzerohot{\ifmmode \nu_{0,\mathrm{h}}\else $\nu_{0,\mathrm{h}}$\fi}
\def\nuzeronear{\ifmmode \nu_{0,\mathrm{n}}\else $\nu_{0,\mathrm{n}}$\fi} 
\def\nuzeroame{\ifmmode \nu_{0,\mathrm{AME}}\else $\nu_{0,\mathrm{AME}}$\fi} 
\def\nuzerosd{\ifmmode \nu_{0,\mathrm{}}\else $\nu_{0,\mathrm{sd}}$\fi} 
\def\nuzerosrc{\ifmmode \nu_{0,\mathrm{src}}\else $\nu_{0,\mathrm{src}}$\fi} 
\def\nup{\ifmmode \nu_{\mathrm{p}}\else $\nu_{\mathrm{p}}$\fi} 
\def\alphasd{\ifmmode \alpha_{\mathrm{sd}}\else $\alpha_{\mathrm{sd}}$\fi} 
\def\Te{\ifmmode T_{\mathrm{e}}\else $T_{\mathrm{e}}$\fi} 
\def\kB{\ifmmode k_\mathrm{B}\else $k_{\mathrm{B}}$\fi} 

% \renewcommand{\topfraction}{1.0}	% max fraction of floats at top
%     \renewcommand{\bottomfraction}{1.0}	% max fraction of floats at bottom
%     %   Parameters for TEXT pages (not float pages):
%     \setcounter{topnumber}{2}
%     \setcounter{bottomnumber}{2}
%     \setcounter{totalnumber}{4}     % 2 may work better
%     \setcounter{dbltopnumber}{2}    % for 2-column pages
%     \renewcommand{\dbltopfraction}{0.9}	% fit big float above 2-col. text
%     \renewcommand{\textfraction}{0.04}	% allow minimal text w. figs
%     %   Parameters for FLOAT pages (not text pages):
%     \renewcommand{\floatpagefraction}{0.9}	% require fuller float pages
% 	% N.B.: floatpagefraction MUST be less than topfraction !!
%     \renewcommand{\dblfloatpagefraction}{0.9}	% require fuller float pages



\begin{document} 

   \title{\bfseries{\Cosmoglobe\ DR2. III. Improved modelling of zodiacal light emission in \textit{COBE}-DIRBE through Bayesian global analysis}}

   %This author list corresponds to \title{Author list for L04\_CMB\_Foregrounds\_Extraction}
%Prepared by M. Lopez-Caniego (Marcos.Lopez.Caniego@sciops.esa.int), ESAC/ESA
%This version is from Thu Jul 12 18:11:48 2018 CET
%\subtitle{There are 152 co-authors in this list}
\newcommand{\oslo}[0]{1}
%\newcommand{\MIT}[0]{2}
\newcommand{\milanoA}[0]{2}
\newcommand{\milanoB}[0]{3}
\newcommand{\milanoC}[0]{4}
\newcommand{\triesteB}[0]{5}
\newcommand{\planetek}[0]{6}
\newcommand{\princeton}[0]{7}
\newcommand{\jpl}[0]{8}
\newcommand{\helsinkiA}[0]{9}
\newcommand{\helsinkiB}[0]{10}
\newcommand{\nersc}[0]{11}
\newcommand{\haverford}[0]{12}
\newcommand{\mpa}[0]{13}
\newcommand{\triesteA}[0]{14}
\newcommand{\iia}[0]{2}

\author{\small
J.~R.~Eskilt\inst{\oslo}\thanks{Corresponding author: J.~R.~Eskilt; \url{j.r.eskilt@astro.uio.no}}
\and
K.~Lee\inst{\oslo}
\and
D.~J.~Watts\inst{\oslo}
\and
S.~Nerval\inst{\oslo}
\and
et al.
}
\institute{\small
        Institute of Theoretical Astrophysics, University of Oslo, Blindern, Oslo, Norway \goodbreak
}


   %\institute{Institute of Theoretical Astrophysics, University of Oslo, Blindern, Oslo, Norway}
  
   % Shortened title, author list for top of page 
   \titlerunning{\Cosmoglobe: Interplanetary dust}
   \authorrunning{M.~San et al.}

   \date{\today}
   

% write an abstract 

   \abstract{We present an improved model of zodiacal light (ZL) emission for \textit{COBE}-DIRBE derived through global Bayesian analysis within \cosmoglobe\ Data Release~2 (DR2). The parametric form of the ZL model is identical to that introduced by Kelsall et al. (1998; K98), but the specific best-fit parameter values are rederived using the combination of DIRBE Calibrated Individual Observations (CIOs), \Planck\ HFI sky maps, and WISE and GAIA compact object catalogs. Furthermore, the ZL parameters are fitted jointly with astrophysical parameters, such as thermal dust and starlight emission, and the new model takes into account excess radiation that appears stationary in solar-centric coordinates as reported in a companion paper. The relative differences in predicted signals at 25$\,\mu\mathrm{m}$ between K98 and our new model are $\lesssim\,2\,\%$ in the Ecliptic plane, and $\lesssim\,5\,\%$ in the Ecliptic poles and along the asteroidal bands. The zero-level of the cleaned DR2 maps are lower than those of the legacy DIRBE Zodiacal light Subtracted Mission Average (ZSMA) maps by $\sim$\,10\,kJy/sr at 1.25--3.5\,$\mu\mathrm{m}$, which is comparable to the entire predicted contribution from high-redshift galaxies to the Cosmic Infrared Background (CIB) at the same wavelengths. The total rms's of each DR2 maps at wavelengths up to and including $60\,\mu\mathrm{m}$ are $\sim$\,30\,\% lower at high Galactic latitudes than the corresponding DIRBE ZSMA maps. The cleaned DR2 maps at 4.9 and 60\,$\mu\mathrm{m}$ are now for the first time visually dominated by Galactic signal at high latitudes rather than by ZL residuals. Even the 100$\,\mu\mathrm{m}$ channel, which has served as a cornerstone for Galactic studies for three decades, appears significantly cleaner in the current processing. Still, obvious ZL residuals can be seen even in several of the DR2 maps, and further work is required to mitigate these. Joint analysis with existing and future high-resolution full-sky surveys such as AKARI, IRAS, \Planck\ HFI, and SPHEREx will be essential both to break key degeneracies in the current model and to determine whether the reported solar-centric excess radiation has a ZL or instrumental origin. On the algorithmic side, more efficient methods for probing massively multi-peaked likelihoods should be explored and implemented. Thus, while the results presented in this paper do re-define the state-of-the-art for DIRBE modelling, it also only represents the first among many steps toward a future optimal Bayesian ZL model.
%     We present the first Bayesian framework for global modelling 
%of zodiacal light and its application to the Diffuse Infrared Background 
%Experiment (DIRBE) time-ordered data (TOD). The framework uses a 
%modified version of the COBE/DIRBE zodiacal light model to estimate the 
%zodiacal light along the observed line-of-sights in the time domain. We 
%obtain a new state-of-the-art zodiacal light model by re-estimating the 
%free parameters in the COBE/DIRBE model and produce DIRBE zodiacal light 
%subtracted mission average (ZSMA) maps with much smaller zodiacal light 
%residuals. We argue that the improved zodiacal light model fit becomes 
%possible through global Bayesian end-to-end analysis of the DIRBE TODs 
%with COBE-FIRAS, GAIA, Planck HFI, and WISE observations, giving us a 
%more accurate instrumental and astophysical characterization of the 
%infrared sky. However, we do note that even though the zodiacal light 
%model represent an improvement with respect to original DIRBE model when 
%it comes to removing ZL from the DIRBE data, this is still a preliminary 
%model. We illustrate the potential of this framework for building 
%zodiacal models by conducting a small study where we further extend and 
%modifcations to the zodiacal light model by adding more asteroidal 
%bands, interstellar dust and using a more physically motivated 
%parameterization of the interplanetary medium as described in 
%\cite{Robinson2013}. A more modern and physical parametrization of the 
%interplanetary medium in combination with joint analysis with 
%complementry infrared experiments such IRAS, AKARI, WISE and SPHEREx 
%will 
}

   \keywords{Zodiacal dust, Interplanetary medium, Cosmology: cosmic background radiation}

   \maketitle

\setcounter{tocdepth}{2}
\tableofcontents
   

\section{Introduction}
Zodiacal light (ZL, sometimes zodiacal emission or interplanetary dust
emission) is the primary source of diffuse radiation observed in the
infrared sky between 1-100 $\mu$m (see, e.g., \citealp{Leinert1998}
and references therein). This radiation comes from scattering and
re-emission of sunlight from interplanetary dust (IPD) grains, and was
first mapped in detail by the IRAS satellite
\citep{neugebauer:1984}.

The inner Solar system is embedded in a Sun centered cloud of IPD,
with a symmetry axis tilted slightly with respect to the Ecliptic,
known as the zodiacal cloud. The ZL is seasonal, and its appearance in
the sky changes as the Earth moves through the IPD distribution. The
most common way to model the observer position-dependent ZL is to
evaluate a line-of-sight integral for each observation directly in the
time-ordered domain. The time-varying and three-dimensional nature of
the ZL makes it one of the most challenging foregrounds to model in
astrophysical and cosmological studies of the infrared sky. The lack
of a high-accuracy ZL model has left a large part of the
electromagnetic spectrum inaccessible to cosmological analysis
attempting to measure the Cosmic Infrared Background (CIB;
\citealp{partridge1967,hauser1998,hauser:2001}).

One of the most widely used ZL models in the field of cosmology is the
\textit{COBE}/DIRBE model by \citet{Kelsall1998}, often simply
referred to as the K98 model. This is a parametric three-dimensional
model that describes the three-dimensional IPD distribution and the
radiative properties of dust using time-dependent measurements from
the \textit{COBE}/DIRBE instrument \citep{hauser1998}. Since then, our
understanding of the infrared sky has improved with new observational
data from experiments like WISE \citep{wright:2010}, Planck HFI
\citep{planck2016-l03}, and GAIA \citep{gaia:2016}. However, these
experiments have largely been analyzed individually, and little or no
coordinated effort has been made toward combining the data from these
experiments into one overall state-of-the-art model of the infrared
sky. The main goal of the current work, summarized in a series of
papers collectively denoted \cosmoglobe\ Data Release 2, is to make
the first step toward such a concordance model by leveraging recent
computational advances in Bayesian cosmological data analysis by the
\BP\ \citep[][and references therein]{BP2023,Galloway2023} and
\cosmoglobe\ \citep{Watts2023} collaborations. The computer code
implementation is called \commanderthree\ \citep{Galloway2023}, which
is a Bayesian Gibbs sampler that was originally designed for
end-to-end analysis of cosmic microwave background (CMB) experiments,
in particular \Planck\ LFI \citep{planck2016-l02} and \WMAP
\citep{bennett2012}. However, as demonstrated in the current work, the
same algorithms are after relatively minor modifications also directly
applicable to infrared measurements.

One of the most important generalizations required for application of
\commanderthree\ to the infrared sky is the implementation of an
accurate time-variable ZL model. In this paper we therefore implement
support for the K98 model in \commanderthree, and we apply this to the
time-domain DIRBE data. This new code is based ZodiPy \citep{San2024},
which is an Astropy-affiliated Python package for ZL simulations. As
an early application of this framework, \cite{San2022} demonstrated
the removal of ZL from the DIRBE TOD with ZodiPy using the K98 model.

The rest of the paper is organized as follows. In
Sect.~\ref{sect:zodi-model}, we introduce the K98 ZL model and discuss
implementation and optimization aspects. Next, in
Sect.~\ref{sect:param-estimation}, we discuss how to optimize the free
parameters in the model within the \commander\ framework, before
presenting an updated ZL model derived from DIRBE time-ordered data in
Sect.~\ref{sect:improved-model}. Finally, we conclude in Sect.~\ref{sec:conclusions}.


\section{Zodiacal light modelling}\label{sect:zodi-model}
ZL is commonly removed directly from timestreams by performing 
line-of-sight integration at each observation through a parametric 
three-dimensional model of the IPD distribution. We use a slightly 
modified version of the K98 ZL model in Commander3. Below we present a 
short introduction to the model parametrization in terms of the 
three-dimensional IPD distribution and the radiative and scattering p
roperties of the IPD. We refer to~\cite{Kelsall1998} for a more detailed 
introduction to the K98 ZL model.


\subsection{Parameterization of interplanetary dust}
The IPD in the zodiacal cloud is smooth and stable, and most of the dust 
is accounted for by a diffuse cloud-like component \citep{Leinert1998}. 
The majority of the dust in this cloud component stems from the 
mass-shedding of Jupiter family comets with low eccentricities and 
inclinations with respect to the ecliptic. However, fine structures 
within the zodiacal cloud exist as a result of collisions and 
fragmentation in asteroids and gravitational resonance in the orbit of 
planets \citep{Low1984, Dermott1984, Dermott1994, Reach1997}. 

We model the IPD distribution as a combination of several zodiacal 
components, denoted by $c$, each described by a heliocentric ecliptic 
number density $n_c(x,y,z)$. Each zodiacal component is allowed to have 
a heliocentric offset $(x_{0,c}, y_{0,c}, z_{0,c})$, such that the 
component-centric coordinates become
\begin{equation}    
    \begin{aligned}
        x_c&= x - x_{0,c}\\
        y_c&= y - y_{0,c}\\
        z_c&= z - z_{0,c}.
    \end{aligned}
\end{equation}
Additionally, each zodiacal component is allowed to have a plane of 
symmetry different from the ecliptic, given by an inclination $i_c$ and 
ascending node $\Omega_c$. In component-centric coordinates, the number 
density is then fully described by the radial distance $r_c$ from the 
origin and the height above the symmetry plane $Z_c$
\begin{align}
    r_c &= \sqrt{x_c^2 + y_c^2 + z_c^2},\\
    Z_c &= x_c\sin{\Omega_c}\sin{i_c} - y_c \cos{\Omega_c}\sin{i_c} + z_c \cos{i_c},\\
    \zeta_c &= \frac{|Z_\mathrm{c}|}{r_\mathrm{c}},
\end{align}
where $\zeta_c$ is the radial height above the symmetry plane.

\subsection{Zodiacal components}
\subsubsection{Smooth cloud}
The cloud component represents the smooth IPD distribution that embeds 
the inner Solar system. Its number density is modeled as
\begin{equation}
    n_\mathrm{C}(x,y,z)=n_{0, \mathrm{C}}r_\mathrm{C}^{-\alpha}f(\zeta_\mathrm{C}),
\end{equation}
where $n_{0, \mathrm{C}}$ is the number density at 1 AU, $\alpha$ is a 
power-law index, $f(\zeta_\mathrm{C})$ is the fan-like vertical 
distribution given as 
\begin{equation}
    f(\zeta_\mathrm{C}) = \exp {\left[-\beta g^\gamma \right]},
\end{equation}
with
\begin{equation}
    g = \begin{cases}
        \zeta^2/2\mu & \mathrm{for}\; \zeta < \mu,\\
        \zeta - \mu/2 & \mathrm{for}\; \zeta \geq \mu,
    \end{cases}
\end{equation}
where $\beta$, $\gamma$ and $\mu$ are shape parameters.

\subsubsection{Dust bands}
Three dust bands are included in the model to represent the observed 
shoulder-like structure in the IRAS scans across the ecliptic plane. 
These bands appear at ecliptic latitudes of 
$\pm \sim 1.4^\circ$, $10^\circ$, and $15^\circ$, and are associated 
with the Themis, Koronis, and Eos asteroid families, respectively. Dust 
bands are modeled as
\begin{align}
    n_{\mathrm{B}_i}(x,y,z) &= \frac{3 n_{0, \mathrm{B}_i}}{r_{\mathrm{B}_i}} \exp \left[-\left(\frac{\zeta_{\mathrm{B}_i}}{\delta_{\zeta_{\mathrm{B}_i}}}\right)^{6}\right]\left[1 + \left(\frac{\zeta_{\mathrm{B}_i}}{\delta_{\zeta_{\mathrm{B}_i}}}\right)^{p}v^{-1}\right] \\
    &\times\left\{1-\exp \left[-\left(\frac{r_{\mathrm{B}_i}}{\delta_{r_{\mathrm{B}_i}}}\right)^{20}\right]\right\},
\end{align}
where $n_{0, \mathrm{B}_i}$ is the number density of band $\mathrm{B}_i$ 
at 3 AU, $\delta_{r_{\mathrm{B}_i}}$ is the inner radial cut-off, and 
$p$, $v$ and $\delta_{\zeta_{\mathrm{B}_i}}$ are shape parameters.
\subsubsection{Circum-solar ring and Earth-trailing feature}
A circum-solar ring component is included in the model to represent dust 
that has accumulated in Earth's orbit due to gravitational effects. This 
component also includes an enhancement to the IPD distribution at 
Earth's wake, known as the Earth-trailing feature. The ring component is 
modeled as
\begin{align}
    n_\mathrm{R}(x, y, z, \theta)&=n_{0, \mathrm{SR}} \exp \left[-\frac{\left(r_\mathrm{R}-r_{0, \mathrm{SR}}\right)^2}{\sigma_{R,\mathrm{SR}} ^2}-\frac{\left| Z_\mathrm{R} \right|}{\sigma_{Z, \mathrm{SR}}}\right],\\
   &+ n_{0, \mathrm{TF}} \exp \left[-\frac{\left(r_\mathrm{R}-r_{0, \mathrm{TF}}\right)^{2}}{\sigma_{R, \mathrm{TF}}^{2}}-\frac{\left|Z_\mathrm{F}\right|}{\sigma_{Z, \mathrm{TF}}}-\frac{\left(\theta-\theta_{0, \mathrm{TF}}\right)^{2}}{\sigma_{\theta,\mathrm{TF}}^{2}}\right],
\end{align}
where $\theta$ is the heliocentric longitude of the Earth, and the 
radial locations $r_{0, \mathrm{SR}}$, $r_{0, \mathrm{TF}}$ specifies 
the distances to the peak densities $n_{0, \mathrm{SR}}$, 
$n_{0, \mathrm{TF}}$. The $\sigma$ parameters are length scales for the 
$r$, $Z$ and $\theta$ parameters, respectively. Note that the 
Earth-trailing feature depends on the position of the Earth and does not 
have a plane symmetry like the other zodiacal components. 


\subsection{Radiative and scattering properties}
Thermal emission from IPD grains is modeled on the form of a blackbody
 modified by an emissivity factor $E_{c, \lambda}$
\begin{equation}
    I^\mathrm{Thermal}_{c,\lambda} = E_{c,\lambda} B_\lambda(T),
\end{equation}
where $B_\lambda$ is the Planck function at a wavelength $\lambda$. The 
temperature $T$ of the IPD falls off with radial distance $r$ from the 
Sun as
\begin{equation}
    T(r) = T_0 r^{-\delta},
\end{equation}
where $T_0$ is the temperature of IPD at 1 AU and $\delta$ the power law 
index. In addition to emitting thermally, IPD grains also scatter 
sunlight in the near-infrared. The contribution to the total signal from 
scattering is modeled as
\begin{equation}\label{eq: scat_term}
    I^\mathrm{Scattering}_{c, \lambda} = A_{c, \lambda} F_\lambda^\odot(r) \Phi_\lambda(\Theta),
\end{equation}
where $A_{c, \lambda}$ is the albedo, or reflectivity of the IPD of a 
given zodiacal component, $F_\lambda^\odot(r)$ the solar flux at a 
radial distance from the Sun, and $\Phi_\lambda(\Theta)$ is the phase 
function for scattering angles $\Theta$, which describes the angular 
distribution of the scattered light.

The total intensity from a single IPD grain is
\begin{align}\label{eq:I_tot}
    I^\mathrm{Total}_{c, \lambda} &= I^\mathrm{Scattering}_{c,\lambda} + I^\mathrm{Thermal}_{c,\lambda}\\
    &= A_{c, \lambda} F_\lambda^\odot \Phi_\lambda + E_{c,\lambda} B_\lambda.
\end{align}

\begin{figure}
  \includegraphics[width=\columnwidth]{figs/band3_instant.pdf}\\
  \includegraphics[width=\columnwidth]{figs/band3_mission.pdf}\\
  \includegraphics[width=\columnwidth]{figs/band3_omega.pdf}
  \caption{}
  \label{fig:band3}
\end{figure}


\subsection{Evaluating the zodiacal model}
The ZL model is evaluated by solving the following line-of-sight 
integral for each observation in the timestream
\begin{equation}\label{eq:los}
    I_{p,t} = \sum_c \int n_c \left[  A_{c, \lambda} F_\lambda^\odot \Phi_\lambda + \left( 1 - A_{c, \lambda} \right) E_{c,\lambda} B_\lambda \right]\,\mathrm {ds}.
\end{equation}
Here, $p$ is a point in the sky, $t$ is the time of observation, and ds 
is a small distance along the line-of-sight $s$ from the observer and 
towards $p$, and $n_c$ is the number density of component $c$ in the 
line-of-sight. Note that there is a factor (1 - $A_{c, \lambda}$) 
difference between equation~\eqref{eq:los} and~\eqref{eq:I_tot}. This 
factor represents a form of extinction where thermal emission is 
sometimes scattered away from the line-of-sight. 


\subsection{Optimizing model evaluations}\label{sect:optimization}
The ZL line-of-sight evaluations are an expensive part of the Cosmoglobe analysis pipeline and significantly increase the time it takes to complete a Gibbs sample when applying and fitting ZL models. Evaluating a line-of-sight for each TOD is computationally feasible in the context of DIRBE due to the low data volume (18 GB when compressed), but a brute for evaluation on this form will not be an option for future data releases where we include data from the likes of AKARI and SPHEREx. There are a few natural and easy optimizations we can use when working with a smoothly spatially- and time-varying signal like the ZL, which we present in the following sections.

The first and perhaps most trivial optimization when working with discrete HEALPix pixelization is to utilize a ZL cache. The cache takes advantage of the smoothly varying nature of the ZL by letting us reuse a previously computed ZL estimate when re-observing a HEALPix pixel within a small timeframe $\Delta t$
\begin{equation}
 s_\mathrm{zodi}(p, t_i) = s_\mathrm{zodi}(p, t_j).
\end{equation}
This cache induces an error on the ZL estimate proportional to $\Delta t$, which comes from the changes to the observer position with respect to the zodiacal cloud within the $\Delta t$ timeframe. For reference, the structure of the ZL on the Ecliptic or Galactic sky maps will appear to move by about $1^\circ$ a day as the Earth orbits the Sun. We find that resetting the cache every 2 hours results in a good compromise between speed and accuracy.

In addition to the positional change along Earth's orbit, the COBE satellite also orbits the Earth at an altitude of 900 km. The specific position of the observer is not taken into account when using the cache. The magnitude of these two errors is illustrated in Figure~\ref{fig:cache-error}. The top panel shows the difference between two instantaneous full-sky views at 25 $\mu$m, one simulated from the north ecliptic pole and the other from the south ecliptic pole, and illustrates the maximum error obtain in each pixel by not using the correct observer position with respect to the Earth. The bottom panel shows a similar difference map, but this time for the error induced by the observer's motion around the Sun with a cache reset time of $\Delta t = 2$hours. In both scenarios, the errors induced by the cache are well below our model residuals.
\begin{figure}
    \includegraphics[width=\columnwidth]{figs/cache_error_delta_t.pdf}
    \includegraphics[width=\columnwidth]{figs/cache_error_z.pdf}
    \caption{Errors induced by the ZL cache. \textit{(top):} Difference between two 
    instantaneous view of the ZL 900 km above Earth's north and south 
    poles. \textit{(bottom):} Difference between two instantaneous views of the ZL at $N_\mathrm{studies}=512$, 
    separated by 2 hours. Pixels with an angular separation $\leq 60^\circ$ from the Sun 
    are masked in both maps.}
    \label{fig:cache-error}
\end{figure}

The smooth nature of the ZL can be exploited in more ways. When sampling parameters for the ZL model it is possible to use a subsample of the full TOD while still resolving the smooth gradient of the ZL structure. In our analysis, we have used a thinning factor of 8, meaning that we effectively fit the data to a timestream with 1 Hz rather than the 8 Hz DIRBE CIOs. 

A final optimization is implemented in the form of spatial downsampling, where the smoothness of the ZL is resolved adequately at a lower HEALPix resolution. We find that using a $N_\mathrm{side} = 64$ down from the original $N_\mathrm{side} = 512$ results in comparable $\chi^2$ measures while further reducing the volume used in the sampling steps by a factor of 64. The downsampling is simulated by projecting the original pointing to the lower HEALPix resolution and using a corresponding low-res cache.

\begin{figure*}
    \centering
    \resizebox{\textwidth}{!}{%
    \includegraphics[height=1cm]{figs/week/week_delta_I_nu.pdf}%
    \includegraphics[height=1cm]{figs/week/week_I_nu.pdf}%
    }\\
    \resizebox{\textwidth}{!}{%
    \includegraphics[height=1cm]{figs/week/week_delta_Z_nu.pdf}%
    \includegraphics[height=1cm]{figs/week/week_Z_nu.pdf}%
    }\\
    \resizebox{\textwidth}{!}{%
    \includegraphics[height=1cm]{figs/week/week_delta_tod_nu.pdf}%
    \includegraphics[height=1cm]{figs/week/week_tod_nu.pdf}%
    }\\
    \caption{Illustration of the basic sky maps involved in the zodiacal 
    light fitting algorithms adopted by the DIRBE (\emph{left column}) 
    and \Cosmoglobe\ (\emph{right column}) pipelines for one week of 
    $25\,\mu\mathrm{m}$ observations and adopting the K98 model. The 
    DIRBE pipeline used exclusively differences between weekly and 
    full-season maps, both for the observed signal, 
    $\Delta I_{\nu} \equiv I_{\nu}-\left<I_{\nu}\right>$ (\emph{top left}), 
    and the zodiacal light model, 
    $\Delta Z_{\nu} = Z_{\nu}-\left<Z_{\nu}\right>$ (\emph{middle left}), 
    where brackets indicate full-survey averages. Correspondingly, the 
    final $\chi^2$ is defined through $\Delta I_{\nu} - \Delta Z_{\nu}$ 
    (\emph{bottom left}), and is by constrution only sensitive to 
    time-variable signals. In contrast, the basic data element in 
    \Cosmoglobe\ is the full sky signal, $I_{\nu}$ (\emph{top right}), 
    which is fitted with the full zodiacal light model, $Z_{\nu}$ 
    (\emph{middle right}), both modelled in time-domain. The $\chi^2$ 
    the minimizes minimize the total signal-minus-model residual, 
    $I_{\nu}-Z_{\nu}$ (\emph{bottom right}). The main advantage of the 
    DIRBE approach is insensitivity to stationary sky signals, in 
    particular thermal dust and CIB, while the main advantage of the 
    \Cosmoglobe\ approach is a much higher effective signal-to-noise 
    ratio, both to zodiacal light parameters and zero-levels.}
    \label{fig:week_vs_full}
\end{figure*}


\begin{figure}
    \centering
    \includegraphics[width=\linewidth]{figs/powell_red_chisq_vs_iter.pdf}
    \caption{Reduced $\chi^2$ as a function of Powell likelihood evaluation count for one single Gibbs chain. Each discrete jump indicates the start of a new Gibbs sample, which is initialized on a new random point that is close to the previous iteration. The following systematic decline within each main Gibbs iteration indicates the non-linear optimization performed by the Powell algorithm.  The solid dark region corresponds to a large number of highly sub-optimal parameter trials. }
    \label{fig:powell_chisq_iter}
\end{figure}


\begin{figure}
    \centering
    \includegraphics[width=\linewidth]{figs/powell_T0_vs_chisq.pdf}
    \caption{Reduced $\chi^2$ as a function of the temperature at 1\,AU, $T_0$. Each curve shows the full set of parameter trials within one single main Gibbs iteration (or Powell call), and different colors indicate different Gibbs iteration. Redder colors are earlier in the chain.}
    \label{fig:powell_T0}
\end{figure}


\section{Data}\label{sect:data}

\subsection{DIRBE Calibrated Individual Observations}

\subsection{Masks}


\begin{figure}
    \centering
    \includegraphics[width=\columnwidth]{figs/tod_zodi.pdf}
    \caption{}
    \label{fig:tod_zodi}
\end{figure}


\begin{figure}
    \centering
    \includegraphics[width=\columnwidth]{figs/zodi_proc_masks.pdf}
    \caption{Three of the processing masks used when estimating ZL parameters. The blue mask is the 
    mask used in the stellar emission dominated 1.25 $\mathrm{\mu m}$ band, the orange mask is used 
    in the ZL dominated 25 $\mathrm{\mu m}$ band, and the green mask is used in the thermal dust dominated 
    240 $\mathrm{\mu m}$ band.}
    \label{fig:zodi-procmask}
\end{figure}





\section{Methods}\label{sect:param-estimation}

\subsection{Bayesian analysis, posterior distribution, and Gibbs sampling}

Estimating the ZL model parameters happens within the Bayesian end-to-end Commander3 framework. An explaination of the algorithms used to fit the full data and sky model, as well as instrumental parameters are described in detail in \cite{CG02_01}. In short, we use Gibbs sampling to map out the joint posterior distribution\citep{Galloway2023}, which allows us to draw samples iteratively from various conditional distributions rather than trying to draw directly from a large and complicated joint distribution. The following equations show each parameter sampled in this Cosmoglobe analysis in order
\begin{alignat}{11}
    \tens{G} &\,\leftarrow P(\tens{G}&\,\mid &\,\dv,&\, &\,\phantom{\tens{G}} &\,\xi_n, &
    \,\beta_{\mathrm{sky}}& \,\a_{\mathrm{sky}}, &\,\zeta_{\mathrm{z}},
    &\,\a_{\mathrm{static}})\label{eq:gibbs_G}\\
    \xi_{\mathrm{n}} &\,\leftarrow P(\xi_{\mathrm{n}}&\,\mid &\,\dv,&\, &\,\tens{G}, &\,\phantom{\xi_n} &
    \,\beta_{\mathrm{sky}}& \,\a_{\mathrm{sky}}, &\,\zeta_{\mathrm{z}},
    &\,\a_{\mathrm{static}})\\
    \beta_{\mathrm{sky}} &\,\leftarrow P(\beta_{\mathrm{sky}}&\,\mid &\,\dv,&\, &\,\tens{G}, &\,\xi_n, &
    \,\phantom{\beta_{\mathrm{sky}}}& \,\a_{\mathrm{sky}}, &\,\zeta_{\mathrm{z}}, &\,\a_{\mathrm{static}})\\
    \a_{\mathrm{sky}} &\,\leftarrow P(\a_{\mathrm{sky}}&\,\mid &\,\dv,&\, &\,\tens{G}, &\,\xi_n, &
    \,\beta_{\mathrm{sky}},& \,\phantom{\a_{\mathrm{sky}},}
    &\,\zeta_{\mathrm{z}}, &\,\a_{\mathrm{static}})\\
    \zeta_{\mathrm{z}} &\,\leftarrow P(\zeta_{\mathrm{z}}&\,\mid &\,\dv,&\, &\,\tens{G}, &\,\xi_n, &
    \,\beta_{\mathrm{sky}},& \,\a_{\mathrm{sky}},
    &\,\phantom{\zeta_{\mathrm{z}},} &\,\a_{\mathrm{static}})\label{eq:gibbs_zodi}\\
    \a_{\mathrm{static}} &\,\leftarrow P(\a_{\mathrm{static}}&\,\mid &\,\dv,&\, &\,\tens{G}, &\,\xi_n, &
    \,\beta_{\mathrm{sky}},& \,\a_{\mathrm{sky}}, &\,\zeta_{\mathrm{z}} &\,\phantom{\a_{\mathrm{static}}})\label{eq:gibbs_static}.
\end{alignat}
Each of the sampling steps in the above Gibbs chain is described by \citet{CG02_01}
and references therin. 

For computational reasons it is sometimes necessary to violate the Gibbs rule,
which is the case for the step in Equation~\ref{eq:gibbs_zodi}, which describes
the sampling of the ZL parameters, denoted by $\zeta_\mathrm{z}$. The K98 ZL 
model contains around 50 parameters to describe the three-dimensional IPD 
distribution and another 30 or more to describe the thermal emission and 
scattered sunlight source functions, many of which are heavily degenerate
due to the difficulties of mapping a three-dimensional structure from a single
data set. N of these parameters were fit to the DIRBE data, in the original DIRBE analysis while the rest were fixed to parameters motivated by studies of the interplanetary medium. In our analysis we have fit M total parameters. A visualization of the degeneracies and the way in which each ZL parameter impacts the final mission-averaged predicted ZL are seen in figures~\ref{fig:atlas1} and~\ref{fig:atlas2} in Appendix~\ref{sec:param-atlas}. This atlas of ZL parameters shows normalized maps of the effects of changing a single ZL parameter by $\pm 5\%$ while keeping the rest of the model fixed. The albedo and emissivity parameters are excluded from the atlas as these represent a direct scaling of the component maps and are degenerate with the $n_0$ maps.
The posterior distribution of this complicated set of parameters exhibits a large number of local minima due to the above mentioned reasons. When experimenting we found that strict Gibbs sampling algorithm quickly became trapped. We have opted to instead 
use a simple non-linear Powell algorithm that is initialized some random parameter distance away from the previous sample, and then searches for the local minimum. Powell's method is a function minimizer, which performs bi-directional searches along a parameter vector. This algorithm is able to escape local minima, but it comes at the cost of larger
uncertainties than what would result from an ideal posterior mapper.


The K98 model was obtained by fitting the model to a set of week-maps. 
The $\chi^2$ measure used by the DIRBE team is given by
\begin{equation}
    \chi^2_\mathrm{K98} = \Delta I_\nu - \Delta Z_\nu,
\end{equation}
where $I_\nu$ and $Z_\nu$ are the full-mission frequency and ZL estimate 
maps, and $\Delta I_\nu = I_\nu - \langle I_\nu \rangle$ and $\Delta Z_\nu = Z_\nu - \langle Z_\nu \rangle$ are the difference between 
the full-mission and a week frequency and ZL map, respectively. These 
maps are illustrated in the left column in Figure~\ref{fig:week_vs_full}. 
$\chi^2_\mathrm{K98}$ is by construction only sensitive to time-variable 
signal, effectively removing the need for a good sky model of the then 
less well-understood infrared sky. Using week maps instead of the full 
timestreams has the additional effect of reducing the overall data 
volume in the fits, which would seem beneficial considering the 
computing resources available during the early DIRBE analysis. A 
downside of this method is that the monopole induced by the smooth ZL is 
removed, resulting in a much weaker signal-to-noise ratio in the ZL 
parameter estimates and to the zero-levels. In contrast, we fit the data 
at each time step to the signal minus model residual
\begin{equation}\label{eq:chisq}
    \chi^2 = I_\nu - Z_\nu.
\end{equation}
To minimize the amount of unmodeled galactic emission that enters our 
ZL parameter estimates, we use strict channel-specific processing masks, 
masking out the brightest galactic regions. The 1.25, 25, and 240 $\mu$m 
masks, corresponding to stellar light, ZL, and thermal dust-dominated 
channels, are shown in Figure~\ref{fig:zodi-procmask}.


\subsection{Non-linear posterior maximization}

The total number of free ZL model parameters in this analysis are N, 
compared to the M parameters fit in K98. We have included a non-zero
scattering contribution at 4.9 $\mu$m. The ZL SED is composition of two
blackbody like spectra with a combined minimum around the $3.5\mu$m band. The first
blackbody spectrum is associated with the solar flux through the scattering 
contribution, while the second with the re-radiated thermal emission. With an
improved sky model and lower residual levels, we believe that it is worth 
including the small non-zero contribution from scattering at 4.9 $\mu$m.
Additionally, we are using the 12 $\mu$m band as our ZL reference band, meaning
that we fix the emissivities at this band to 1 \textcolor{red}{(why?)}. The
reference ZL band in K98 is $25 \mu$m. Similar to K98, we fit one emissivity 
for the smooth cloud, and one joint emissivity for the dust bands, while we 
only fit one overall albedo for all components, at each frequency band. Where,
K98 fixed emissivities to the cloud at certian bands, we have kept all free.

We can improve our error estimates on the ZL parameters by including the 
Powell search history in our error estimation. For each bi-directional 
search, we output both the parameter values and the intermediate $\chi^2$
values which lets us find the 95\% confidence interval in the search. This
is used jointly with the uncertienties determined from the Gibbs chain.
\textcolor{red}{Do this with Duncan}

\subsection{Computational costs and optimization}


\section{Results}\label{sect:improved-model}
Here we present our best-fit ZL model from having run Commander3 for six independant
Markov Chains for N samples. Our best-fit parameters for the geometrical IPD, emissivities 
and albedo parameters are listed in tables~\ref{table:zodi-params-geo} 
and~\ref{table:zodi-params-spectral}, where they are compared to the values in the K98 model.

% \setlength{\tabcolsep}{28.5pt} % Default value: 6pt
\renewcommand{\arraystretch}{1.5} % Default value: 1
\begin{table*}
    \small
    \centering
    \newcolumntype{C}{ @{}>{${}}r<{{}$}@{} }
    \begin{tabular}{l l *2{rCl}}
    \multicolumn{8}{c}{Interplanetary dust parameters}\\
    \hline
    \hline
     Parameter & Description & \multicolumn{3}{c}{DIRBE} & \multicolumn{3}{c}{DR2} \\ 
     \hline
     \multicolumn{8}{c}{Smooth Cloud}\\
     \hline
     $n_{0, \mathrm{C}}$ [$10^{-8}$ AU$^{-1}$]\dotfill & Number density at 1 AU & 11.3 &\pm& 0.064 & 8.539 & \pm & 0.379\\
     $\alpha$\dotfill & Radial power-law exponent \quad& 1.34 &\pm& 0.022 & 1.282 & \pm & 0.015\\
     $\beta$\dotfill & Vertical shape parameter & 4.14 &\pm& 0.067 & 4.006 & \pm & 0.049\\
     $\gamma$\dotfill & Vertical power-law exponent & 0.942 &\pm& 0.025 & 1.068 & \pm & 0.026\\
     $\mu$\dotfill & Widening parameter & 0.189 &\pm& 0.014 & 0.238 & \pm & 0.013\\
     $i$ [deg]\dotfill & Inclination & 2.03 &\pm& 0.017 & 2.833 & \pm & 0.030\\
     $\Omega$ [deg]\dotfill & Ascending node & 77.7 &\pm& 0.6 & 80.931 & \pm & 0.310\\
     $x_0$ [$10^{-2}$ AU]\dotfill & x-offset from the Sun  & 1.19 &\pm& 0.11 & 1.596 & \pm & 0.104\\
     $y_0$ [$10^{-3}$ AU]\dotfill & y-offset from the Sun &  5.48 &\pm& 0.77 & 9.466 & \pm & 0.806\\
     $z_0$ [$10^{-3}$ AU]\dotfill & z-offset from the Sun & -2.22 &\pm& 0.43 & -5.256 & \pm & 0.301\\
     \hline
     \multicolumn{8}{c}{Dust band 1}\\
     \hline
     $n_{0, \mathrm{B}_1}$ [$10^{-9}$ AU$^{-1}$]\dotfill & Number density at 1 AU & 0.559 &\pm& 0.072 & 7.215 & \pm & 0.790\\
     $\delta_{\zeta_{\mathrm{B}_1}}$ [deg]\dotfill & Shape parameter & 8.78 && Fixed & 9.494 & \pm & 0.142\\
     $v_{\mathrm{B}_1}$\dotfill & Shape parameter & 0.10 && Fixed & 1.486 & \pm & 0.287\\
     $p_{\mathrm{B}_1}$\dotfill & Shape parameter & 4 && Fixed & 4.0007 & \pm & 0.0007\\
     $i_{\mathrm{B}_1}$[deg] \dotfill & Inclination & 0.56 && Fixed & 1.229 & \pm & 0.075\\
     $\Omega_{\mathrm{B}_1}$ [deg]\dotfill & Ascending node & 80 && Fixed & 60.764 & \pm & 2.415\\
     $\delta_{R_{\mathrm{B}_1}}$ [AU]\dotfill & Inner radial cutoff & 1.5 && Fixed & 0.942 & \pm & 0.005\\
     $x_0$ [$10^{-2}$ AU]\dotfill & x-offset from the Sun  &  &&  & -3.003 & \pm & 0.358\\
     $y_0$ [$10^{-2}$ AU]\dotfill & y-offset from the Sun &  &&  & -1.078 & \pm & 0.078\\
     $z_0$ [$10^{-2}$ AU]\dotfill & z-offset from the Sun &  &&  & 1.390 & \pm & 0.103\\
     \hline
     \multicolumn{8}{c}{Dust band 2}\\
     \hline
     $n_{0, \mathrm{B}_2}$ [$10^{-9}$ AU$^{-1}$]\dotfill & Number density at 1 AU & 1.99 &\pm& 0.128 & 3.921 & \pm & 0.253\\
     $\delta_{\zeta_{\mathrm{B}_2}}$ [deg]\dotfill & Shape parameter & 1.99 && Fixed & 2.605 & \pm & 0.0197\\
     $v_{\mathrm{B}_2}$\dotfill & Shape parameter & 0.90 && Fixed & 2.500 & \pm & 0.000009\\
     $p_{\mathrm{B}_2}$\dotfill & Shape parameter & 4 && Fixed & 4.001 & \pm & 0.000004\\
     $i_{\mathrm{B}_2}$ [deg]\dotfill & Inclination & 1.2 && Fixed & 1.690 & \pm & 0.028\\
     $\Omega_{\mathrm{B}_2}$ [deg]\dotfill & Ascending node & 30.3 && Fixed & 44.826 & \pm & 1.377\\
     $\delta_{R_{\mathrm{B}_2}}$ [AU]\dotfill & Inner radial cutoff & 0.94 &\pm& 0.025 & 0.992 & \pm & 0.008\\
     $x_0$ [$10^{-2}$ AU]\dotfill & x-offset from the Sun  &  &&  & -8.313 & \pm & 0.476\\
     $y_0$ [$10^{-2}$ AU]\dotfill & y-offset from the Sun &  &&  & -1.588 & \pm & 0.387\\
     $z_0$ [$10^{-3}$ AU]\dotfill & z-offset from the Sun &  &&  & 8.719 & \pm & 0.455\\
     \hline
     \multicolumn{8}{c}{Dust band 3}\\
     \hline
     $n_{0, \mathrm{B}_3}$ [$10^{-10}$ AU$^{-1}$]\dotfill & Number density at 1 AU & 1.44 &\pm& 0.234 & 2.674 & \pm & 0.185\\
     \hline
     &&&&&&\\
    \end{tabular}
    \caption{Best-fit interplanetary dust parameter estimates and uncertianties in the DR2 analysis,
     comparing values with the K98 model. Only parameters fit in the DR2 analysis are presented.}
    \label{table:zodi-params-geo}
    \end{table*}

% \setlength{\tabcolsep}{10pt} % Default value: 6pt
\begin{table*}
    \small
    \centering
    \newcolumntype{C}{ @{}>{${}}r<{{}$}@{} }
    \begin{tabular}{l l *2{rCl}}
    \multicolumn{8}{c}{Source function parameters}\\
    \hline
    \hline
    Parameter & Description & \multicolumn{3}{l}{DIRBE} & \multicolumn{3}{c}{DR2} \\ 
    \hline
    \multicolumn{8}{c}{All zodiacal components}\\
    \hline
    $T_0$ (K)\dotfill & IPD temperature at 1 AU  & 286 && Fixed & 283.29 &\pm& 2.70\\
    $A_1$ \dotfill & Albedo at 1.25$\mu $m & 0.204 &\pm& 0.0013 & 0.2087 &\pm& 0.0086\\
    $A_2$ \dotfill & Albedo at 2.2$\mu $m & 0.255 &\pm& 0.0017 & 0.2712 &\pm& 0.0109\\
    $A_3$ \dotfill & Albedo at 3.5$\mu $m & 0.210 &\pm& 0.019 & 0.3430 &\pm& 0.0088\\
    $A_4$ \dotfill & Albedo at 4.9$\mu $m  & 0 && Fixed & 0.4496 &\pm& 0.0314\\
    $A_5$ \dotfill & Albedo at 12$\mu $m  & 0 && Fixed & 0 && Fixed\\
    $A_6$ \dotfill & Albedo at 25$\mu $m  & 0 && Fixed & 0 && Fixed\\
    $A_7$ \dotfill & Albedo at 60$\mu $m  & 0 && Fixed & 0 && Fixed\\
    $A_8$ \dotfill & Albedo at 100$\mu $m  & 0 && Fixed & 0 && Fixed\\
    $A_9$ \dotfill & Albedo at 140$\mu $m  & 0 && Fixed & 0 && Fixed\\
    $A_{10}$ \dotfill & Albedo at 240$\mu $m  & 0 && Fixed & 0 && Fixed\\

    \hline
    \multicolumn{8}{c}{Smooth Cloud}\\
    \hline
    $E_1$\dotfill & Emissivity at 1.25$\mu $m  & 1 && Fixed & 1 & & Fixed\\
    $E_2$\dotfill & Emissivity at 2.2$\mu $m  & 1 && Fixed & 1 & & Fixed\\
    $E_3$\dotfill & Emissivity at 3.5$\mu $m  & 1.66 &\pm& 0.088 & 1 & & Fixed\\
    $E_4$\dotfill & Emissivity at 4.9$\mu $m  & 0.997 &\pm& 0.0036 & 1.9002 &\pm& 0.1317\\
    $E_5$\dotfill & Emissivity at 12$\mu $m  & 0.958 &\pm& 0.0026 & 1 & & Fixed\\
    $E_6$\dotfill & Emissivity at 25$\mu $m  &  1 && Fixed & 1.0005 &\pm& 0.0146\\
    $E_7$\dotfill & Emissivity at 60$\mu $m  & 0.733 &\pm& 0.0055 & 0.6471 &\pm& 0.0179\\
    $E_8$\dotfill & Emissivity at 100$\mu $m  & 0.647 &\pm& 0.012 & 0.7121 &\pm& 0.0277\\
    $E_9$\dotfill & Emissivity at 140$\mu $m  & 0.677 &&  & 0.6981 &\pm& 0.1388\\
    $E_{10}$\dotfill & Emissivity at 240$\mu$m  & 0.519 &&  & 0.4950 &\pm& 0.1729\\
    \hline
    \multicolumn{8}{c}{Dust bands}\\
    \hline
    $E_1$\dotfill & Emissivity at 1.25$\mu $m  & 1 && Fixed & 1 & & Fixed\\
    $E_2$\dotfill & Emissivity at 2.2$\mu $m  & 1 && Fixed & 1 & & Fixed\\
    $E_3$\dotfill & Emissivity at 3.5$\mu $m  & 1.66 && Fixed to smooth cloud & 1 & & Fixed\\
    $E_4$\dotfill & Emissivity at 4.9$\mu $m  & 0.359 &\pm& 0.054 & 1.7264 &\pm& 0.1062\\
    $E_5$\dotfill & Emissivity at 12$\mu $m  & 1.01 &\pm& 0.15 & 1 && Fixed\\
    $E_6$\dotfill & Emissivity at 25$\mu $m  & 1 && Fixed & 0.9826 &\pm& 0.0220\\
    $E_7$\dotfill & Emissivity at 60$\mu $m  & 1.25 &\pm& 0.30 & 0.8758  &\pm& 0.0439\\
    $E_8$\dotfill & Emissivity at 100$\mu $m  & 1.52 &\pm& 0.65 & 0.9151 &\pm& 0.0555\\
    $E_9$\dotfill & Emissivity at 140$\mu $m  & 1.13 &&  & 0.5815 &\pm& 0.3701\\
    $E_{10}$\dotfill & Emissivity at 240$\mu $m  & 1.40 &&  & 0.6355 &\pm& 0.3439\\
    \hline
    &&&&&&\\
    \end{tabular}
    \caption{Best-fit source function parameter estimates and uncertianties in the DR2 analysis,
     comparing values with the K98 model. Only parameters fit in the DR2 analysis are presented.}
\label{table:zodi-params-source}
\end{table*}


\subsection{Markov chains}

\begin{figure*}
    \centering
    \includegraphics[width=1\textwidth]{figs/total_trace.pdf}
    \caption{Trace plot of all geometrical interplanetary dust paramteres for six independant Markov chains.}
    \label{fig:trace-ipd}
\end{figure*}

\begin{figure*}
    \centering
    \includegraphics[width=1\textwidth]{figs/emissivity_and_albedo_trace.pdf}
    \caption{Trace plot of all geometrical interplanetary dust paramteres for six independant Markov chains.}
    \label{fig:trace-emissivity-albedo}
\end{figure*}


\subsection{Updated ZL model and goodness-of-fit}

\begin{figure}
    \centering
    \includegraphics[width=\columnwidth]{figs/zodi_comp.pdf}
    \caption{Simulated ZL intensity on January 1, 2024 as a function of wavelength from the best-fit 
    Cosmoglobe ZL model, made with ZodiPy. The ZL SED is directional 
    and seasonal due to temperature variations along independent line-of-sights and variations in IPD composition. 
    The black curve shows the mean sky intensity in a HEALPix map with resolution
    $N_\mathrm{side}= 64$ where pixels with an angular separation of less than
    $60^\circ$ are masked out. The colored dashed lines represent the ZL 
    intensity
    }
    \label{fig:zodi-intensity}
\end{figure}

\begin{figure}
    \centering
    \includegraphics[width=\columnwidth]{figs/zodi_reldiff.pdf}
    \caption{}
    \label{fig:reldiff}
\end{figure}


\begin{table}
\newdimen\tblskip \tblskip=5pt
\caption{}
\label{tab:chisq}
\vskip -4mm
\footnotesize
\setbox\tablebox=\vbox{
 \newdimen\digitwidth
 \setbox0=\hbox{\rm 0}
 \digitwidth=\wd0
 \catcode`*=\active
 \def*{\kern\digitwidth}
%
  \newdimen\dpwidth
  \setbox0=\hbox{.}
  \dpwidth=\wd0
  \catcode`!=\active
  \def!{\kern\dpwidth}
%
  \halign{\hbox to 2cm{#\leaderfil}\tabskip 2em&
    \hfil$#$\hfil \tabskip 2em&
    \hfil$#$\hfil \tabskip 2em&
    \hfil$#$\hfil \tabskip 0em\cr
\noalign{\doubleline}
\omit\sc $\lambda$ ($\mu\mathrm{m}$)\hfil& N_{\mathrm{samp}} (10^3) & \sigma_{0}^{\mathrm{(a)}} [\mathrm{MJy/sr}] & \chi^2_{\mathrm{red}} \cr
\noalign{\vskip 3pt\hrule\vskip 5pt}
*1.25   & *70 & *0.031 & 0.756 \cr 
*2.2    & *51 & *0.031 & 0.770 \cr 
*3.5    & *51 & *0.026 & 1.019 \cr 
*4.9    & *99 & *0.029 & 1.150 \cr 
*12     & 225 & *0.102 & 2.649 \cr
*25     & \cdots & *0.190 & \cdots \cr 
*60     & *97 & *0.322 & 1.501 \cr 
100     & *85 & *0.381 & 1.415 \cr 
140     & 187 & 32.1*  & 1.015 \cr 
240     & 177 & 18.3*  & 1.024 \cr 
\noalign{\vskip 5pt\hrule\vskip 5pt}}}
  \endPlancktable
  \tablenote {{\rm a}} Mission average TOD noise rms per 8\,Hz sample.\par
\par
\end{table}


\begin{figure*}[t]
    \centering
    \includegraphics[width=0.22\linewidth]{figs/compare_zodi_res/cosmoglobe_res_01a.pdf}%
    \includegraphics[width=0.22\linewidth]{figs/compare_zodi_res/cosmoglobe_res_01b.pdf}%
    \includegraphics[width=23mm,angle=90]{figs/compare_zodi_res/cbar_01.pdf}\hspace*{3mm}
    \includegraphics[width=0.22\linewidth]{figs/compare_zodi_res/cosmoglobe_res_02a.pdf}%
    \includegraphics[width=0.22\linewidth]{figs/compare_zodi_res/cosmoglobe_res_02b.pdf}%
    \includegraphics[width=23mm,angle=90]{figs/compare_zodi_res/cbar_02.pdf}\\
    \includegraphics[width=0.22\linewidth]{figs/compare_zodi_res/cosmoglobe_res_03a.pdf}%
    \includegraphics[width=0.22\linewidth]{figs/compare_zodi_res/cosmoglobe_res_03b.pdf}%
    \includegraphics[width=23mm,angle=90]{figs/compare_zodi_res/cbar_03.pdf}\hspace*{3mm}
    \includegraphics[width=0.22\linewidth]{figs/compare_zodi_res/cosmoglobe_res_04a.pdf}%
    \includegraphics[width=0.22\linewidth]{figs/compare_zodi_res/cosmoglobe_res_04b.pdf}%
    \includegraphics[width=23mm,angle=90]{figs/compare_zodi_res/cbar_04.pdf}\\
    \includegraphics[width=0.22\linewidth]{figs/compare_zodi_res/cosmoglobe_res_05a.pdf}%
    \includegraphics[width=0.22\linewidth]{figs/compare_zodi_res/cosmoglobe_res_05b.pdf}%
    \includegraphics[width=23mm,angle=90]{figs/compare_zodi_res/cbar_05.pdf}\hspace*{3mm}
    \includegraphics[width=0.22\linewidth]{figs/compare_zodi_res/cosmoglobe_res_06a.pdf}%
    \includegraphics[width=0.22\linewidth]{figs/compare_zodi_res/cosmoglobe_res_06b.pdf}%
    \includegraphics[width=23mm,angle=90]{figs/compare_zodi_res/cbar_06.pdf}\\
    \includegraphics[width=0.22\linewidth]{figs/compare_zodi_res/cosmoglobe_res_07a.pdf}%
    \includegraphics[width=0.22\linewidth]{figs/compare_zodi_res/cosmoglobe_res_07b.pdf}%
    \includegraphics[width=23mm,angle=90]{figs/compare_zodi_res/cbar_07.pdf}\hspace*{3mm}
    \includegraphics[width=0.22\linewidth]{figs/compare_zodi_res/cosmoglobe_res_08a.pdf}%
    \includegraphics[width=0.22\linewidth]{figs/compare_zodi_res/cosmoglobe_res_08b.pdf}%
    \includegraphics[width=23mm,angle=90]{figs/compare_zodi_res/cbar_08.pdf}\\
    \includegraphics[width=0.22\linewidth]{figs/compare_zodi_res/cosmoglobe_res_09a.pdf}%
    \includegraphics[width=0.22\linewidth]{figs/compare_zodi_res/cosmoglobe_res_09b.pdf}%
    \includegraphics[width=23mm,angle=90]{figs/compare_zodi_res/cbar_09.pdf}\hspace*{3mm}
    \includegraphics[width=0.22\linewidth]{figs/compare_zodi_res/cosmoglobe_res_10a.pdf}%
    \includegraphics[width=0.22\linewidth]{figs/compare_zodi_res/cosmoglobe_res_10b.pdf}%
    \includegraphics[width=23mm,angle=90]{figs/compare_zodi_res/cbar_10.pdf}%
    \caption{Half-mission split residual maps, smoothed by a 0.5 degrees or 15 arcmin beam.}
    \label{fig:half-mission-res2}
\end{figure*}



We illustrate the effectiveness of our model in figures~\ref{fig:dr2-zsma-compare1} 
and~\ref{fig:dr2-zsma-compare2}, which shows the data minus model residuals for both models.
We obtain cleaner residuals in all DIRBE channels with our new ZL model, most notably in the 12 and 25 $\mu$m  
ZL dominated channels.

Figure~\ref{fig:zodi-intensity} shows the interpolated ZL signal as a function of wavelength. The SED will be a composite of many independant 


\subsection{Comparison of ZSMA maps}

\textcolor{red}{(Fill in once the zodi paper gets further along and we have our results)}
\begin{figure*}
    \centering

    \resizebox{\textwidth}{!}{%
    \includegraphics[height=1cm]{figs/compare_freq_maps/cosmoglobe_ma_01.pdf}%
    \includegraphics[width=1cm,angle=90]{figs/compare_freq_maps/cbar_tot_01.pdf}%
    \includegraphics[height=1cm]{figs/compare_freq_maps/dirbe_zsma_01.pdf}%
    \includegraphics[height=1cm]{figs/compare_freq_maps/cosmoglobe_zsma_01.pdf}%
    \includegraphics[width=1cm,angle=90]{figs/compare_freq_maps/cbar_01.pdf}%
    }\\


    \resizebox{\textwidth}{!}{%
    \includegraphics[height=1cm]{figs/compare_freq_maps/cosmoglobe_ma_02.pdf}%
    \includegraphics[width=1cm,angle=90]{figs/compare_freq_maps/cbar_tot_02.pdf}%
    \includegraphics[height=1cm]{figs/compare_freq_maps/dirbe_zsma_02.pdf}%
    \includegraphics[height=1cm]{figs/compare_freq_maps/cosmoglobe_zsma_02.pdf}%
    \includegraphics[width=1cm,angle=90]{figs/compare_freq_maps/cbar_02.pdf}%
    }\\

    \resizebox{\textwidth}{!}{%
    \includegraphics[height=1cm]{figs/compare_freq_maps/cosmoglobe_ma_03.pdf}%
    \includegraphics[width=1cm,angle=90]{figs/compare_freq_maps/cbar_tot_03.pdf}%
    \includegraphics[height=1cm]{figs/compare_freq_maps/dirbe_zsma_03.pdf}%
    \includegraphics[height=1cm]{figs/compare_freq_maps/cosmoglobe_zsma_03.pdf}%
    \includegraphics[width=1cm,angle=90]{figs/compare_freq_maps/cbar_03.pdf}%
    }\\

    \resizebox{\textwidth}{!}{%
    \includegraphics[height=1cm]{figs/compare_freq_maps/cosmoglobe_ma_04.pdf}%
    \includegraphics[width=1cm,angle=90]{figs/compare_freq_maps/cbar_tot_04.pdf}%
    \includegraphics[height=1cm]{figs/compare_freq_maps/dirbe_zsma_04.pdf}%
    \includegraphics[height=1cm]{figs/compare_freq_maps/cosmoglobe_zsma_04.pdf}%
    \includegraphics[width=1cm,angle=90]{figs/compare_freq_maps/cbar_04.pdf}%
    }\\

    \resizebox{\textwidth}{!}{%
    \includegraphics[height=1cm]{figs/compare_freq_maps/cosmoglobe_ma_05.pdf}%
    \includegraphics[width=1cm,angle=90]{figs/compare_freq_maps/cbar_tot_05.pdf}%
    \includegraphics[height=1cm]{figs/compare_freq_maps/dirbe_zsma_05.pdf}%
    \includegraphics[height=1cm]{figs/compare_freq_maps/cosmoglobe_zsma_05.pdf}%
    \includegraphics[width=1cm,angle=90]{figs/compare_freq_maps/cbar_05.pdf}%
    }\\

    \resizebox{\textwidth}{!}{%
    \includegraphics[height=1cm]{figs/compare_freq_maps/cosmoglobe_ma_06.pdf}%
    \includegraphics[width=1cm,angle=90]{figs/compare_freq_maps/cbar_tot_06.pdf}%
    \includegraphics[height=1cm]{figs/compare_freq_maps/dirbe_zsma_06.pdf}%
    \includegraphics[height=1cm]{figs/compare_freq_maps/cosmoglobe_zsma_06.pdf}%
    \includegraphics[width=1cm,angle=90]{figs/compare_freq_maps/cbar_06.pdf}%
    }\\

    \resizebox{\textwidth}{!}{%
    \includegraphics[height=1cm]{figs/compare_freq_maps/cosmoglobe_ma_07.pdf}%
    \includegraphics[width=1cm,angle=90]{figs/compare_freq_maps/cbar_tot_07.pdf}%
    \includegraphics[height=1cm]{figs/compare_freq_maps/dirbe_zsma_07.pdf}%
    \includegraphics[height=1cm]{figs/compare_freq_maps/cosmoglobe_zsma_07.pdf}%
    \includegraphics[width=1cm,angle=90]{figs/compare_freq_maps/cbar_07.pdf}%
    }\\
    \caption{Comparison between ZL subtractions with our best-fit ZL model and with K98. 
    \textit{(left column):} Mission-averaged frequency maps containing ZL after data selection at our native HEALPix resolution 
    of $N_\mathrm{side} = 512$. \textit{(middle column):} DIRBE ZSMA maps
    at $N_\mathrm{side} = 256$.\textit{(right column):} Our ZL subtracted mission-average 
    maps, downgraded to $N_\mathrm{side} = 256$. The rows list to DIRBE frequency channels, from top to 
    bottom. We observe a clear improved ZL solution at all DIRBE channels. \textcolor{red}{(Replace left column with correct frequency maps)}
    }
    \label{fig:dr2-zsma-compare1}
\end{figure*}

\begin{figure*}
    \centering
    \resizebox{\textwidth}{!}{%
    \includegraphics[height=1cm]{figs/compare_freq_maps/cosmoglobe_ma_08.pdf}%
    \includegraphics[width=1cm,angle=90]{figs/compare_freq_maps/cbar_tot_08.pdf}%
    \includegraphics[height=1cm]{figs/compare_freq_maps/dirbe_zsma_08.pdf}%
    \includegraphics[height=1cm]{figs/compare_freq_maps/cosmoglobe_zsma_08.pdf}%
    \includegraphics[width=1cm,angle=90]{figs/compare_freq_maps/cbar_08.pdf}%
    }\\

    \resizebox{\textwidth}{!}{%
    \includegraphics[height=1cm]{figs/compare_freq_maps/cosmoglobe_ma_09.pdf}%
    \includegraphics[width=1cm,angle=90]{figs/compare_freq_maps/cbar_tot_09.pdf}%
    \includegraphics[height=1cm]{figs/compare_freq_maps/dirbe_zsma_09.pdf}%
    \includegraphics[height=1cm]{figs/compare_freq_maps/cosmoglobe_zsma_09.pdf}%
    \includegraphics[width=1cm,angle=90]{figs/compare_freq_maps/cbar_09.pdf}%
    }\\

    \resizebox{\textwidth}{!}{%
    \includegraphics[height=1cm]{figs/compare_freq_maps/cosmoglobe_ma_10.pdf}%
    \includegraphics[width=1cm,angle=90]{figs/compare_freq_maps/cbar_tot_10.pdf}%
    \includegraphics[height=1cm]{figs/compare_freq_maps/dirbe_zsma_10.pdf}%
    \includegraphics[height=1cm]{figs/compare_freq_maps/cosmoglobe_zsma_10.pdf}%
    \includegraphics[width=1cm,angle=90]{figs/compare_freq_maps/cbar_10.pdf}%
    }\\

    \caption{Comparison between ZL subtractions with our best-fit ZL model and with K98. 
    \textit{(left column):} Mission-averaged frequency maps containing ZL after data selection at our native HEALPix resolution 
    of $N_\mathrm{side} = 512$. \textit{(middle column):} DIRBE ZSMA maps
    at $N_\mathrm{side} = 256$. The rows list to DIRBE frequency channels, from top to 
    bottom. \textit{(right column):} Our ZL subtracted mission-average 
    maps, downgraded to $N_\mathrm{side} = 256$. We observe a clear improved ZL solution at all DIRBE channels.
    }    
    \label{fig:dr2-zsma-compare2}
\end{figure*}







\begin{figure}
    \centering
    \includegraphics[width=\linewidth]{figs/zodi_mean_diff_DIRBE_DR2.pdf}
    \caption{Monopole difference between official DIRBE and
      \cosmoglobe\ DR2 ZSMA maps, evaluated as the average of the
      difference between the maps shown in the second and third
      columns in Figs.~\ref{fig:dr2-zsma-compare1} and
      \ref{fig:dr2-zsma-compare2} over the DR2 analysis masks.}
    \label{fig:zsma_mean}
\end{figure}

\begin{figure}
    \centering
    \includegraphics[width=\linewidth]{figs/zodi_rms_ratio_DIRBE_DR2_v2.pdf}
    \caption{Rms ratio between \cosmoglobe\ DR2 and DIRBE ZSMA, ,
      $\sigma_{\mathrm{DR2}}/\sigma_\mathrm{DIRBE}$,
      as evaluated outside the DR2 analysis masks.}
    \label{fig:zsma_rms}
\end{figure}



\begin{figure*}
    \centering
    \includegraphics[width=\linewidth]{figs/tod_zodi_residuals.pdf}
    \caption{}
    \label{fig:res_vs_b}
\end{figure*}


Subtracting the ZL with our best-fit-model yields lower sky residuals at all bands 
compared to the DIRBE ZSMA maps. A comparison between the ZL residuals can be seen 
in Figure~\ref{fig:dr2-zsma-compare}. Here the first column shows the frequency maps of the
four most ZL dominated DIREB bands (4.9, 12, 25, and 60 $\mu$m), the second shows the ZL
subtracted maps with our model, and column three shows the DIRBE ZSMA.




\section{Conclusions}
\label{sec:conclusions}


\begin{acknowledgements}
 The current work has received funding from the European
  Union’s Horizon 2020 research and innovation programme under grant
  agreement numbers 819478 (ERC; \textsc{Cosmoglobe}) and 772253 (ERC;
  \textsc{bits2cosmology}). Some of the results in this paper have been 
  derived using the HEALPix \citep{Gorski2005} package.
  We acknowledge the use of the Legacy Archive for Microwave Background 
  Data
  Analysis (LAMBDA), part of the High Energy Astrophysics Science
  Archive Center
  (HEASARC). HEASARC/LAMBDA is a service of the Astrophysics Science 
  Division at the NASA Goddard Space Flight Center.  
\end{acknowledgements}


%-------------------------------------------------------------
%                                       Table with references 
%-------------------------------------------------------------
%

\bibliographystyle{aa}
\bibliography{../../common/CG_bibliography,references,../../common/Planck_bib,references}
%\bibliography{references}

\appendix
\onecolumn

\section{Interplanetary dust parameter atlas}
\label{sec:param-atlas}
In this Appendix, we present an atlas of mission-averaged ZL parameter maps.
Each map represents the effect of changing one ZL model parameter by $\pm 5\%$ while holding the other fixed. The maps are normalized.

\begin{figure*}[hbt]
    \centering
    \resizebox{0.82\textwidth}{!}{%
    \includegraphics[height=1cm]{figs/atlas/C_01.pdf}%
    \includegraphics[height=1cm]{figs/atlas/B1_01.pdf}%
    \includegraphics[height=1cm]{figs/atlas/B2_01.pdf}%
    \includegraphics[height=1cm]{figs/atlas/B3_01.pdf}%
    }\\
    \resizebox{0.82\textwidth}{!}{%
    \includegraphics[height=1cm]{figs/atlas/C_02.pdf}%
    \includegraphics[height=1cm]{figs/atlas/B1_02.pdf}%
    \includegraphics[height=1cm]{figs/atlas/B2_02.pdf}%
    \includegraphics[height=1cm]{figs/atlas/B3_02.pdf}%
    }\\
    \resizebox{0.82\textwidth}{!}{%
    \includegraphics[height=1cm]{figs/atlas/C_03.pdf}%
    \includegraphics[height=1cm]{figs/atlas/B1_03.pdf}%
    \includegraphics[height=1cm]{figs/atlas/B2_03.pdf}%
    \includegraphics[height=1cm]{figs/atlas/B3_03.pdf}%
    }\\
    \resizebox{0.82\textwidth}{!}{%
    \includegraphics[height=1cm]{figs/atlas/C_04.pdf}%
    \includegraphics[height=1cm]{figs/atlas/B1_04.pdf}%
    \includegraphics[height=1cm]{figs/atlas/B2_04.pdf}%
    \includegraphics[height=1cm]{figs/atlas/B3_04.pdf}%
    }\\
    \resizebox{0.82\textwidth}{!}{%
    \includegraphics[height=1cm]{figs/atlas/C_05.pdf}%
    \includegraphics[height=1cm]{figs/atlas/B1_05.pdf}%
    \includegraphics[height=1cm]{figs/atlas/B2_05.pdf}%
    \includegraphics[height=1cm]{figs/atlas/B3_05.pdf}%
    }\\
    \resizebox{0.82\textwidth}{!}{%
    \includegraphics[height=1cm]{figs/atlas/C_06.pdf}%
    \includegraphics[height=1cm]{figs/atlas/B1_06.pdf}%
    \includegraphics[height=1cm]{figs/atlas/B2_06.pdf}%
    \includegraphics[height=1cm]{figs/atlas/B3_06.pdf}%
    }\\
    \resizebox{0.82\textwidth}{!}{%
    \includegraphics[height=1cm]{figs/atlas/C_07.pdf}%
    \includegraphics[height=1cm]{figs/atlas/B1_07.pdf}%
    \includegraphics[height=1cm]{figs/atlas/B2_07.pdf}%
    \includegraphics[height=1cm]{figs/atlas/B3_07.pdf}%
    }\\
    \resizebox{0.82\textwidth}{!}{%
    \includegraphics[height=1cm]{figs/atlas/C_08.pdf}%
    \includegraphics[height=1cm]{figs/atlas/B1_08.pdf}%
    \includegraphics[height=1cm]{figs/atlas/B2_08.pdf}%
    \includegraphics[height=1cm]{figs/atlas/B3_08.pdf}%
    }\\
    \resizebox{0.82\textwidth}{!}{%
    \includegraphics[height=1cm]{figs/atlas/C_09.pdf}%
    \includegraphics[height=1cm]{figs/atlas/B1_09.pdf}%
    \includegraphics[height=1cm]{figs/atlas/B2_09.pdf}%
    \includegraphics[height=1cm]{figs/atlas/B3_09.pdf}%
    }\\
    \resizebox{0.82\textwidth}{!}{%
    \includegraphics[height=1cm]{figs/atlas/C_10.pdf}%
    \includegraphics[height=1cm]{figs/atlas/B1_10.pdf}%
    \includegraphics[height=1cm]{figs/atlas/B2_10.pdf}%
    \includegraphics[height=1cm]{figs/atlas/B3_10.pdf}%
    }\\
   
    \caption{ZL parameter atlas showing the difference between increasing 
    and lowering each ZL model parameter by $5\%$ in the form of normalized 
    mission-averaged ZL maps. Columns list, from left to right parameters of
    1) the smooth cloud; 2) dust band 1; 3) dust band 2; and 4) dust band 3.}
    \label{fig:atlas1}
\end{figure*}
\begin{figure*}
    \centering
    \resizebox{0.82\textwidth}{!}{%
    \includegraphics[height=1cm]{figs/atlas/SR_01.pdf}%
    \includegraphics[height=1cm]{figs/atlas/SR_06.pdf}%
    \includegraphics[height=1cm]{figs/atlas/TF_01.pdf}%
    \includegraphics[height=1cm]{figs/atlas/TF_06.pdf}%
    }\\   
    \resizebox{0.82\textwidth}{!}{%
    \includegraphics[height=1cm]{figs/atlas/SR_02.pdf}%
    \includegraphics[height=1cm]{figs/atlas/SR_07.pdf}%
    \includegraphics[height=1cm]{figs/atlas/TF_02.pdf}%
    \includegraphics[height=1cm]{figs/atlas/TF_07.pdf}%
    }\\
    \resizebox{0.82\textwidth}{!}{%
    \includegraphics[height=1cm]{figs/atlas/SR_03.pdf}%
    \includegraphics[height=1cm]{figs/atlas/SR_08.pdf}%
    \includegraphics[height=1cm]{figs/atlas/TF_03.pdf}%
    \includegraphics[height=1cm]{figs/atlas/TF_08.pdf}%
    }\\
    \resizebox{0.82\textwidth}{!}{%
    \includegraphics[height=1cm]{figs/atlas/SR_04.pdf}%
    \includegraphics[height=1cm]{figs/atlas/SR_09.pdf}%
    \includegraphics[height=1cm]{figs/atlas/TF_04.pdf}%
    \includegraphics[height=1cm]{figs/atlas/TF_09.pdf}%
    }\\
    \resizebox{0.82\textwidth}{!}{%
    \includegraphics[height=1cm]{figs/atlas/SR_05.pdf}%
    \includegraphics[height=1cm]{figs/atlas/TF_05.pdf}%
    \includegraphics[height=1cm]{figs/atlas/TF_10.pdf}%
    \includegraphics[height=1cm]{figs/atlas/TF_11.pdf}%
    }\\
    \resizebox{0.41\textwidth}{!}{%
    \includegraphics[height=1cm]{figs/atlas/general_01.pdf}%
    \includegraphics[height=1cm]{figs/atlas/general_02.pdf}%
    }
    %\hspace*{0.47\textwidth}
    \caption{ZL parameter atlas showing the difference between increasing 
    and lowering each ZL model parameter by $5\%$ in the form of normalized 
    mission-averaged ZL maps. Columns list, from left to right parameters of
    1) the smooth cloud; 2) dust band 1; 3) dust band 2; and 4) dust band 3.}
    \label{fig:atlas2}
\end{figure*}


\clearpage
\section{Component-wise zodiacal light maps and number density cross-sections}
\label{sec:zodi-comps}

% \noindent\begin{minipage}{\textwidth}
In this Appendix, we present maps of visualizations of our best-fit ZL 
light model. Such figures can help illustrate the physical properties 
of the model and help validate how physical our models are. The ZL 
component-wise maps, both the mission-averaged and the instantaneous maps 
in~\ref{fig:mission-averaged-comp-maps} 
and~\ref{fig:mission-averaged-inst-maps}, respectively, are compared to 
the K98 model. The IPD number density visualization for the K98 model, 
corresponding to Figure~\ref{fig:ipd-number-density} can be seen in 
Figure~X in~\cite{San2022} .


% \end{minipage}

\begin{figure*}
    \centering
    \includegraphics[width=\textwidth]{figs/number_density.pdf}
    \caption{Visualization of the IPD number density of the four fitted zodiacal components in our model. The number densities are shown as a cross-section of the Solar system in the xz-plane. \textit{(top left):} The smooth cloud. \textit{(top right):} Dust band 1. \textit{(bottom left):} Dust band 2. \textit{(bottom right):} Dust band 3. The gray dotted line represents the ecliptic plane and helps illustrate the variations in the components symmetry planes.}
    \label{fig:ipd-number-density}
\end{figure*}


\begin{figure*}
    \centering
    \resizebox{\textwidth}{!}{%
    \includegraphics[height=1cm]{figs/comp_maps/K98_0_inst.pdf}%
    \includegraphics[height=1cm]{figs/comp_maps/CG_0_inst.pdf}%
    \includegraphics[width=1cm,angle=90]{figs/comp_maps/cbar_0_inst.pdf}%
    }\\
    \resizebox{\textwidth}{!}{%
    \includegraphics[height=1cm]{figs/comp_maps/K98_1_inst.pdf}%
    \includegraphics[height=1cm]{figs/comp_maps/CG_1_inst.pdf}%
    \includegraphics[width=1cm,angle=90]{figs/comp_maps/cbar_1_inst.pdf}%
    }\\
    \resizebox{\textwidth}{!}{%
    \includegraphics[height=1cm]{figs/comp_maps/K98_2_inst.pdf}%
    \includegraphics[height=1cm]{figs/comp_maps/CG_2_inst.pdf}%
    \includegraphics[width=1cm,angle=90]{figs/comp_maps/cbar_2_inst.pdf}%
    }\\
    \resizebox{\textwidth}{!}{%
    \includegraphics[height=1cm]{figs/comp_maps/K98_3_inst.pdf}%
    \includegraphics[height=1cm]{figs/comp_maps/CG_3_inst.pdf}%
    \includegraphics[width=1cm,angle=90]{figs/comp_maps/cbar_3_inst.pdf}%
    }\\
    \caption{Full-sky component-wise ZL maps (January 1, 2024) at $25\mu$m made with ZodiPy. 
    \textit{(left column:)} The K98 model. \textit{(right column:)} Best-fit Cosmoglobe ZL model. 
    Rows list the zodiacal components, from top to bottom, 1) smooth cloud; 2) dust band 1; 3) 
    dust band 2; 4) dust band 3. The maps are in ecliptic coordinates, with the Sun marked as 
    an orange circle.}
    \label{fig:mission-averaged-inst-maps}
\end{figure*}

\begin{figure*}[hbt]
    \centering
    \resizebox{\textwidth}{!}{%
    \includegraphics[height=1cm]{figs/comp_maps/K98_0.pdf}%
    \includegraphics[height=1cm]{figs/comp_maps/DR2_0.pdf}%
    \includegraphics[width=1cm,angle=90]{figs/comp_maps/cbar_0.pdf}%
    }\\
    \resizebox{\textwidth}{!}{%
    \includegraphics[height=1cm]{figs/comp_maps/K98_1.pdf}%
    \includegraphics[height=1cm]{figs/comp_maps/DR2_1.pdf}%
    \includegraphics[width=1cm,angle=90]{figs/comp_maps/cbar_1.pdf}%
    }\\
    \resizebox{\textwidth}{!}{%
    \includegraphics[height=1cm]{figs/comp_maps/K98_2.pdf}%
    \includegraphics[height=1cm]{figs/comp_maps/DR2_2.pdf}%
    \includegraphics[width=1cm,angle=90]{figs/comp_maps/cbar_2.pdf}%
    }\\
    \resizebox{\textwidth}{!}{%
    \includegraphics[height=1cm]{figs/comp_maps/K98_3.pdf}%
    \includegraphics[height=1cm]{figs/comp_maps/DR2_3.pdf}%
    \includegraphics[width=1cm,angle=90]{figs/comp_maps/cbar_3.pdf}%
    }\\
    \caption{Mission-averaged component-wise ZL maps at $25\mu$m made with ZodiPy. 
    \textit{(left column:)} The K98 model. \textit{(right column:)} Best-fit Cosmoglobe ZL model.
    Rows list the zodiacal components, from top to bottom, 1) smooth cloud; 2) dust band 1; 3) 
    dust band 2; 4) dust band 3. The maps are in galactic coordinates.}
    \label{fig:mission-averaged-comp-maps}
\end{figure*}



\end{document}
%%%% End of aa.dem
