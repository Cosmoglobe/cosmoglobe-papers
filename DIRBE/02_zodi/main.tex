%                                                                 aa.dem
% AA vers. 9.1, LaTeX class for Astronomy & Astrophysics
% demonstration file
%                                                       (c) EDP Sciences
%-----------------------------------------------------------------------
%
% \documentclass[referee]{aa} % for a referee version
%\documentclass[onecolumn]{aa} % for a paper on 1 column  
%\documentclass[longauth]{aa} % for the long lists of affiliations 
%\documentclass[letter]{aa} % for the letters 
%\documentclass[bibyear]{aa} % if the references are not structured 
%                              according to the author-year natbib style

%

\documentclass{aa}  

%
\usepackage{graphicx}
\usepackage{amsmath,amsfonts,amssymb}
\usepackage{natbib}
\usepackage{tabularx}
\usepackage{collcell}
\usepackage{array}
\usepackage{booktabs}

%%%%%%%%%%%%%%%%%%%%%%%%%%%%%%%%%%%%%%%%
\usepackage{txfonts}
\usepackage{xcolor}
\usepackage{blindtext}
%%%%%%%%%%%%%%%%%%%%%%%%%%%%%%%%%%%%%%%%
% \usepackage[options]{hyperref}
% To add links in your PDF file, use the package "hyperref"
% with options according to your LaTeX or PDFLaTeX drivers.
\usepackage{float}
%\usepackage{stfloats}
\usepackage{dblfloatfix}
\usepackage{afterpage}
\usepackage{ifthen}
\usepackage[morefloats=12]{morefloats}

\usepackage{placeins}
\usepackage{multicol}
%\usepackage[breaklinks,colorlinks,citecolor=blue]{hyperref}
\bibpunct{(}{)}{;}{a}{}{,}
\usepackage[switch]{lineno}
\definecolor{linkcolor}{rgb}{0.6,0,0}
\definecolor{citecolor}{rgb}{0,0,0.75}
\definecolor{urlcolor}{rgb}{0.12,0.46,0.7}
\usepackage[breaklinks, colorlinks, urlcolor=urlcolor,
linkcolor=linkcolor,citecolor=citecolor,pdfencoding=auto]{hyperref}
\hypersetup{linktocpage}
\usepackage{bold-extra}



\def\setsymbol#1#2{\expandafter\def\csname #1\endcsname{#2}}
\def\getsymbol#1{\csname #1\endcsname}

\def\Planck{\textit{Planck}}

\def\HeJT{$^4$He-JT}

\def\allearlypapers{\nocite{planck2011-1.1, planck2011-1.3, planck2011-1.4, planck2011-1.5, planck2011-1.6, planck2011-1.7, planck2011-1.10, planck2011-1.10sup, planck2011-5.1a, planck2011-5.1b, planck2011-5.2a, planck2011-5.2b, planck2011-5.2c, planck2011-6.1, planck2011-6.2, planck2011-6.3a, planck2011-6.4a, planck2011-6.4b, planck2011-6.6, planck2011-7.0, planck2011-7.2, planck2011-7.3, planck2011-7.7a, planck2011-7.7b, planck2011-7.12, planck2011-7.13}}

\def\alltwentythirteenresultspapers{\nocite{planck2013-p01, planck2013-p02, planck2013-p02a, planck2013-p02d, planck2013-p02b, planck2013-p03, planck2013-p03c, planck2013-p03f, planck2013-p03d, planck2013-p03e, planck2013-p01a, planck2013-p06, planck2013-p03a, planck2013-pip88, planck2013-p08, planck2013-p11, planck2013-p12, planck2013-p13, planck2013-p14, planck2013-p15, planck2013-p05b, planck2013-p17, planck2013-p09, planck2013-p09a, planck2013-p20, planck2013-p19, planck2013-pipaberration, planck2013-p05, planck2013-p05a, planck2013-pip56, planck2013-p06b, planck2013-p01a}}

\def\alltwentyfifteenresultspapers{\nocite{planck2014-a01, planck2014-a03, planck2014-a04, planck2014-a05, planck2014-a06, planck2014-a07, planck2014-a08, planck2014-a09, planck2014-a11, planck2014-a12, planck2014-a13, planck2014-a14, planck2014-a15, planck2014-a16, planck2014-a17, planck2014-a18, planck2014-a19, planck2014-a20, planck2014-a22, planck2014-a24, planck2014-a26, planck2014-a28, planck2014-a29, planck2014-a30, planck2014-a31, planck2014-a35, planck2014-a36, planck2014-a37, planck2014-ES}}

\newbox\tablebox    \newdimen\tablewidth
\def\leaderfil{\leaders\hbox to 5pt{\hss.\hss}\hfil}
\def\endPlancktable{\tablewidth=\columnwidth 
    $$\hss\copy\tablebox\hss$$
    \vskip-\lastskip\vskip -2pt}
\def\endPlancktablewide{\tablewidth=\textwidth 
    $$\hss\copy\tablebox\hss$$
    \vskip-\lastskip\vskip -2pt}
\def\tablenote#1 #2\par{\begingroup \parindent=0.8em
    \abovedisplayshortskip=0pt\belowdisplayshortskip=0pt
    \noindent
    $$\hss\vbox{\hsize\tablewidth \hangindent=\parindent \hangafter=1 \noindent
    \hbox to \parindent{$^#1$\hss}\strut#2\strut\par}\hss$$
    \endgroup}
\def\doubleline{\vskip 3pt\hrule \vskip 1.5pt \hrule \vskip 5pt}

\def\L2{\ifmmode L_2\else $L_2$\fi}
\def\dtt{\Delta T/T}
\def\DeltaT{\ifmmode \Delta T\else $\Delta T$\fi}
\def\deltat{\ifmmode \Delta t\else $\Delta t$\fi}
\def\fknee{\ifmmode f_{\rm knee}\else $f_{\rm knee}$\fi}
\def\Fmax{\ifmmode F_{\rm max}\else $F_{\rm max}$\fi}
\def\solar{\ifmmode{\rm M}_{\mathord\odot}\else${\rm M}_{\mathord\odot}$\fi}
\def\Msolar{\ifmmode{\rm M}_{\mathord\odot}\else${\rm M}_{\mathord\odot}$\fi}
\def\Lsolar{\ifmmode{\rm L}_{\mathord\odot}\else${\rm L}_{\mathord\odot}$\fi}
\def\inv{\ifmmode^{-1}\else$^{-1}$\fi}
\def\mo{\ifmmode^{-1}\else$^{-1}$\fi}
\def\sup#1{\ifmmode ^{\rm #1}\else $^{\rm #1}$\fi}
\def\expo#1{\ifmmode \times 10^{#1}\else $\times 10^{#1}$\fi}
\def\,{\thinspace}
\def\lsim{\mathrel{\raise .4ex\hbox{\rlap{$<$}\lower 1.2ex\hbox{$\sim$}}}}
\def\gsim{\mathrel{\raise .4ex\hbox{\rlap{$>$}\lower 1.2ex\hbox{$\sim$}}}}
\let\lea=\lsim
\let\gea=\gsim
\def\simprop{\mathrel{\raise .4ex\hbox{\rlap{$\propto$}\lower 1.2ex\hbox{$\sim$}}}}
\def\deg{\ifmmode^\circ\else$^\circ$\fi}
\def\pdeg{\ifmmode $\setbox0=\hbox{$^{\circ}$}\rlap{\hskip.11\wd0 .}$^{\circ}
          \else \setbox0=\hbox{$^{\circ}$}\rlap{\hskip.11\wd0 .}$^{\circ}$\fi}
\def\arcs{\ifmmode {^{\scriptstyle\prime\prime}}
          \else $^{\scriptstyle\prime\prime}$\fi}
\def\arcm{\ifmmode {^{\scriptstyle\prime}}
          \else $^{\scriptstyle\prime}$\fi}
\newdimen\sa  \newdimen\sb
\def\parcs{\sa=.07em \sb=.03em
     \ifmmode \hbox{\rlap{.}}^{\scriptstyle\prime\kern -\sb\prime}\hbox{\kern -\sa}
     \else \rlap{.}$^{\scriptstyle\prime\kern -\sb\prime}$\kern -\sa\fi}
\def\parcm{\sa=.08em \sb=.03em
     \ifmmode \hbox{\rlap{.}\kern\sa}^{\scriptstyle\prime}\hbox{\kern-\sb}
     \else \rlap{.}\kern\sa$^{\scriptstyle\prime}$\kern-\sb\fi}
\def\ra[#1 #2 #3.#4]{#1\sup{h}#2\sup{m}#3\sup{s}\llap.#4}
\def\dec[#1 #2 #3.#4]{#1\deg#2\arcm#3\arcs\llap.#4}
\def\deco[#1 #2 #3]{#1\deg#2\arcm#3\arcs}
\def\rra[#1 #2]{#1\sup{h}#2\sup{m}}
\def\page{\vfill\eject}
\def\dots{\relax\ifmmode \ldots\else $\ldots$\fi}
\def\WHzsr{\ifmmode $W\,Hz\mo\,sr\mo$\else W\,Hz\mo\,sr\mo\fi}
\def\mHz{\ifmmode $\,mHz$\else \,mHz\fi}
\def\GHz{\ifmmode $\,GHz$\else \,GHz\fi}
\def\mKs{\ifmmode $\,mK\,s$^{1/2}\else \,mK\,s$^{1/2}$\fi}
\def\muKs{\ifmmode \,\mu$K\,s$^{1/2}\else \,$\mu$K\,s$^{1/2}$\fi}
\def\muKRJs{\ifmmode \,\mu$K$_{\rm RJ}$\,s$^{1/2}\else \,$\mu$K$_{\rm RJ}$\,s$^{1/2}$\fi}
\def\muKHz{\ifmmode \,\mu$K\,Hz$^{-1/2}\else \,$\mu$K\,Hz$^{-1/2}$\fi}
\def\MJysr{\ifmmode \,$MJy\,sr\mo$\else \,MJy\,sr\mo\fi}
\def\MJysrmK{\ifmmode \,$MJy\,sr\mo$\,mK$_{\rm CMB}\mo\else \,MJy\,sr\mo\,mK$_{\rm CMB}\mo$\fi}
\def\microns{\ifmmode \,\mu$m$\else \,$\mu$m\fi}
\def\micron{\microns}
\def\muK{\ifmmode \,\mu$K$\else \,$\mu$\hbox{K}\fi}
\def\microK{\ifmmode \,\mu$K$\else \,$\mu$\hbox{K}\fi}
\def\muW{\ifmmode \,\mu$W$\else \,$\mu$\hbox{W}\fi}
\def\kms{\ifmmode $\,km\,s$^{-1}\else \,km\,s$^{-1}$\fi}
\def\kmsMpc{\ifmmode $\,\kms\,Mpc\mo$\else \,\kms\,Mpc\mo\fi}

\providecommand{\sorthelp}[1]{}


% Custom definitions
\def\Cosmoglobe{\textsc{Cosmoglobe}}
\def\cosmoglobe{\textsc{Cosmoglobe}}
\def\Planck{\textit{Planck}}


% \renewcommand{\topfraction}{1.0}	% max fraction of floats at top
%     \renewcommand{\bottomfraction}{1.0}	% max fraction of floats at bottom
%     %   Parameters for TEXT pages (not float pages):
%     \setcounter{topnumber}{2}
%     \setcounter{bottomnumber}{2}
%     \setcounter{totalnumber}{4}     % 2 may work better
%     \setcounter{dbltopnumber}{2}    % for 2-column pages
%     \renewcommand{\dbltopfraction}{0.9}	% fit big float above 2-col. text
%     \renewcommand{\textfraction}{0.04}	% allow minimal text w. figs
%     %   Parameters for FLOAT pages (not text pages):
%     \renewcommand{\floatpagefraction}{0.9}	% require fuller float pages
% 	% N.B.: floatpagefraction MUST be less than topfraction !!
%     \renewcommand{\dblfloatpagefraction}{0.9}	% require fuller float pages



\begin{document} 

   \title{\bfseries{\Cosmoglobe\ DR2. II. Bayesian global modelling of zodiacal light\\ with a first application to \textit{COBE}-DIRBE}}

   %This author list corresponds to \title{Author list for L04\_CMB\_Foregrounds\_Extraction}
%Prepared by M. Lopez-Caniego (Marcos.Lopez.Caniego@sciops.esa.int), ESAC/ESA
%This version is from Thu Jul 12 18:11:48 2018 CET
%\subtitle{There are 152 co-authors in this list}
\newcommand{\oslo}[0]{1}
%\newcommand{\MIT}[0]{2}
\newcommand{\milanoA}[0]{2}
\newcommand{\milanoB}[0]{3}
\newcommand{\milanoC}[0]{4}
\newcommand{\triesteB}[0]{5}
\newcommand{\planetek}[0]{6}
\newcommand{\princeton}[0]{7}
\newcommand{\jpl}[0]{8}
\newcommand{\helsinkiA}[0]{9}
\newcommand{\helsinkiB}[0]{10}
\newcommand{\nersc}[0]{11}
\newcommand{\haverford}[0]{12}
\newcommand{\mpa}[0]{13}
\newcommand{\triesteA}[0]{14}
\newcommand{\iia}[0]{2}

\author{\small
J.~R.~Eskilt\inst{\oslo}\thanks{Corresponding author: J.~R.~Eskilt; \url{j.r.eskilt@astro.uio.no}}
\and
K.~Lee\inst{\oslo}
\and
D.~J.~Watts\inst{\oslo}
\and
S.~Nerval\inst{\oslo}
\and
et al.
}
\institute{\small
        Institute of Theoretical Astrophysics, University of Oslo, Blindern, Oslo, Norway \goodbreak
}


   \institute{Institute of Theoretical Astrophysics, University of Oslo, Blindern, Oslo, Norway}
  
   % Shortened title, author list for top of page 
   \titlerunning{\Cosmoglobe: Interplanetary dust}
   \authorrunning{M.~San et al.}

   \date{\today}
   

% write an abstract 

\abstract{We present the first Bayesian framework for global modelling of zodiacal light in the time domain and its application to the Diffuse Infrared Background Experiment (DIRBE) time-ordered data (TOD). The framework uses a modified version of the COBE/DIRBE zodiacal light model to evaluate the zodiacal light brightness along the observed line-of-sights in the time domain. We obtain a new state-of-the-art zodiacal light model by re-estimating the free parameters in the COBE/DIRBE model and produce DIRBE zodiacal light subtracted mission average (ZSMA) maps with much smaller zodiacal light residuals. We argue that the improved zodiacal light model fit becomes possible through global Bayesian end-to-end analysis of the DIRBE TODs with COBE-FIRAS, GAIA, Planck HFI, and WISE observations, giving us a more accurate instrumental and astophysical characterization of the infrared sky.
However, we do note that even though the zodiacal light model represent an improvement with respect to original DIRBE model when it comes to removing ZL from the DIRBE data, this is still a preliminary model. We illustrate the potential of this framework for building zodiacal models by conducting a small study where we further extend and modifcations to the zodiacal light model by adding more asteroidal bands, interstellar dust and using a more physically motivated parameterization of the interplanetary medium as described in \cite{RRM}. A more modern and physical parametrization of the interplanetary medium in combination with joint analysis with complementry infrared experiments such IRAS, AKARI, WISE and SPHEREx will 

}

%   \abstract{We present a new and improved interplanetary dust model. The interplanetary dust model is a re-estimation of the parameters in the Kelsall et al. (1998) model in addition to an interstellar dust component inspired by Robinson and May (200?). In addition, other small improvements such as using modern solar irradiance models are included. The model parameters are re-estimated using Commander, where we have added zodiacal parameters as an additional gibbs step. The 180 total parameters in the model are estimated using Gibbs sampling. We demonstrate the use of the new interplanetary model on the binned DIRBE CIOs along with the \Cosmoglobe\ sky model to produce the cleanest to date DIRBE sky maps. The Cosmoglobe model which is valid between 1.25 $\mu m$ and 240 $\mu m$ is added added as the new default interplanetary dust model in ZodiPy.}

   \keywords{Zodiacal dust, Interplanetary medium, Cosmology: cosmic background radiation}

   \maketitle

\setcounter{tocdepth}{3}
%\tableofcontents
   
\section{Introduction}
Zodiacal light (ZL, sometimes zodiacal emission or interplanetary dust emission) is the primary source of diffuse radiation observed in the infrared sky between 1-100 $\mu$m (\cite{leinert1998} and references therein). The light comes from scattering and re-emission of sunlight from interplanetary dust (IPD) grains. The inner Solar system is enveloped in a sun-centered cloud of IPD, with a symmetry axis tilted slightly with respect to the Ecliptic, known as the zodiacal cloud. The ZL is seasonal, and its appearance in the sky changes as the Earth moves through the IPD distribution. The most common way to model the observer position-dependent ZL is to evaluate a line-of-sight integral for each observation directly in the time-ordered domain. The time-varying and three-dimensional nature of the ZL makes it one of the most challenging foregrounds to model in astrophysical and cosmological studies of the infrared sky. The lack of a good ZL model has left a vast portion of the electromagnetic spectrum inaccessible to cosmological analysis attempting to measure the Cosmic Optical and Infrared Backgrounds (COB, CIB). 

The state-of-the-art ZL model in the field of cosmology is the \cite{Kelsall1998} model (K98, sometimes referred to as the \textit{COBE}/DIRBE or just the DIRBE model). K98 is a parametric ZL model describing the three-dimensional IPD distribution and the radiative properties of the dust. The model was developed by the \textit{COBE}/DIRBE team to remove ZL from their data in the late 1900s. Since then, our understanding of the infrared sky has improved with new observational data from experiments like Planck HFI, GAIA, and WISE. Combining recent computational advances in Bayesian cosmological analysis \citep{BP2023, Galloway2023, Watts2023}, more readily available computing power and new complementary infrared datasets allow us to extract significantly more information about the true nature of the ZL from the DIRBE data than was possible at the time of the original DIRBE analysis.

This paper is the second in a series of six describing the methods and results of our re-analysis of the time-ordered \textit{COBE}/DIRBE data within a global end-to-end Bayesian analysis framework. Here, we describe the ZL modeling approach in Commander3 and demonstrate the benefits of fitting a ZL model directly within a global end-to-end Bayesian framework. The basis of the newly implemented ZL module in Commander3 is ZodiPy\citep{San2024}, which is an Astropy-affiliated Python package for ZL simulations. The development of ZodiPy was motivated by two factors. It would serve as a basis for the code development in Commander3, and secondly, it would provide the astrophysical and cosmology communities with an accessible, easy-to-use interface to existing and coming ZL models. As an early application, \cite{San2022} demonstrates the removal of ZL from the DIRBE TOD with ZodiPy using the K98 model.

The following sections are organized as follows. In Sect. \ref{sect:zodi-model}, we introduce the K98 ZL model and discuss implementation details and some optimizations regarding evaluating ZL models on HEALPix grids. Next, in Sect. \ref{sect:param-estimation}, we discuss ZL parameter estimation and our implementation, which involves the use of a function minimizer on a sub-sample of the DIRBE TOD. Finally, in Sect. \ref{sect:improved-model}, we present a greatly improved ZL model, yielding lower residuals at all ten DIRBE channels. The improved model uses only a slightly different parametrization than the K98 model. We recommend using this model over K98 for all infrared community members.

\section{Zodiacal light modelling}\label{sect:zodi-model}
In the following we present a very brief overview of the Zodiacal light model as implemented in Commander and used during the \Cosmoglobe\ DR2 reanalysis of the DIRBE data. The implemented model is a modified version of the K98 model. For more a details on the model and the implementation, we refer to \cite{K98, ZODIPY}.
\subsection{Parameterization of interplanetary dust}
The IPD distribution in the zodiacal cloud is both smooth and stable and most of the dust can be accounted for by a diffuse cloud-like component. However, fine structures are produced within the otherwise smooth cloud, from processes such as asteroidal collisions, cometary trails, and gravitational resonance in the orbit of the planets. The full IPD distribution is therefore a sum of several zodiacal components, denoted by $c$, each described by a heliocentric ecliptic number density $n_c(x,y,z)$. We allow each zodiacal component to have a heliocentric offset $(x_{0,c}, y_{0,c}, z_{0,c})$, such that the component-centric coordinates become
\begin{equation}    
    \begin{aligned}
        x_c&= x - x_{0,c}\\
        y_c&= y - y_{0,c}\\
        z_c&= z - z_{0,c}.
    \end{aligned}
\end{equation}
Additionally, each zodiacal component are allowed to have a plane of symmetry different to the smooth zodiacal cloud component which is described through an inclination $i_c$ and ascending node $\Omega_c$. Due to gravitational dynamics the dust in a zodiacal component is expected to be symmetrically distributed about its plane. In the component-centric coordinates the density is then fully described by the radial distance $r_c$ from the origin, and the height above the symmetry plane $Z_c$
\begin{align}
    r_c &= \sqrt{x_c^2 + y_c^2 + z_c^2},\\
    Z_c &= x_c\sin{\Omega_c}\sin{i_c} - y_c \cos{\Omega_c}\sin{i_c} + z_c \cos{i_c}.
\end{align}


\subsection{DIRBE/K98 zodiacal components}
K98 models the full IPD distribution out to the orbit of Jupiter at 5.2 AU as the sum of a diffuse cloud component, three asteroidal dust bands, and a circumsolar-ring component in Earth's orbit with an enhancement in the Earth-trailing region.
\subsubsection{Diffuse cloud}
The diffuse cloud component represents the smooth IPD distribution in the zodiacal cloud and is parametrized as
\begin{equation}
    n_\mathrm{C}(x,y,z)=n_{0, \mathrm{C}}r_\mathrm{C}^{-\alpha}f(\zeta_\mathrm{C}),
\end{equation}
where $n_{0, \mathrm{C}}$ is the density at 1 AU, $\alpha$ is a power-law index, $f(\zeta_\mathrm{C})$ is the fan-like vertical distribution given as $f(\zeta_\mathrm{C}) = \exp {\left[-\beta g^\gamma \right]}$, where $\zeta_\mathrm{c} = |Z_\mathrm{c}|/r_\mathrm{c}$ is the radial height above the symmetry plane, 
\begin{equation}
    g = \begin{cases}
        \zeta^2/2\mu & \mathrm{for}\; \zeta < \mu,\\
        \zeta - \mu/2 & \mathrm{for}\; \zeta \geq \mu,
    \end{cases}
\end{equation}
and  $\beta$, $\gamma$ and $\mu$ are all shape parameters.

\subsubsection{Asteroidal dust bands}
The majority of the small scale structure in the model is represented by three asteroidal dust bands associated with asteroid families and collisions within the asteroid belt. These were parametrized as 
\begin{align}
    n_{\mathrm{B}_i}(x,y,z) &= \frac{3 n_{0, \mathrm{B}_i}}{r_{\mathrm{B}_i}} \exp \left[-\left(\frac{\zeta_{\mathrm{B}_i}}{\delta_{\zeta_{\mathrm{B}_i}}}\right)^{6}\right]\left[1 + \left(\frac{\zeta_{\mathrm{B}_i}}{\delta_{\zeta_{\mathrm{B}_i}}}\right)^{p}v^{-1}\right] \\
    &\times\left\{1-\exp \left[-\left(\frac{r_{\mathrm{B}_i}}{\delta_{r_{\mathrm{B}_i}}}\right)^{20}\right]\right\},
\end{align}
where $n_{0, \mathrm{B}_i}$ is the density of band $\mathrm{B}_i$ at 3 AU, $\delta_{r_{\mathrm{B}_i}}$ is the inner radial cut-off, and $p$, $v$ and $\delta_{\zeta_{\mathrm{B}_i}}$ are shape parameters.

\subsubsection{Circum-solar ring and Earth-trailing feature}
Finally, a circum-solar ring component was included, representing dust accumulated in Earth's orbit with an enhancment in Earth's wake, known as the trailing feature. This is parameterized as
\begin{align}
    n_\mathrm{R}(x, y, z, \theta)&=n_{0, \mathrm{SR}} \exp \left[-\frac{\left(r_\mathrm{R}-r_{0, \mathrm{SR}}\right)^2}{\sigma_{R,\mathrm{SR}} ^2}-\frac{\left| Z_\mathrm{R} \right|}{\sigma_{Z, \mathrm{SR}}}\right],\\
   &+ n_{0, \mathrm{TF}} \exp \left[-\frac{\left(r_\mathrm{R}-r_{0, \mathrm{TF}}\right)^{2}}{\sigma_{R, \mathrm{TF}}^{2}}-\frac{\left|Z_\mathrm{F}\right|}{\sigma_{Z, \mathrm{TF}}}-\frac{\left(\theta-\theta_{0, \mathrm{TF}}\right)^{2}}{\sigma_{\theta,\mathrm{TF}}^{2}}\right],
\end{align}
where $\theta$ is the heliocentric longitude of the Earth, and the radial locations $r_{0, \mathrm{SR}}$, $r_{0, \mathrm{TF}}$ specifies the distances to the peak densities $n_{0, \mathrm{SR}}$, $n_{0, \mathrm{TF}}$. The $\sigma$ parameters are length scales for the $r$, $Z$ and $\theta$ parameters, respectively. This is the only zodiacal component that is not perfectly symmetric in the plane due to the Earth-trailing feature.

\subsection{Radiative and scattering properties}
IPD grains are assumed to emit thermal emission on the form of a blackbody modified by an emissivity factor $E_{c, \lambda}$
\begin{equation}
    I^\mathrm{Thermal}_{c,\lambda} = E_{c,\lambda} B_\lambda(T).
\end{equation}
where $B_\lambda$ is the Planck function at a wavelength $\lambda$. The temperature $T$ of the IPD is assumed to fall off with radial distance $r$ from the Sun
\begin{equation}
    T(r) = T_0 R^{-\delta},
\end{equation}
where $T_0$ is the temperature at 1 AU and $\delta$ the power law index. The IPD dust grains will scatter sunlight at wavelengths similar to the grain sizes. This contribution to the observed zodiacal light is modelled as
\begin{equation}\label{eq: scat_term}
    I^\mathrm{Scattering}_{c, \lambda} = A_{c, \lambda} F_\lambda^\odot(r) \Phi_\lambda(\Theta).
\end{equation}
where, $A_{c, \lambda}$ is the albedo,  $F_\lambda^\odot(r)$ the solar flux at a heliocentric distance from the Sun $r$, and $\Phi_\lambda(\Theta)$ the phase function for scattering angles $\Theta$. The total intensity from a single grain of IPD is then
\begin{align}
    I^\mathrm{Total}_{c, \lambda} &= I^\mathrm{Scattering}_{c,\lambda} + I^\mathrm{Thermal}_{c,\lambda}\\
    &= A_{c, \lambda} F_\lambda^\odot(R) \Phi_\lambda(\Theta) + E_{c,\lambda} B_\lambda(T(R)).
\end{align}
\subsection{Evaluating the zodiacal model}

To evaluate the zodiacal light model we solve the brightness integral
\begin{equation}\label{eq:los}
    I_{p,t} = \sum_c \int n_c \left[  A_{c, \lambda} F_\lambda^\odot \Phi_\lambda + \left( 1 - A_{c, \lambda} \right) E_{c,\lambda} B_\lambda \right]\,\mathrm {ds},
\end{equation}
where $p$ is a point on the sky, $\lambda$ is the observational wavelength, and $t$ the time of observation and ds a small distance along the time-dependent line-of-sight. Note that a factor $\left( 1 - A_{c, \lambda} \right)$ is multiplied with the thermal contribution to the intensity, which accounts for the thermal emission scattered away from the line-of-sight. 

In Commander, we perform the line-of-sight integral in Eq. \eqref{eq:los} for each observation in the time-ordered datasets. This is one of the most expensive parts of the Commander pipeline in terms of computational time. We are able to make it somewhat more efficient and reduce the overall number of line-of-sight integrals computed by taking advantage of the smoothness of the ZL. The ZL moves by about $1^\circ$ on the sky each day due to Earth's motion through the Solar system. This means that observations towards the same point on the sky at proximate times $\Delta_t$ can be caches and reused at the detector levels. Additionally, the DIRBE detectors mostly view the same part of the sky meaning that we can cache and reuse the IPD densities $n_c$ for each detector. The blackbody evaluations $B_\lambda(T)$ are precomputed for a grid of Solar system temperatures and interpolated over for each detector, further reducing the computational cost.


\subsection{Geocentric stationary zodiacal light}
The circumsolar-ring and Earth-trailing feature are by definition distributed with respect to the Earth. In an Earth-centric, or geocentric, reference frame, the signal from such components should be stationary on the sky. 
A geocentric map of the sky where we have subtract of the sky model and the non-geocentric zodiacal components is shown in Figure \ref{fig: geomap} for the DIRBE 25$\mu$ detector. This is essentially then the weighted mean of any un-modelled sky signal in geocentric coordinates which should be dominated by the circumsolar-ring component and stray light from sidelobes. Rather than modelling the circum-solar ring component through line-of-sight integrals, we use the geocentric stationary maps as a look-up template for the circum-solar ring and Earth-trailing feature component. In practice this equates to computing the geocentric coordinates for each DIRBE TOD and subtracting the corresponding pixel value on the geocentric stationary map directly from the TOD. This has the additional benefit of correcting for some of the sidelobe emission which we are currently not explicitly modelling.
\begin{figure}
    \centering
    \includegraphics[width=\columnwidth]{figs/zodi_static.pdf}
    \caption{Geocentric map $M^{\bigoplus}_\nu$ of the full survey, where we have subtracted away the sky model in addition to the diffuse cloud and the asteroidal dust bands. What we are left with is the stationary emission from the circumsolar ring and the Earth-trailing and possible other contamination from sidelobes.}
    \label{fig: geomap}
\end{figure}

\section{Bayesian zodiacal light parameter estimation}\label{sect:param-estimation}
\subsection{Data selection and masks}
\subsection{Time-ordered sampling vs week maps}
\subsection{Zodiacal light parameter atlas}
\subsection{Gibbs sampling of zodiacal light parameters}

\section{Improved zodiacal light model}\label{sect:improved-model}
\subsection{Re-estimated DIRBE/K98 model}
\subsection{Exploring alternative zodiacal light models}
\subsubsection{Freeing up frozen K98 parameters}
\subsubsection{Inclusion of more asteroidal dust bands}
\subsubsection{The Rowan-Robinson and May parametrization}


%\subsection{A time-varying foreground}

% HKE: Commented out for now, since it's already shown in the Zodipy paper
%\begin{figure}
%    \centering
%      \includegraphics[width=0.7\linewidth]{figs/illustration.pdf}
%      \caption{Illustration showing that the integrated IPD along a line of sight toward a point on the celestial sphere as seen from Earth (blue circles) changes as Earth orbits the Sun (yellow circle).}
%      \label{fig: illustration}
%  \end{figure}

\subsubsection{Survey of zodiacal light parameter variations}

\begin{figure*}
    \centering
    \includegraphics[width=0.22\textwidth]{figs/zodi/zodi_10_tot.pdf}\includegraphics[width=0.22\textwidth]{figs/zodi/zodi_10_a.pdf}\includegraphics[width=0.22\textwidth]{figs/zodi/zodi_01_b.pdf}\includegraphics[width=0.22\textwidth]{figs/zodi/zodi_10_a-b.pdf} 
    \vspace{-0.3cm}

    \includegraphics[width=0.22\textwidth]{figs/zodi/zodi_09_tot.pdf}\includegraphics[width=0.22\textwidth]{figs/zodi/zodi_09_a.pdf}\includegraphics[width=0.22\textwidth]{figs/zodi/zodi_02_b.pdf}\includegraphics[width=0.22\textwidth]{figs/zodi/zodi_09_a-b.pdf}
    \vspace{-0.3cm}

    \includegraphics[width=0.22\textwidth]{figs/zodi/zodi_08_tot.pdf}\includegraphics[width=0.22\textwidth]{figs/zodi/zodi_08_a.pdf}\includegraphics[width=0.22\textwidth]{figs/zodi/zodi_03_b.pdf}\includegraphics[width=0.22\textwidth]{figs/zodi/zodi_08_a-b.pdf}
    \vspace{-0.3cm}

    \includegraphics[width=0.22\textwidth]{figs/zodi/zodi_07_tot.pdf}\includegraphics[width=0.22\textwidth]{figs/zodi/zodi_07_a.pdf}\includegraphics[width=0.22\textwidth]{figs/zodi/zodi_04_b.pdf}\includegraphics[width=0.22\textwidth]{figs/zodi/zodi_07_a-b.pdf}
    \vspace{-0.3cm}

    \includegraphics[width=0.22\textwidth]{figs/zodi/zodi_06_tot.pdf}\includegraphics[width=0.22\textwidth]{figs/zodi/zodi_06_a.pdf}\includegraphics[width=0.22\textwidth]{figs/zodi/zodi_05_b.pdf}\includegraphics[width=0.22\textwidth]{figs/zodi/zodi_06_a-b.pdf}
    \vspace{-0.3cm}

    \includegraphics[width=0.22\textwidth]{figs/zodi/zodi_05_tot.pdf}\includegraphics[width=0.22\textwidth]{figs/zodi/zodi_05_a.pdf}\includegraphics[width=0.22\textwidth]{figs/zodi/zodi_06_b.pdf}\includegraphics[width=0.22\textwidth]{figs/zodi/zodi_05_a-b.pdf}
    \vspace{-0.3cm}

    \includegraphics[width=0.22\textwidth]{figs/zodi/zodi_04_tot.pdf}\includegraphics[width=0.22\textwidth]{figs/zodi/zodi_04_a.pdf}\includegraphics[width=0.22\textwidth]{figs/zodi/zodi_07_b.pdf}\includegraphics[width=0.22\textwidth]{figs/zodi/zodi_04_a-b.pdf}
    \vspace{-0.3cm}

    \includegraphics[width=0.22\textwidth]{figs/zodi/zodi_03_tot.pdf}\includegraphics[width=0.22\textwidth]{figs/zodi/zodi_03_a.pdf}\includegraphics[width=0.22\textwidth]{figs/zodi/zodi_08_b.pdf}\includegraphics[width=0.22\textwidth]{figs/zodi/zodi_03_a-b.pdf}
    \vspace{-0.3cm}

    \includegraphics[width=0.22\textwidth]{figs/zodi/zodi_02_tot.pdf}\includegraphics[width=0.22\textwidth]{figs/zodi/zodi_02_a.pdf}\includegraphics[width=0.22\textwidth]{figs/zodi/zodi_09_b.pdf}\includegraphics[width=0.22\textwidth]{figs/zodi/zodi_02_a-b.pdf}
    \vspace{-0.3cm}

    \includegraphics[width=0.22\textwidth]{figs/zodi/zodi_01_tot.pdf}\includegraphics[width=0.22\textwidth]{figs/zodi/zodi_01_a.pdf}\includegraphics[width=0.22\textwidth]{figs/zodi/zodi_10_b.pdf}\includegraphics[width=0.22\textwidth]{figs/zodi/zodi_01_a-b.pdf}

    \caption{Zodi frequency maps}
    \label{fig:zodi_freq}
  \end{figure*}

  \begin{figure*}
    \centering
    \includegraphics[width=0.88\columnwidth]{figs/zodi_comps/zodi_06_cloud_week.pdf}\includegraphics[width=0.88\columnwidth]{figs/zodi_comps/zodi_06_cloud_full.pdf}

    \vspace{-0.6cm}

    \includegraphics[width=0.88\columnwidth]{figs/zodi_comps/zodi_06_band1_week.pdf}\includegraphics[width=0.88\columnwidth]{figs/zodi_comps/zodi_06_band1_full.pdf}

    \vspace{-0.6cm}

    \includegraphics[width=0.88\columnwidth]{figs/zodi_comps/zodi_06_band2_week.pdf}\includegraphics[width=0.88\columnwidth]{figs/zodi_comps/zodi_06_band2_full.pdf}

    \vspace{-0.6cm}

    \includegraphics[width=0.88\columnwidth]{figs/zodi_comps/zodi_06_band3_week.pdf}\includegraphics[width=0.88\columnwidth]{figs/zodi_comps/zodi_06_band3_full.pdf}

    \vspace{-0.6cm}

    \includegraphics[width=0.88\columnwidth]{figs/zodi_comps/zodi_06_ring_week.pdf}\includegraphics[width=0.88\columnwidth]{figs/zodi_comps/zodi_06_ring_full.pdf}

    \vspace{-0.6cm}

    \includegraphics[width=0.88\columnwidth]{figs/zodi_comps/zodi_06_feature_week.pdf}\includegraphics[width=0.88\columnwidth]{figs/zodi_comps/zodi_06_feature_full.pdf}

    \caption{week map of comps*}
    \label{fig: comp week}
  \end{figure*}

\begin{figure*}
\centering
\includegraphics[width=0.88\columnwidth]{figs/zodi_comps/zodi_sky_98_week.pdf}\includegraphics[width=0.88\columnwidth]{figs/zodi_comps/zodi_cloud_98_week.pdf}
\vspace{-0.6cm}

\includegraphics[width=0.88\columnwidth]{figs/zodi_comps/zodi_zodi_98_week.pdf}\includegraphics[width=0.88\columnwidth]{figs/zodi_comps/zodi_bands_98_week.pdf}
\vspace{-0.6cm}

\includegraphics[width=0.88\columnwidth]{figs/zodi_comps/zodi_res_98_week.pdf}\includegraphics[width=0.88\columnwidth]{figs/zodi_comps/zodi_ring+feature_98_week.pdf}
\caption{week map of comps*}
\label{fig: K98 week comparison}
\end{figure*}


% % INTRODUCTION
% %-------------------------------------------------------------------
% \section{Introduction}
% Zodiacal light (ZL, sometimes zodiacal emission or interplanetary dust emission) is the primary source of diffuse radiation observed in the infrared sky between 1-100 $\mu$m (\cite{leinert1998} and references therein). The light comes from scattering and re-emission of sunlight from interplanetary dust grains. ZL has been the most problematic source of foreground contamination in studies of the Extragalactic Background Light (EBL) the infrared sky due, mainly due to its time-varying nature. Improving our understanding of the interplanetary medium and building better ZL models is arguably the most critical step in making a large part of the infrared sky accessible to cosmological analysis. 

% The current state-of-the-art ZL model is the Kelsall et al. 1998 (K98), which was developed by the DIRBE team for the purpose of removing ZL from their data. It consists of a parametric three-dimensional model of the interplanetary dust distribution with several distinguishable components such as 
% These components were assumed to emit like modified blackbodies and could be evaluated through line-of-sight integrations to simulate the observed zodiacal light. 


% Ever since it was first understood in the 17th century \citep{cassini}, zodiacal light has been a driving force for the exploration of the interplanetary medium. The Diffuse Infrared Background Experiment (DIRBE) instrument, onboard the Cosmic Background Explorer (COBE), found that the zodiacal light could be effectively characterized in the infrared \citep{mather:1994, hauser:1998}. The DIRBE team developed a geometric model that represented the interplanetary medium and its identifiable components.
% These components were assumed to emit like modified blackbodies and could be evaluated through line-of-sight integrations to simulate the observed zodiacal light. This model, detailed in \cite{K98}, (here-after K98), has demonstrated its effectiveness in describing zodiacal light in the infrared and sub-millimeter domains and has been the default modeling used in the cosmology community for the past twenty years. The Planck Collaboration \citep{PLANCK2013, PLANCK2015, PLANCK2018} recently utilized the DIRBE model in their analysis of the High Frequency Instrument (HFI) data. They adapted the model to be applicable at sub-terahertz frequencies by evaluating the K98 model with the HFI scanning strategy and fitting an overall amplitude to model components.

% The ZL is considered a local foreground in CMB studies with the emission originating from the near vicinity of the observer. This is a contrast to more common foregrounds such as the CMB and galactic thermal dust emission, which we assume to be stationary in the sky. Galactic and extragalactic foregrounds can be modeled with a single template describing the structure of the component at some reference frequency. The template can then be scaled to arbitrary frequencies given a description of the component's spectral energy density (SED). Simple models like this does not apply to the zodiacal light, which highly depends on the position and time of observation.
% It is impossible to describe and model the zodiacal light foreground through a single template, applicable to all experiments. Instead, the zodiacal light must be dynamically modeled on a per-experiment basis taking into account the position of the observer within the solar system and the scanning strategy. 

% The main highlight of this work is that we are able to produce a better zodiacal light model with much smaller residuals in the frequency maps using only the same zodiacal light-contaminated data used by the DIRBE team in their original analysis. We attribute most of this success to the Cosmoglobe effort of joint global Bayesian analysis of the time-ordered DIRBE data along with maps and point source catalogs from HFI, FIRAS, and WISE. This results in a much better constrained sky model than what was possible at the time of the original DIRBE analysis, making it easier to distinguish the zodiacal light from other signal sources. Additionally, when fitting the zodiacal light model parameters, the DIRBE team used week maps differenced by the full survey sky map. While this removes all static signals from the sky, it also kills much of the effective signal-to-noise ratio for both zodiacal light parameters and the zero levels. In the Cosmoglobe approach, we fit all zodiacal light parameters directly to the timestreams, making it easier to resolve the degeneracies of the geometric interplanetary dust parameters. 

% This goes to show that even when only using archival data, it is possible to create more robust descriptions of the interplanetary medium and the observed zodiacal light. In coming Cosmoglobe data releases we will include more zodiacal light-contaminated time-ordered data from experiments such as AKARI and IRAS, which will help break many of the geometric parameter degeneracies in the interplanetary dust model due to the complimentary scanning strategies of the respective experiments.

% In this paper, we will detail the zodiacal light modeling approach used in the Commander framework during the production of Cosmoglobe DR2. Furthermore, the perhaps greatest result from this works comes from the fitting of the model parameters from a Bayesian perspective, utilizing all DIRBE bands jointly along with HFI, FIRAS and WISE data to produce the best zodiacal light model of the infrared sky. 
% In Sect. 2 we describe zodiacal light model in terms of the interplanetary dust models, source functions, and line-of-sight integrations. 
% Additionally, introduce and interpret the zodiacal light in the DIRBE time-ordered data. Finally we discuss the sampling techniques used to fit the many zodiacal light model parameters from a Bayesian perspective. In Sect. 3 we reanalyze the DIRBE data using the K98 model as derived by the DIRBE team, and see that we recover their results. 
% In Sect. 4 we explore extensions to the K98 model by lifting some of the constraints set on the model parameter by the DIRBE team, and including some of the more physical descriptions of the zodiacal components from \cite{RRM}. These extended models are referred to as model A and B, where model A is the relaxed K98 model, while model B is a more complex model with additional modifications.


% \section{Modeling zodiacal light in the DIRBE data}
% Introduction to the foreground and why it differs from typical CMB foregrounds with time-variations. More or less a copy of \cite{ZODIPY} section 3

% \subsection{Zodiacal light models}
% The zodiacal light models we use are two-fold. The first part considers the three-dimensional parameteric description of how the interplanetary dust is distributed in the solar system, consisting of several zodiacal components. The second part of the model considers the emission mechanisms of the dust. 
% The three-dimensional model is defined in cartesian heliocentric coordinates $(x, y, z)$ with each zodiacal component $c$ being allowed to have an offset from the sun $(x_0, y_0, z_0)$. The component-centric coordinates are then
% \begin{equation}    
%     \begin{aligned}
%         x_c&= x - x_{0,c}\\
%         y_c&= y - y_{0,c}\\
%         z_c&= z - z_{0,c}.
%     \end{aligned}
% \end{equation}
% Additionally, each component is allowed to have its own symmetry-plane described by the inclination $i$ and the ascending node $\Omega$. 
% In the component-centric coordinates we can then describe the full density distribution of the component $n_c(r_c, z_c)$ with the two parameters $r_c$ and $z_c$
% \begin{align}
%     r_c &= \sqrt{x_c^2 + y_c^2 + z_c^2}\\
%     z_c &= x_c\sin{\Omega_c}\sin{i_c} - y_c \cos{\Omega_c}\sin{i_c} + z_c \cos{i_c},
% \end{align}
% where $r_c$ is the radial distance from the components center $(x_{0,c}, y_{0,c}, z_{0,c})$ and $z_c$ height above the symmetry plane. Note that all distances are in units of AU.


% \subsection{Emission mechanisms}
% We assume that the thermal emission from the interplanetary dust grains are on the form of a blackbody modified by an emissivity factor $E_{c, \lambda}$
% \begin{equation}
%     I^\mathrm{Thermal}_{c,\lambda} = E_{c,\lambda} B_\lambda(T).
% \end{equation}
% where $B_\lambda(T)$ is the Planck function with the temperature $T$ given by the heliocentric distance from the sun $r$
% \begin{equation}
%     T(r) = T_0 R^{-\delta},
% \end{equation}
% where $\delta$ describes how the temperature decreases with distance from the sun.

% The scattered light is described by the solar flux $F_\lambda^\odot$, the phase function $\Phi_\lambda(\Theta)$ representing the probability of a photon being scattered at an angle $\Theta$, and the reflectivity or albedo $A_{c, \lambda}$ of the dust grains
% \begin{equation}\label{eq: scat_term}
%     I^\mathrm{Scattering}_{c, \lambda} = A_{c, \lambda} F_\lambda^\odot(R) \Phi_\lambda(\Theta).
% \end{equation}
% We refer to \cite{ZODIPY} for a more details. The total intensity from the zodiacal light is then the sum of the scattered and thermal emission
% \begin{align}
%     I^\mathrm{Total}_{c, \lambda} &= I^\mathrm{Scattering}_{c,\lambda} + I^\mathrm{Thermal}_{c,\lambda}\\
%     &= A_{c, \lambda} F_\lambda^\odot(R) \Phi_\lambda(\Theta) + \left( 1 - A_{c, \lambda} \right) E_{c,\lambda} B_\lambda(T(R)).
% \end{align}

% \subsection{Line-of-sight integration}
% When evaluating the zodiacal emission at a time $t$ we integrating along a line-of-sight $ds$ from the observer towards the observed pixel $p$
% \begin{equation}\label{eq: intensity}
%     I_{p,t} = \sum_c \int n_c \left[  A_{c, \lambda} F_\lambda^\odot \Phi_\lambda + \left( 1 - A_{c, \lambda} \right) E_{c,\lambda} B_\lambda \right]\,\mathrm ds,
% \end{equation}

% \subsection{Novel techniques for zodiacal light modelling}
% \subsubsection{Geocentric static zodiacal emission}
% The circumsolar-ring and Earth-trailing feature are by definition distributed with respect to the Earth. In an Earth-centric, or geocentric, reference frame, the signal from such components will be stationary on the sky, if these are infact perfectly following the Earth around in the orbit. 
% We can therefore make a geocentric map of the full survey, where we have subtracted away the sky model in addition to the diffuse cloud and the asteroidal dust bands. What we are left with is the static zodiacal emission coming from the circumsolar ring and the Earth-trailing feature and other potential earth centric interplanetary dust. 
% This map can then be used as a lookup map for the combined emission from the circumsolar ring and the Earth-trailing feature components. The projection of the signal from this map to the timestream at a DIRBE frequency $\nu$, at a pixel $p$ in galactic coordinates, and at at a time $t$ is then a matter of a simple lookup
% \begin{equation}
%     S_{\nu, t, p}^{\mathrm{Ring + Feature}} = M^{\bigoplus}_{\nu, p'},
% \end{equation}
% where $M^{\bigoplus}_{\nu, p'}$ is the geocentric lookup map and $p'$ is the corresponding pixel index in the geocentric reference frame. An example of such a map is shown in figure \ref{fig: geomap}.

% \begin{figure}
%     \centering
%     \includegraphics[width=\columnwidth]{figs/zodi_static.pdf}
%     \caption{Geocentric map $M^{\bigoplus}_\nu$ of the full survey, where we have subtracted away the sky model in addition to the diffuse cloud and the asteroidal dust bands. What we are left with is the stationary emission from the circumsolar ring and the Earth-trailing and possible other contamination from sidelobes.}
%     \label{fig: geomap}
% \end{figure}

% \subsubsection{Pixel Fourier interpolation}
% Write this if we end up using it.

% \section{Reference models}
% Below is a brief overview of the six zodiacal components in the K98 and RRM models. For a more detailed description of the components, see \cite{K98} and \cite{RRM}.
% \subsection{DIRBE/Kelsall (K98) model}
% The K98 model includes six zodiacal components. For a visual representation of the components, see Figure 1 and 2 in \cite{ZODIPY}. 

% \subsubsection{The diffuse cloud}
% The diffuse cloud is the primary source of interplanetary dust in the model and is a cloud-like distribution of dust that encapsulates most of the inner solar system. 
% The density of the cloud is described on a seperable form with a radial and a vertical component motived by the Pointing-Robertson effect
% \begin{equation}
%     n_\mathrm{C}(r_\mathrm{C}, z_\mathrm{C}) = n_{0,\mathrm{C}} r_\mathrm{C}^{-\alpha} f(\zeta_\mathrm{C}),
% \end{equation}
% where 
% \begin{equation}
%     f(\zeta) = \exp \left(-\beta g^{\gamma} \right),
% \end{equation}
% and $\zeta \equiv |z_c|/r_c$, $g$ is
% \begin{equation}
%     g = \begin{cases}
%         \zeta^2/2\mu, & \text{for } \zeta < \mu\\
%         \zeta - \mu/2, & \text{for } \zeta \geq \mu,
%     \end{cases}
% \end{equation}
% where $\mu$, $\alpha$, $\beta$, and $\gamma$ are all shape parameters.

% \subsubsection{Asteroidal dust bands}
% The K98 model includes three asteroidal dust band pairs, each associated with a different asteroidal family. 
% The bands are described as a sum of two Gaussian distributions in the radial and vertical direction
% \begin{equation}
%     \begin{aligned}
%         n_{\mathrm{B}_j}(r_{\mathrm{B}_j}, z_{\mathrm{B}_j})=& \frac{3 n_{0, \mathrm{B}_j}}{r_{\mathrm{B}_j}} \exp \left[-\left(\frac{\zeta_{\mathrm{B}_j}}{\delta_{\zeta_{\mathrm B_j}}}\right)^{6}\right]\left[1 + \left(\frac{\zeta_{\mathrm{B}_j}}{\delta_{\zeta_{\mathrm{B}_j}}}\right)^{p_{\mathrm{B}_j}}v^{-1}_{\mathrm{B}_j}\right] \\
%         & \times\left\{1-\exp \left[-\left(\frac{r_{\mathrm{B}_j}}{\delta_{r_{\mathrm{B}_j}}}\right)^{20}\right]\right\},
%     \end{aligned}
% \end{equation}
% \subsubsection{Circumsolar ring and Earth-trailing feature}
% The circumsolar ring and Earth-trailing feature are two components that are associated with the Earth's orbit around the sun. 
% The circumsolar ring is a toroidal distribution of dust that is centered around the sun, while the Earth-trailing is an enhancement to this distribution in the region that trails the Earth.
% \begin{equation}\label{eq: ring}
%     n_\mathrm{R}(r_\mathrm{R}, z_\mathrm{R})=n_{\mathrm{R},0} \exp \left[-\frac{\left(r_\mathrm{R}-r_{0, \mathrm{R}}\right)^2}{\sigma_{\mathrm{R}, r} ^2}-\frac{\left| z_\mathrm{R} \right|}{\sigma_{\mathrm{R}, z}}\right],
% \end{equation}
% \begin{equation}\label{eq: feature}
%    n_\mathrm{F}(r_\mathrm{F}, z_\mathrm{F}, \theta_\mathrm{F}) = n_{\mathrm{F}, 0} \exp \left[-\frac{\left(r_\mathrm{F}-r_{\mathrm{F}, 0}\right)^{2}}{\sigma_{\mathrm{F}, r}^{2}}-\frac{\left|z_\mathrm{F}\right|}{\sigma_{\mathrm{F}, z}}-\frac{\left(\theta_\mathrm{F}-\theta_{\mathrm{F}, 0}\right)^{2}}{\sigma_{\mathrm{F}, \theta }^{2}}\right].
% \end{equation}
% The Earth-trailing feature is the only part of the zodiacal light model that is not symmetric in the solar system and requires the observer to input the angle $\theta_\mathrm{F}$ which describes the longitude of the Earth. Note that in K98 these two densities was joined to one single component.

% \subsection{Rowan-Robinson and May (RRM) model}
% In the RRM model, the components are described in terms of the radial distance from the component center $r_c$ and the the latitude above the symmetry-plane $\beta_0$. The transformation between the two angular and cartesian representations are given by $z_c = r_c \sin{\beta_0}$, 
% \begin{equation}
%     \beta_0 = \arctan\left(\frac{z_c}{r_c}\right).
% \end{equation}

% \subsubsection{The fan}
% The fan is the RRM equivalent of the diffuse cloud in the K98 model. The density of the fan is described in a separable form similar to the diffuse cloud
% \begin{equation}
%     n_{\mathrm{Fan}}\left(r_\mathrm{Fan}, \beta_0\right)=n_0 r_\mathrm{Fan}^{-\gamma} f\left(\beta_0\right),
% \end{equation}
% but the vertical distribution is rather given as 
% \begin{equation}
% f\left(\beta_0\right)=\left(\cos \beta_0\right)^Q \exp \left(-P \sin \left|\beta_0\right|^{\xi}\right),
% \end{equation}
% where
% \begin{equation}\ 
% \xi= \begin{cases}
%     2 - |z_{\mathrm{Fan}}/z_{\mathrm{Fan}, 0}| \mid & \text { for }|z_\mathrm{Fan}|<z_{\mathrm{Fan}, 0} \\     1 & \text { otherwise },
% \end{cases}
% \end{equation}
% Add more components if we use them.
    
% \begin{figure}
%     \centering
%          \includegraphics[width=\linewidth]{figs/zodi_obs_diff.pdf}
%         \caption{Difference in simulated zodiacal light between an observed at the center of Earth and an observer moved 900km in the positive z-direction from the center of Earth.}
%     \label{fig: z}
% \end{figure}

% \begin{figure}
%     \centering
%         \includegraphics[width=\columnwidth]{figs/mask_zodi_fitting.pdf}
%         \caption{Masks applied when fitting zodiacal light parameters for the $2.2\mathrm{\mu m}$, $25\mathrm{\mu m}$ and $240\mathrm{\mu m}$ bands.}
%     \label{fig:masks}
% \end{figure}

% \begin{figure*}
%     \centering
%     \includegraphics[width=0.22\textwidth]{figs/zodi/zodi_10_tot.pdf}\includegraphics[width=0.22\textwidth]{figs/zodi/zodi_10_a.pdf}\includegraphics[width=0.22\textwidth]{figs/zodi/zodi_01_b.pdf}\includegraphics[width=0.22\textwidth]{figs/zodi/zodi_10_a-b.pdf} 
%     \vspace{-0.3cm}

%     \includegraphics[width=0.22\textwidth]{figs/zodi/zodi_09_tot.pdf}\includegraphics[width=0.22\textwidth]{figs/zodi/zodi_09_a.pdf}\includegraphics[width=0.22\textwidth]{figs/zodi/zodi_02_b.pdf}\includegraphics[width=0.22\textwidth]{figs/zodi/zodi_09_a-b.pdf}
%     \vspace{-0.3cm}

%     \includegraphics[width=0.22\textwidth]{figs/zodi/zodi_08_tot.pdf}\includegraphics[width=0.22\textwidth]{figs/zodi/zodi_08_a.pdf}\includegraphics[width=0.22\textwidth]{figs/zodi/zodi_03_b.pdf}\includegraphics[width=0.22\textwidth]{figs/zodi/zodi_08_a-b.pdf}
%     \vspace{-0.3cm}

%     \includegraphics[width=0.22\textwidth]{figs/zodi/zodi_07_tot.pdf}\includegraphics[width=0.22\textwidth]{figs/zodi/zodi_07_a.pdf}\includegraphics[width=0.22\textwidth]{figs/zodi/zodi_04_b.pdf}\includegraphics[width=0.22\textwidth]{figs/zodi/zodi_07_a-b.pdf}
%     \vspace{-0.3cm}

%     \includegraphics[width=0.22\textwidth]{figs/zodi/zodi_06_tot.pdf}\includegraphics[width=0.22\textwidth]{figs/zodi/zodi_06_a.pdf}\includegraphics[width=0.22\textwidth]{figs/zodi/zodi_05_b.pdf}\includegraphics[width=0.22\textwidth]{figs/zodi/zodi_06_a-b.pdf}
%     \vspace{-0.3cm}

%     \includegraphics[width=0.22\textwidth]{figs/zodi/zodi_05_tot.pdf}\includegraphics[width=0.22\textwidth]{figs/zodi/zodi_05_a.pdf}\includegraphics[width=0.22\textwidth]{figs/zodi/zodi_06_b.pdf}\includegraphics[width=0.22\textwidth]{figs/zodi/zodi_05_a-b.pdf}
%     \vspace{-0.3cm}

%     \includegraphics[width=0.22\textwidth]{figs/zodi/zodi_04_tot.pdf}\includegraphics[width=0.22\textwidth]{figs/zodi/zodi_04_a.pdf}\includegraphics[width=0.22\textwidth]{figs/zodi/zodi_07_b.pdf}\includegraphics[width=0.22\textwidth]{figs/zodi/zodi_04_a-b.pdf}
%     \vspace{-0.3cm}

%     \includegraphics[width=0.22\textwidth]{figs/zodi/zodi_03_tot.pdf}\includegraphics[width=0.22\textwidth]{figs/zodi/zodi_03_a.pdf}\includegraphics[width=0.22\textwidth]{figs/zodi/zodi_08_b.pdf}\includegraphics[width=0.22\textwidth]{figs/zodi/zodi_03_a-b.pdf}
%     \vspace{-0.3cm}

%     \includegraphics[width=0.22\textwidth]{figs/zodi/zodi_02_tot.pdf}\includegraphics[width=0.22\textwidth]{figs/zodi/zodi_02_a.pdf}\includegraphics[width=0.22\textwidth]{figs/zodi/zodi_09_b.pdf}\includegraphics[width=0.22\textwidth]{figs/zodi/zodi_02_a-b.pdf}
%     \vspace{-0.3cm}

%     \includegraphics[width=0.22\textwidth]{figs/zodi/zodi_01_tot.pdf}\includegraphics[width=0.22\textwidth]{figs/zodi/zodi_01_a.pdf}\includegraphics[width=0.22\textwidth]{figs/zodi/zodi_10_b.pdf}\includegraphics[width=0.22\textwidth]{figs/zodi/zodi_01_a-b.pdf}

%     \caption{Zodi frequency maps}
%     \label{fig:zodi_freq}
%   \end{figure*}

%   \begin{figure*}
%     \centering
%     \includegraphics[width=0.9\columnwidth]{figs/zodi_comps/zodi_06_cloud_week.pdf}\includegraphics[width=0.9\columnwidth]{figs/zodi_comps/zodi_06_cloud_full.pdf}

%     \vspace{-0.6cm}

%     \includegraphics[width=0.9\columnwidth]{figs/zodi_comps/zodi_06_band1_week.pdf}\includegraphics[width=0.9\columnwidth]{figs/zodi_comps/zodi_06_band1_full.pdf}

%     \vspace{-0.6cm}

%     \includegraphics[width=0.9\columnwidth]{figs/zodi_comps/zodi_06_band2_week.pdf}\includegraphics[width=0.9\columnwidth]{figs/zodi_comps/zodi_06_band2_full.pdf}

%     \vspace{-0.6cm}

%     \includegraphics[width=0.9\columnwidth]{figs/zodi_comps/zodi_06_band3_week.pdf}\includegraphics[width=0.9\columnwidth]{figs/zodi_comps/zodi_06_band3_full.pdf}

%     \vspace{-0.6cm}

%     \includegraphics[width=0.9\columnwidth]{figs/zodi_comps/zodi_06_ring_week.pdf}\includegraphics[width=0.9\columnwidth]{figs/zodi_comps/zodi_06_ring_full.pdf}

%     \vspace{-0.6cm}

%     \includegraphics[width=0.9\columnwidth]{figs/zodi_comps/zodi_06_feature_week.pdf}\includegraphics[width=0.9\columnwidth]{figs/zodi_comps/zodi_06_feature_full.pdf}

%     \caption{week map of comps*}
%     \label{fig: comp week}
%   \end{figure*}

% \begin{figure*}
% \centering
% \includegraphics[width=0.9\columnwidth]{figs/zodi_comps/zodi_sky_98_week.pdf}\includegraphics[width=0.9\columnwidth]{figs/zodi_comps/zodi_cloud_98_week.pdf}
% \vspace{-0.6cm}

% \includegraphics[width=0.9\columnwidth]{figs/zodi_comps/zodi_zodi_98_week.pdf}\includegraphics[width=0.9\columnwidth]{figs/zodi_comps/zodi_bands_98_week.pdf}
% \vspace{-0.6cm}

% \includegraphics[width=0.9\columnwidth]{figs/zodi_comps/zodi_res_98_week.pdf}\includegraphics[width=0.9\columnwidth]{figs/zodi_comps/zodi_ring+feature_98_week.pdf}
% \caption{week map of comps*}
% \label{fig: K98 week comparison}
% \end{figure*}


% \begin{figure*}
%   \centering
%    	\includegraphics[width=0.8\linewidth]{figs/atlas_1_v2.pdf}
%   	\caption{Atlas 1}
% 	\label{fig: atlas1}
% \end{figure*}

% \begin{figure*}
%     \centering
%          \includegraphics[width=0.8\linewidth]{figs/atlas_2_v2.pdf}
%         \caption{Atlas 2}
%       \label{fig: atlas2}
%   \end{figure*}


% \subsection{DIRBE data}
% Discuss data used for sampling of the model. Talk about time-order processing, downsampling, thinning, etc.

% %\subsection{Difficulties with sampling and degeneracies in the interplanetary dust model parameters}
% See atlases in figure \ref{fig: atlas1} and \ref{fig: atlas2} for degeneracies in the interplanetary dust model parameters.

% \subsection{Sampling techniques}

% \begin{figure*}
%     \centering
%     \includegraphics{figs/zodi_params_new.pdf}
%     \caption{A subset of the estimated zodiacal light parameters fit in this work.}
%     \label{fig: zodi_trace}

% \end{figure*}

\begin{figure*}
    \centering
    \includegraphics[width=0.49\linewidth]{figs/maptot_06a_week_minus_full.pdf}
    \includegraphics[width=0.49\linewidth]{figs/maptot_06a_week.pdf}\\
    \includegraphics[width=0.49\linewidth]{figs/mapzodi_06a_week_minus_full.pdf}
    \includegraphics[width=0.49\linewidth]{figs/mapzodi_06a_week.pdf}
    \includegraphics[width=0.49\linewidth]{figs/map_06a_week_minus_full.pdf}
    \includegraphics[width=0.49\linewidth]{figs/map_06a_week.pdf}
    \caption{Illustration of the basic sky maps involved in the zodiacal light fitting algorithms adopted by the DIRBE (\emph{left column}) and \Cosmoglobe\ (\emph{right column}) pipelines for one week of $25\,\mu\mathrm{m}$ observations and adopting the K98 model. The DIRBE pipeline used exclusively differences between weekly and full-season maps, both for the observed signal, $\Delta I_{\nu} \equiv I_{\nu}-\left<I_{\nu}\right>$ (\emph{top left}), and the zodiacal light model, $\Delta Z_{\nu} = Z_{\nu}-\left<Z_{\nu}\right>$ (\emph{middle left}), where brackets indicate full-survey averages. Correspondingly, the final $\chi^2$ is defined through $\Delta I_{\nu} - \Delta Z_{\nu}$ (\emph{bottom left}), and is by constrution only sensitive to time-variable signals. 
    In contrast, the basic data element in \Cosmoglobe\ is the full sky signal, $I_{\nu}$ (\emph{top right}), which is fitted with the full zodiacal light model, $Z_{\nu}$ (\emph{middle right}), both modelled in time-domain. The $\chi^2$ the minimizes minimize the total signal-minus-model residual, $I_{\nu}-Z_{\nu}$ (\emph{bottom right}). The main advantage of the DIRBE approach is insensitivity to stationary sky signals, in particular thermal dust and CIB, while the main advantage of the \Cosmoglobe\ approach is a much higher effective signal-to-noise ratio, both to zodiacal light parameters and zero-levels.}
    \label{fig:week_vs_full}
  \end{figure*}


% The approach we will use Gibbs sample each of the model parameters. This means that we propose a change to one model parameter, estimate the zodiacal emission over the full timestream of each ten DIRBE bands, compute a chi-squared and accept/reject. We perform N such proposals before we move on the the next parameter. Since the zodiacal emission is mostly very smooth on the sky, we can afford to downsample the TOD timestream before evaluating the zodiacal emission. For the diffuse cloud and the dust bands we downsample the TODS by X. For the circum-solar ring and the Earth-trailing feature, which are less smooth, and harder to constrain, we downsample the TODS by X. The downsampled TODS are subtracted by the \Cosmoglobe\ skymodel, and recomputed after after each parameter evaluation. 








% \section{Reanalysis of the K98 zodiacal light model}



% \begin{figure*}
% 	\centering
% 	\includegraphics[width=0.8\linewidth]{figs/zodi_diff.pdf}
% 	\caption{Half-mission model. Columns 1 and 2 are the prediction of the zodiacal emission for each band for the first and second half of the DIRBE mission. Column 3 is the difference between these two columns, and column 4 is the difference between the two zodi-subtracted half-mission maps}
% 	\label{fig: zodi_HM}
% \end{figure*}

% \begin{figure*}
%     \centering
% 	 \includegraphics[width=0.8\linewidth]{figs/tod_zodi_residuals.pdf}
% 	\caption{Data-minus-model residual for \cosmoglobe\ results (CG, black) and the official \cite{K98} model (K98, orange), as a function of ecliptic latitude, Galactic latitude, and Solar elongation. The offset for \cosmoglobe\ and the K98 model are listed on the left and right axes of each row, respectively. The \cosmoglobe\ and K98 values are horizontally offset left and right respectively for clarity.}
%       \label{fig: zodi_timestream}
%   \end{figure*}

% \begin{table*}
%     % \renewcommand{\arraystretch}{1.1} % Default value: 1
%     \begin{center}
%     \small
%     \caption{Best fit geometrical interplanetary dust parameters as fit by K98 and us.}
%     \label{table:zodi parameters}
%     \begin{tabular}{
%         l 
%         l 
%         >{\collectcell{}}r<{\endcollectcell}
%         @{${}\pm{}$}
%         >{\collectcell{}}l<{\endcollectcell}
%         >{\collectcell{}}r<{\endcollectcell}
%         @{${}\pm{}$}
%         >{\collectcell{}}l<{\endcollectcell}
%         >{\collectcell{}}r<{\endcollectcell}
%         @{${}\pm{}$}
%         >{\collectcell{}}l<{\endcollectcell}
%     }
%     \hline \hline
%     Parameter & Description & \multicolumn{2}{c}{K98} & \multicolumn{2}{c}{Model A} & \multicolumn{2}{c}{Model B} \\
%     \hline
%     \multicolumn{8}{c}{All zodiacal components}\\
%     \hline
%     $T_0$ [K]     & Temperature at 1 AU    & \multicolumn{2}{c}{286} & \multicolumn{2}{c}{286} & \multicolumn{2}{c}{286}\\
%     $\delta$      & Temperature power-law exponent    & \multicolumn{2}{c}{0.467} & \multicolumn{2}{c}{0.467} & \multicolumn{2}{c}{0.467}\\
%     \hline
%     \multicolumn{8}{c}{Diffuse cloud}\\
%     \hline
%     $n_0$ [$10^{-7}$ AU$^{-1}$]     & Density at 1 AU               & 1.13 & 0.0064           & 1.13 & 0.0064         & 1.13 & 0.0064\\
%     $\alpha$                        & Radial power-law exponent     & 1.34 & 0.022            & 1.34 & 0.022          & 1.34 & 0.022\\
%     $\beta$                         & Vertical shape parameter      & 4.14 & 0.067            & 4.14 & 0.067          & 4.14 & 0.067\\
%     $\gamma$                        & Vertical power-law exponent   & 0.942 & 0.025           & 0.942 & 0.025         & 0.942 & 0.025\\
%     $\mu$                           & Widening parameter            & 0.189 & 0.014           & 0.189 & 0.014         & 0.189 & 0.014\\
%     $i$ [deg]                       & Inclination                   & 2.03 & 0.017            & 2.03 & 0.017          & 2.03 & 0.017\\
%     $\Omega$ [deg]                  & Ascending node                & 77.7 & 0.6              & 77.7 & 0.6            & 77.7 & 0.6\\
%     $X_0$ [$10^{-3}$ AU]            & x offset from Sun             & 11.9 & 1.1 & 11.9 & 11.9 & 1.1 & 11.9 \\ 
%     $Y_0$ [$10^{-3}$ AU]            & y offset from Sun             & 5.48 & 0.77 & 5.48 & 5.48 & 0.77 & 5.48 \\ 
%     $Z_0$ [$10^{-3}$ AU]            & z offset from Sun             & 2.15 & 0.43 & 2.15 & 2.15 & 0.43 & 2.15 \\ 
%     \hline
%     \multicolumn{8}{c}{Asteroidal dust band 1}\\
%     \hline
%     $n_0$ [$10^{-10}$ AU$^{-1}$]  & Density at 3 AU               & 5.59 & 0.72                 & 5.59 & 0.72               & 5.59 & 0.72\\
%     $\delta_{\zeta_{B}}$ [deg]    & Shape parameter               & \multicolumn{2}{c}{8.78}    & \multicolumn{2}{c}{8.78}  & \multicolumn{2}{c}{8.78}\\
%     $v_{B}$                       & Shape parameter               & \multicolumn{2}{c}{0.1}     & \multicolumn{2}{c}{0.1}   & \multicolumn{2}{c}{0.1}\\
%     $p_{B}$                       & Shape parameter               & \multicolumn{2}{c}{4}       & \multicolumn{2}{c}{4}     & \multicolumn{2}{c}{4}\\
%     $i_{B}$ [deg]                 & Inclination                   & \multicolumn{2}{c}{0.56}    & \multicolumn{2}{c}{0.56}  & \multicolumn{2}{c}{0.56}\\
%     $\Omega_{B}$ [deg]            & Ascending node                & \multicolumn{2}{c}{80}      & \multicolumn{2}{c}{80}    & \multicolumn{2}{c}{80}\\
%     $\delta_{R_{B}}$ [AU]         & Inner radial cutoff           & \multicolumn{2}{c}{1.5}     & \multicolumn{2}{c}{1.5}   & \multicolumn{2}{c}{1.5}\\
%     \hline
%     \multicolumn{8}{c}{Asteroidal dust band 2}\\
%     \hline
%     $n_0$ [$10^{-9}$ AU$^{-1}$]   & Density at 3 AU               & 1.99 & 0.128                & 1.99 & 0.128              & 1.99 & 0.128\\
%     $\delta_{\zeta_{B}}$ [deg]    & Shape parameter               & \multicolumn{2}{c}{8.78}    & \multicolumn{2}{c}{8.78}  & \multicolumn{2}{c}{8.78}\\
%     $v_{B}$                       & Shape parameter               & \multicolumn{2}{c}{0.9}     & \multicolumn{2}{c}{0.9}   & \multicolumn{2}{c}{0.9}\\
%     $p_{B}$                       & Shape parameter               & \multicolumn{2}{c}{4}       & \multicolumn{2}{c}{4}     & \multicolumn{2}{c}{4}\\
%     $i_{B}$ [deg]                 & Inclination                   & \multicolumn{2}{c}{1.2}     & \multicolumn{2}{c}{1.2}   & \multicolumn{2}{c}{1.2}\\
%     $\Omega_{B}$ [deg]            & Ascending node                & \multicolumn{2}{c}{30.3}    & \multicolumn{2}{c}{30.3}  & \multicolumn{2}{c}{30.3}\\
%     $\delta_{R_{B}}$ [AU]         & Inner radial cutoff           & \multicolumn{2}{c}{0.94}    & \multicolumn{2}{c}{0.94}  & \multicolumn{2}{c}{0.94}\\
%     \hline
%     \multicolumn{8}{c}{Asteroidal dust band 3}\\
%     \hline
%     $n_0$ [$10^{-10}$ AU$^{-1}$]  & Density at 3 AU               & 1.44 & 0.234                & 1.44 & 0.234              & 1.44 & 0.234  \\
%     $\delta_{\zeta_{B}}$ [deg]    & Shape parameter               & \multicolumn{2}{c}{15}      & \multicolumn{2}{c}{15}    & \multicolumn{2}{c}{15}\\
%     $v_{B}$                       & Shape parameter               & \multicolumn{2}{c}{0.05}    & \multicolumn{2}{c}{0.05}  & \multicolumn{2}{c}{0.05}\\
%     $p_{B}$                       & Shape parameter               & \multicolumn{2}{c}{4}       & \multicolumn{2}{c}{4}     & \multicolumn{2}{c}{4}\\
%     $i_{B}$ [deg]                 & Inclination                   & \multicolumn{2}{c}{0.8}     & \multicolumn{2}{c}{0.8}   & \multicolumn{2}{c}{0.8}\\
%     $\Omega_{B}$ [deg]            & Ascending node                & \multicolumn{2}{c}{80}      & \multicolumn{2}{c}{80}    & \multicolumn{2}{c}{80}\\
%     $\delta_{R_{B}}$ [AU]         & Inner radial cutoff           & \multicolumn{2}{c}{1.5}     & \multicolumn{2}{c}{1.5}   & \multicolumn{2}{c}{1.5}\\
%     \hline
%     \multicolumn{8}{c}{Circumsolar Ring}\\
%     \hline
%     $n_\mathrm{SR}$ [$10^{-8}$ AU$^{-1}$]   & Density at 1 AU           & 1.83 & 0.127              & 1.83 & 0.127              & 1.83 & 0.127\\
%     $R_\mathrm{SR}$ [AU]                    & Radius of peak density    & 1.03 & 0.00016            & 1.03 & 0.00016            & 1.03 & 0.00016\\
%     $\sigma_\mathrm{rSR}$ [AU]              & Radial dispersion         & \multicolumn{2}{c}{0.025} & \multicolumn{2}{c}{0.025} & \multicolumn{2}{c}{0.025}\\
%     $\sigma_\mathrm{zSR}$ [AU]              & Vertical dispersion       & 0.054 & 0.0066            & 0.054 & 0.0066            & 0.054 & 0.0066\\
%     $i_\mathrm{RB}$ [deg]                   & Inclination               & 0.49  & 0.063             & 0.49  & 0.063             & 0.49  & 0.063\\
%     $\Omega_\mathrm{RB}$ [deg]              & Ascending node            & 22.3  & 0.0014            & 22.3 & 0.0014             & 22.3 & 0.0014\\
%     \hline
%     \multicolumn{8}{c}{Trailing Feature}\\
%     \hline
%     $n_\mathrm{TB}$ [$10^{-8}$ AU$^{-1}$]   & Density at 1 AU                   & 1.9 & 0.142 & 1.9 & 0.142 & 1.9 & 0.142\\
%     $R_\mathrm{TB}$ [AU]                    & Radius of peak density            & 1.06 & 0.011 & 1.06 & 0.011 & 1.06 & 0.011\\
%     $\sigma_\mathrm{rTB}$ [AU]              & Radial dispersion                 &  0.10 & 0.0097 & 0.10 & 0.0097 & 0.10 & 0.0097\\
%     $\sigma_\mathrm{zTB}$ [AU]              & Vertical dispersion               & 0.091 &  0.013 & 0.091 &  0.013 & 0.091 &  0.013\\
%     $\theta_\mathrm{TB}$ [deg]              & Longitude with respect to Earth   & \multicolumn{2}{c}{-10} & \multicolumn{2}{c}{-10} & \multicolumn{2}{c}{-10}\\
%     $\sigma_\mathrm{\theta TB}$ [deg]       & Longitude dispersion              & 12.1 & 3.4 & 12.1 & 3.4 & 12.1 & 3.4\\
%     \hline    
%     \end{tabular}
%     \end{center}
% \end{table*}


% \begin{table*}
%     % \renewcommand{\arraystretch}{1.1} % Default value: 1
%     \begin{center}
%     \small
%     \caption{Best fit zodiacal light source parameters from this analysis and the K98 model.}
%     \label{table:zodi parameters}
%     \begin{tabular}{
%         l 
%         >{\collectcell\Num}r<{\endcollectcell}
%         @{${}\pm{}$}
%         >{\collectcell\Num}l<{\endcollectcell}
%         >{\collectcell\Num}r<{\endcollectcell}
%         @{${}\pm{}$}
%         >{\collectcell\Num}l<{\endcollectcell}
%         >{\collectcell\Num}r<{\endcollectcell}
%         @{${}\pm{}$}
%         >{\collectcell\Num}l<{\endcollectcell}
%         >{\collectcell\Num}r<{\endcollectcell}
%         @{${}\pm{}$}
%         >{\collectcell\Num}l<{\endcollectcell}
%         >{\collectcell\Num}r<{\endcollectcell}
%         @{${}\pm{}$}
%         >{\collectcell\Num}l<{\endcollectcell}
%         >{\collectcell\Num}r<{\endcollectcell}
%         @{${}\pm{}$}
%         >{\collectcell\Num}l<{\endcollectcell}
%     }
%     \hline \hline
%     Channel [$\mu$m] & \multicolumn{2}{c}{Diffuse Cloud} & \multicolumn{2}{c}{Dust band 1} & \multicolumn{2}{c}{Dust band 2} & \multicolumn{2}{c}{Dust band 3} & \multicolumn{2}{c}{Circumsolar ring} & \multicolumn{2}{c}{Trailing feature} \\
%     \hline
%     \multicolumn{13}{c}{Emissivity}\\
%     \hline
%     3.5  & 1.66 & 0.088 & \multicolumn{2}{c}{1} & \multicolumn{2}{c}{1} & \multicolumn{2}{c}{1} & \multicolumn{2}{c}{1} & \multicolumn{2}{c}{1} \\
%     4.9  & 0.997 & 0.0036 & \multicolumn{2}{c}{1} & \multicolumn{2}{c}{1} & \multicolumn{2}{c}{1} & \multicolumn{2}{c}{1} & \multicolumn{2}{c}{1} \\
%     12  & 0.958 & 0.002 & \multicolumn{2}{c}{1} & \multicolumn{2}{c}{1} & \multicolumn{2}{c}{1} & \multicolumn{2}{c}{1} & \multicolumn{2}{c}{1} \\
%     25  & \multicolumn{2}{c}{1} & \multicolumn{2}{c}{1} & \multicolumn{2}{c}{1} & \multicolumn{2}{c}{1} & \multicolumn{2}{c}{1} & \multicolumn{2}{c}{1} \\
%     60  & 0.733 & 0.0055 & \multicolumn{2}{c}{1} & \multicolumn{2}{c}{1} & \multicolumn{2}{c}{1} & \multicolumn{2}{c}{1} & \multicolumn{2}{c}{1} \\
%     100  & 0.647 & 0.012 & \multicolumn{2}{c}{1} & \multicolumn{2}{c}{1} & \multicolumn{2}{c}{1} & \multicolumn{2}{c}{1} & \multicolumn{2}{c}{1} \\
%     140  & \multicolumn{2}{c}{1} & \multicolumn{2}{c}{1} & \multicolumn{2}{c}{1} & \multicolumn{2}{c}{1} & \multicolumn{2}{c}{1} & \multicolumn{2}{c}{1} \\
%     240  & \multicolumn{2}{c}{1} & \multicolumn{2}{c}{1} & \multicolumn{2}{c}{1} & \multicolumn{2}{c}{1} & \multicolumn{2}{c}{1} & \multicolumn{2}{c}{1} \\
%     \hline
%     \multicolumn{13}{c}{Albedo}\\
%     \hline
%     1.25  & \multicolumn{2}{c}{1} & \multicolumn{2}{c}{1} & \multicolumn{2}{c}{1} & \multicolumn{2}{c}{1} & \multicolumn{2}{c}{1} & \multicolumn{2}{c}{1} \\
%     2.2  & \multicolumn{2}{c}{1} & \multicolumn{2}{c}{1} & \multicolumn{2}{c}{1} & \multicolumn{2}{c}{1} & \multicolumn{2}{c}{1} & \multicolumn{2}{c}{1} \\
%     3.5  & \multicolumn{2}{c}{1} & \multicolumn{2}{c}{1} & \multicolumn{2}{c}{1} & \multicolumn{2}{c}{1} & \multicolumn{2}{c}{1} & \multicolumn{2}{c}{1} \\
%     \end{tabular}
%     \end{center}
% \end{table*}


% \clearpage
% \section{Extended zodiacal light models}

% \subsection{Generalized K98 modelling}

% \subsection{RRM modelling}

% %\subsection{Interplanetary dust model}
% Describe the model components used and tabulate all parameters fit.

% %\subsection{Spectral parameters (emissivities / albedos)}

% %\subsection{Zodi subtracted DIRBE maps and timestreams}

\section{Conclusions}


\begin{acknowledgements}
 The current work has received funding from the European
  Union’s Horizon 2020 research and innovation programme under grant
  agreement numbers 819478 (ERC; \textsc{Cosmoglobe}) and 772253 (ERC;
  \textsc{bits2cosmology}). Some of the results in this paper have been derived using the HEALPix \citep{HEALPIX} package.
  We acknowledge the use of the Legacy Archive for Microwave Background Data
  Analysis (LAMBDA), part of the High Energy Astrophysics Science Archive Center
  (HEASARC). HEASARC/LAMBDA is a service of the Astrophysics Science Division at
  the NASA Goddard Space Flight Center.  
\end{acknowledgements}


%-------------------------------------------------------------
%                                       Table with references 
%-------------------------------------------------------------
%

\bibliographystyle{aa}
\bibliography{references}
\end{document}
%%%% End of aa.dem
