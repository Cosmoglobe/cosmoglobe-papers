%                                                                 aa.dem
% AA vers. 9.1, LaTeX class for Astronomy & Astrophysics
% demonstration file
%                                                       (c) EDP Sciences
%-----------------------------------------------------------------------
%
% \documentclass[referee]{aa} % for a referee version
%\documentclass[onecolumn]{aa} % for a paper on 1 column  
%\documentclass[longauth]{aa} % for the long lists of affiliations 
%\documentclass[letter]{aa} % for the letters 
%\documentclass[bibyear]{aa} % if the references are not structured 
%                              according to the author-year natbib style

%

\documentclass[twocolumn]{aa}  

%
\usepackage{graphicx}
\usepackage{amsmath,amsfonts,amssymb}
\usepackage{natbib}
\usepackage{tabularx}
\usepackage{collcell}
\usepackage{array}
\usepackage{booktabs}
\usepackage{subfigure}
%%%%%%%%%%%%%%%%%%%%%%%%%%%%%%%%%%%%%%%%
\usepackage{txfonts}
\usepackage{xcolor}
\usepackage{blindtext}
%%%%%%%%%%%%%%%%%%%%%%%%%%%%%%%%%%%%%%%%
% \usepackage[options]{hyperref}
% To add links in your PDF file, use the package "hyperref"
% with options according to your LaTeX or PDFLaTeX drivers.
\usepackage{float}
%\usepackage{stfloats}
\usepackage{dblfloatfix}
\usepackage{afterpage}
\usepackage{ifthen}
\usepackage[morefloats=12]{morefloats}
\usepackage{tabularx}
\usepackage{placeins}
\usepackage{multicol}
%\usepackage[breaklinks,colorlinks,citecolor=blue]{hyperref}
\bibpunct{(}{)}{;}{a}{}{,}
\usepackage[switch]{lineno}
\definecolor{linkcolor}{rgb}{0.6,0,0}
\definecolor{citecolor}{rgb}{0,0,0.75}
\definecolor{urlcolor}{rgb}{0.12,0.46,0.7}
\usepackage[breaklinks, colorlinks, urlcolor=urlcolor,
linkcolor=linkcolor,citecolor=citecolor,pdfencoding=auto]{hyperref}
\hypersetup{linktocpage}
\usepackage{bold-extra}
\usepackage{ifthen}


\def\setsymbol#1#2{\expandafter\def\csname #1\endcsname{#2}}
\def\getsymbol#1{\csname #1\endcsname}

\def\Planck{\textit{Planck}}

\def\HeJT{$^4$He-JT}

\def\allearlypapers{\nocite{planck2011-1.1, planck2011-1.3, planck2011-1.4, planck2011-1.5, planck2011-1.6, planck2011-1.7, planck2011-1.10, planck2011-1.10sup, planck2011-5.1a, planck2011-5.1b, planck2011-5.2a, planck2011-5.2b, planck2011-5.2c, planck2011-6.1, planck2011-6.2, planck2011-6.3a, planck2011-6.4a, planck2011-6.4b, planck2011-6.6, planck2011-7.0, planck2011-7.2, planck2011-7.3, planck2011-7.7a, planck2011-7.7b, planck2011-7.12, planck2011-7.13}}

\def\alltwentythirteenresultspapers{\nocite{planck2013-p01, planck2013-p02, planck2013-p02a, planck2013-p02d, planck2013-p02b, planck2013-p03, planck2013-p03c, planck2013-p03f, planck2013-p03d, planck2013-p03e, planck2013-p01a, planck2013-p06, planck2013-p03a, planck2013-pip88, planck2013-p08, planck2013-p11, planck2013-p12, planck2013-p13, planck2013-p14, planck2013-p15, planck2013-p05b, planck2013-p17, planck2013-p09, planck2013-p09a, planck2013-p20, planck2013-p19, planck2013-pipaberration, planck2013-p05, planck2013-p05a, planck2013-pip56, planck2013-p06b, planck2013-p01a}}

\def\alltwentyfifteenresultspapers{\nocite{planck2014-a01, planck2014-a03, planck2014-a04, planck2014-a05, planck2014-a06, planck2014-a07, planck2014-a08, planck2014-a09, planck2014-a11, planck2014-a12, planck2014-a13, planck2014-a14, planck2014-a15, planck2014-a16, planck2014-a17, planck2014-a18, planck2014-a19, planck2014-a20, planck2014-a22, planck2014-a24, planck2014-a26, planck2014-a28, planck2014-a29, planck2014-a30, planck2014-a31, planck2014-a35, planck2014-a36, planck2014-a37, planck2014-ES}}

\newbox\tablebox    \newdimen\tablewidth
\def\leaderfil{\leaders\hbox to 5pt{\hss.\hss}\hfil}
\def\endPlancktable{\tablewidth=\columnwidth 
    $$\hss\copy\tablebox\hss$$
    \vskip-\lastskip\vskip -2pt}
\def\endPlancktablewide{\tablewidth=\textwidth 
    $$\hss\copy\tablebox\hss$$
    \vskip-\lastskip\vskip -2pt}
\def\tablenote#1 #2\par{\begingroup \parindent=0.8em
    \abovedisplayshortskip=0pt\belowdisplayshortskip=0pt
    \noindent
    $$\hss\vbox{\hsize\tablewidth \hangindent=\parindent \hangafter=1 \noindent
    \hbox to \parindent{$^#1$\hss}\strut#2\strut\par}\hss$$
    \endgroup}
\def\doubleline{\vskip 3pt\hrule \vskip 1.5pt \hrule \vskip 5pt}

\def\L2{\ifmmode L_2\else $L_2$\fi}
\def\dtt{\Delta T/T}
\def\DeltaT{\ifmmode \Delta T\else $\Delta T$\fi}
\def\deltat{\ifmmode \Delta t\else $\Delta t$\fi}
\def\fknee{\ifmmode f_{\rm knee}\else $f_{\rm knee}$\fi}
\def\Fmax{\ifmmode F_{\rm max}\else $F_{\rm max}$\fi}
\def\solar{\ifmmode{\rm M}_{\mathord\odot}\else${\rm M}_{\mathord\odot}$\fi}
\def\Msolar{\ifmmode{\rm M}_{\mathord\odot}\else${\rm M}_{\mathord\odot}$\fi}
\def\Lsolar{\ifmmode{\rm L}_{\mathord\odot}\else${\rm L}_{\mathord\odot}$\fi}
\def\inv{\ifmmode^{-1}\else$^{-1}$\fi}
\def\mo{\ifmmode^{-1}\else$^{-1}$\fi}
\def\sup#1{\ifmmode ^{\rm #1}\else $^{\rm #1}$\fi}
\def\expo#1{\ifmmode \times 10^{#1}\else $\times 10^{#1}$\fi}
\def\,{\thinspace}
\def\lsim{\mathrel{\raise .4ex\hbox{\rlap{$<$}\lower 1.2ex\hbox{$\sim$}}}}
\def\gsim{\mathrel{\raise .4ex\hbox{\rlap{$>$}\lower 1.2ex\hbox{$\sim$}}}}
\let\lea=\lsim
\let\gea=\gsim
\def\simprop{\mathrel{\raise .4ex\hbox{\rlap{$\propto$}\lower 1.2ex\hbox{$\sim$}}}}
\def\deg{\ifmmode^\circ\else$^\circ$\fi}
\def\pdeg{\ifmmode $\setbox0=\hbox{$^{\circ}$}\rlap{\hskip.11\wd0 .}$^{\circ}
          \else \setbox0=\hbox{$^{\circ}$}\rlap{\hskip.11\wd0 .}$^{\circ}$\fi}
\def\arcs{\ifmmode {^{\scriptstyle\prime\prime}}
          \else $^{\scriptstyle\prime\prime}$\fi}
\def\arcm{\ifmmode {^{\scriptstyle\prime}}
          \else $^{\scriptstyle\prime}$\fi}
\newdimen\sa  \newdimen\sb
\def\parcs{\sa=.07em \sb=.03em
     \ifmmode \hbox{\rlap{.}}^{\scriptstyle\prime\kern -\sb\prime}\hbox{\kern -\sa}
     \else \rlap{.}$^{\scriptstyle\prime\kern -\sb\prime}$\kern -\sa\fi}
\def\parcm{\sa=.08em \sb=.03em
     \ifmmode \hbox{\rlap{.}\kern\sa}^{\scriptstyle\prime}\hbox{\kern-\sb}
     \else \rlap{.}\kern\sa$^{\scriptstyle\prime}$\kern-\sb\fi}
\def\ra[#1 #2 #3.#4]{#1\sup{h}#2\sup{m}#3\sup{s}\llap.#4}
\def\dec[#1 #2 #3.#4]{#1\deg#2\arcm#3\arcs\llap.#4}
\def\deco[#1 #2 #3]{#1\deg#2\arcm#3\arcs}
\def\rra[#1 #2]{#1\sup{h}#2\sup{m}}
\def\page{\vfill\eject}
\def\dots{\relax\ifmmode \ldots\else $\ldots$\fi}
\def\WHzsr{\ifmmode $W\,Hz\mo\,sr\mo$\else W\,Hz\mo\,sr\mo\fi}
\def\mHz{\ifmmode $\,mHz$\else \,mHz\fi}
\def\GHz{\ifmmode $\,GHz$\else \,GHz\fi}
\def\mKs{\ifmmode $\,mK\,s$^{1/2}\else \,mK\,s$^{1/2}$\fi}
\def\muKs{\ifmmode \,\mu$K\,s$^{1/2}\else \,$\mu$K\,s$^{1/2}$\fi}
\def\muKRJs{\ifmmode \,\mu$K$_{\rm RJ}$\,s$^{1/2}\else \,$\mu$K$_{\rm RJ}$\,s$^{1/2}$\fi}
\def\muKHz{\ifmmode \,\mu$K\,Hz$^{-1/2}\else \,$\mu$K\,Hz$^{-1/2}$\fi}
\def\MJysr{\ifmmode \,$MJy\,sr\mo$\else \,MJy\,sr\mo\fi}
\def\MJysrmK{\ifmmode \,$MJy\,sr\mo$\,mK$_{\rm CMB}\mo\else \,MJy\,sr\mo\,mK$_{\rm CMB}\mo$\fi}
\def\microns{\ifmmode \,\mu$m$\else \,$\mu$m\fi}
\def\micron{\microns}
\def\muK{\ifmmode \,\mu$K$\else \,$\mu$\hbox{K}\fi}
\def\microK{\ifmmode \,\mu$K$\else \,$\mu$\hbox{K}\fi}
\def\muW{\ifmmode \,\mu$W$\else \,$\mu$\hbox{W}\fi}
\def\kms{\ifmmode $\,km\,s$^{-1}\else \,km\,s$^{-1}$\fi}
\def\kmsMpc{\ifmmode $\,\kms\,Mpc\mo$\else \,\kms\,Mpc\mo\fi}

\providecommand{\sorthelp}[1]{}


% Custom definitions
\def\Cosmoglobe{\textsc{Cosmoglobe}}
\def\cosmoglobe{\textsc{Cosmoglobe}}
\def\Planck{\textit{Planck}}

\renewcommand{\a}[0]{\vec{a}}
% \newcommand{\g}[0]{\vec{g}}
\newcommand{\phm}{\phantom{-}}
\newcommand{\dv}[0]{\vec{d}}
\renewcommand{\t}[0]{\vec{t}}
\newcommand{\A}[0]{\tens{A}}
\newcommand{\B}[0]{\tens{B}}
\newcommand{\Y}[0]{\tens{Y}}
\newcommand{\G}[0]{\tens{G}}
\newcommand{\n}[0]{\vec{n}}
\newcommand{\red}[0]{\color{red}}
\newcommand{\green}[0]{\color{green}}
\newcommand{\s}[0]{\vec{s}}
\renewcommand{\a}[0]{\vec{a}}
\newcommand{\m}[0]{\vec{m}}
\newcommand{\bv}[0]{\vec{b}}
\newcommand{\f}[0]{\vec{f}}
\newcommand{\F}[0]{\tens{F}}
\newcommand{\T}[0]{\tens{T}}
\newcommand{\Cp}[0]{\tens{C}}
\renewcommand{\L}[0]{\tens{L}}
\newcommand{\g}[0]{\vec{g}}
\newcommand{\N}[0]{\tens{N}}
\newcommand{\M}[0]{\tens{M}}
\newcommand{\iN}[0]{\tens{N}^{-1}}
\newcommand{\iM}[0]{\tens{M}^{-1}}
\newcommand{\w}[0]{\vec{w}}
\renewcommand{\S}[0]{\tens{S}}
\renewcommand{\r}[0]{\vec{r}}
\renewcommand{\u}[0]{\vec{u}}
\newcommand{\q}[0]{\vec{q}}
\renewcommand{\v}[0]{\vec{v}}
\renewcommand{\P}[0]{\tens{P}}
\newcommand{\dt}[0]{d_t}
\newcommand{\di}[0]{d_i}
\newcommand{\nt}[0]{n_t}
\newcommand{\st}[0]{s_t}
\newcommand{\mt}[0]{m_t}
\newcommand{\ft}[0]{f_t}
\newcommand{\Te}[0]{T_{\rm e}}
\newcommand{\EM}[0]{\rm EM}
\newcommand{\mathsc}[1]{{\normalfont\textsc{#1}}}
\newcommand{\hi}{\ensuremath{\mathsc {Hi}}}
\newcommand{\bpbold}{\bfseries{\scshape{BeyondPlanck}}}
\newcommand{\BP}{\textsc{BeyondPlanck}}
\newcommand{\bp}{\textsc{BeyondPlanck}}
\newcommand{\lfi}[0]{LFI}
\newcommand{\hfi}[0]{HFI}
\newcommand{\npipe}[0]{\texttt{NPIPE}}
\newcommand{\K}[0]{\textit K}
\newcommand{\Ka}[0]{\textit{Ka}}
\newcommand{\Q}[0]{\textit Q}
\newcommand{\V}[0]{\textit V}
\newcommand{\W}[0]{\textit W}
\newcommand{\e}{\mathrm e}
\newcommand{\cvar}{\ensuremath{c(\vartheta, \varphi, \psi)}}


% \renewcommand{\topfraction}{1.0}	% max fraction of floats at top
%     \renewcommand{\bottomfraction}{1.0}	% max fraction of floats at bottom
%     %   Parameters for TEXT pages (not float pages):
%     \setcounter{topnumber}{2}
%     \setcounter{bottomnumber}{2}
%     \setcounter{totalnumber}{4}     % 2 may work better
%     \setcounter{dbltopnumber}{2}    % for 2-column pages
%     \renewcommand{\dbltopfraction}{0.9}	% fit big float above 2-col. text
%     \renewcommand{\textfraction}{0.04}	% allow minimal text w. figs
%     %   Parameters for FLOAT pages (not text pages):
%     \renewcommand{\floatpagefraction}{0.9}	% require fuller float pages
% 	% N.B.: floatpagefraction MUST be less than topfraction !!
%     \renewcommand{\dblfloatpagefraction}{0.9}	% require fuller float pages



\begin{document} 

   \title{\bfseries{\Cosmoglobe\ DR2. III. Bayesian global modelling of zodiacal light\\ with a first application to \textit{COBE}-DIRBE}}

   %This author list corresponds to \title{Author list for L04\_CMB\_Foregrounds\_Extraction}
%Prepared by M. Lopez-Caniego (Marcos.Lopez.Caniego@sciops.esa.int), ESAC/ESA
%This version is from Thu Jul 12 18:11:48 2018 CET
%\subtitle{There are 152 co-authors in this list}
\newcommand{\oslo}[0]{1}
%\newcommand{\MIT}[0]{2}
\newcommand{\milanoA}[0]{2}
\newcommand{\milanoB}[0]{3}
\newcommand{\milanoC}[0]{4}
\newcommand{\triesteB}[0]{5}
\newcommand{\planetek}[0]{6}
\newcommand{\princeton}[0]{7}
\newcommand{\jpl}[0]{8}
\newcommand{\helsinkiA}[0]{9}
\newcommand{\helsinkiB}[0]{10}
\newcommand{\nersc}[0]{11}
\newcommand{\haverford}[0]{12}
\newcommand{\mpa}[0]{13}
\newcommand{\triesteA}[0]{14}
\newcommand{\iia}[0]{2}

\author{\small
J.~R.~Eskilt\inst{\oslo}\thanks{Corresponding author: J.~R.~Eskilt; \url{j.r.eskilt@astro.uio.no}}
\and
K.~Lee\inst{\oslo}
\and
D.~J.~Watts\inst{\oslo}
\and
S.~Nerval\inst{\oslo}
\and
et al.
}
\institute{\small
        Institute of Theoretical Astrophysics, University of Oslo, Blindern, Oslo, Norway \goodbreak
}


   \institute{Institute of Theoretical Astrophysics, University of Oslo, Blindern, Oslo, Norway}
  
   % Shortened title, author list for top of page 
   \titlerunning{\Cosmoglobe: Interplanetary dust}
   \authorrunning{M.~San et al.}

   \date{\today}
   

% write an abstract 

\abstract{We present the first Bayesian framework for global modelling 
of zodiacal light and its application to the Diffuse Infrared Background 
Experiment (DIRBE) time-ordered data (TOD). The framework uses a 
modified version of the COBE/DIRBE zodiacal light model to estimate the 
zodiacal light along the observed line-of-sights in the time domain. We 
obtain a new state-of-the-art zodiacal light model by re-estimating the 
free parameters in the COBE/DIRBE model and produce DIRBE zodiacal light 
subtracted mission average (ZSMA) maps with much smaller zodiacal light 
residuals. We argue that the improved zodiacal light model fit becomes 
possible through global Bayesian end-to-end analysis of the DIRBE TODs 
with COBE-FIRAS, GAIA, Planck HFI, and WISE observations, giving us a 
more accurate instrumental and astophysical characterization of the 
infrared sky. However, we do note that even though the zodiacal light 
model represent an improvement with respect to original DIRBE model when 
it comes to removing ZL from the DIRBE data, this is still a preliminary 
model. We illustrate the potential of this framework for building 
zodiacal models by conducting a small study where we further extend and 
modifcations to the zodiacal light model by adding more asteroidal 
bands, interstellar dust and using a more physically motivated 
parameterization of the interplanetary medium as described in 
\cite{Robinson2013}. A more modern and physical parametrization of the 
interplanetary medium in combination with joint analysis with 
complementry infrared experiments such IRAS, AKARI, WISE and SPHEREx 
will 
}

   \keywords{Zodiacal dust, Interplanetary medium, Cosmology: cosmic background radiation}

   \maketitle

\setcounter{tocdepth}{2}
\tableofcontents
   

\section{Introduction}
Zodiacal light (ZL, sometimes zodiacal emission or interplanetary dust 
emission) is the primary source of diffuse radiation observed in the 
infrared sky between 1-100 $\mu$m (\cite{Leinert1998} and references 
therein). The light comes from scattering and re-emission of sunlight 
from interplanetary dust (IPD) grains. The inner Solar system is 
embedded in a sun-centered cloud of IPD, with a symmetry axis tilted 
slightly with respect to the Ecliptic, known as the zodiacal cloud. The 
ZL is seasonal, and its appearance in the sky changes as the Earth moves 
through the IPD distribution. The most common way to model the observer 
position-dependent ZL is to evaluate a line-of-sight integral for each 
observation directly in the time-ordered domain. The time-varying and 
three-dimensional nature of the ZL makes it one of the most challenging 
foregrounds to model in astrophysical and cosmological studies of the 
infrared sky. The lack of a good ZL model has left a vast portion of the 
electromagnetic spectrum inaccessible to cosmological analysis 
attempting to measure the Cosmic Optical and Infrared Backgrounds (COB, 
CIB). 

The state-of-the-art ZL model in the field of cosmology is the 
\cite{Kelsall1998} model (K98, sometimes referred to as the 
\textit{COBE}/DIRBE or just the DIRBE model). K98 is a parametric ZL 
model describing the three-dimensional IPD distribution and the 
radiative properties of the dust. The model was developed by the 
\textit{COBE}/DIRBE team to remove ZL from their data in the late 1900s.
 Since then, our understanding of the infrared sky has improved with new 
 observational data from experiments like Planck HFI, GAIA, and WISE. 
 Combining recent computational advances in Bayesian cosmological 
 analysis \citep{BP2023, Galloway2023, Watts2023}, more readily 
 available computing power and new complementary infrared datasets 
 allow us to extract significantly more information about the true 
 nature of the ZL from the DIRBE data than was possible at the time 
 of the original DIRBE analysis.

This paper is the third in a series of six describing the methods and 
results of our re-analysis of the time-ordered \textit{COBE}/DIRBE data 
within a global end-to-end Bayesian analysis framework. Here, we 
describe the ZL modeling approach in Commander3 and demonstrate the 
benefits of fitting a ZL model directly within a global end-to-end 
Bayesian framework. The basis of the newly implemented ZL module in 
Commander3 is ZodiPy \citep{San2024}, which is an Astropy-affiliated 
Python package for ZL simulations. The development of ZodiPy was 
motivated by two factors. It would serve as a basis for the code 
development in Commander3, and secondly, it would provide the 
astrophysical and cosmology communities with an accessible, easy-to-use 
interface to existing and coming ZL models. As an early application, 
\cite{San2022} demonstrates the removal of ZL from the DIRBE TOD with 
ZodiPy using the K98 model.

The following sections are organized as follows. In Sect.~\ref{sect:zodi-model}
, we introduce the K98 ZL model and discuss 
implementation details and some optimizations regarding evaluating ZL 
models on HEALPix grids. Next, in Sect.~\ref{sect:param-estimation}, we 
discuss ZL parameter estimation and our implementation, which involves 
the use of a function minimizer on a sub-sample of the DIRBE TOD. 
Finally, in Sect.~\ref{sect:improved-model}, we present a greatly 
improved ZL model, yielding lower residuals at all ten DIRBE channels. 
The improved model uses only a slightly different parametrization than 
the K98 model. We recommend using this model over K98 for all infrared 
community members.


\section{Zodiacal light modelling}\label{sect:zodi-model}
ZL is commonly removed directly from timestreams by performing 
line-of-sight integration at each observation through a parametric 
three-dimensional model of the IPD distribution. We use a slightly 
modified version of the K98 ZL model in Commander3. Below we present a 
short introduction to the model parametrization in terms of the 
three-dimensional IPD distribution and the radiative and scattering p
roperties of the IPD. We refer to~\cite{Kelsall1998} for a more detailed 
introduction to the K98 ZL model.


\subsection{Parameterization of interplanetary dust}
The IPD in the zodiacal cloud is smooth and stable, and most of the dust 
is accounted for by a diffuse cloud-like component \citep{Leinert1998}. 
The majority of the dust in this cloud component stems from the 
mass-shedding of Jupiter family comets with low eccentricities and 
inclinations with respect to the ecliptic. However, fine structures 
within the zodiacal cloud exist as a result of collisions and 
fragmentation in asteroids and gravitational resonance in the orbit of 
planets \citep{Low1984, Dermott1984, Dermott1994, Reach1997}. 

We model the IPD distribution as a combination of several zodiacal 
components, denoted by $c$, each described by a heliocentric ecliptic 
number density $n_c(x,y,z)$. Each zodiacal component is allowed to have 
a heliocentric offset $(x_{0,c}, y_{0,c}, z_{0,c})$, such that the 
component-centric coordinates become
\begin{equation}    
    \begin{aligned}
        x_c&= x - x_{0,c}\\
        y_c&= y - y_{0,c}\\
        z_c&= z - z_{0,c}.
    \end{aligned}
\end{equation}
Additionally, each zodiacal component is allowed to have a plane of 
symmetry different from the ecliptic, given by an inclination $i_c$ and 
ascending node $\Omega_c$. In component-centric coordinates, the number 
density is then fully described by the radial distance $r_c$ from the 
origin and the height above the symmetry plane $Z_c$
\begin{align}
    r_c &= \sqrt{x_c^2 + y_c^2 + z_c^2},\\
    Z_c &= x_c\sin{\Omega_c}\sin{i_c} - y_c \cos{\Omega_c}\sin{i_c} + z_c \cos{i_c}.
\end{align}


\subsection{Zodiacal components}
\subsubsection{Smooth cloud}
The cloud component represents the smooth IPD distribution that embeds 
the inner Solar system. Its number density is modeled as
\begin{equation}
    n_\mathrm{C}(x,y,z)=n_{0, \mathrm{C}}r_\mathrm{C}^{-\alpha}f(\zeta_\mathrm{C}),
\end{equation}
where $n_{0, \mathrm{C}}$ is the number density at 1 AU, $\alpha$ is a 
power-law index, $f(\zeta_\mathrm{C})$ is the fan-like vertical 
distribution given as 
$f(\zeta_\mathrm{C}) = \exp {\left[-\beta g^\gamma \right]}$, where 
$\zeta_\mathrm{c} = |Z_\mathrm{c}|/r_\mathrm{c}$ is the radial height 
above the symmetry plane, 
\begin{equation}
    g = \begin{cases}
        \zeta^2/2\mu & \mathrm{for}\; \zeta < \mu,\\
        \zeta - \mu/2 & \mathrm{for}\; \zeta \geq \mu,
    \end{cases}
\end{equation}
and  $\beta$, $\gamma$ and $\mu$ are shape parameters.

\subsubsection{Dust bands}
Three dust bands are included in the model to represent the observed 
shoulder-like structure in the IRAS scans across the ecliptic plane. 
These bands appear at ecliptic latitudes of 
$\pm \sim 1.4^\circ$, $10^\circ$, and $15^\circ$, and are associated 
with the Themis, Koronis, and Eos asteroid families, respectively. Dust 
bands are modeled as
\begin{align}
    n_{\mathrm{B}_i}(x,y,z) &= \frac{3 n_{0, \mathrm{B}_i}}{r_{\mathrm{B}_i}} \exp \left[-\left(\frac{\zeta_{\mathrm{B}_i}}{\delta_{\zeta_{\mathrm{B}_i}}}\right)^{6}\right]\left[1 + \left(\frac{\zeta_{\mathrm{B}_i}}{\delta_{\zeta_{\mathrm{B}_i}}}\right)^{p}v^{-1}\right] \\
    &\times\left\{1-\exp \left[-\left(\frac{r_{\mathrm{B}_i}}{\delta_{r_{\mathrm{B}_i}}}\right)^{20}\right]\right\},
\end{align}
where $n_{0, \mathrm{B}_i}$ is the number density of band $\mathrm{B}_i$ 
at 3 AU, $\delta_{r_{\mathrm{B}_i}}$ is the inner radial cut-off, and 
$p$, $v$ and $\delta_{\zeta_{\mathrm{B}_i}}$ are shape parameters.
\subsubsection{Circum-solar ring and Earth-trailing feature}
A circum-solar ring component is included in the model to represent dust 
that has accumulated in Earth's orbit due to gravitational effects. This 
component also includes an enhancement to the IPD distribution at 
Earth's wake, known as the Earth-trailing feature. The ring component is 
modeled as
\begin{align}
    n_\mathrm{R}(x, y, z, \theta)&=n_{0, \mathrm{SR}} \exp \left[-\frac{\left(r_\mathrm{R}-r_{0, \mathrm{SR}}\right)^2}{\sigma_{R,\mathrm{SR}} ^2}-\frac{\left| Z_\mathrm{R} \right|}{\sigma_{Z, \mathrm{SR}}}\right],\\
   &+ n_{0, \mathrm{TF}} \exp \left[-\frac{\left(r_\mathrm{R}-r_{0, \mathrm{TF}}\right)^{2}}{\sigma_{R, \mathrm{TF}}^{2}}-\frac{\left|Z_\mathrm{F}\right|}{\sigma_{Z, \mathrm{TF}}}-\frac{\left(\theta-\theta_{0, \mathrm{TF}}\right)^{2}}{\sigma_{\theta,\mathrm{TF}}^{2}}\right],
\end{align}
where $\theta$ is the heliocentric longitude of the Earth, and the 
radial locations $r_{0, \mathrm{SR}}$, $r_{0, \mathrm{TF}}$ specifies 
the distances to the peak densities $n_{0, \mathrm{SR}}$, 
$n_{0, \mathrm{TF}}$. The $\sigma$ parameters are length scales for the 
$r$, $Z$ and $\theta$ parameters, respectively. Note that the 
Earth-trailing feature depends on the position of the Earth and does not 
have a plane symmetry like the other zodiacal components. 

\begin{figure}
    \centering
    \includegraphics[width=\linewidth]{figs/powell_red_chisq_vs_iter.pdf}
    \caption{Reduced $\chi^2$ as a function of Powell likelihood evaluation count for one single Gibbs chain. Each discrete jump indicates the start of a new Gibbs sample, which is initialized on a new random point that is close to the previous iteration. The following systematic decline within each main Gibbs iteration indicates the non-linear optimization performed by the Powell algorithm.  The solid dark region corresponds to a large number of highly sub-optimal parameter trials. }
    \label{fig:powell_chisq_iter}
\end{figure}


\begin{figure}
    \centering
    \includegraphics[width=\linewidth]{figs/powell_T0_vs_chisq.pdf}
    \caption{Reduced $\chi^2$ as a function of the temperature at 1\,AU, $T_0$. Each curve shows the full set of parameter trials within one single main Gibbs iteration (or Powell call), and different colors indicate different Gibbs iteration. Redder colors are earlier in the chain.}
    \label{fig:powell_T0}
\end{figure}


\subsection{Radiative and scattering properties}
Thermal emission from IPD grains is modeled on the form of a blackbody
 modified by an emissivity factor $E_{c, \lambda}$
\begin{equation}
    I^\mathrm{Thermal}_{c,\lambda} = E_{c,\lambda} B_\lambda(T),
\end{equation}
where $B_\lambda$ is the Planck function at a wavelength $\lambda$. The 
temperature $T$ of the IPD falls off with radial distance $r$ from the 
Sun as
\begin{equation}
    T(r) = T_0 r^{-\delta},
\end{equation}
where $T_0$ is the temperature of IPD at 1 AU and $\delta$ the power law 
index. In addition to emitting thermally, IPD grains also scatter 
sunlight in the near-infrared. The contribution to the total signal from 
scattering is modeled as
\begin{equation}\label{eq: scat_term}
    I^\mathrm{Scattering}_{c, \lambda} = A_{c, \lambda} F_\lambda^\odot(r) \Phi_\lambda(\Theta),
\end{equation}
where $A_{c, \lambda}$ is the albedo, or reflectivity of the IPD of a 
given zodiacal component, $F_\lambda^\odot(r)$ the solar flux at a 
radial distance from the Sun, and $\Phi_\lambda(\Theta)$ is the phase 
function for scattering angles $\Theta$, which describes the angular 
distribution of the scattered light.

The total intensity from a single IPD grain is
\begin{align}\label{eq:I_tot}
    I^\mathrm{Total}_{c, \lambda} &= I^\mathrm{Scattering}_{c,\lambda} + I^\mathrm{Thermal}_{c,\lambda}\\
    &= A_{c, \lambda} F_\lambda^\odot \Phi_\lambda + E_{c,\lambda} B_\lambda.
\end{align}

\subsection{Evaluating the zodiacal model}
The ZL model is evaluated by solving the following line-of-sight 
integral for each observation in the timestream
\begin{equation}\label{eq:los}
    I_{p,t} = \sum_c \int n_c \left[  A_{c, \lambda} F_\lambda^\odot \Phi_\lambda + \left( 1 - A_{c, \lambda} \right) E_{c,\lambda} B_\lambda \right]\,\mathrm {ds}.
\end{equation}
Here, $p$ is a point in the sky, $t$ is the time of observation, and ds 
is a small distance along the line-of-sight $s$ from the observer and 
towards $p$, and $n_c$ is the number density of component $c$ in the 
line-of-sight. Note that there is a factor (1 - $A_{c, \lambda}$) 
difference between equation~\eqref{eq:los} and~\eqref{eq:I_tot}. This 
factor represents a form of extinction where thermal emission is 
sometimes scattered away from the line-of-sight. 


\subsection{Optimizing model evaluations}\label{sect:optimization}
The line-of-sight evaluations are an expensive part of the analysis 
pipeline and significantly increase the time that it takes to complete a 
single Gibbs sample. Additionally, when estimating ZL parameters we need 
to make a pass over the entire data set for each proposal. The DIRBE 
data set is very light when compared with modern observations, meaning 
that we can get away with some additional computational time. But for 
future data releases with AKARI or SPHEREx data, this will become 
computationally unfeasible. As such, we have implemented a few 
optimizations regarding ZL evaluation. 
\begin{figure}
    \includegraphics[width=\columnwidth]{figs/cache_error_delta_t.pdf}
    \includegraphics[width=\columnwidth]{figs/cache_error_z.pdf}
    \caption{Errors induced by the ZL cache. \textit{(top):} Difference between two 
    instantaneous view of the ZL 900 km above Earth's north and south 
    poles. \textit{(bottom):} Difference between two instantaneous views of the ZL at $N_\mathrm{studies}=512$, 
    separated by 2 hours. Pixels with an angular separation $\leq 60^\circ$ from the Sun 
    are masked in both maps.}
    \label{fig:cache-error}
\end{figure}
The main optimization comes from the implementation of a ZL cache. The 
cache takes advantage of the smoothly varying nature of the ZL, and lets 
us reuse computed ZL estimates for proximate re-observations. 
Internally, Commander3 works with instrument pointing on the form of 
discrete pixel indices, meaning that re-observerations of a given pixel, 
within a short timeframe $\Delta t$, can be reused
\begin{equation}
    s_\mathrm{zodi}(p, t_i) = s_\mathrm{zodi}(p, t_j).
\end{equation}
The error induces from this cache is directly proportional to the 
distance that the observer has moved through the $\Delta t$ parameter. 
As a reference, the ZL moves by about $1^\circ$ on the sky each day as 
the Earth orbits the Sun. In addition to the positional change along 
Earth's orbit, the COBE satellite also orbits the Earth at an altitude 
of 900 km. This motion is lost when using the cache. These magnitude of 
the error induced by the cache is illustrated in 
Figure~\ref{fig:cache-error}.  We found that resetting the cache every 
2 hours resulted in a good compromise between speed and accuracy. Note 
that a cache on this form is only efficient for experiments with scanning 
strategies with partially overlapping subsequent scans. The magnitude of
the errors are well below the the residuals, justifying the use of the 
cache.

The smooth nature of the ZL can be exploited in more ways. When sampling 
ZL parameters is is possible to use a subsample of the full TOD while 
still resolving the smooth ZL structure. In our analysis we have used a 
thinning factor of 8 meaning that we fit the data to a timestream with 1 
Hz. Another optimization comes in form of spatial downsampling, where 
the smoothness of the ZL is resolved adequatly at a HEALPix resolution 
of $n_\mathrm{SIDE} = 64$ down from the original 512, further reducing 
the volume of the data used in the fits by a factor of 64. The 
downsampling is simulated by projecting all the original pointing to the 
lower HEALPix resolution, and using a corresponding low-res cache.


\section{Bayesian zodiacal light parameter estimation}\label{sect:param-estimation}
Estimating the ZL model parameters happens within the Bayesian end-to-end Commander3 framework. An explaination of the algorithms used to fit the full data and sky model, as well as instrumental parameters are described in detail in \cite{CG02_01}. In short, we use Gibbs sampling to map out the joint posterior distribution\citep{Galloway2023}, which allows us to draw samples iteratively from various conditional distributions rather than trying to draw directly from a large and complicated joint distribution. The following equations show each parameter sampled in this Cosmoglobe analysis in order
\begin{alignat}{11}
    \G &\,\leftarrow P(\G&\,\mid &\,\dv,&\, &\,\phantom{\G} &\,\xi_n, &
    \,\beta_{\mathrm{sky}}& \,\a_{\mathrm{sky}}, &\,\zeta_{\mathrm{z}},
    &\,\a_{\mathrm{static}})\label{eq:gibbs_G}\\
    \xi_{\mathrm{n}} &\,\leftarrow P(\xi_{\mathrm{n}}&\,\mid &\,\dv,&\, &\,\G, &\,\phantom{\xi_n} &
    \,\beta_{\mathrm{sky}}& \,\a_{\mathrm{sky}}, &\,\zeta_{\mathrm{z}},
    &\,\a_{\mathrm{static}})\\
    \beta_{\mathrm{sky}} &\,\leftarrow P(\beta_{\mathrm{sky}}&\,\mid &\,\dv,&\, &\,\G, &\,\xi_n, &
    \,\phantom{\beta_{\mathrm{sky}}}& \,\a_{\mathrm{sky}}, &\,\zeta_{\mathrm{z}}, &\,\a_{\mathrm{static}})\\
    \a_{\mathrm{sky}} &\,\leftarrow P(\a_{\mathrm{sky}}&\,\mid &\,\dv,&\, &\,\G, &\,\xi_n, &
    \,\beta_{\mathrm{sky}},& \,\phantom{\a_{\mathrm{sky}},}
    &\,\zeta_{\mathrm{z}}, &\,\a_{\mathrm{static}})\\
    \zeta_{\mathrm{z}} &\,\leftarrow P(\zeta_{\mathrm{z}}&\,\mid &\,\dv,&\, &\,\G, &\,\xi_n, &
    \,\beta_{\mathrm{sky}},& \,\a_{\mathrm{sky}},
    &\,\phantom{\zeta_{\mathrm{z}},} &\,\a_{\mathrm{static}})\label{eq:gibbs_zodi}\\
    \a_{\mathrm{static}} &\,\leftarrow P(\a_{\mathrm{static}}&\,\mid &\,\dv,&\, &\,\G, &\,\xi_n, &
    \,\beta_{\mathrm{sky}},& \,\a_{\mathrm{sky}}, &\,\zeta_{\mathrm{z}} &\,\phantom{\a_{\mathrm{static}}})\label{eq:gibbs_static}.
\end{alignat}
Each of the sampling steps in the above Gibbs chain is described by \citet{CG02_01}
and references therin. 

For computational reasons it is sometimes necessary to violate the Gibbs rule,
which is the case for the step in Equation~\ref{eq:gibbs_zodi}, which describes
the sampling of the ZL parameters, denoted by $\zeta_\mathrm{z}$. The K98 ZL 
model contains around 50 parameters to describe the three-dimensional IPD 
distribution and another 30 or more to describe the thermal emission and 
scattered sunlight source functions, many of which are heavily degenerate
due to the difficulties of mapping a three-dimensional structure from a single
data set. N of these parameters were fit to the DIRBE data, in the original DIRBE analysis while the rest were fixed to parameters motivated by studies of the interplanetary medium. In our analysis we have fit M total parameters. A visualization of the degeneracies and the way in which each ZL parameter impacts the final mission-averaged predicted ZL are seen in figures~\ref{fig:atlas1} and~\ref{fig:atlas2} in Appendix~\ref{sec:param-atlas}. This atlas of ZL parameters shows normalized maps of the effects of changing a single ZL parameter by $\pm 5\%$ while keeping the rest of the model fixed. The albedo and emissivity parameters are excluded from the atlas as these represent a direct scaling of the component maps and are degenerate with the $n_0$ maps.
The posterior distribution of this complicated set of parameters exhibits a large number of local minima due to the above mentioned reasons. When experimenting we found that strict Gibbs sampling algorithm quickly became trapped. We have opted to instead 
use a simple non-linear Powell algorithm that is initialized some random parameter distance away from the previous sample, and then searches for the local minimum. Powell's method is a function minimizer, which performs bi-directional searches along a parameter vector. This algorithm is able to escape local minima, but it comes at the cost of larger
uncertainties than what would result from an ideal posterior mapper.


\begin{figure*}
    \centering
    \resizebox{\textwidth}{!}{%
    \includegraphics[height=1cm]{figs/week/week_delta_I_nu.pdf}%
    \includegraphics[height=1cm]{figs/week/week_I_nu.pdf}%
    }\\
    \resizebox{\textwidth}{!}{%
    \includegraphics[height=1cm]{figs/week/week_delta_Z_nu.pdf}%
    \includegraphics[height=1cm]{figs/week/week_Z_nu.pdf}%
    }\\
    \resizebox{\textwidth}{!}{%
    \includegraphics[height=1cm]{figs/week/week_delta_tod_nu.pdf}%
    \includegraphics[height=1cm]{figs/week/week_tod_nu.pdf}%
    }\\
    \caption{Illustration of the basic sky maps involved in the zodiacal 
    light fitting algorithms adopted by the DIRBE (\emph{left column}) 
    and \Cosmoglobe\ (\emph{right column}) pipelines for one week of 
    $25\,\mu\mathrm{m}$ observations and adopting the K98 model. The 
    DIRBE pipeline used exclusively differences between weekly and 
    full-season maps, both for the observed signal, 
    $\Delta I_{\nu} \equiv I_{\nu}-\left<I_{\nu}\right>$ (\emph{top left}), 
    and the zodiacal light model, 
    $\Delta Z_{\nu} = Z_{\nu}-\left<Z_{\nu}\right>$ (\emph{middle left}), 
    where brackets indicate full-survey averages. Correspondingly, the 
    final $\chi^2$ is defined through $\Delta I_{\nu} - \Delta Z_{\nu}$ 
    (\emph{bottom left}), and is by constrution only sensitive to 
    time-variable signals. In contrast, the basic data element in 
    \Cosmoglobe\ is the full sky signal, $I_{\nu}$ (\emph{top right}), 
    which is fitted with the full zodiacal light model, $Z_{\nu}$ 
    (\emph{middle right}), both modelled in time-domain. The $\chi^2$ 
    the minimizes minimize the total signal-minus-model residual, 
    $I_{\nu}-Z_{\nu}$ (\emph{bottom right}). The main advantage of the 
    DIRBE approach is insensitivity to stationary sky signals, in 
    particular thermal dust and CIB, while the main advantage of the 
    \Cosmoglobe\ approach is a much higher effective signal-to-noise 
    ratio, both to zodiacal light parameters and zero-levels.}
    \label{fig:week_vs_full}
\end{figure*}
\begin{figure}
    \centering
    \includegraphics[width=\columnwidth]{figs/mask_zodi_fitting.pdf}
    \caption{Masks applied when fitting zodiacal light parameters for 
    the $2.2\mathrm{\mu m}$, $25\mathrm{\mu m}$ and $240\mathrm{\mu m}$ 
    bands.}
    \label{fig:masks}
\end{figure}
The K98 model was obtained by fitting the model to a set of week-maps. 
The $\chi^2$ measure used by the DIRBE team is given by
\begin{equation}
    \chi^2_\mathrm{K98} = \Delta I_\nu - \Delta Z_\nu,
\end{equation}
where $I_\nu$ and $Z_\nu$ are the full-mission frequency and ZL estimate 
maps, and $\Delta I_\nu = I_\nu - \langle I_\nu \rangle$ and $\Delta Z_\nu = Z_\nu - \langle Z_\nu \rangle$ are the difference between 
the full-mission and a week frequency and ZL map, respectively. These 
maps are illustrated in the left column in Figure~\ref{fig:week_vs_full}. 
$\chi^2_\mathrm{K98}$ is by construction only sensitive to time-variable 
signal, effectively removing the need for a good sky model of the then 
less well-understood infrared sky. Using week maps instead of the full 
timestreams has the additional effect of reducing the overall data 
volume in the fits, which would seem beneficial considering the 
computing resources available during the early DIRBE analysis. A 
downside of this method is that the monopole induced by the smooth ZL is 
removed, resulting in a much weaker signal-to-noise ratio in the ZL 
parameter estimates and to the zero-levels. In contrast, we fit the data 
at each time step to the signal minus model residual
\begin{equation}\label{eq:chisq}
    \chi^2 = I_\nu - Z_\nu.
\end{equation}

To minimize the amount of unmodeled galactic emission that enters our 
ZL parameter estimates, we use strict channel-specific processing masks, 
masking out the brightest galactic regions. The 1.25, 25, and 240 $\mu$m 
masks, corresponding to stellar light, ZL, and thermal dust-dominated 
channels, are shown in Figure~\ref{fig:masks}.


\begin{figure*}
    \centering
    \includegraphics[width=1\textwidth]{figs/total_trace.pdf}
    \caption{Trace plot of all geometrical interplanetary dust paramteres for six independant Markov chains.}
    \label{fig:trace-ipd}
\end{figure*}

\begin{figure*}
    \centering
    \includegraphics[width=1\textwidth]{figs/emissivity_and_albedo_trace.pdf}
    \caption{Trace plot of all geometrical interplanetary dust paramteres for six independant Markov chains.}
    \label{fig:trace-emissivity-albedo}
\end{figure*}
The total number of free ZL model parameters in this analysis are N, 
compared to the M parameters fit in K98. We have included a non-zero
scattering contribution at 4.9 $\mu$m. The ZL SED is composition of two
blackbody like spectra with a combined minimum around the $3.5\mu$m band. The first
blackbody spectrum is associated with the solar flux through the scattering 
contribution, while the second with the re-radiated thermal emission. With an
improved sky model and lower residual levels, we believe that it is worth 
including the small non-zero contribution from scattering at 4.9 $\mu$m.
Additionally, we are using the 12 $\mu$m band as our ZL reference band, meaning
that we fix the emissivities at this band to 1 \textcolor{red}{(why?)}. The
reference ZL band in K98 is $25 \mu$m. Similar to K98, we fit one emissivity 
for the smooth cloud, and one joint emissivity for the dust bands, while we 
only fit one overall albedo for all components, at each frequency band. Where,
K98 fixed emissivities to the cloud at certian bands, we have kept all free.

We can improve our error estimates on the ZL parameters by including the 
Powell search history in our error estimation. For each bi-directional 
search, we output both the parameter values and the intermediate $\chi^2$
values which lets us find the 95\% confidence interval in the search. This
is used jointly with the uncertienties determined from the Gibbs chain.
\textcolor{red}{Do this with Duncan}


\section{Improved zodiacal light model}\label{sect:improved-model}
Here we present our best-fit ZL model from having run Commander3 for six independant
Markov Chains for N samples. Our best-fit parameters for the geometrical IPD, emissivities 
and albedo parameters are listed in tables~\ref{table:zodi-params-geo} 
and~\ref{table:zodi-params-spectral}, where they are compared to the values in the K98 model.

% \setlength{\tabcolsep}{28.5pt} % Default value: 6pt
\renewcommand{\arraystretch}{1.5} % Default value: 1
\begin{table*}
    \small
    \centering
    \newcolumntype{C}{ @{}>{${}}r<{{}$}@{} }
    \begin{tabular}{l l *2{rCl}}
    \multicolumn{8}{c}{Interplanetary dust parameters}\\
    \hline
    \hline
     Parameter & Description & \multicolumn{3}{c}{DIRBE} & \multicolumn{3}{c}{DR2} \\ 
     \hline
     \multicolumn{8}{c}{Smooth Cloud}\\
     \hline
     $n_{0, \mathrm{C}}$ [$10^{-8}$ AU$^{-1}$]\dotfill & Number density at 1 AU & 11.3 &\pm& 0.064 & 8.539 & \pm & 0.379\\
     $\alpha$\dotfill & Radial power-law exponent \quad& 1.34 &\pm& 0.022 & 1.282 & \pm & 0.015\\
     $\beta$\dotfill & Vertical shape parameter & 4.14 &\pm& 0.067 & 4.006 & \pm & 0.049\\
     $\gamma$\dotfill & Vertical power-law exponent & 0.942 &\pm& 0.025 & 1.068 & \pm & 0.026\\
     $\mu$\dotfill & Widening parameter & 0.189 &\pm& 0.014 & 0.238 & \pm & 0.013\\
     $i$ [deg]\dotfill & Inclination & 2.03 &\pm& 0.017 & 2.833 & \pm & 0.030\\
     $\Omega$ [deg]\dotfill & Ascending node & 77.7 &\pm& 0.6 & 80.931 & \pm & 0.310\\
     $x_0$ [$10^{-2}$ AU]\dotfill & x-offset from the Sun  & 1.19 &\pm& 0.11 & 1.596 & \pm & 0.104\\
     $y_0$ [$10^{-3}$ AU]\dotfill & y-offset from the Sun &  5.48 &\pm& 0.77 & 9.466 & \pm & 0.806\\
     $z_0$ [$10^{-3}$ AU]\dotfill & z-offset from the Sun & -2.22 &\pm& 0.43 & -5.256 & \pm & 0.301\\
     \hline
     \multicolumn{8}{c}{Dust band 1}\\
     \hline
     $n_{0, \mathrm{B}_1}$ [$10^{-9}$ AU$^{-1}$]\dotfill & Number density at 1 AU & 0.559 &\pm& 0.072 & 7.215 & \pm & 0.790\\
     $\delta_{\zeta_{\mathrm{B}_1}}$ [deg]\dotfill & Shape parameter & 8.78 && Fixed & 9.494 & \pm & 0.142\\
     $v_{\mathrm{B}_1}$\dotfill & Shape parameter & 0.10 && Fixed & 1.486 & \pm & 0.287\\
     $p_{\mathrm{B}_1}$\dotfill & Shape parameter & 4 && Fixed & 4.0007 & \pm & 0.0007\\
     $i_{\mathrm{B}_1}$[deg] \dotfill & Inclination & 0.56 && Fixed & 1.229 & \pm & 0.075\\
     $\Omega_{\mathrm{B}_1}$ [deg]\dotfill & Ascending node & 80 && Fixed & 60.764 & \pm & 2.415\\
     $\delta_{R_{\mathrm{B}_1}}$ [AU]\dotfill & Inner radial cutoff & 1.5 && Fixed & 0.942 & \pm & 0.005\\
     $x_0$ [$10^{-2}$ AU]\dotfill & x-offset from the Sun  &  &&  & -3.003 & \pm & 0.358\\
     $y_0$ [$10^{-2}$ AU]\dotfill & y-offset from the Sun &  &&  & -1.078 & \pm & 0.078\\
     $z_0$ [$10^{-2}$ AU]\dotfill & z-offset from the Sun &  &&  & 1.390 & \pm & 0.103\\
     \hline
     \multicolumn{8}{c}{Dust band 2}\\
     \hline
     $n_{0, \mathrm{B}_2}$ [$10^{-9}$ AU$^{-1}$]\dotfill & Number density at 1 AU & 1.99 &\pm& 0.128 & 3.921 & \pm & 0.253\\
     $\delta_{\zeta_{\mathrm{B}_2}}$ [deg]\dotfill & Shape parameter & 1.99 && Fixed & 2.605 & \pm & 0.0197\\
     $v_{\mathrm{B}_2}$\dotfill & Shape parameter & 0.90 && Fixed & 2.500 & \pm & 0.000009\\
     $p_{\mathrm{B}_2}$\dotfill & Shape parameter & 4 && Fixed & 4.001 & \pm & 0.000004\\
     $i_{\mathrm{B}_2}$ [deg]\dotfill & Inclination & 1.2 && Fixed & 1.690 & \pm & 0.028\\
     $\Omega_{\mathrm{B}_2}$ [deg]\dotfill & Ascending node & 30.3 && Fixed & 44.826 & \pm & 1.377\\
     $\delta_{R_{\mathrm{B}_2}}$ [AU]\dotfill & Inner radial cutoff & 0.94 &\pm& 0.025 & 0.992 & \pm & 0.008\\
     $x_0$ [$10^{-2}$ AU]\dotfill & x-offset from the Sun  &  &&  & -8.313 & \pm & 0.476\\
     $y_0$ [$10^{-2}$ AU]\dotfill & y-offset from the Sun &  &&  & -1.588 & \pm & 0.387\\
     $z_0$ [$10^{-3}$ AU]\dotfill & z-offset from the Sun &  &&  & 8.719 & \pm & 0.455\\
     \hline
     \multicolumn{8}{c}{Dust band 3}\\
     \hline
     $n_{0, \mathrm{B}_3}$ [$10^{-10}$ AU$^{-1}$]\dotfill & Number density at 1 AU & 1.44 &\pm& 0.234 & 2.674 & \pm & 0.185\\
     \hline
     &&&&&&\\
    \end{tabular}
    \caption{Best-fit interplanetary dust parameter estimates and uncertianties in the DR2 analysis,
     comparing values with the K98 model. Only parameters fit in the DR2 analysis are presented.}
    \label{table:zodi-params-geo}
    \end{table*}

% \setlength{\tabcolsep}{10pt} % Default value: 6pt
\begin{table*}
    \small
    \centering
    \newcolumntype{C}{ @{}>{${}}r<{{}$}@{} }
    \begin{tabular}{l l *2{rCl}}
    \multicolumn{8}{c}{Source function parameters}\\
    \hline
    \hline
    Parameter & Description & \multicolumn{3}{l}{DIRBE} & \multicolumn{3}{c}{DR2} \\ 
    \hline
    \multicolumn{8}{c}{All zodiacal components}\\
    \hline
    $T_0$ (K)\dotfill & IPD temperature at 1 AU  & 286 && Fixed & 283.29 &\pm& 2.70\\
    $A_1$ \dotfill & Albedo at 1.25$\mu $m & 0.204 &\pm& 0.0013 & 0.2087 &\pm& 0.0086\\
    $A_2$ \dotfill & Albedo at 2.2$\mu $m & 0.255 &\pm& 0.0017 & 0.2712 &\pm& 0.0109\\
    $A_3$ \dotfill & Albedo at 3.5$\mu $m & 0.210 &\pm& 0.019 & 0.3430 &\pm& 0.0088\\
    $A_4$ \dotfill & Albedo at 4.9$\mu $m  & 0 && Fixed & 0.4496 &\pm& 0.0314\\
    $A_5$ \dotfill & Albedo at 12$\mu $m  & 0 && Fixed & 0 && Fixed\\
    $A_6$ \dotfill & Albedo at 25$\mu $m  & 0 && Fixed & 0 && Fixed\\
    $A_7$ \dotfill & Albedo at 60$\mu $m  & 0 && Fixed & 0 && Fixed\\
    $A_8$ \dotfill & Albedo at 100$\mu $m  & 0 && Fixed & 0 && Fixed\\
    $A_9$ \dotfill & Albedo at 140$\mu $m  & 0 && Fixed & 0 && Fixed\\
    $A_{10}$ \dotfill & Albedo at 240$\mu $m  & 0 && Fixed & 0 && Fixed\\

    \hline
    \multicolumn{8}{c}{Smooth Cloud}\\
    \hline
    $E_1$\dotfill & Emissivity at 1.25$\mu $m  & 1 && Fixed & 1 & & Fixed\\
    $E_2$\dotfill & Emissivity at 2.2$\mu $m  & 1 && Fixed & 1 & & Fixed\\
    $E_3$\dotfill & Emissivity at 3.5$\mu $m  & 1.66 &\pm& 0.088 & 1 & & Fixed\\
    $E_4$\dotfill & Emissivity at 4.9$\mu $m  & 0.997 &\pm& 0.0036 & 1.9002 &\pm& 0.1317\\
    $E_5$\dotfill & Emissivity at 12$\mu $m  & 0.958 &\pm& 0.0026 & 1 & & Fixed\\
    $E_6$\dotfill & Emissivity at 25$\mu $m  &  1 && Fixed & 1.0005 &\pm& 0.0146\\
    $E_7$\dotfill & Emissivity at 60$\mu $m  & 0.733 &\pm& 0.0055 & 0.6471 &\pm& 0.0179\\
    $E_8$\dotfill & Emissivity at 100$\mu $m  & 0.647 &\pm& 0.012 & 0.7121 &\pm& 0.0277\\
    $E_9$\dotfill & Emissivity at 140$\mu $m  & 0.677 &&  & 0.6981 &\pm& 0.1388\\
    $E_{10}$\dotfill & Emissivity at 240$\mu$m  & 0.519 &&  & 0.4950 &\pm& 0.1729\\
    \hline
    \multicolumn{8}{c}{Dust bands}\\
    \hline
    $E_1$\dotfill & Emissivity at 1.25$\mu $m  & 1 && Fixed & 1 & & Fixed\\
    $E_2$\dotfill & Emissivity at 2.2$\mu $m  & 1 && Fixed & 1 & & Fixed\\
    $E_3$\dotfill & Emissivity at 3.5$\mu $m  & 1.66 && Fixed to smooth cloud & 1 & & Fixed\\
    $E_4$\dotfill & Emissivity at 4.9$\mu $m  & 0.359 &\pm& 0.054 & 1.7264 &\pm& 0.1062\\
    $E_5$\dotfill & Emissivity at 12$\mu $m  & 1.01 &\pm& 0.15 & 1 && Fixed\\
    $E_6$\dotfill & Emissivity at 25$\mu $m  & 1 && Fixed & 0.9826 &\pm& 0.0220\\
    $E_7$\dotfill & Emissivity at 60$\mu $m  & 1.25 &\pm& 0.30 & 0.8758  &\pm& 0.0439\\
    $E_8$\dotfill & Emissivity at 100$\mu $m  & 1.52 &\pm& 0.65 & 0.9151 &\pm& 0.0555\\
    $E_9$\dotfill & Emissivity at 140$\mu $m  & 1.13 &&  & 0.5815 &\pm& 0.3701\\
    $E_{10}$\dotfill & Emissivity at 240$\mu $m  & 1.40 &&  & 0.6355 &\pm& 0.3439\\
    \hline
    &&&&&&\\
    \end{tabular}
    \caption{Best-fit source function parameter estimates and uncertianties in the DR2 analysis,
     comparing values with the K98 model. Only parameters fit in the DR2 analysis are presented.}
\label{table:zodi-params-source}
\end{table*}


We illustrate the effectiveness of our model in figures~\ref{fig:dr2-zsma-compare1} 
and~\ref{fig:dr2-zsma-compare2}, which shows the data minus model residuals for both models.
We obtain cleaner residuals in all DIRBE channels with our new ZL model, most notably in the 12 and 25 $\mu$m  
ZL dominated channels.

Figure~\ref{fig:zodi-intensity} shows the interpolated ZL signal as a function of wavelength. The SED will be a composite of many independant 

\textcolor{red}{(Fill in once the zodi paper gets further along and we have our results)}
\begin{figure*}
    \centering

    \resizebox{\textwidth}{!}{%
    \includegraphics[height=1cm]{figs/compare_freq_maps/cosmoglobe_ma_01.pdf}%
    \includegraphics[width=1cm,angle=90]{figs/compare_freq_maps/cbar_tot_01.pdf}%
    \includegraphics[height=1cm]{figs/compare_freq_maps/cosmoglobe_zsma_01.pdf}%
    \includegraphics[height=1cm]{figs/compare_freq_maps/dirbe_zsma_01.pdf}%
    \includegraphics[width=1cm,angle=90]{figs/compare_freq_maps/cbar_01.pdf}%
    }\\


    \resizebox{\textwidth}{!}{%
    \includegraphics[height=1cm]{figs/compare_freq_maps/cosmoglobe_ma_02.pdf}%
    \includegraphics[width=1cm,angle=90]{figs/compare_freq_maps/cbar_tot_02.pdf}%
    \includegraphics[height=1cm]{figs/compare_freq_maps/cosmoglobe_zsma_02.pdf}%
    \includegraphics[height=1cm]{figs/compare_freq_maps/dirbe_zsma_02.pdf}%
    \includegraphics[width=1cm,angle=90]{figs/compare_freq_maps/cbar_02.pdf}%
    }\\

    \resizebox{\textwidth}{!}{%
    \includegraphics[height=1cm]{figs/compare_freq_maps/cosmoglobe_ma_03.pdf}%
    \includegraphics[width=1cm,angle=90]{figs/compare_freq_maps/cbar_tot_03.pdf}%
    \includegraphics[height=1cm]{figs/compare_freq_maps/cosmoglobe_zsma_03.pdf}%
    \includegraphics[height=1cm]{figs/compare_freq_maps/dirbe_zsma_03.pdf}%
    \includegraphics[width=1cm,angle=90]{figs/compare_freq_maps/cbar_03.pdf}%
    }\\

    \resizebox{\textwidth}{!}{%
    \includegraphics[height=1cm]{figs/compare_freq_maps/cosmoglobe_ma_04.pdf}%
    \includegraphics[width=1cm,angle=90]{figs/compare_freq_maps/cbar_tot_04.pdf}%
    \includegraphics[height=1cm]{figs/compare_freq_maps/cosmoglobe_zsma_04.pdf}%
    \includegraphics[height=1cm]{figs/compare_freq_maps/dirbe_zsma_04.pdf}%
    \includegraphics[width=1cm,angle=90]{figs/compare_freq_maps/cbar_04.pdf}%
    }\\

    \resizebox{\textwidth}{!}{%
    \includegraphics[height=1cm]{figs/compare_freq_maps/cosmoglobe_ma_05.pdf}%
    \includegraphics[width=1cm,angle=90]{figs/compare_freq_maps/cbar_tot_05.pdf}%
    \includegraphics[height=1cm]{figs/compare_freq_maps/cosmoglobe_zsma_05.pdf}%
    \includegraphics[height=1cm]{figs/compare_freq_maps/dirbe_zsma_05.pdf}%
    \includegraphics[width=1cm,angle=90]{figs/compare_freq_maps/cbar_05.pdf}%
    }\\

    \resizebox{\textwidth}{!}{%
    \includegraphics[height=1cm]{figs/compare_freq_maps/cosmoglobe_ma_06.pdf}%
    \includegraphics[width=1cm,angle=90]{figs/compare_freq_maps/cbar_tot_06.pdf}%
    \includegraphics[height=1cm]{figs/compare_freq_maps/cosmoglobe_zsma_06.pdf}%
    \includegraphics[height=1cm]{figs/compare_freq_maps/dirbe_zsma_06.pdf}%
    \includegraphics[width=1cm,angle=90]{figs/compare_freq_maps/cbar_06.pdf}%
    }\\

    \resizebox{\textwidth}{!}{%
    \includegraphics[height=1cm]{figs/compare_freq_maps/cosmoglobe_ma_07.pdf}%
    \includegraphics[width=1cm,angle=90]{figs/compare_freq_maps/cbar_tot_07.pdf}%
    \includegraphics[height=1cm]{figs/compare_freq_maps/cosmoglobe_zsma_07.pdf}%
    \includegraphics[height=1cm]{figs/compare_freq_maps/dirbe_zsma_07.pdf}%
    \includegraphics[width=1cm,angle=90]{figs/compare_freq_maps/cbar_07.pdf}%
    }\\
    \caption{Comparison between ZL subtractions with our best-fit ZL model and with K98. 
    \textit{(left column):} Mission-averaged containing ZL at native HEALPix resolution 
    of $N_\mathrm{side} = 512$. \textit{(middle column):} Our ZL subtracted mission-average 
    maps, downgraded to $N_\mathrm{side} = 256$. \textit{(right column):} DIRBE ZSMA maps
    at $N_\mathrm{side} = 256$. The rows list to DIRBE frequency channels, from top to 
    bottom. We observe a clear improved ZL solution at all DIRBE channels.
    }
    \label{fig:dr2-zsma-compare1}
\end{figure*}

\begin{figure*}
    \centering
    \resizebox{\textwidth}{!}{%
    \includegraphics[height=1cm]{figs/compare_freq_maps/cosmoglobe_ma_08.pdf}%
    \includegraphics[width=1cm,angle=90]{figs/compare_freq_maps/cbar_tot_08.pdf}%
    \includegraphics[height=1cm]{figs/compare_freq_maps/cosmoglobe_zsma_08.pdf}%
    \includegraphics[height=1cm]{figs/compare_freq_maps/dirbe_zsma_08.pdf}%
    \includegraphics[width=1cm,angle=90]{figs/compare_freq_maps/cbar_08.pdf}%
    }\\

    \resizebox{\textwidth}{!}{%
    \includegraphics[height=1cm]{figs/compare_freq_maps/cosmoglobe_ma_09.pdf}%
    \includegraphics[width=1cm,angle=90]{figs/compare_freq_maps/cbar_tot_09.pdf}%
    \includegraphics[height=1cm]{figs/compare_freq_maps/cosmoglobe_zsma_09.pdf}%
    \includegraphics[height=1cm]{figs/compare_freq_maps/dirbe_zsma_09.pdf}%
    \includegraphics[width=1cm,angle=90]{figs/compare_freq_maps/cbar_09.pdf}%
    }\\

    \resizebox{\textwidth}{!}{%
    \includegraphics[height=1cm]{figs/compare_freq_maps/cosmoglobe_ma_10.pdf}%
    \includegraphics[width=1cm,angle=90]{figs/compare_freq_maps/cbar_tot_10.pdf}%
    \includegraphics[height=1cm]{figs/compare_freq_maps/cosmoglobe_zsma_10.pdf}%
    \includegraphics[height=1cm]{figs/compare_freq_maps/dirbe_zsma_10.pdf}%
    \includegraphics[width=1cm,angle=90]{figs/compare_freq_maps/cbar_10.pdf}%
    }\\

    \caption{Comparison between ZL subtractions with our best-fit ZL model and with K98. 
    \textit{(left column):} Mission-averaged containing ZL at native HEALPix resolution 
    of $N_\mathrm{side} = 512$. \textit{(middle column):} Our ZL subtracted mission-average 
    maps, downgraded to $N_\mathrm{side} = 256$. \textit{(right column):} DIRBE ZSMA maps
    at $N_\mathrm{side} = 256$. The rows list to DIRBE frequency channels, from top to 
    bottom. We observe a clear improved ZL solution at all DIRBE channels.
    }    
    \label{fig:dr2-zsma-compare2}
\end{figure*}

Subtracting the ZL with our best-fit-model yields lower sky residuals at all bands 
compared to the DIRBE ZSMA maps. A comparison between the ZL residuals can be seen 
in Figure~\ref{fig:dr2-zsma-compare}. Here the first column shows the frequency maps of the
four most ZL dominated DIREB bands (4.9, 12, 25, and 60 $\mu$m), the second shows the ZL
subtracted maps with our model, and column three shows the DIRBE ZSMA.

\begin{figure}
    \centering
    \includegraphics[width=\columnwidth]{figs/zodi_intensity.pdf}
    \caption{Simulated ZL intensity on January 1, 2024 as a function of wavelength from the best-fit 
    Cosmoglobe ZL model, made with ZodiPy. The ZL SED is directional 
    and seasonal due to temperature variations along independent line-of-sights. 
    The black curve shows the mean sky intensity in a HEALPix map with resolution
    $N_\mathrm{side}= 64$ where pixels with an angular separation of less than
    $60^\circ$ are masked out. The colored dashed lines represent the ZL 
    intensity
    }
    \label{fig:zodi-intensity}
\end{figure}

\section{Conclusions}


\begin{acknowledgements}
 The current work has received funding from the European
  Union’s Horizon 2020 research and innovation programme under grant
  agreement numbers 819478 (ERC; \textsc{Cosmoglobe}) and 772253 (ERC;
  \textsc{bits2cosmology}). Some of the results in this paper have been 
  derived using the HEALPix \citep{Gorski2005} package.
  We acknowledge the use of the Legacy Archive for Microwave Background 
  Data
  Analysis (LAMBDA), part of the High Energy Astrophysics Science
  Archive Center
  (HEASARC). HEASARC/LAMBDA is a service of the Astrophysics Science 
  Division at the NASA Goddard Space Flight Center.  
\end{acknowledgements}


%-------------------------------------------------------------
%                                       Table with references 
%-------------------------------------------------------------
%

\bibliographystyle{aa}
% \bibliography{../../common/CG_bibliography,references,../../common/Planck_bib}
\bibliography{references}

\appendix
\onecolumn

\section{Interplanetary dust atlas}
\label{sec:param-atlas}
In this Appendix, we present an atlas of mission-averaged ZL parameter maps.
Each map represents the effect of changing one ZL model parameter by $\pm 5\%$ while holding the other fixed. The maps are normalized.

\begin{figure*}[hbt]
    \centering
    \resizebox{0.82\textwidth}{!}{%
    \includegraphics[height=1cm]{figs/atlas/C_01.pdf}%
    \includegraphics[height=1cm]{figs/atlas/B1_01.pdf}%
    \includegraphics[height=1cm]{figs/atlas/B2_01.pdf}%
    \includegraphics[height=1cm]{figs/atlas/B3_01.pdf}%
    }\\
    \resizebox{0.82\textwidth}{!}{%
    \includegraphics[height=1cm]{figs/atlas/C_02.pdf}%
    \includegraphics[height=1cm]{figs/atlas/B1_02.pdf}%
    \includegraphics[height=1cm]{figs/atlas/B2_02.pdf}%
    \includegraphics[height=1cm]{figs/atlas/B3_02.pdf}%
    }\\
    \resizebox{0.82\textwidth}{!}{%
    \includegraphics[height=1cm]{figs/atlas/C_03.pdf}%
    \includegraphics[height=1cm]{figs/atlas/B1_03.pdf}%
    \includegraphics[height=1cm]{figs/atlas/B2_03.pdf}%
    \includegraphics[height=1cm]{figs/atlas/B3_03.pdf}%
    }\\
    \resizebox{0.82\textwidth}{!}{%
    \includegraphics[height=1cm]{figs/atlas/C_04.pdf}%
    \includegraphics[height=1cm]{figs/atlas/B1_04.pdf}%
    \includegraphics[height=1cm]{figs/atlas/B2_04.pdf}%
    \includegraphics[height=1cm]{figs/atlas/B3_04.pdf}%
    }\\
    \resizebox{0.82\textwidth}{!}{%
    \includegraphics[height=1cm]{figs/atlas/C_05.pdf}%
    \includegraphics[height=1cm]{figs/atlas/B1_05.pdf}%
    \includegraphics[height=1cm]{figs/atlas/B2_05.pdf}%
    \includegraphics[height=1cm]{figs/atlas/B3_05.pdf}%
    }\\
    \resizebox{0.82\textwidth}{!}{%
    \includegraphics[height=1cm]{figs/atlas/C_06.pdf}%
    \includegraphics[height=1cm]{figs/atlas/B1_06.pdf}%
    \includegraphics[height=1cm]{figs/atlas/B2_06.pdf}%
    \includegraphics[height=1cm]{figs/atlas/B3_06.pdf}%
    }\\
    \resizebox{0.82\textwidth}{!}{%
    \includegraphics[height=1cm]{figs/atlas/C_07.pdf}%
    \includegraphics[height=1cm]{figs/atlas/B1_07.pdf}%
    \includegraphics[height=1cm]{figs/atlas/B2_07.pdf}%
    \includegraphics[height=1cm]{figs/atlas/B3_07.pdf}%
    }\\
    \resizebox{0.82\textwidth}{!}{%
    \includegraphics[height=1cm]{figs/atlas/C_08.pdf}%
    \includegraphics[height=1cm]{figs/atlas/B1_08.pdf}%
    \includegraphics[height=1cm]{figs/atlas/B2_08.pdf}%
    \includegraphics[height=1cm]{figs/atlas/B3_08.pdf}%
    }\\
    \resizebox{0.82\textwidth}{!}{%
    \includegraphics[height=1cm]{figs/atlas/C_09.pdf}%
    \includegraphics[height=1cm]{figs/atlas/B1_09.pdf}%
    \includegraphics[height=1cm]{figs/atlas/B2_09.pdf}%
    \includegraphics[height=1cm]{figs/atlas/B3_09.pdf}%
    }\\
    \resizebox{0.82\textwidth}{!}{%
    \includegraphics[height=1cm]{figs/atlas/C_10.pdf}%
    \includegraphics[height=1cm]{figs/atlas/B1_10.pdf}%
    \includegraphics[height=1cm]{figs/atlas/B2_10.pdf}%
    \includegraphics[height=1cm]{figs/atlas/B3_10.pdf}%
    }\\
   
    \caption{ZL parameter atlas showing the difference between increasing 
    and lowering each ZL model parameter by $5\%$ in the form of normalized 
    mission-averaged ZL maps. Columns list, from left to right parameters of
    1) the smooth cloud; 2) dust band 1; 3) dust band 2; and 4) dust band 3.}
    \label{fig:atlas1}
\end{figure*}
\begin{figure*}
    \centering
    \resizebox{0.82\textwidth}{!}{%
    \includegraphics[height=1cm]{figs/atlas/SR_01.pdf}%
    \includegraphics[height=1cm]{figs/atlas/SR_06.pdf}%
    \includegraphics[height=1cm]{figs/atlas/TF_01.pdf}%
    \includegraphics[height=1cm]{figs/atlas/TF_06.pdf}%
    }\\   
    \resizebox{0.82\textwidth}{!}{%
    \includegraphics[height=1cm]{figs/atlas/SR_02.pdf}%
    \includegraphics[height=1cm]{figs/atlas/SR_07.pdf}%
    \includegraphics[height=1cm]{figs/atlas/TF_02.pdf}%
    \includegraphics[height=1cm]{figs/atlas/TF_07.pdf}%
    }\\
    \resizebox{0.82\textwidth}{!}{%
    \includegraphics[height=1cm]{figs/atlas/SR_03.pdf}%
    \includegraphics[height=1cm]{figs/atlas/SR_08.pdf}%
    \includegraphics[height=1cm]{figs/atlas/TF_03.pdf}%
    \includegraphics[height=1cm]{figs/atlas/TF_08.pdf}%
    }\\
    \resizebox{0.82\textwidth}{!}{%
    \includegraphics[height=1cm]{figs/atlas/SR_04.pdf}%
    \includegraphics[height=1cm]{figs/atlas/SR_09.pdf}%
    \includegraphics[height=1cm]{figs/atlas/TF_04.pdf}%
    \includegraphics[height=1cm]{figs/atlas/TF_09.pdf}%
    }\\
    \resizebox{0.82\textwidth}{!}{%
    \includegraphics[height=1cm]{figs/atlas/SR_05.pdf}%
    \includegraphics[height=1cm]{figs/atlas/TF_05.pdf}%
    \includegraphics[height=1cm]{figs/atlas/TF_10.pdf}%
    \includegraphics[height=1cm]{figs/atlas/TF_11.pdf}%
    }\\
    \resizebox{0.41\textwidth}{!}{%
    \includegraphics[height=1cm]{figs/atlas/general_01.pdf}%
    \includegraphics[height=1cm]{figs/atlas/general_02.pdf}%
    }
    %\hspace*{0.47\textwidth}
    \caption{ZL parameter atlas showing the difference between increasing 
    and lowering each ZL model parameter by $5\%$ in the form of normalized 
    mission-averaged ZL maps. Columns list, from left to right parameters of
    1) the smooth cloud; 2) dust band 1; 3) dust band 2; and 4) dust band 3.}
    \label{fig:atlas2}
\end{figure*}

\clearpage
\section{Half-mission residuals}
\label{sec:zodi-comps}

\begin{figure*}[hbt]
    \centering
    \resizebox{0.98\textwidth}{!}{%
    \includegraphics[height=1cm]{figs/compare_zodi_res/cosmoglobe_res_01a.pdf}%
    \includegraphics[height=1cm]{figs/compare_zodi_res/cosmoglobe_res_01b.pdf}%
    \includegraphics[width=1cm,angle=90]{figs/compare_zodi_res/cbar_01.pdf}%
    }\\
    \resizebox{0.98\textwidth}{!}{%
    \includegraphics[height=1cm]{figs/compare_zodi_res/cosmoglobe_res_02a.pdf}%
    \includegraphics[height=1cm]{figs/compare_zodi_res/cosmoglobe_res_02b.pdf}%
    \includegraphics[width=1cm,angle=90]{figs/compare_zodi_res/cbar_02.pdf}%
    }\\
    \resizebox{0.98\textwidth}{!}{%
    \includegraphics[height=1cm]{figs/compare_zodi_res/cosmoglobe_res_03a.pdf}%
    \includegraphics[height=1cm]{figs/compare_zodi_res/cosmoglobe_res_03b.pdf}%
    \includegraphics[width=1cm,angle=90]{figs/compare_zodi_res/cbar_03.pdf}%
    }\\
    \resizebox{0.98\textwidth}{!}{%
    \includegraphics[height=1cm]{figs/compare_zodi_res/cosmoglobe_res_04a.pdf}%
    \includegraphics[height=1cm]{figs/compare_zodi_res/cosmoglobe_res_04b.pdf}%
    \includegraphics[width=1cm,angle=90]{figs/compare_zodi_res/cbar_04.pdf}%
    }\\
    \resizebox{0.98\textwidth}{!}{%
    \includegraphics[height=1cm]{figs/compare_zodi_res/cosmoglobe_res_05a.pdf}%
    \includegraphics[height=1cm]{figs/compare_zodi_res/cosmoglobe_res_05b.pdf}%
    \includegraphics[width=1cm,angle=90]{figs/compare_zodi_res/cbar_05.pdf}%
    }\\
    \caption{Half-mission split residual maps, smoothed by a 0.5 degrees or 15 arcmin beam.}
    \label{fig:half-mission-res1}
\end{figure*}

\begin{figure*}[hbt]
    \centering
    \resizebox{0.98\textwidth}{!}{%
    \includegraphics[height=1cm]{figs/compare_zodi_res/cosmoglobe_res_06a.pdf}%
    \includegraphics[height=1cm]{figs/compare_zodi_res/cosmoglobe_res_06b.pdf}%
    \includegraphics[width=1cm,angle=90]{figs/compare_zodi_res/cbar_06.pdf}%
    }\\
    \resizebox{0.98\textwidth}{!}{%
    \includegraphics[height=1cm]{figs/compare_zodi_res/cosmoglobe_res_07a.pdf}%
    \includegraphics[height=1cm]{figs/compare_zodi_res/cosmoglobe_res_07b.pdf}%
    \includegraphics[width=1cm,angle=90]{figs/compare_zodi_res/cbar_07.pdf}%
    }\\
    \resizebox{0.98\textwidth}{!}{%
    \includegraphics[height=1cm]{figs/compare_zodi_res/cosmoglobe_res_08a.pdf}%
    \includegraphics[height=1cm]{figs/compare_zodi_res/cosmoglobe_res_08b.pdf}%
    \includegraphics[width=1cm,angle=90]{figs/compare_zodi_res/cbar_08.pdf}%
    }\\
    \resizebox{0.98\textwidth}{!}{%
    \includegraphics[height=1cm]{figs/compare_zodi_res/cosmoglobe_res_09a.pdf}%
    \includegraphics[height=1cm]{figs/compare_zodi_res/cosmoglobe_res_09b.pdf}%
    \includegraphics[width=1cm,angle=90]{figs/compare_zodi_res/cbar_09.pdf}%
    }\\
    \resizebox{0.98\textwidth}{!}{%
    \includegraphics[height=1cm]{figs/compare_zodi_res/cosmoglobe_res_10a.pdf}%
    \includegraphics[height=1cm]{figs/compare_zodi_res/cosmoglobe_res_10b.pdf}%
    \includegraphics[width=1cm,angle=90]{figs/compare_zodi_res/cbar_10.pdf}%
    }\\
    \caption{Half-mission split residual maps, smoothed by a 0.5 degrees or 15 arcmin beam.}
    \label{fig:half-mission-res2}
\end{figure*}

\clearpage
\section{Component-wise maps and number density}
\label{sec:zodi-comps}

% \noindent\begin{minipage}{\textwidth}
In this Appendix, we present maps of visualizations of our best-fit ZL 
light model. Such figures can help illustrate the physical properties 
of the model and help validate how physical our models are. The ZL 
component-wise maps, both the mission-averaged and the instantaneous maps 
in~\ref{fig:mission-averaged-comp-maps} 
and~\ref{fig:mission-averaged-inst-maps}, respectively, are compared to 
the K98 model. The IPD number density visualization for the K98 model, 
corresponding to Figure~\ref{fig:ipd-number-density} can be seen in 
Figure~X in~\cite{San2022}.


% \end{minipage}

\begin{figure*}[hbt]
    \centering
    \resizebox{\textwidth}{!}{%
    \includegraphics[height=1cm]{figs/comp_maps/CG_0.pdf}%
    \includegraphics[height=1cm]{figs/comp_maps/K98_0.pdf}%
    \includegraphics[width=1cm,angle=90]{figs/comp_maps/cbar_0.pdf}%
    }\\
    \resizebox{\textwidth}{!}{%
    \includegraphics[height=1cm]{figs/comp_maps/CG_1.pdf}%
    \includegraphics[height=1cm]{figs/comp_maps/K98_1.pdf}%
    \includegraphics[width=1cm,angle=90]{figs/comp_maps/cbar_1.pdf}%
    }\\
    \resizebox{\textwidth}{!}{%
    \includegraphics[height=1cm]{figs/comp_maps/CG_2.pdf}%
    \includegraphics[height=1cm]{figs/comp_maps/K98_2.pdf}%
    \includegraphics[width=1cm,angle=90]{figs/comp_maps/cbar_2.pdf}%
    }\\
    \resizebox{\textwidth}{!}{%
    \includegraphics[height=1cm]{figs/comp_maps/CG_3.pdf}%
    \includegraphics[height=1cm]{figs/comp_maps/K98_3.pdf}%
    \includegraphics[width=1cm,angle=90]{figs/comp_maps/cbar_3.pdf}%
    }\\
    \caption{Mission-averaged component-wise ZL maps at $25\mu$m made with ZodiPy. 
    \textit{(left column:)} Best-fit Cosmoglobe ZL model. \textit{(right column:)} The K98 model.
    Rows list the zodiacal components, from top to bottom, 1) smooth cloud; 2) dust band 1; 3) 
    dust band 2; 4) dust band 3. The maps are in galactic coordinates.}
    \label{fig:mission-averaged-comp-maps}
\end{figure*}

\begin{figure*}
    \centering
    \resizebox{\textwidth}{!}{%
    \includegraphics[height=1cm]{figs/comp_maps/CG_0_inst.pdf}%
    \includegraphics[height=1cm]{figs/comp_maps/K98_0_inst.pdf}%
    \includegraphics[width=1cm,angle=90]{figs/comp_maps/cbar_0_inst.pdf}%
    }\\
    \resizebox{\textwidth}{!}{%
    \includegraphics[height=1cm]{figs/comp_maps/CG_1_inst.pdf}%
    \includegraphics[height=1cm]{figs/comp_maps/K98_1_inst.pdf}%
    \includegraphics[width=1cm,angle=90]{figs/comp_maps/cbar_1_inst.pdf}%
    }\\
    \resizebox{\textwidth}{!}{%
    \includegraphics[height=1cm]{figs/comp_maps/CG_2_inst.pdf}%
    \includegraphics[height=1cm]{figs/comp_maps/K98_2_inst.pdf}%
    \includegraphics[width=1cm,angle=90]{figs/comp_maps/cbar_2_inst.pdf}%
    }\\
    \resizebox{\textwidth}{!}{%
    \includegraphics[height=1cm]{figs/comp_maps/CG_3_inst.pdf}%
    \includegraphics[height=1cm]{figs/comp_maps/K98_3_inst.pdf}%
    \includegraphics[width=1cm,angle=90]{figs/comp_maps/cbar_3_inst.pdf}%
    }\\
    \caption{Full-sky component-wise ZL maps (January 1, 2024) at $25\mu$m made with ZodiPy. 
    \textit{(left column:)} Best-fit Cosmoglobe ZL model. \textit{(right column:)} The K98 model.
    Rows list the zodiacal components, from top to bottom, 1) smooth cloud; 2) dust band 1; 3) 
    dust band 2; 4) dust band 3. The maps are in ecliptic coordinates, with the Sun marked as 
    an orange circle.}
    \label{fig:mission-averaged-inst-maps}
\end{figure*}

\begin{figure*}
    \centering
    \includegraphics[width=\textwidth]{figs/number_density.pdf}
    \caption{Visualization of the IPD number density of the four fitted zodiacal components in our model. The number densities are shown as a cross-section of the Solar system in the xz-plane. \textit{(top left):} The smooth cloud. \textit{(top right):} Dust band 1. \textit{(bottom left):} Dust band 2. \textit{(bottom right):} Dust band 3. The gray dotted line represents the ecliptic plane and helps illustrate the variations in the components symmetry planes.}
    \label{fig:ipd-number-density}
\end{figure*}


\end{document}
%%%% End of aa.dem
